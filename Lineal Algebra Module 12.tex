\documentclass[a4paper,12pt]{article}

% Кодировка и язык
\usepackage[utf8]{inputenc}
\usepackage[russian]{babel}

% Математические пакеты
\usepackage{amsmath,amsfonts,amssymb}

% Графика
\usepackage{graphicx}
\usepackage{tikz}
\usetikzlibrary{shapes.geometric, calc}
\usepackage{pgfplots}
\pgfplotsset{compat=1.18} % Добавлено для устранения предупреждения

% Геометрия страницы
\usepackage{geometry}
\geometry{top=2cm, bottom=2cm, left=2.5cm, right=2.5cm}

% Гиперссылки
\usepackage{hyperref}

% Плавающие объекты
\usepackage{float}

% Дополнительные пакеты
\usepackage{venndiagram}

% Настройки заголовка
\title{Домашнее задание}
\author{Студент: \textbf{Ваше Имя Фамилия}}
\date{\today}

\begin{document}

% Титульный лист
\begin{titlepage}
    \centering
    \vspace*{1cm}

    \Huge
    \textbf{Домашнее задание}

    \vspace{0.5cm}
    \LARGE
    По курсу: \textbf{Название Курса}

    \vspace{1.5cm}

    \textbf{Студент: Ростислав Лохов}

    \vfill

    \Large
    АНО ВО Центральный университет\\
    \vspace{0.3cm}
    \today

\end{titlepage}

% Содержание
\tableofcontents
\newpage

% Основной текст
\section{Объёмы и движения в евклидовом пространстве}

\subsection{Задача 1}

\[
G = \begin{pmatrix}
    7 & -4 \\
    -4 & 7 \\
\end{pmatrix}
\]

\[
V_1 = \sqrt{\det(G)} = \sqrt{33}
\]

\[
G = \begin{pmatrix}
    7 & -4 & -3 \\
    -4 & 7 & -1 \\
    -3 & -1 & 4 \\
\end{pmatrix}
\]

\[
V_2 = \sqrt{\det(G)} = \sqrt{38}
\]

\subsection{Задача 3}

а) \( ||v_1|| = \sqrt{15} \land ||v_2|| = \sqrt{13} \)

б) \( \cos(\varphi) = \arccos\left(\frac{-7}{\sqrt{195}}\right) \)

в) \( ||v_1 - v_2|| = \sqrt{42} \)

\subsection{Задача 7}

Пусть 
\[
v_1 = \begin{pmatrix}
    2 \\ 0 \\ 0 \\ 0
\end{pmatrix}
\]
Тогда 
\[
v_2 = \begin{pmatrix}
    e \\ f \\ g \\ h
\end{pmatrix}
\]

\[
\langle v_1, v_2 \rangle = 2e = 3 \Rightarrow e = 1.5 \land \langle v_2, v_2 \rangle = f^2 + g^2 + h^2 = 4.75 \Rightarrow h = 0 \land f = \frac{\sqrt{19}}{2} \land g = 0
\]

\[
v_2 = \begin{pmatrix}
    1.5 \\ \frac{\sqrt{19}}{2} \\ 0 \\ 0
\end{pmatrix}
\]

\[
v_3 = \begin{pmatrix}
    k \\ l \\ m \\ n
\end{pmatrix}
\]

\[
\langle v_1, v_3 \rangle = 1 \Rightarrow k = 0.5 \land \langle v_3, v_3 \rangle = 7 \Rightarrow l^2 + m^2 + n^2 = 6.75
\]

\[
v_3 = \begin{pmatrix} 0.5 \\ \dfrac{13}{2\sqrt{19}} \\ \dfrac{\sqrt{1634}}{38} \\ 0 \end{pmatrix}
\]

\subsection{Задача 8}

\[
\det(A-\lambda I) = 0 \Rightarrow 3\lambda^2 +2\lambda^2 -2\lambda - 3 = 0 \Rightarrow \lambda_1 = 1, \quad \lambda_2 = \dfrac{-5 + i\sqrt{11}}{6}, \quad \lambda_3 = \dfrac{-5 - i\sqrt{11}}{6}
\]

\[
v_1 = \begin{pmatrix}
    1 \\ 3 \\ 1
\end{pmatrix} \Rightarrow e_1 = \begin{pmatrix}
    \dfrac{1}{\sqrt{11}} \\[6pt]
    \dfrac{3}{\sqrt{11}} \\[6pt]
    \dfrac{1}{\sqrt{11}}
\end{pmatrix}
\]
Отбросим комплексное решение, поскольку поворот в вещественном поле. Тогда выберем произвольные векторы ортогональные \( e_1 \) и ортонормируем с помощью Грама-Шмидта.

Пусть \( u_1 = \begin{pmatrix} 1 \\ 0 \\ -1 \end{pmatrix} \), тогда \( \langle e_1, u_1 \rangle = 0 \), т.е. ортогонален, после нормирования получим \( e_2 = \begin{pmatrix} \dfrac{1}{\sqrt{2}} \\ 0 \\ -\dfrac{1}{\sqrt{2}} \end{pmatrix} \).

Пусть \( u_3 = \begin{pmatrix} -3 \\ 2 \\ -3 \end{pmatrix} \), тогда \( \langle e_1, u_3 \rangle = 0 \), нормируем и составляем матрицу ортонормированного оператора в каноническом виде. 

\[
Q =
\begin{pmatrix}
\dfrac{1}{\sqrt{11}} & \dfrac{1}{\sqrt{2}} & -\dfrac{3}{\sqrt{22}} \\
\dfrac{3}{\sqrt{11}} & 0 & \dfrac{2}{\sqrt{22}} \\
\dfrac{1}{\sqrt{11}} & -\dfrac{1}{\sqrt{2}} & -\dfrac{3}{\sqrt{22}}
\end{pmatrix}
\]

\[
A' = Q^T A Q
\]

\[
A' =
\begin{pmatrix}
1 & 0 & 0 \\
0 & -\dfrac{5}{6} & -\dfrac{\sqrt{11}}{6} \\
0 & \dfrac{\sqrt{11}}{6} & -\dfrac{5}{6}
\end{pmatrix}
\]

\subsection{Задача 9}

\[
Av_1 = v_2
\]

Нормализируем вектора, получаем третий вектор как ортогональный первым двум, находится через векторное произведение. Строим базис.

\[
A =
\begin{bmatrix}
-\dfrac{2\sqrt{17}}{17} & \dfrac{5\sqrt{1122}}{1122} & \dfrac{7\sqrt{66}}{66} \\
\dfrac{3\sqrt{17}}{17} & -\dfrac{8\sqrt{1122}}{561} & \dfrac{2\sqrt{66}}{33} \\
\dfrac{2\sqrt{17}}{17} & \dfrac{29\sqrt{1122}}{1122} & \dfrac{\sqrt{66}}{66}
\end{bmatrix}
\]

\subsection{Задача 10}

Матрицы неотрицательны, нормированны, также должно выполняться \( Q^T Q = I \), таким образом данным условиям удовлетворяют только перестановочные матрицы.

\subsection{Задача 11}

\[
G = \begin{pmatrix}
2 & 0 \\
0 & \dfrac{2}{3} \\
\end{pmatrix}.
\]

\[
V = \sqrt{\det(G)} = \sqrt{\dfrac{4}{3}}
\]

\[
G = \begin{pmatrix}
2 & 0 & \dfrac{2}{3} \\
0 & \dfrac{2}{3} & 0 \\
\dfrac{2}{3} & 0 & \dfrac{2}{5} \\
\end{pmatrix}.
\]

\[
V = \sqrt{\det(G)} = \sqrt{\dfrac{32}{135}}
\]

\subsection{Задача 12}

Для существования скалярного произведения необходимо, чтобы матрица \( G \) была положительно определена. Проверим, \( \det(G) = 0 \).

В таком случае существует линейная зависимость матрицы Грама, таким образом один вектор может быть выражен через другие, однако произведение по определению положительно определено. Если появляется ненулевой вектор с нулевой нормой, то это не скалярное произведение.

\subsection{Задача 13}

\[
\det(v_1|v_2) = -7
\]

\[
\det(BA) = 0
\]

Просуммируем главные миноры второго порядка для определения коэффициентов при втором члене характеристического многочлена, получаем, что все собственные значения - 0, 2, -2. Мы знаем, что характеристический многочлен от коммутирующих матриц не меняется за исключением нулей. Тогда:

\[
\det(v_1|v_2) \det(AB) = 28
\]

\subsection{Задача 14}

Длины \( v_1 \) и \( u_2 \) совпадают. Длины \( v_2 \) и \( u_2 \) не совпадают. Проверим \( v_3 \) и \( u_2 \) - совпадают. Проверим угол между \( v_1 \) и \( v_3 \): \( \varphi = 39^\circ \Rightarrow \cos(\varphi) = \dfrac{39}{\sqrt{14}\sqrt{114}} \).

Проверим угол для \( u_1 \) и \( u_2 \): \( \cos(\varphi) = \dfrac{39}{\sqrt{114}\sqrt{14}} \) совпадает. Тогда можно построить ортонормированные базы и дополнить до базиса. Тогда существует ортогональная матрица \( R \), такая что \( Rv_1 = u_1 \) и \( Rv_3 = u_2 \). Образы \( v_2 \) и \( v_4 \) определяются тем же ортогональным преобразованием \( R \). Однако выбор третьего направления не единственен, так как мы завершим базис до ортонормированного. Следовательно, \( R \) не единственно определен по предыдущим условиям. Следовательно, решение существует и оно не единственно.

\end{document}
