\documentclass[a4paper,12pt]{article}

% Кодировка и язык
\usepackage[utf8]{inputenc}
\usepackage[russian]{babel}

% Математические пакеты
\usepackage{amsmath,amsfonts,amssymb}

% Графика
\usepackage{graphicx}
\usepackage{tikz}
\usetikzlibrary{shapes.geometric, calc}
\usepackage{pgfplots}

% Геометрия страницы
\usepackage{geometry}
\geometry{top=2cm, bottom=2cm, left=2.5cm, right=2.5cm}

% Гиперссылки
\usepackage{hyperref}

% Плавающие объекты
\usepackage{float}

% Дополнительные пакеты
\usepackage{venndiagram}

% Настройки заголовка
\title{Домашнее задание}
\author{Студент: \textbf{Лохов Ростислав}}
\date{\today}

\begin{document}

% Титульный лист
\begin{titlepage}
    \centering
    \vspace*{1cm}

    \Huge
    \textbf{Домашнее задание}

    \vspace{0.5cm}
    \LARGE
    По курсу: \textbf{Математический Анализ}

    \vspace{1.5cm}

    \textbf{Студент: Ростислав Лохов}

    \vfill

    \Large
    АНО ВО Центральный университет\\
    \vspace{0.3cm}
    \today

\end{titlepage}

% Содержание
\tableofcontents
\newpage

% Основной текст
\section{Правило Лопиталя}

\subsection{Задача 4}
\[
\lim_{x \to 1} \frac{70x^9-70}{20x^4-20} = \lim_{x \to 1} \frac{630x^8}{80x^3} = \frac{63}{8}
\]

\subsection{Задача 5}
\[
\lim_{x \to 0} \frac{(1+x)^{-1}-1}{3x^2} = \frac{-1}{3}
\]

\subsection{Задача 6}
\[
\lim_{x \to 0} \frac{\cos(x) - (\cos(x) - x\sin(x))}{3\sin^2(x)\cos(x)} = \frac{1}{3}
\]

\subsection{Задача 7}
\[
\frac{2x^3}{1/3x^3} = 6
\]

\subsection{Задача 8}
\[
-2\arcsin(\sqrt{2}-1) = \frac{-\pi}{4}
\]

\subsection{Задача 9}
\[
\frac{2x + cos(x)}{2x-cos(x)}
\]

\[
\frac{2 + sin(x)}{2-sin(x)}
\]
т.к на бесконечности sinx не определен, т.е осцилирует, в таком случае правило Лопеталя не работает, продолжая

\subsection{Задача 10}
\[
\lim_{x \to \infty} \frac{2\ln(x)/x}{0.5x^{-0.5}} = 0 
\]

\subsection{Задача 11}
\[
\lim_{x \to \infty}\frac{a!}{b^ae^{bx}} = 0
\]

\subsection{Задача 12}
\[
\lim_{x \to \infty} \frac{a!}{b} \frac{1}{x^b} = 0
\]

\subsection{Задача 13}

Предела не существует т.к показатель $e^{-1/x^3}$ зависит от знака, т.е левостороний предел не равен правостороннему, разрыв второго рода.

\subsection{Задача 14}
Заметим, что числитель при данном пределе равен нулю, следовательно функция непрерывна в нуле. Рассмотрим первую производную: 
\[
\frac{de^{-1/x^2}}{dx} = \frac{2}{x^3}e^{-1/x^2}
\]

\[
\lim_{t \to \infty}|f'(x)| = |2t^3e^{-t^2}| = 0
\]

ММИ:

Предположим, что для всех производных до порядка n-1 включительно верно $f^{(k)}(0)=0$ Покажем, что $f^{(n)}(0) = 0$

По определению: 

\[
f^{(n)}(0) = \lim_{x \to 0} \frac{f^{(n-1)}(x) - f^{(n-1)}(0)}{x-0}.
\]

По предположению индукции:
\[
f^{(n)}(0) = \lim_{x \to 0}\frac{f^{(n-1)}(x)}{x}.
\]

\[
f^{(n-1)}(x) = P_{n-1}(1/x) e^{-1/x^2}
\]

где $P_{n-1}$ - любой многочлен степени n-1

\[
    f^{(n)}(0) = \lim_{x \to 0}\frac{P_{n-1}(1/x)e^{-1/x^2}}{x} = \lim_{x \to 0}P_{n-1}(1/x)e^{-1/x^2}\frac{1}{x}.
\]

\[
    f^{(n)}(0) = \lim_{x \to 0}\frac{P_{n-1}(1/x)e^{-1/x^2}}{x} = \lim_{x \to 0}P_{n-1}(1/x)e^{-1/x^2}\frac{1}{x}.
\]

\[
    f^{(n)}(0) = \lim_{t \to \infty} P_{n-1}(t) e^{-t^2} t.
\]

т.к экспонента убывает быстрее любой полиномиальной функции:
\[
    \lim_{t \to \infty} P_{n-1}(t)t e^{-t^2} = 0.
\]

Следовательно 
\[
    f^{(n)}(0) = 0
\]

\subsection{Задача 15}

\[
\lim_{x \to \pi/2} \tan(x)\ln(\sin(x))
\]

Сделаем замену

\[
\lim_{t \to 0} \frac{1}{t}(-\frac{t^2}{2}) = 0
\]

cледовательно предел 1, т.к мы искали предел экспоненты

\subsection{Задача 16}
Логарифмируем
\[
    \ln L = \lim_{x \to \infty} \frac{\ln\left( x + \sqrt{x^2 + 1} \right)}{\ln x}
\]

\[
    \ln L = \lim_{x \to \infty} \frac{\frac{1 + \frac{x}{\sqrt{x^2 + 1}}}{x + \sqrt{x^2 + 1}}}{\frac{1}{x}} = \lim_{x \to \infty} \frac{\left(1 + \frac{x}{\sqrt{x^2 + 1}}\right) \cdot x}{x + \sqrt{x^2 + 1}}
\]

\[
    \ln L = \lim_{x \to \infty} \frac{2x}{2x} = 1
\]

\[
    L = e^{\ln L} = e^1 = e
\]

\subsection{Задача 17}
Используем разложение Тейлора около а до второго порядка

числитель:
\begin{align*}
    &\left[ f(a) + 3h f'(a) + \frac{9h^2}{2} f''(a) + o(h^2) \right] \\
    &\quad - 3\left[ f(a) + 2h f'(a) + 2h^2 f''(a) + o(h^2) \right] \\
    &\quad + 3\left[ f(a) + h f'(a) + \frac{h^2}{2} f''(a) + o(h^2) \right] \\
    &\quad - f(a) = o(h^2)
\end{align*}

Тогда:

\[
    \lim_{h \to 0} \frac{o(h^2)}{h^2} = 0
\]

\subsection{Задача 18}
\[
    \text{Sector}(QPR) = \frac{r^2 \theta}{2} \land B(\theta) = \frac{1}{2} (r)(r)\sin\theta = \frac{r^2}{2}\sin\theta
\]

\[
    A(\theta) = \text{Sector}(QPR) - \text{Triangle}(PQR) = \frac{r^2 \theta}{2} - \frac{r^2 \sin\theta}{2} = \frac{r^2}{2}(\theta - \sin\theta)
\]

\[
    \frac{A(\theta)}{B(\theta)} = \frac{\frac{r^2}{2}(\theta - \sin\theta)}{\frac{r^2}{2}\sin\theta} = \frac{\theta - \sin\theta}{\sin\theta}
\]

Тейлор около 0:

\[
    \sin\theta = \theta - \frac{\theta^3}{6} + o(\theta^3)
\]


\[
    \theta - \sin\theta = \theta - \left(\theta - \frac{\theta^3}{6} + o(\theta^3)\right) = \frac{\theta^3}{6} + o(\theta^3)
\]

\[
    \frac{\theta - \sin\theta}{\sin\theta} = \frac{\frac{\theta^3}{6} + o(\theta^3)}{\theta - \frac{\theta^3}{6} + o(\theta^3)}.
\]

\[
    \frac{\theta - \sin\theta}{\sin\theta} \approx \frac{\frac{\theta^3}{6}}{\theta} = \frac{\theta^2}{6}.
\]

\[
    \frac{\theta - \sin\theta}{\sin\theta} \approx \frac{\frac{\theta^3}{6}}{\theta} = \frac{\theta^2}{6}.
\]

\[
    \lim_{\theta \to 0} \frac{A(\theta)}{B(\theta)} = \lim_{\theta \to 0} \frac{\theta^2}{6} = 0.
\]
Как будто бы очев
\end{document}
