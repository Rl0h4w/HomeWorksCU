\documentclass[a4paper,12pt]{article}

% Кодировка и язык
\usepackage[utf8]{inputenc}
\usepackage[russian]{babel}

% Математические пакеты
\usepackage{amsmath,amsfonts,amssymb}

% Графика
\usepackage{graphicx}
\usepackage{tikz}
\usetikzlibrary{shapes.geometric, calc}
\usepackage{pgfplots}

% Геометрия страницы
\usepackage{geometry}
\geometry{top=2cm, bottom=2cm, left=2.5cm, right=2.5cm}

% Гиперссылки
\usepackage{hyperref}

% Плавающие объекты
\usepackage{float}

% Дополнительные пакеты
\usepackage{venndiagram}

% Настройки заголовка
\title{Домашнее задание}
\author{Студент: \textbf{Ваше Имя Фамилия}}
\date{\today}

\begin{document}

% Титульный лист
\begin{titlepage}
    \centering
    \vspace*{1cm}

    \Huge
    \textbf{Домашнее задание}

    \vspace{0.5cm}
    \LARGE
    По курсу: \textbf{Линейная Алгебра}

    \vspace{1.5cm}

    \textbf{Студент: Ростислав Лохов}

    \vfill

    \Large
    АНО ВО Центральный университет\\
    \vspace{0.3cm}
    \today

\end{titlepage}

% Содержание
\tableofcontents
\newpage

\subsection{Задача 1}

\[
A\cdot B = \begin{pmatrix}
    5 & 6 \\
    -4 & -5
\end{pmatrix} \Rightarrow \det(AB) = -1
\]

Матрица обратима, столбцы \( B \) линейно независимы и строки \( A \) линейно независимы, значит можно искать оператор проекции на \( U \) вдоль \( W \):

\[
(AB)^{-1} = \begin{pmatrix}
    5 & 6 \\
    -4 & -5
\end{pmatrix}
\]

\[
B(AB)^{-1}Av = \begin{pmatrix}
    -3 & -3 & -7 & -3 \\
    4 & 4 & 10 & 4 \\
    0 & 0 & 1 & 0 \\
    0 & 0 & -3 & 0
\end{pmatrix} \cdot \begin{pmatrix}
    2 \\
    -1 \\
    1 \\
    -3
\end{pmatrix} = \begin{pmatrix}
    -1 \\
    2 \\
    1 \\
    -3
\end{pmatrix}
\]

\[
w = v - u = \begin{pmatrix}
    2 \\ -1 \\ 1 \\ -3
\end{pmatrix} - \begin{pmatrix}
    -1 \\ 2 \\ 1 \\ -3 
\end{pmatrix} = \begin{pmatrix}
    3 \\ -3 \\ 0 \\ 0
\end{pmatrix}
\]

\[
v = \begin{pmatrix}
    -1 \\ 2 \\ 1 \\ -3 
\end{pmatrix} + \begin{pmatrix}
    3 \\ -3 \\ 0 \\ 0
\end{pmatrix}
\]

\subsection{Задача 7}

\[
Ac = b, \quad A = 
\begin{pmatrix}
    \langle \mathbf{u}_1, \mathbf{u}_1 \rangle & \langle \mathbf{u}_1, \mathbf{u}_2 \rangle & \langle \mathbf{u}_1, \mathbf{u}_3 \rangle \\
    \langle \mathbf{u}_2, \mathbf{u}_1 \rangle & \langle \mathbf{u}_2, \mathbf{u}_2 \rangle & \langle \mathbf{u}_2, \mathbf{u}_3 \rangle \\
    \langle \mathbf{u}_3, \mathbf{u}_1 \rangle & \langle \mathbf{u}_3, \mathbf{u}_2 \rangle & \langle \mathbf{u}_3, \mathbf{u}_3 \rangle 
\end{pmatrix},
\quad
\mathbf{c} =
\begin{pmatrix}
    c_1 \\
    c_2 \\
    c_3
\end{pmatrix},
\quad
\mathbf{b} =
\begin{pmatrix}
    \langle \mathbf{v}, \mathbf{u}_1 \rangle \\
    \langle \mathbf{v}, \mathbf{u}_2 \rangle \\
    \langle \mathbf{v}, \mathbf{u}_3 \rangle 
\end{pmatrix}
\]

\[
\left[\begin{array}{ccc|c}
1 & 1 & 1 & 1 \\
4 & 10 & -2 & -8 \\
4 & -2 & 10 & 16
\end{array}\right]
\]

Система имеет бесконечно много решений

\[
c = \begin{pmatrix}
    -1 \\ 0 \\ 2 
\end{pmatrix}
\]

\[
v_{\text{proj}} = -u_1 + 2u_3
\]

\[
v_{\text{proj}} = \begin{pmatrix}
    1 \\ -1 \\ -1 \\ 5
\end{pmatrix}
\]

\subsection{Задача 8}

\[
AA^T = 
\begin{pmatrix}
15 & -12 & -6 \\
-12 & 15 & 21 \\
-6 & 21 & 51 
\end{pmatrix}
\]

\[
Av = \begin{pmatrix}
    11\\ -13 \\ -17
\end{pmatrix}
\]

\[
AA^Ty=Av
\]

\[
Pv = \begin{pmatrix}
    \frac{8}{3} \\ 1 \\ -\frac{5}{3} \\ -\frac{2}{3}
\end{pmatrix}
\]

\[
v - Pv = 
\begin{pmatrix}
    \frac{1}{3} \\ -2 \\ -\frac{4}{3} \\ \frac{5}{3}
\end{pmatrix}
\]

\subsection{Задача 9}

\[
Av = \begin{pmatrix}
    11 \\ 16
\end{pmatrix}
\]

\[
AA^T = \begin{pmatrix}
    10 & 18 \\
    18 & 40 \\
\end{pmatrix}
\]

\[
A^T(AA^T)^{-1}Av = \begin{pmatrix}
    2 \\ -0.5
\end{pmatrix}
\]

\[
p = A^T \cdot A^T(AA^T)^{-1}Av = \begin{pmatrix}
    3 \\ 2 \\ 1 \\ 0
\end{pmatrix}
\]

\[
\|p\| = \sqrt{14}
\]

\subsection{Задача 10}
\[
A^TAx = A^Tb
\]

\[
A^TA = \begin{pmatrix}
-18 & -3 & 21 \\
-3 & 18 & -21 \\
21 & -21 & 42 
\end{pmatrix}
\]

\[
A^Tb =
\begin{pmatrix}
5 \\ 5 \\ 0
\end{pmatrix}
\]

\[
x = \begin{pmatrix}
    \frac{1}{3} \\ \frac{1}{3} \\ 0  
\end{pmatrix} + t \begin{pmatrix}
    -1 \\ 1 \\ 1
\end{pmatrix}
\]

\subsection{Задача 11}

\[
u_1 = 1, \quad 
\langle u_1, u_1 \rangle = 2
\]

\[
u_2 = x - \frac{\langle x, u_1 \rangle}{\langle u_1, u_1 \rangle} u_1 = x, \quad 
\langle u_2, u_2 \rangle = \frac{2}{3}
\]

\[
u_3 = x^2 - \frac{1}{3}, \quad 
\langle u_3, u_3 \rangle = \frac{8}{45}
\]

\[
u_4 = x^3 - 0.6x, \quad 
\langle u_4, u_4 \rangle = \frac{8}{175}
\]

\[
D =  \sqrt{ \|x^4\|^2 - \sum_{i=1}^4 \frac{|\langle x^4, u_i \rangle|^2}{\|u_i\|^2} }
\]

\[
\|x^4\|^2 = \int_{-1}^{1} x^8 \, dx = \frac{2}{9}
\]

\[
D = \frac{8\sqrt{2}}{105}
\]

\subsection{Задача 12}
\[
\|Ax - b\|_2^2 = (Ax - b)^T(Ax - b) = x^TA^TAx - 2b^TAx + b^Tb
\]

\[
\lambda \|x\|^2 = \lambda x^Tx
\]

\[
f(x) = \|Ax - b\|^2 + \lambda \|x\|^2 = x^T(A^TA + \lambda I)x - 2b^TAx + b^Tb 
\]

Т.к \( b^Tb = \text{const} \), то необходимо найти приближенное решение.

\[
f(x) = x^T(A^TA + \lambda I)x - 2b^TAx
\]

\subsection{Задача 13}

\[
\langle \pi(v), u \rangle = \langle v, \pi^*(u) \rangle \quad \forall v, u \in V.
\]
Т.к \( \pi \) является проекцией,
\[
\pi^*(v) = \text{Proj}_{U^\perp}^{W^\perp}(v), \quad \forall v \in V.
\]

\end{document}
