\documentclass[a4paper,12pt]{article}
\usepackage[utf8]{inputenc}
\usepackage[russian]{babel}
\usepackage{amsmath,amsfonts,amssymb}
\usepackage{graphicx}
\usepackage{geometry}
\usepackage{hyperref}
\usepackage{venndiagram}
\usepackage{tikz}
\usetikzlibrary{shapes.geometric, calc}
\usepackage{pgfplots}
\usepackage{float}

% Параметры страницы
\geometry{top=2cm,bottom=2cm,left=2.5cm,right=2.5cm}
\geometry{a4paper, margin=1in}

% Заголовок документа
\title{Домашнее задание}
\author{Студент: Лохов Ростислав}
\date{\today}

\begin{document}

% Титульный лист
\begin{titlepage}
    \centering
    \vspace{1cm}

    \Huge
    \textbf{Домашнее задание}

    \vspace{0.5cm}
    \LARGE
    По курсу: \textbf{Линейная алгебра}

    \vspace{1.5cm}

    \textbf{Студент: Лохов Ростислав}

    \vfill

    \Large
    АНО ВО Центральный Университет\\
    \vspace{0.3cm}
    \today

\end{titlepage}

% Содержание
\tableofcontents
\newpage

% Основной текст
\section{Задания}

\subsection*{ВСЕ ЗАДАЧИ}

\subsubsection*{Задачи для самостоятельного решения}

\begin{enumerate}

    \item \textbf{(3 балла)} Пусть $\phi: \mathbb{R}^3 \to \mathbb{R}^2$ — линейное отображение, заданное в стандартном базисе матрицей
    \[
    A = \begin{pmatrix}
    1 & 2 & 0 \\
    -1 & 0 & 2 
    \end{pmatrix}.
    \]
    Пусть 
    \[
    f_1 = \begin{pmatrix} 1 \\ 1 \\ 1 \end{pmatrix}, \quad
    f_2 = \begin{pmatrix} 1 \\ 1 \\ 2 \end{pmatrix}, \quad
    f_3 = \begin{pmatrix} 1 \\ 2 \\ 3 \end{pmatrix}, \quad
    x = \begin{pmatrix} 2 \\ 1 \\ -3 \end{pmatrix}
    \]
    — векторы в $\mathbb{R}^3$; 
    \[
    g_1 = \begin{pmatrix} 1 \\ 2 \end{pmatrix}, \quad
    g_2 = \begin{pmatrix} 1 \\ 1 \end{pmatrix}
    \]
    — векторы в $\mathbb{R}^2$.
    \begin{enumerate}
        \item \textbf{(1 балл)} Покажи, что $f_1, f_2, f_3$ образуют базис в $\mathbb{R}^3$, и найди координаты вектора $x$ в этом базисе.
        \item \textbf{(1 балл)} Покажи, что $g_1, g_2$ образуют базис в $\mathbb{R}^2$, и найди координаты вектора $\phi(x)$ в этом базисе.
        \item \textbf{(1 балл)} Найди матрицу отображения $\phi$ в базисах $f_1, f_2, f_3$ и $g_1, g_2$.
    \end{enumerate}

\textbf{Решение: }
1)приведем обьединенную матрицу f к ступенчатому виду

\[
F = \begin{pmatrix}
    1 & 0 & 0 \\
    0 & 1 & 0 \\
    0 & 0 & 1 \\
\end{pmatrix}
\] - базис пространства $R^3$

Теперь решим уравнение вида FY = x чтобы найти координаты вектора x в этом базисе

\[
Y = \begin{pmatrix}
    6 \\
    -3\\
    -1
\end{pmatrix}
\]

Проверим на линейную независимость $g_1|g_2 \Rightarrow det(G)= -1$ - линейно независимы 

Найдем отображение: $Ax = \begin{pmatrix}
    4\\
    -8
\end{pmatrix}$

Найдём координаты базиса: 

\[
\begin{pmatrix}
    1 & 2 \\
    2 & 1 
\end{pmatrix} \cdot
c = \begin{pmatrix}
    4\\
    -8
\end{pmatrix}
\]
\[
c = \begin{pmatrix}
    -12 \\
    16
\end{pmatrix}
\]

3)
\[
AF=GH\text{, где H - матрица отображения в базисе G, тогда} H = G^-1AF \Rightarrow -1\cdot \begin{pmatrix}
    1 & -1 \\
    -2 & 1
\end{pmatrix} \cdot 

\begin{pmatrix}
    3 & 3 & 5 \\
    1 & 3 & 5 \\
\end{pmatrix} = 

\begin{pmatrix}
    -2 & 0 & 0 \\
    5 & 3 & 5 
\end{pmatrix} 
\]


    \item \textbf{ЗАДАЧА 5} \textbf{(2.5 балла)} Найди матрицу линейного отображения $\varphi: \mathbb{R}^3 \to \mathbb{R}^2$, такого, что
    \[
    \ker \varphi = \text{span} \left\{ 
    \begin{pmatrix} 1 \\ 1 \\ 3 \end{pmatrix}, 
    \begin{pmatrix} -1 \\ -2 \\ -4 \end{pmatrix} 
    \right\}
    \]
    и
    \[
    \text{Im} \varphi = \text{span} \left\{ 
    \begin{pmatrix} 2 \\ 3 \end{pmatrix} 
    \right\}.
    \]

\textbf{Решение: }
ПУсть у нас есть матрица 2x3: 

\[
A = \begin{pmatrix}
    a_{11} & a_{12} & a_{13} \\
    a_{23} & a_{22} & a_{23} \\
\end{pmatrix}
\]

Т.к образ $\varphi$ - линейная оболочка вектора(2, 3), то столбцы должны быть пропорциональны этому вектору. Это означает, что каждый столбец A должен быть кратен (2, 3)

\[
A = \begin{pmatrix}
    2a & 2b & 2c \\
    3a & 3b & 3c
\end{pmatrix}
\]

т.к ядро А состоит из всех векторов размерности 3, для которых Ax=0.

\[
ax_1 + bx_2 + cx_3 = 0
\]

\begin{cases}
    a + b + 3c = 0 \\
    a + 2b + 4c = 0
\end{cases}
a = -2c, b = -c

\[
A = \begin{pmatrix}
    -4 & -2 & 2 \\
    -6 & -3 & 3 
\end{pmatrix}
\]

    \item \textbf{ЗАДАЧА 6} \textbf{(1.5 балла)} Приведи пример или докажи, что данного линейного отображения не существует:
    \begin{enumerate}
        \item $\phi: \mathbb{R}^3 \to \mathbb{R}^3$, такое, что $\text{Im} \phi = \ker \phi$;
        \item $\phi: \mathbb{R}^4 \to \mathbb{R}^4$, такое, что $\text{Im} \phi = \ker \phi$.
    \end{enumerate}

\textbf{Решение: }
\[
dim(lm(\varphi)) + dim(ker(\varphi)) = n
\]

\[
2dim(lm(\varphi)) = 3
\]

- не имеет решения в целых числах

\[
2dim(lm(\varphi)) = 4
\]

имеет решение в целых числах

\[
dim(lm(\varphi)) + dim(ker(\varphi)) = 2
\]


\[
A = \begin{pmatrix}
    0 & 0 & 1 & 0 \\
    0 & 0 & 0 & 1 \\
    0 & 0 & 0 & 0 \\
    0 & 0 & 0 & 0 
\end{pmatrix}
\]

    \item \textbf{ЗАДАЧА 7} \textbf{(3 балла)} Пусть $R[x]_{\leq n}$ — пространство многочленов степени не выше $n$. Пусть на нём задано линейное отображение, определённое правилом:
    \[
    f \mapsto (x + 1)f'(x) - 2f(x) + \frac{1}{x} \int_{0}^{x} f(t) \, dt.
    \]
    Найди матрицу этого линейного отображения в базисе $1, x, x^2, \dots, x^n$.

    \textbf{Решение: }
    \[
    f(x^n) = (x^n+1)(nx^{n-1})-2x^n+\frac{x^n}{n+1}=(n-2\frac{1}{n+1})x^n + nx^{n-1}=\frac{n^2-n-1}{n+1}x^n - nx^{n-1} 
    \]
    Проверим для базиса  $B = {1, x, x^2, x^3}$:
    
    $f(0) = 2(0)-2+1=-1$

    $f(1)= -\frac{1}{2}x-1$
    
    $f(2)=-\frac{1}{3}x^2 - 2x$

    $f(3)=\frac{5}{4}x^3-3x^2$

    $f(4)=\frac{11}{5}x^4-4x^3$

    $\frac{19}{6}x^5-5x^4$

    \[
    A = \begin{pmatrix}
    -1 & 1 & 0 & 0 & \cdots & 0 \\
    0 & -\frac{1}{2} & 2 & 0 & \cdots & 0 \\
    0 & 0 & \frac{2^2 - 2 - 1}{3} & 3 & \cdots & 0 \\
    0 & 0 & 0 & \frac{3^2 - 3 - 1}{4} & \ddots & \vdots \\
    \vdots & \vdots & \vdots & \ddots & \ddots & n \\
    0 & 0 & 0 & \cdots & 0 & \frac{n^2 - n - 1}{n+1}
    \end{pmatrix}
    \]

    
    \item \textbf{ЗАДАЧА 8} \textbf{(2 балла)} Пусть $R[x]_{\leq n}$ — множество всех многочленов с вещественными коэффициентами степени не больше $n$. Пусть отображение
    \[
    \frac{d}{dx} : R[x]_{\leq n} \to R[x]_{\leq n}
    \]
    — дифференцирование многочлена по переменной $x$.
    \begin{enumerate}
        \item \textbf{(0.5 балла)} Найди матрицу этого отображения в базисе $1, x, \dots, x^n$.
        \item \textbf{(0.7 балла)} Пусть $p \in R[x]$ и $a, b \in \mathbb{R}$. Докажи, что уравнение $a f'' + b f' + f = p$ имеет единственное решение. (Указание: понадобится результат задачи 13 из домашнего задания второй недели «Обратные матрицы. Матричные уравнения. LU-разложение».)
        \item \textbf{(0.8 балла)} Пусть $p \in R[x]$ и $a \in \mathbb{R}$. Докажи, что уравнение $a f'' + f' = p$ имеет решение.
    \end{enumerate}
    \textbf{Решение: }
    У нас есть базис ${1, x, ...., x^n}$ Дифференцирование многочлена $\frac{d}{dx} = kx^{k-1}$, тогда матрица отображения: 
    \[
    \begin{pmatrix}
        0 & 1 & 0 &... & 0 \\
        0 & 0 & 2 &... & 0 \\
        0 & 0 & 0 &... & 0 \\
        0 & 0 & 0 &... & n \\
        0 & 0 & 0 &... & 0 \\
    \end{pmatrix}
    \]
    Двойная производная от многочлена будет равна $k(k-1)x^{k-2}$ Тогда матрица отображения: 
    \[
    \begin{pmatrix}
        0 & 0 & 2 & 0 & 0 & ... & 0 \\
        0 & 0 & 0 & 6 & 0 & ... & 0 \\
        0 & 0 & 0 & 0 & 12 & ... & 0 \\
        0 & 0 & 0 & 0 & 0 & ... & 0 \\
        0 & 0 & 0 & 0 & 0 & ... & n(n-1) \\
        0 & 0 & 0 & 0 & 0 & ... & 0 \\
    \end{pmatrix}
    \]
    Плохой подход, попробуем записать в дифференциалах уравнение: 
    \[
    a\frac{d^2f}{dx^2} + b\frac{df}{dx} + f = p
    \]

    \[
    A = aD^2+bD + I 
    \]
    Матрица A  является верхнедиагональной, причём все элементы на главной диагонали не равны 0, т.к у нас D и $D^2$ - верхнетреугольная матрица и на главной диагонали - нули. Т.е при прибавлении к $aD^2+bD$ единичной матрицы мы получаем верхнедиагональную матрицу с ненулевыми элементами на главной диагонали - полный ранг. Имеет единственное решение.


    \[
    aD^2+D = A
    \]
    т.к сумма двух верхнетреугольных матриц является верхнетреугольной, то А - верхнетреугольная. А значит имеет решения.
    

    \item \textbf{ЗАДАЧА 9} \textbf{(2 балла)} Пусть $U, V \subseteq E$ — подпространства некоторого векторного пространства $E$.
    \begin{enumerate}
        \item Покажи, что множество линейных операторов вида
        \[
        \{ \varphi: E \to E \mid U \subseteq \ker \varphi, \ \text{Im} \varphi \subseteq V \}
        \]
        является векторным подпространством в пространстве всех линейных отображений из $E$ в себя. \textbf{(1 балл)}
        \item Найди размерность этого подпространства. \textbf{(1 балл)}
    \end{enumerate}

    \textbf{Решение: }
    
    Наличие нулевого элемента - $0: E \to E 0(x)=0 \forall x \in E: U \subseteq ker0 \land Im(0) = {0}\subseteq V \Rightarrow 0 \in S$
    
    Замкнутость относительно сложения - $(\varphi, \psi \in S) \Rightarrow \psi + \varphi: E \to E \Longleftrightarrow U \subseteq ker(\psi + \varphi): (\varphi + \psi)(u) = \varphi(u)+\psi(u) = 0 + 0 = 0 \forall u \in U \land Im(\varphi + \psi) \subseteq Im(\varphi) + Im(\psi) \subseteq V + V = V \Rightarrow \varphi + \psi \in S$

    Замкнутость относительно умножения на скаляры - $(\varphi \in S)(c \in R) \Rightarrow c\phi E \to E \longleftrightarrow U \subseteq ker(c\varphi) \Leftrightarrow (c\phi)(u) = c \cdot \varphi(u) = c \cdot 0 = 0 \forall u \in U \land Im(c\varphi) = c Im(\varphi) \subseteq V \Leftrightarrow c \cdot V \subseteq V \Rightarrow c\varphi \in S$

    Исходя из того, что все три пункта выполняются S - векторное подпространство

    Выберем базис в Е такой что перывые $k=dim(U)$ векторов образуют базис подпространства U, а оставшиеся $n-k$ векторов дополняют его до базиса $E$ Пусть $\dim(V)=l$. Т.к $U \subseteq ker(\varphi)$ первые k столбцов матрицы оператора $\varphi$ равны нулю. Образ каждого из оставшихся $n-k$ базисных векторов может быть любым элементом из V т.е определяется l параметрами. Таким образом общее число независимых параметров равно $(n-dim(U))\cdot dim(V)$



    

    \item \textbf{ЗАДАЧА 10} \textbf{(3 балла)} Пусть $\varphi: V \to V$ — линейное отображение векторного пространства $V$ в себя.
    \begin{enumerate}
        \item \textbf{(1 балл)} Докажи эквивалентность следующих условий:
        \begin{enumerate}
            \item $\text{Im} \varphi \cap \ker \varphi = \{0\}$.
            \item $\text{Im} \varphi + \ker \varphi = V$.
            \item В некотором базисе пространства $V$ отображение $\varphi$ имеет блочный вид
            \[
            \begin{pmatrix}
            A & 0 \\
            0 & 0 
            \end{pmatrix},
            \]
            где $A$ — невырожденная матрица.
        \end{enumerate}
        \item \textbf{(1 балл)} Докажи эквивалентность следующих условий:
        \begin{enumerate}
            \item $\text{Im} \varphi \subseteq \ker \varphi$.
            \item $\varphi^2 = 0$.
        \end{enumerate}
    \end{enumerate}
    
    \textbf{Решение: }

    $ker(\varphi)={x \in V| \varphi(x)=0}$

    $Im(\varphi)={y\in V|\exists x \in V: \varphix=y}$

    1) $Im(\varphi) \cap ker(\varphi) = 0$

    $dim(ker(\varphi)) + dim(Im(\varphi)) = dim(V)$
    
    т.к размерность ядра отображения при отображении пространства в само себя равно 0, то это означает, что $dim(Im(\varphi)) + 0 = dim(V)$ т.е из 1ого следует второе

    2) поскольку 1 и 2 пункт эквивалентны, то матрица A невырождена. из этого следует что пункт 1 эквивалентен пункту 2 и пункту 3.

    3) $Im(\phi ) \subseteq ker(\phi) \forall v \in V: \phi^2(v) = \phi(\phi(v))=0$ Поскольку $\phi(v) \in \ker(\phi) \Rightarrow \varphi^2 = 0$

    4) $\varphi(\varphi(v))=0 \Rightarrow \varphi(v) \in ker(\phi) \Rightarrow Im(\varphi) \subseteq ker(\varphi)$

\end{enumerate}

\end{document}
