\documentclass[a4paper,12pt]{article}
\usepackage[utf8]{inputenc}
\usepackage[russian]{babel}
\usepackage{amsmath,amsfonts,amssymb}
\usepackage{graphicx}
\usepackage{geometry}
\usepackage{hyperref}
\usepackage{venndiagram}
\usepackage{tikz}
\usetikzlibrary{shapes.geometric, calc}
\usepackage{pgfplots}
\usepackage{float}

% Параметры страницы
\geometry{top=2cm,bottom=2cm,left=2.5cm,right=2.5cm}
\geometry{a4paper, margin=1in}

% Заголовок документа
\title{Домашнее задание}
\author{Студент: Лохов Ростислав}
\date{\today}

\begin{document}

% Титульный лист
\begin{titlepage}
    \centering
    \vspace*{1cm}

    \Huge
    \textbf{Домашнее задание}

    \vspace{0.5cm}
    \LARGE
    По курсу: \textbf{Математический Анализ}

    \vspace{1.5cm}

    \textbf{Студент: Лохов Ростислав}

    \vfill

    \Large
    АНО ВО Центральный Университет\\
    \vspace{0.3cm}
    \today

\end{titlepage}

% Содержание
\tableofcontents
\newpage

% Основной текст
\section{Предел последовательности}

\subsection{Частичный предел}

\subsubsection{Задача 1 \hfill 0,5 балла}

\textbf{Условие задачи:}

Найди все частичные пределы последовательности \( z_n = \left( (-1)^n + 1 \right) n \).

\textbf{Решение: }
Заметим, что при нечетном n $z_n$ получаем всегда 0, при четном - бесконечность, т.е при n=2k 
$z_2k = 0, z_2k+1 + \infty $

\vspace{1cm}

\subsubsection{Задача 2 \hfill 1 балл}

\textbf{Условие задачи:}

Известно, что \( \lim\limits_{k \to \infty} x_{3k} = \alpha \), \( \lim\limits_{k \to \infty} x_{3k-1} = \beta \), \( \lim\limits_{k \to \infty} x_{3k-2} = \gamma \). Докажи, что у последовательности \( \{ x_n \} \) нет частичных пределов, отличных от \( \alpha \), \( \beta \) и \( \gamma \).

\textbf{Решение: }
Представим, что существует такой предел, что $\lim_{j \to \infty}x_n_j = L$, где $L \ne \alpha, \beta, \gamma$, это означает, что существует такая подпоследовательность, которая сходится к L. т.к каждый индекс при делении на 3 дает остатки 0 1 2, то подпоследовательность должна соответствовать одной из трех типов последовательностей, что противоречит первоначальному предположению. Таким образом наше первоначальное предположение неверно и отсюда следует, что у последовательности нет частичных пределов отличных от $\alpha, \beta, \gamma$
\vspace{1cm}

\subsection{Теорема Больцано-Вейерштрасса}

\subsubsection{Задача 3 \hfill 0,5 балла}

\textbf{Условие задачи:}

Приведи пример последовательности, имеющей ровно 7 частичных пределов.

\vspace{1cm}
\textbf{Решение: }
\{1, 2, 3, 4, 5, 6, 7, 1, 2, 3....\}
\subsubsection{Задача 4 \hfill 1 балл}

\textbf{Условие задачи:}

Найди все частичные пределы последовательности \( \{ 0; 1; 0; \dfrac{1}{2}; 1; 0; \dfrac{1}{3}; \dfrac{2}{3}; 1; 0; \dfrac{1}{4}; \dfrac{2}{4}; \dfrac{3}{4}; 1; \dots \} \).

\textbf{Решение: }
Каждый блок для каждого k содержит k+1 элементов, 0 1; 0 1/2 1; 0 1/3 2/3 1;...
Хочется написать очев, что ответом будет множество действительных чисел, но надо обосновать предположение.
Рассмотрим число x из интервала 0-1, оно может быть:

Рациональным: тогда можно представить x как m/k, для натуральных m и k, тогда x появляется в блоках с номерами k, 2k, 3k и тд. Следовательно существует подпоследовательность сходящаяся к x.

Иррациональным: тогда в каждом блоке k есть элемент, максимально близкий к x, и при увеличении k мы будем получать как можно более точное стремление к покрытию 0-1 в иррациональном представлении. т.е при $\lim_{k \to \infty} $ мы получаем множество всех действительных чисел.  
\vspace{1cm}

\subsubsection{Задача 5 \hfill 1,5 балла}

\textbf{Условие задачи:}

У последовательности \( \{ a_n \} \) подпоследовательности \( \{ a_{2k} \} \), \( \{ a_{2k-1} \} \), \( \{ a_{3k} \} \) сходятся. Докажи, что \( \{ a_n \} \) сходится.

\textbf{Решение: }
Методом пристального взгляда можно сказать, что тут пересекаются элементы подпоследовательностей.
пусть $\lim_{k\to \infty}a_{2k}, \lim_{k\to \infty}a_{2k-1}, \lim_{k\to \infty}a_{3k} = L_1, L_2, L_3$
Несложно заметить, что $\lim_{k\to \infty}a_{6k} = L_1 = L_2$, $ \lim_{k\to \infty}a_{6k+1} a_6k = L_2 = L_3$ таким образом $L_1=L_2=L_3$ и, если все подпоследовательности сходятся к одному пределу, то и вся последовательность сходится.

\vspace{1cm}

\subsection{Верхний и нижний пределы}

\subsubsection{Задача 6 \hfill 1,5 балла}

\textbf{Условие задачи:}

Для последовательности \( \{ x_n \} \) найди \( \lim\limits_{n \to \infty} x_n \), \( \lim\limits_{n \to \infty} x_n \), \( \sup\limits_{n \in \mathbb{N}} x_n \), \( \inf\limits_{n \in \mathbb{N}} x_n \), если:

а) \( x_n = \dfrac{11 (-1)^{n(n - 1)}}{n} + \dfrac{1 + (-1)^{3n}}{5} \);

б) \( x_n = \dfrac{(\cos \pi n - 1) n + \ln n}{\ln 12n} \).

\textbf{Решение: }
Путём нехитрых преобразований получим

\begin{enumerate}
    \item $x_n=\frac{11}{n}+\frac{1+(-1)^n}{5}$
    \begin{enumerate}
        \item $\lim_{n \to \infty} x_n$ будет осцилировать между двумя значениями 0 и 0.4 т.е верхний предел - 0.4, нижний - 0. Cупремум будет равен 11, при нечетных n, инфинум будет 0 тоже при нечетных n. Общего предела нет
    \end{enumerate}
    \item \( x_n = \dfrac{(\cos \pi n - 1) n + \ln n}{\ln 12n} \).
    \begin{enumerate}
        \item очевидно, что $\cos(\pi n ) = (-1)^n$
        \item при четном n получим, что  $\frac{\ln(n)}{\ln(12) + \ln(n)}$ это будет стремится к 1, тк $ln(12)$ особо значения не имеет при больших n. 
        \item при нечетном n получаем $\frac{-2n + \ln(n)}{\ln(12) + \ln(n)}$ также при больших n будем получать $\frac{-2n}{\ln(n)}$ и т.к линейная функция растёт быстрее логарифмической, то пределом будет $-\infty$
        \item супремум будет 1, достигается при четных n, инфинум будет $-\infty$ (можно было сократить на $ln(n)$ числитель и знаменатель)
    \end{enumerate}
\end{enumerate}

\vspace{1cm}

\subsubsection{Задача 7 \hfill 2 балла}

\textbf{Условие задачи:}

Последовательность \( \{ x_n \} \) ограничена, \( \lim\limits_{n \to \infty} x_n = b \), \( \lim\limits_{n \to \infty} x_n = a \), \( a < b \), \( \lim\limits_{n \to \infty} (x_{n+1} - x_n) = 0 \). Докажи, что множеством частичных пределов последовательности \( \{ x_n \} \) является отрезок \( [a; b] \).

\textbf{Решение: }

\begin{enumerate}
    \item последовательность находится между a и b 
    \item последовательность постепенно приближается к любому промежуточному значению( из предела разностей соседних членов)
    \item для любой точки c принадлежащей в промежутке от а до b можно найти такие члены последовтельности, которые будут стремиться к c из за малой разницы между соседними членами при n стремящйейся к бесконечности
    \item поскольку для любого  с можно достигнуть через подпоследовательость, множество частичных пределов включает отрезок от a до b
    \item из ограниченности последовательности следует, что частичные пределы не выходят за границы
    \item таким образом множество частичных пределов совпадает с отрезком от a до b
\end{enumerate}
\vspace{1cm}

\subsection{Критерий Коши сходимости последовательности}

\subsubsection{Задача 8 \hfill 0,5 балла}

\textbf{Условие задачи:}

Докажи, что последовательность \( a_n = (-1)^n \left( 1 + \dfrac{1}{n} \right)^n \) расходится.

\textbf{Решение: }

\begin{enumerate}
    \item последовательность при $n \to \infty $ можно представить как $\lim_{n \to \infty}  a_n = (-1)^n e$
    \item при четных n будет e, при нечетных n будет -e. 
    \item таким образом последовательность будет чередоваться между e и -e 
    \item таким образом имеем две подпоследовательности при разных четностях и мы не имеем одного общего предела.
\end{enumerate}
\vspace{1cm}

\subsubsection{Задача 9 \hfill 2 балла}

\textbf{Условие задачи:}

Исследуй последовательность \( \{ x_n \} \) на сходимость, используя критерий Коши:

а) \( x_n = \sum\limits_{k=1}^{n} \dfrac{1}{k!} \);

б) \( x_n = \sum\limits_{k=1}^{n} \dfrac{1}{k} \).

\textbf{Решение: }

а)
\[
\sum\limits_{k=n+1}^{\infty} \frac{1}{k!} < \frac{1}{n!} 
\]

\[
|x_m - x_n| = \sum\limits_{k=n+1}^{m} \dfrac{1}{k!} < \sum\limits_{k=n+1}^{\infty} \dfrac{1}{k!} < \dfrac{1}{n!} < \varepsilon
\]
т.е сходится

б)
используем факт, что
\[
\sum\limits_{k=n+1}^{2n} \dfrac{1}{k} > \sum\limits_{k=n+1}^{2n} \dfrac{1}{2n} = \dfrac{n}{2n} = \dfrac{1}{2}
\]
посскольку мы не можем найти такой номер, что начиная с которого для любого n, m больше этого номера модуль разницы был меньше условного эпсилон меньше единицы, то последовательность расходится

\vspace{1cm}

\subsubsection{Задача 10 \hfill 1,5 балла}

\textbf{Условие задачи:}

Приведи пример такой расходящейся последовательности \( \{ a_n \} \), что для любого \( p \in \mathbb{N} \) выполнено \( \lim\limits_{n \to \infty} |a_{n+p} - a_n| = 0 \).

\textbf{Решение: }

логарифмическая последовательность, она расходится. $   |a_{n+p} - a_n| = |\ln(n+p) - \ln(n)| = \ln\left(1 + \frac{p}{n}\right)$ При n стремящемся к бесконечности, можем убрать единицу и, используя свойство перехода от логарифма, можем сказать, что $p/n=0$ таким образом последовательность удовлетворяет условию и является расходящейся.

\vspace{1cm}

\subsubsection{Задача 11 \hfill 1 балл}

\textbf{Условие задачи:}

Верно ли, что если \( \forall \varepsilon > 0 \ \exists N : \ \forall n > N \ |x_n - x_N| < \varepsilon \), то последовательность \( \{ x_n \} \) фундаментальна?

\vspace{1cm}

\subsubsection{Задача 12 \hfill 2 балла}

\textbf{Условие задачи:}

Для некоторой последовательности \( \{ x_n \} \) существуют числа \( \alpha \in (0; 1) \) и \( C > 0 \) такие, что \( \forall n \in \mathbb{N} \ |x_{n+1} - x_n| < C \alpha^n \). Можно ли утверждать, что последовательность \( \{ x_n \} \) сходится?

\textbf{Решение:}

Используя неравенство треугольника получим: $   |x_m - x_n| \leq |x_m - x_N| + |x_n - x_N| < \frac{\varepsilon}{2} + \frac{\varepsilon}{2} = \varepsilon$ Таким образом условие выполняется. Т.е последовательность фундаментальная.

\vspace{1cm}

\subsection{Число \( e \)}

\subsubsection{Задача 13 \hfill 2 балла}

\textbf{Условие задачи:}

Докажи, что последовательность \( y_n = \left( 1 + \dfrac{1}{n} \right)^n \) строго возрастает. Для этого достаточно показать, что отношение \( \dfrac{y_{n+1}}{y_n} > 1 \).

\textbf{Решение: }

\[
\frac{y_{n+1}}{y_n} = (1-\frac{1}{(n+1)^2})\cdot (1+\frac{1}{n+1})=\frac{n^3+4n^2+4n}{(n+1)^3}
\]
Функция бесконечно убывает и имеет предел = 1, супремум 1.125, наименьшее - 1, инфинум $\infty$ отсюда следует что всегда > 1. чтд
\vspace{1cm}

\subsubsection{Задача 14 \hfill 2 балла}

\textbf{Условие задачи:}

Для каждой из последовательностей \( a_n = \sum\limits_{k=0}^{n} \dfrac{1}{k!} \) и \( y_n = \left( 1 + \dfrac{1}{n} \right)^n \) оцени номер члена \( N \), начиная с которого выполняется неравенство \( |e - a_n| < 10^{-5} \) (или \( |e - y_n| < 10^{-5} \)). Какая из последовательностей сходится «быстрее»? Найди приближение числа \( e \) с точностью до четырёх знаков после запятой.

\textbf{Решение: }
Очев, что грубо решить такую задачу не получится, очевидно что ряд суммы обратных факториалов быстрее сходится. Попробуем аналитически:
\[
e = \sum\limits_{k=0}^{n} \dfrac{1}{k!} + \dfrac{\theta}{n! \cdot n}, \quad \theta \in (0;1)
\]

\[
R_n = e - a_n = \dfrac{\theta}{n! \cdot n} < \dfrac{1}{n! \cdot n}
\]

\[
n! \cdot n > 10^{5}
\]

при n = 8

\[
a_8 = \sum\limits_{k=0}^{8} \dfrac{1}{k!} = 1 + 1 + \dfrac{1}{2} + \dfrac{1}{6} + \dfrac{1}{24} + \dfrac{1}{120} + \dfrac{1}{720} + \dfrac{1}{5040} + \dfrac{1}{40320} \approx 2.71827877
\]

\[
e \approx 2.7183
\]

б)
Записываем в двух неравенствах, слева просто степень, посередине е, справа степень + 1, тогда
\[
0 < e - y_n < \left(1 + \dfrac{1}{n}\right)^{n+1} - \left(1 + \dfrac{1}{n}\right)^n = \left(1 + \dfrac{1}{n}\right)^n \cdot \dfrac{1}{n} < \dfrac{e}{n}
\]

\[
e - y_n < \dfrac{e}{n}
\]

\[
n > \frac{e}{10^{-5}} = 271,8282
\]
Настолько грубо, что если пользоваться рядами и логарифмированием мы отличаемся в 2 раза. Понимаю, что мне запретят это сделать, но я всё равно это сделаю.

\[
\ln(y_n) = n \cdot \ln(1+\frac{1}{n})
\]

Разложив $\ln(1+x)$ в ряд Тейлора около точки x=0

\[
\ln(1+x) = x - \frac{x^2}{2}+\frac{x^3}{3}...
\]

Подставив, домножив на n, экспоненциировав, получим:

\[
e^{1-\frac{1}{2n}+\frac{1}{3n^2}...}
\]

\[
e-y_n \approx e(\frac{1}{2n}-\frac{1}{3n^2})
\]

Можем отбросить вычитаемое т.к влияние незначительно при больших n. Тогда получим n = 135,914, что больше правильного ответа на 1, и значительно меньше, чем предыдущий подход
\vspace{1cm}

\subsection{Теорема Штольца}

\subsubsection{Задача 15 \hfill 1,5 балла}

\textbf{Условие задачи:}

Найди предел \( \lim\limits_{n \to \infty} \dfrac{1 + \sqrt{2} + \sqrt{3} + \dots + \sqrt{n}}{n} \).


\textbf{Решение: }
Как будто можно устно, но надо что то написать. Теорема Stolz-Cesaro гласит, что если $\lim_{n \to \infty} \frac{\Delta X}{\Delta y}$ существует, то существует и $\frac{x}{y}$ и наоборот. Дискретный аналог Лопеталя для устранения неопределенностей вида частное бесконечностей и нулей. Вообще это краткая выжимка из полноценной ГЕНЕРАЛЬНОЙ формы. Генеральная связывает супремумы и инфинумы неравенствами. Забыл упомянуть, что y монотонна и неограничена. Вернемся обратно к заданию. Получим, что сверху разница сумм корней, по разнице индексов получим $\frac{\sqrt{n+1}}{1}$ предел которой равен бесконечности. А значит, что и вся последовательность, первоначальная имеет предел равный бесконечости
\vspace{1cm}

\subsubsection{Задача 16 \hfill 2 балла}

\textbf{Условие задачи:}

Найди пределы:

а) \( \lim\limits_{n \to \infty} \left( \dfrac{1}{n^4} \sum\limits_{k=1}^{n} k^3 \right) \);

б) \( \lim\limits_{n \to \infty} \left( \dfrac{1}{n^{p+1}} \sum\limits_{k=1}^{n} k^p \right) \), \( p \in \mathbb{N} \).

\textbf{Решение: }

a) Есть формула суммы кубов натуральных чисел, для произвольной степени чуть позже
\[
\frac{1}{n^4} \cdot \frac{n^2(n+1)^2}{4} = \frac{n^2+2n+1}{4n^2}
\]

Предел будет 0.25

б) Можно вывести формулу, она реккурентная и доказывается через $(n-1)^p$. Есть менее затратная формула Бернулли, но в ней используются числа Бернулли. В знаменателе у чисел бернули будет p+1, в числителе будет $n^p$ n раз, т.е можно будет сократить числитель и знаменатель на $n^{p+1}$, тогда снизу останется p+1, сверху будет 1 при большом n

\[
\frac{(n^p + (n-1)^{p}+(n-2)^p...+1)}{n^{p}n}
\]
Очевидно, что $n^p$ встречается n раз, тогда смотря на формулу бернулли, получим

\[
\frac{(n+B)^{p+1}-B^{p+1}}{n^{p+1}(p+1)}
\]
Методом пристального взгляда понимаем, что можно сократить числитель и знаменатель на одно и то же $n^{p+1}$
Дамы и господа, я был не прав, забыл про Штольца поскольку решаю это глубокой ночью, простите, Теперь решаем нормально.

Поскольку знаменатель монотонно возрастает можем применить теорему Штольца. 

\[
\frac{(n+1)^p}{(n+1)^{p+1} - n^{p+1}} = \frac{(n+1)^p}{(p+1)n^p+0.5p(p+1)n^{p-1}...}=\frac{1}{p+1}
\]


\vspace{1cm}

\subsubsection{Задача 17 \hfill 1,5 балла}

\textbf{Условие задачи:}

Найди предел \( \lim\limits_{n \to \infty} \left( \dfrac{1}{\sqrt{n}} \sum\limits_{k=1}^{n} \dfrac{1}{\sqrt{k}} \right) \).

\textbf{Решение: }

Штольцем, т.к знаменатель монотонно возрастает и не ограничен.

\[
\frac{1}{n+1-\sqrt{n(n+1)}}
\]

\[
\frac{n+1+\sqrt{n^2+n}}{n^2+2n+1-n^2-n}=\frac{n+1+\sqrt{n^2+n}}{n+1}= 1+ \frac{n\sqrt{1+1/n}}{n(1+1/n)}=2
\]

\vspace{1cm}

\subsubsection{Задача 18 \hfill 1,5 балла}

\textbf{Условие задачи:}

Пусть \( \{ a_n \} \) — последовательность положительных чисел, сходящаяся к числу \( A > 0 \). Применяя теорему Штольца к последовательностям \( x_n = \ln(a_1 a_2 \dots a_n) \) и \( y_n = n \), докажи, что \( \lim\limits_{n \to \infty} \sqrt[n]{a_1 a_2 \dots a_n} = A \).

\textbf{Решение: }
Ну это из контрольной(В простонародье - баян)

\[
x_n = \sum_{k=1}^{n} \ln{a_k}; y_n = n
\]

\[
\frac{x_n}{y_n}=\frac{\Delta x}{\Delta y} 
\]

\[
\frac{\Delta x}{\Delta y}=\frac{\ln{a_{n+1}}}{1}=\ln(A)=\frac{x_n}{y_n}=\frac{\ln{a_1a_2...a_n}}{n}=ln{\sqrt[n]{a_1 a_2 \dots a_n}}=ln(A)
\]
Используя свойство непрерывности логарифма и экспоненты, получим, что $e^{\ln{A}}=A$
\vspace{1cm}

\end{document}