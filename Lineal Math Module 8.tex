\documentclass[a4paper,12pt]{article}

% Кодировка и язык
\usepackage[utf8]{inputenc}
\usepackage[russian]{babel}

% Математические пакеты
\usepackage{amsmath,amsfonts,amssymb}

% Графика
\usepackage{graphicx}
\usepackage{tikz}
\usetikzlibrary{shapes.geometric, calc}
\usepackage{pgfplots}

% Геометрия страницы
\usepackage{geometry}
\geometry{top=2cm, bottom=2cm, left=2.5cm, right=2.5cm}

% Гиперссылки
\usepackage{hyperref}

% Плавающие объекты
\usepackage{float}

% Дополнительные пакеты
\usepackage{venndiagram}

% Настройки заголовка
\title{Домашнее задание}
\author{Студент: \textbf{Ваше Имя Фамилия}}
\date{\today}

\begin{document}

% Титульный лист
\begin{titlepage}
    \centering
    \vspace*{1cm}

    \Huge
    \textbf{Домашнее задание}

    \vspace{0.5cm}
    \LARGE
    По курсу: \textbf{Линейная алгебра}

    \vspace{1.5cm}

    \textbf{Студент: Ростислав Лохов}

    \vfill

    \Large
    АНО ВО Центральный университет\\
    \vspace{0.3cm}
    \today

\end{titlepage}

% Содержание
\tableofcontents
\newpage

% Основной текст
\section{Квадратичные формы}

\subsection{Задача 1}
\textbf{Условие задачи:}

Пусть билинейная форма $\beta : \mathbb{R}^3 \times \mathbb{R}^3 \to \mathbb{R}$ определена правилом $\beta(x,y) = x^T By$, где
$$
B =
\begin{pmatrix}
-4 & -3 & -4 \\
2 & 3 & 4 \\
1 & 1 & 1
\end{pmatrix}.
$$
Найди матрицу билинейной формы $\beta$ в следующем базисе:
$$
f_1 =
\begin{pmatrix}
1 \\
1 \\
-1
\end{pmatrix}, \quad
f_2 =
\begin{pmatrix}
1 \\
2 \\
-2
\end{pmatrix}, \quad
f_3 =
\begin{pmatrix}
-1 \\
-2 \\
3
\end{pmatrix}.
$$

\textbf{Решение:}

\[
B' =  F^TBF = \begin{pmatrix}
    -3 & -3 & 2 \\
    -3 & -4 & 6 \\
    4 & 5 & -6 
\end{pmatrix}
\]

\vspace{1cm}

\subsection{Задача 2}
\textbf{Условие задачи:}

Найди какой-нибудь базис, в котором следующая билинейная форма в $\mathbb{R}^3$ имеет диагональный вид:
$$
\beta(x, y) = 2x_1y_1 - x_1y_2 + x_1y_3 - x_2y_1 + x_3y_1 + 3x_3y_3.
$$
Определи её сигнатуру.

\textbf{Решение:}

\vspace{1cm}

\subsection{Задача 3}
\textbf{Условие задачи:}

Методом Якоби определи сигнатуру и знак квадратичной формы $Q: \mathbb{R}^4 \to \mathbb{R}$, заданной правилом $Q(x) = x^T B x$, где
$$
B =
\begin{pmatrix}
-4 & 5 & 7 & -5 \\
5 & -4 & -5 & 7 \\
7 & -5 & -7 & 9 \\
-5 & 7 & 9 & -7
\end{pmatrix}.
$$

\textbf{Решение:}
Матрица симметрична

$D_1$ = -4; $D_2$ = -9; $D_3$ = 9; $D_4$ = 0; (p, q) = 1, 2

\vspace{1cm}

\subsection{Задача 4}
\textbf{Условие задачи:}

Пусть в пространстве $\mathbb{R}^3$ заданы билинейная форма $\beta$ и линейный оператор $\phi$ по правилам $\beta(x, y) = x^T By$ и $\phi(x) = Ax$, где
$$
A =
\begin{pmatrix}
1 & -1 & -1 \\
1 & -2 & -2 \\
-1 & 2 & 1
\end{pmatrix}, \quad
B =
\begin{pmatrix}
-1 & 3 & 1 \\
2 & 1 & 2 \\
3 & -1 & -1
\end{pmatrix}.
$$
\begin{enumerate}
    \item Найди матрицу билинейной формы $\gamma(x, y) = \beta(\phi(x), \phi^{-1}(y))$ в стандартном базисе.
    \item Найди квадратичную форму $Q_\gamma$, соответствующую $\gamma$ в координатах стандартного базиса.
\end{enumerate}

\textbf{Решение:}

% Здесь вставьте решение задачи

\vspace{1cm}

\subsection{Задача 5}
\textbf{Условие задачи:}

Найди сигнатуру следующей квадратичной формы:
$$
Q(x, y, z) = 5x^2 - 6xy - 10xz - 8y^2 + 6yz + 5z^2.
$$

\textbf{Решение:}

% Здесь вставьте решение задачи

\vspace{1cm}

\subsection{Задача 6}
\textbf{Условие задачи:}

Пусть квадратичная форма $Q: \mathbb{R}^3 \to \mathbb{R}$ задана матрицей
$$
\begin{pmatrix}
\alpha & 0 & 1 \\
0 & \alpha + 3 & -4 \\
3 & 0 & \alpha - 1
\end{pmatrix}.
$$
Найдите при каком значении параметра $\alpha \in \mathbb{R}$ форма положительно определена, а при каких отрицательно.

\textbf{Решение:}

% Здесь вставьте решение задачи

\vspace{1cm}

\subsection{Задача 7}
\textbf{Условие задачи:}

В пространстве $V = \mathbb{R}[x]_{\leq 2}$ задана билинейная форма $\beta : V \times V \to \mathbb{R}$ по правилу
$$
\beta(f, g) = \int_{-1}^{1} f(t)g(t) \, dt.
$$
Найди матрицу этой формы в базисе $1, x, x^2$.

\textbf{Решение:}

\[
e_1(x) = x \vspace e_2(x) = x, \vspace e_3(x) = x^2
\]

\[
A_{ij} = \beta(e_i, e_j) = \int_{-1}^{1} e_i(t)e_j(t) dt
\]

\[
A =
\begin{pmatrix}
2 & 0 & \frac{2}{3} \\
0 & \frac{2}{3} & 0 \\
\frac{2}{3} & 0 & \frac{2}{5}
\end{pmatrix}
\]

$D_1 = 2$; $D_2 = \frac{4}{3}$; $D_3 = \frac{8}{15}$, т.е сигнатура $3, 0, 0$

\vspace{1cm}

\subsection{Задача 8}
\textbf{Условие задачи:}

В пространстве $V = \text{span}(\sin(x), \cos(x), 1)$ задана билинейная форма по правилу
$$
\beta(f, g) = \int_{0}^{\pi} f(x)g(x) \sin(x) \, dx.
$$
\begin{enumerate}
    \item Найди матрицу билинейной формы в базисе $\sin(x)$, $\cos(x)$, $1$.
    \item Проверь, что форма симметрична, и найди сигнатуру $\beta$.
    \item Найди подпространство $U$, ортогональное $\sin(x)$, и определите $\dim U$.
\end{enumerate}

\textbf{Решение:}

\[
e_1(x) = \sin(x) \vspace e_2(x) = \cos(x), \vspace e_3(x) = 1
\]

\[
A_{ij} = \beta(e_i, e_j) = \int_{-1}^{1} e_i(t)e_j(t) dt
\]

\[
A = 
\begin{pmatrix}
\frac{4}{3} & 0 & \frac{\pi}{2} \\
0 & \frac{2}{3} & 0 \\
\frac{\pi}{2} & 0 & 2 \\
\end{pmatrix}
\]

$D_1 = \frac{4}{3}$; $D_2=\frac{8}{9}$; $D_3=\frac{16}{9}-\frac{\pi^2}{6}$  т.е сигнатура $(3, 0, 0)$

\[
g(x) = a\sin(x) + b\cos(x) + c\cdot 1 \Longrightarrow \int_{0}^{\pi} \sin^2(x)(a\sin(x) + b\cos(x) + c) dx = 0 \Rightarrow a = -\frac{3 \pi}{8}c
\]

\[
g(x) = c(-\frac{3 \pi}{8}\sin(x) + 1) + b\cos(x)
\]

\[
U = \span(-\frac{3 \pi}{8}\sin(x)+1;\cos(x)) \land \dim(U) = 2
\]
\vspace{1cm}

\subsection{Задача 9}
\textbf{Условие задачи:}

Пусть $Q: V \to \mathbb{R}$ — невырожденная квадратичная форма и пусть существует ненулевой вектор $v \in V$ такой, что $Q(v) = 0$. Докажи, что отображение $Q: V \to \mathbb{R}$ сюръективно, то есть каждый элемент множества $\mathbb{R}$ является образом некоторого элемента из $V$ (смотри рисунок 1).

\begin{figure}[H]
    \centering
    % Здесь вставьте рисунок 1
    \caption{Сюръективное отображение}
\end{figure}

\textbf{Решение:}

% Здесь вставьте решение задачи

\vspace{1cm}

\subsection{Задача 10}
\textbf{Условие задачи:}

В пространстве $\mathbb{R}^4$ задана билинейная форма
$$
\beta(x, y) = 2x_2y_1 + x_4y_4.
$$
По ней построили квадратичную форму $Q: \mathbb{R}^4 \to \mathbb{R}$. После этого $Q$ ограничили на подпространство
$$
V = \{x \in \mathbb{R}^4 \mid x_1 - 3x_2 - 3x_3 + x_4 = 0\}.
$$
Найди сигнатуру $Q$ и сигнатуру ограничения $Q$ на $V$.

\textbf{Решение:}
\[
u = x_1 + x_2 \land v = x_1 - x_2 \Rightarrow 2x_1x_2 = \frac{u^2-v^2}{2} \Rightarrow Q(x) = \frac{u^2-v^2}{2} + x_4^2
\]

таким образом сигнатура Q на $\mathbb{R}^4 = (2, 1, 1)$

\[
Q|_V = 2(3x_2 + 3x_3 - x_4)x_2 + x_4^2 = 6x_2^2 + 6x_2x_3 - 2x_2x_4 + x_4^2
\]

\[
F = 
\begin{pmatrix}
6 & 3 & -1 \\
3 & 0 & 0 \\
-1 & 0 & 1
\end{pmatrix}
\]

\[
F_d = 
\begin{pmatrix}
6 & 0 & 0 \\
0 & -1.5 & 0 \\
0 & 0 & 1
\end{pmatrix}
\]

Таким образом сигнатура ограниченной формы (2, 1, 0)

\vspace{1cm}

\subsection{Задача 11}
\textbf{Условие задачи:}

Пусть $V = \mathbb{R}[t]_{\leq 3}$ — пространство многочленов степени не больше 3 и пусть $Q: V \to \mathbb{R}$ — квадратичная форма, заданная по правилу
$$
Q(f) = f(1)f(2).
$$
Определи сигнатуру этой формы.

\textbf{Решение:}

\[
f(1) = a + b + c + d \land f(2) = a + 2b + 4c + 8d
\]

\[
Q(f)=a^2+3ab+5ac+9ad+2b ^2+6bc+12bd+4c^2+12cd+8d^2
\]

\[
M = \begin{pmatrix}
1 & 1.5 & 2.5 & 4.5 \\
1.5 & 2 & 3 & 5 \\
2.5 & 3 & 4 & 6 \\
4.5 & 5 & 6 & 8 \\
\end{pmatrix}
\]

\[
M_s =
\begin{pmatrix}
1 & 1.5 & 2.5 & 4.5 \\
0 & -0.25 & -0.75 & -1.75 \\
0 & 0 & 0 & 0 \\
0 & 0 & 0 & 0 \\
\end{pmatrix}
\]

Таким образом сигнатура (1, 1, 2)



\vspace{1cm}

\subsection{Задача 12}
\textbf{Условие задачи:}

Пусть $\beta : V \times V \to \mathbb{R}$ — билинейная форма такая, что для любых векторов $v, u \in V$ условие $\beta(v, u) = 0$ равносильно свойству $\beta(u, v) = 0$. Покажи, что $\beta$ либо симметрична, либо кососимметрична.

\textbf{Решение:}

Пользуясь свойством того, что любую билинейную форму можно единственным образом представить как сумму симметричной и кососимметричной формы

\[
\beta(u, v) = s(u, v) + a(u, v)
\]

\[
s(u, v) = \frac{\beta(u, v) + \beta(v, u)}{2} \quad \text{(симметричная часть)},
\]

\[
a(u, v) = \frac{\beta(u, v) - \beta(v, u)}{2} \quad \text{(кососимметричная часть)}.
\]

Тогда:

\[
\begin{cases}
s(u, v) + a(u, v) = 0, \\
s(u, v) - a(u, v) = 0.
\end{cases}
\]

\[
a(u, v) = 0 \lor s(u, v) = 0.
\]

Если $s(u, v) = 0 \forall u, v \in V$, то \(\beta(u, v) = a(u, v)\), то есть \(\beta\) является кососимметричной.
Если $a(u, v) = 0 \forall u, v \in V$, то \(\beta(u, v) = s(u, v)\), то есть \(\beta\) является симметричной.


\vspace{1cm}

\subsection{Задача 13}
\textbf{Условие задачи:}

Пусть $Q: \mathbb{R}^3 \to \mathbb{R}$ — квадратичная форма, заданная в виде
$$
Q(x) = x^T A x,
$$
где
$$
A =
\begin{pmatrix}
0 & 0 & -2 \\
2 & -3 & 4 \\
0 & 2 & -3
\end{pmatrix}.
$$
Если $\phi: \mathbb{R}^2 \to \mathbb{R}^3$ — некоторое линейное отображение, то определим $Q'(v) = Q(\phi(v))$, где $v \in \mathbb{R}^2$, квадратичную форму на $\mathbb{R}^2$.
\begin{enumerate}
    \item Существует ли такое $\phi$, что $Q'$ имеет сигнатуру $(1, -1)$?
    \item Существует ли такое $\phi$, что $Q'$ имеет сигнатуру $(1, 1)$?
\end{enumerate}

\textbf{Решение:}

Симметризируем матрицу, получим: 

\[ B = 
\begin{pmatrix}
    0 & 1 & -1 \\
    1 & -3 & 3 \\
    -1 & 3 & -3 \\
\end{pmatrix}
\]

Через хар многочлен $D_1 = -\lambda^3-6\lambda^2$ $D_2= \lambda$ $D_3 = \lambda$ $\lambda(\lambda - \frac{-6\pm \sqrt{44}}{2})$ сигнатура (1, 1, 0), ранг - 2, ранг на $R^2$ меньше двух, т.к $N = M^TBM$ где $\phi(v) = Mv$ M - матрица 3 на 2.

Итоговые ответы: да, нет
\vspace{1cm}

\subsection{Задача 14}
\textbf{Условие задачи:}

Докажи, что если в симметричной матрице некоторый главный минор порядка $r$ отличен от нуля, а все окаймляющие его главные миноры порядков $r + 1$ и $r + 2$ равны нулю, то ранг этой матрицы равен $r$.

\textbf{Решение:}

Поскольку существует ненулевой главный минор порядка r, то существует r на r подматрица А, определитель которой ненулевой, следовательно матрциа А имеет по крайней мере ранг r. Все окаймляющие миноры порядка r+1 и т.д равны нулю. В симметричной матрице ранг определяется как максимальный порядок ненулевого главного минора. А т.е все миноры большего порядка которые содержат $A_r$ равны нулю, то не существует ненулевых главных миноров порядка r+1 и выше, следовательно равнг матрицы А равен r

\vspace{1cm}

\end{document}
