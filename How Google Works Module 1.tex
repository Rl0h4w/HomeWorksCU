\documentclass[a4paper,12pt]{article}
\usepackage[utf8]{inputenc}
\usepackage[russian]{babel}
\usepackage{amsmath,amsfonts,amssymb}
\usepackage{graphicx}
\usepackage{geometry}
\usepackage{hyperref}
\usepackage{venndiagram}
\usepackage{tikz}
\usetikzlibrary{shapes.geometric, calc}
\usepackage{pgfplots}
\usepackage{float}
\usepackage{listings}
\usepackage{hyperref}

\lstset{
    language=Python,
    basicstyle=\ttfamily\small,
    keywordstyle=\color{blue},
    stringstyle=\color{red},
    commentstyle=\color{gray},
    morecomment=[l][\color{magenta}]{\#},
    numbers=left,
    numberstyle=\tiny\color{gray},
    stepnumber=1,
    numbersep=10pt,
    backgroundcolor=\color{lightgray!10},
    tabsize=4,
    showspaces=false,
    showstringspaces=false
}

% Параметры страницы
\geometry{top=2cm,bottom=2cm,left=2.5cm,right=2.5cm}
\geometry{a4paper, margin=1in}

% Заголовок документа
\title{Домашнее задание}
\author{Студент: Лохов Ростислав Алексеевич}
\date{\today}

\begin{document}

% Титульный лист
\begin{titlepage}
    \centering
    \vspace*{1cm}

    \Huge
    \textbf{Домашнее задание}

    \vspace{0.5cm}
    \LARGE
    По курсу: \textbf{How Google Works}

    \vspace{1.5cm}

    \textbf{Студент: Лохов Ростислав Алексеевич}

    \vfill

    \Large
    АНО ВО Центральный Университет\\
    \vspace{0.3cm}
    \today

\end{titlepage}

% Содержание
\tableofcontents
\newpage

% Основной текст
\section{Поиск как продукт}

\subsection{Задача 1}
\textbf{Условие задачи:}
\begin{itemize}
    \item[a)] Приведи два конкретных и реальных примера того, чем современный поисковик (Яндекс или Google) отличается от базового варианта с лекции. Для каждого примера покажи, на какую метрику из дерева метрик он влияет. Если необходимо, можешь добавить дополнительную метрику в дерево метрик. Хорошие примеры: сценарий «задать вопрос голосом», который мы упомянули в лонгриде; наличие рекламных страниц в топе выдачи.
    \item[б)] Предложи одну новую идею, которая могла бы улучшить существующие поисковые системы. Покажи, на какую метрику из дерева метрик она может повлиять.
\end{itemize}

\textbf{Решение:}

\textbf{а) Решение:}
\begin{itemize}
    \item[1.] Google Lens для поиска по фото.
    \item[2.] Отличным дополнением аудиопоиска будет интеграция по поиску музыки (например, Shazam) (реализовано у Google).
    \item[3.] У Яндекса и Google есть свои сервисы, которые помогают приводить данные к одному профилю с разных устройств, тем самым повышая релевантность выдачи запросов. Допустим, у поисковика DuckDuckGo нет таких сервисов, что делает его более анонимным, но менее точным в выдаче.
\end{itemize}

\textbf{б) Решение:}
\begin{itemize}
    \item[1.] Можно добавить функцию суммирования информации с топ-5 сайтов по запросу и выводить итоговый ответ прямо на странице поиска, не заходя на сайты.
    \item[2.] Это могло бы повысить количество пользователей, так как AI и ML — топовые технологии 2024 года, которые позволяют улучшить UX/UI. Однако это может привести к увеличению стоимости одного запроса.
\end{itemize}

\vspace{1cm}

\subsection{Задача 2}
\textbf{Условие задачи:}
\begin{itemize}
    \item[a)] Постройте матрицу смежности для графа Б (рисунок 1). Считайте, что вершины упорядочены по алфавиту.
    \item[б)] Постройте граф по матрице смежности.
    \item[в)] Постройте матрицу смежности графа А (рисунок 1), считая его ненаправленным. Определите уникальное свойство такой матрицы.
\end{itemize}

\textbf{Решение:}

\textbf{а) Матрица смежности:}
\[
\begin{array}{c|cccccccc}
      & A & B & C & D & E & F & G & H \\ \hline
    A & 0 & 1 & 1 & 0 & 0 & 0 & 0 & 0 \\
    B & 0 & 0 & 0 & 1 & 1 & 0 & 0 & 0 \\
    C & 0 & 0 & 0 & 0 & 0 & 1 & 1 & 0 \\
    D & 1 & 0 & 0 & 0 & 0 & 0 & 0 & 1 \\
    E & 0 & 0 & 0 & 0 & 0 & 0 & 0 & 1 \\
    F & 1 & 0 & 0 & 0 & 0 & 0 & 0 & 0 \\
    G & 1 & 0 & 0 & 0 & 0 & 0 & 0 & 0 \\
    H & 1 & 0 & 0 & 0 & 0 & 0 & 0 & 0 \\
\end{array}
\]

\textbf{б) Построение графа:}
\begin{tikzpicture}[scale=1, transform shape]
    \node[circle, draw] (A) at (0, 2) {A};
    \node[circle, draw] (B) at (-2, 0) {B};
    \node[circle, draw] (C) at (2, 0) {C};
    \node[circle, draw] (D) at (-2, -2) {D};
    \node[circle, draw] (E) at (0, -2) {E};
    \node[circle, draw] (F) at (2, -2) {F};

    \draw[->] (A) -- (B);
    \draw[->] (A) -- (C);
    \draw[->] (B) -- (A);
    \draw[->] (B) -- (C);
    \draw[->] (B) -- (D);
    \draw[->] (B) -- (E);
    \draw[->] (C) -- (E);
    \draw[->] (D) -- (A);
    \draw[->] (D) -- (E);
    \draw[->] (E) -- (D);
    \draw[->] (F) -- (E);
\end{tikzpicture}

\textbf{в) Матрица смежности ненаправленного графа:}
\[
\begin{array}{c|cccc}
      & A & B & C & D \\ \hline
    A & 0 & 0 & 1 & 0 \\
    B & 0 & 0 & 1 & 1 \\
    C & 1 & 1 & 0 & 1 \\
    D & 0 & 1 & 1 & 0 \\
\end{array} 
\]
Матрица симметрична относительно главной диагонали.

\vspace{1cm}

% Добавляйте дополнительные разделы и задачи по мере необходимости

\section{PageRank. Авторитетность страницы}

\subsection{Задача 3}
\textbf{Условие задачи:}
\begin{itemize}
    \item[a)] Подсчитайте авторитетность каждой страницы графа Б после голосования. Чему равна сумма итоговых авторитетностей всех страниц?
\end{itemize}

\textbf{Решение:}

\begin{lstlisting}[language=Python]
d = {'A': ['B', 'C'],
     'B': ['D', 'E'],
     'C': ['F', 'G'],
     'D': ['A', 'H'],
     'E': ['A', 'H'],
     'F': ['A'],
     'G': ['A'],
     'H': ['A']}
res = {i: 0. for i in d.keys()}
for key in d.keys():
    for val in d[key]:
        res[val] += 1/len(d[key])
print(res)
\end{lstlisting}

Результат выполнения этого кода:

\[
\text{{'A': 4.0, 'B': 0.5, 'C': 0.5, 'D': 0.5, 'E': 0.5, 'F': 0.5, 'G': 0.5, 'H': 1.0}}
\]

Сумма авторитетностей всех страниц: 8.0.

\vspace{1cm}

\subsection{Задача 4}
\textbf{Условие задачи:}
\begin{itemize}
    \item[a)] ПК сожалению, идея 1 сама по себе не даёт хорошего решения задачи определения авторитетности. Вернёмся к сайтам aviasales.ru и nonametickets.ru. Что должен сделать владелец
nonametickets.ru, чтобы по итогам голосования получить авторитетность выше, чем у сайта
aviasales.ru?
В этой задаче мы считаем, что каждый сайт состоит из одной страницы.
\end{itemize}

\textbf{Решение:}

\textbf{а) Решение:}
\begin{itemize}
    \item[1.] Наделать сайтов с ссылками на наш "Плохой" сайт
\end{itemize}

\vspace{1cm}

\subsection{Задача 5}
\textbf{Условие задачи:}
\begin{itemize}
    \item[a)] Подсчитай авторитетность каждой страницы графа Б на рисунке 1 после второго раунда голосования. Чему равна сумма авторитетностей всех страниц?
\end{itemize}



\textbf{а) Решение:}

\begin{lstlisting}[language=Python]
d = {'A': ['B', 'C'],
     'B': ['D', 'E'],
     'C': ['F', 'G'],
     'D': ['A', 'H'],
     'E': ['A', 'H'],
     'F': ['A'],
     'G': ['A'],
     'H': ['A']}

res = {i: 0. for i in d.keys()}

for key in d.keys():
    for val in d[key]:
        res[val] += 1/len(d[key])

new_res = {i: 0. for i in d.keys()}

for key in d.keys():
    for val in d[key]:
        new_res[val] += res[key] / len(d[key])

print(sum(new_res.values()))
\end{lstlisting}

Результат выполнения этого кода:

\[
\text{{'A': 2.5, 'B': 2.0, 'C': 2.0, 'D': 0.25, 'E': 0.25, 'F': 0.25, 'G': 0.25, 'H': 0.5}}
\]

Сумма авторитетностей всех страниц: 8.0.
\vspace{1cm}
\subsection{Задача 6}
\textbf{Условие задачи:}
\begin{itemize}
    \item[a)] Проведи еще одну итерацию голосования на графе А, начиная со значений из таблицы 3. Прокомментируй результат.
\end{itemize}

\textbf{а) Решение:}
\[
PR_i^{(k+1)} = \sum_{j \in M_i} \frac{PR_j^{(k)}}{L_j}
\]

\[
\exists PR_i: \ \lim_{k \to \infty} PR_i^{(k)} = PR_i
\]

\[
\forall \epsilon > 0 \ \exists K \in \mathbb{N} \ \forall k > K: \ |PR_i^{(k+1)} - PR_i^{(k)}| < \epsilon, \ \forall i
\]
\vspace{1cm}

\subsection{Задача 7}
\textbf{Условие задачи:}
\begin{itemize}
    \item[a)] Найди предельные значения авторитетностей (k → ∞) для каждого из графов на рисунке 2, не
используя Python или другой язык программирования. Есть ли проблема в полученных результатах?
Если есть, то как нужно изменить алгоритм Basic PageRank, чтобы ее исправить?
\end{itemize}
\textbf{а) Решение:}
Проблемы:\\

Тупик в графе \( a \) вершина \( G \) не имеет исходящих ссылок, что приводит к утечке авторитетности.\\
  
Циклы \( b \) присутствуют циклы, такие как \( F \to G \to F \), что может вызвать замкнутое движение авторитетности и препятствовать сходимости.\\

Решение проблем:

Использование демпфирующего фактора \( d \) в уравнении PageRank:

\[
PR_i = \frac{1 - d}{N} + d \sum_{j \in M_i} \frac{PR_j}{L_j}
\]

где:
- \( N \) — общее количество страниц,
- \( M_i \) — множество страниц, ссылающихся на страницу \( i \),
- \( L_j \) — количество исходящих ссылок у страницы \( j \),
- \( d \) — демпфирующий фактор, обычно \( d = 0.85 \).
\vspace{1cm}
\section{Crowler, BFS, DFS}

\subsection{Задача 8}
\textbf{Условие задачи:}
\begin{itemize}
    \item[a)]  Получи финальные значения авторитетностей с точностью до двух знаков после запятой для графа б при к=1000
    \item[б)] Построй график, в котором по оси X отложены итерации, а по оси Y — значения авторитетностей
каждой из страниц. Прокомментируй результат.
    \item[в)] В LMS приложи ссылку на Google Colab, открытый для чтения, Jupyter Notebook или другой
файл, в котором ты проводил(-а) вычисления.
\end{itemize}

\textbf{Решение:}
\href{https://colab.research.google.com/drive/1H1bMEVws8LbPNbGYGrFPi2vDS5lYgGls?usp=sharing}{COLAB}
\vspace{1cm}
\subsection{Задача 9}
\textbf{Условие задачи:}
\begin{itemize}
    \item[a)]  В каком порядке краулер со стратегией BFS будет посещать страницы на графе Б (смотри
рисунок 1)? Краулер начинает исследование с вершины E и при прочих равных присваивает
порядковый номер страницам в алфавитном порядке.                \item[б)] . Альтернативная стратегия выбора страниц для посещения — Depth-First Search. Найди описание этой стратегии в поисковике и опиши её своими словами. Покажи, в каком порядке краулер
со стратегией DFS будет посещать страницы на ненаправленном графе из рисунка 4. Считай,
что краулер начинает исследование с вершины A и при прочих равных выбирает страницы
для посещения в алфавитном порядке.
\end{itemize}

\textbf{Решение:}
\begin{itemize}
    \item[a)] ['E', 'H', 'A', 'B', 'C', 'D', 'F', 'G']\\
    \item[б)] ['A', 'B', 'D', 'E', 'H', 'F', 'C', 'G']
\end{itemize}
\end{document}
