\documentclass[a4paper,12pt]{article}

% Кодировка и язык
\usepackage[utf8]{inputenc}
\usepackage[russian]{babel}

% Математические пакеты
\usepackage{amsmath,amsfonts,amssymb}

% Графика
\usepackage{graphicx}
\usepackage{tikz}
\usetikzlibrary{shapes.geometric, calc}
\usepackage{pgfplots}

% Геометрия страницы
\usepackage{geometry}
\geometry{top=2cm, bottom=2cm, left=2.5cm, right=2.5cm}

% Гиперссылки
\usepackage{hyperref}

% Плавающие объекты
\usepackage{float}

% Дополнительные пакеты
\usepackage{venndiagram}

% Настройки заголовка
\title{Домашнее задание}
\author{Студент: \textbf{Ростислав Алексеевич Лохов}}
\date{\today}

\begin{document}

% Титульный лист
\begin{titlepage}
    \centering
    \vspace*{1cm}

    \Huge
    \textbf{Домашнее задание}

    \vspace{0.5cm}
    \LARGE
    По курсу: \textbf{Дифференцирование. Часть 1}

    \vspace{1.5cm}

    \textbf{Студент: Ростислав Алексеевич Лохов}

    \vfill

    \Large
    АНО ВО "Центральный Университет"\\
    \vspace{0.3cm}
    \today

\end{titlepage}

% Содержание
\tableofcontents
\newpage

% Основной текст
\section{Дифференцирование. Часть 1}

\subsection{Дифференцируемые функции и производная}

\subsubsection{Задача 1 (1 балл)}
Вычисли
\[
\lim_{x \to 0} \frac{\tan(\sin(x + a)) - \tan(\sin(a - x))}{x}.
\]

\textbf{Решение:}

% Здесь вставьте ваше решение задачи 1

\vspace{1cm}

\subsection{Односторонние производные}

\subsubsection{Задача 2 (1 балл)}
Найди константы \( a \) и \( b \), чтобы функция
\[
f(x) =
\begin{cases}
\frac{x^3 - 2x^2 - 9x + 18}{x^2 + x - 6}, & \text{если } x < 2, \\
a, & \text{если } x = 2, \\
\frac{a}{2x} + b, & \text{если } x > 2.
\end{cases}
\]
была непрерывной при \( x = 2 \). Определи, для каких \( x \) эта функция будет дифференцируемой при найденных значениях \( a \) и \( b \).

\textbf{Решение:}

% Здесь вставьте ваше решение задачи 2

\vspace{1cm}

\subsection{Правила дифференцирования}

\subsubsection{Задача 3 (2 балла)}
Найди производную функции на её области определения:
\begin{enumerate}
    \item[a)] \( y(x) = \frac{\sqrt{4} \cdot (8x - 3)^3}{x^2 - 1} \);
    \item[b)] \( y(x) = \ln(\tan x) + \frac{1}{2} \cot 2x \);
    \item[c)] \( y(x) = x \cdot 2^{1 - x^2} \);
    \item[d)] \( y(x) = \frac{1}{3} \ln\left(\sqrt{9 + 4x^2}\right) - \frac{3}{x} \).
\end{enumerate}

\textbf{Решение:}

% Здесь вставьте ваше решение задачи 3

\vspace{1cm}

\subsection{Частичные суммы и производные}

\subsubsection{Задача 4 (0,5 балла)}
Дана частичная сумма:
\[
f_n(x) = 1 + 2x + 3x^2 + \ldots + nx^{n-1}.
\]
Найди \( f'_n(x) \).

\textbf{Решение:}

% Здесь вставьте ваше решение задачи 4

\vspace{1cm}

\subsection{Производная в конкретной точке}

\subsubsection{Задача 5 (1 балл)}
Найди производную в точке \( x = 3 \):
\[
y(x) = \left(\cot\left(x^3\right)\right)^{\sqrt{3}} e^{x} - e^{-x}.
\]

\textbf{Решение:}

% Здесь вставьте ваше решение задачи 5

\vspace{1cm}

\subsection{Обратные функции и их производные}

\subsubsection{Задача 6 (1 балл)}
Найди производную обратной функции к \( y(x) = x^2 + 2x + 2 \) в точках \( y = 1 \) и \( y = 2 \).

\textbf{Решение:}

% Здесь вставьте ваше решение задачи 6

\vspace{1cm}

\subsection{Понятие производной}

\subsubsection{Задача 7 (1,5 балла)}
Пользуясь определением, найди производные функций:
\begin{enumerate}
    \item[a)] \( y(x) = x^3 \);
    \item[b)] \( y(x) = \frac{1}{x^2} \);
    \item[c)] \( y(x) = \cos x \).
\end{enumerate}

\textbf{Решение:}
a)

\[
\lim_{\delta \to 0}\frac{f(x + \delta) - f(x)}{\delta}
\]

\[
\lim_{\delta \to 0} \frac{(x+\delta)^3-x^3}{\delta} = \lim_{\delta \to 0} \frac{3x^2\delta + 2x\delta^2+\delta^3}{\delta} = 3x^2
\]

b)

\[
\lim_{\delta \to 0} \frac{(x+\delta)^{-2}-x^{-2}}{
\delta} = \lim_{\delta \to 0} \frac{-2x-\delta}{x^4-2x^3\delta + x^3\delta^2} = \frac{-2}{x^3}
\]

c)

\[
\lim_{\delta \to 0} \frac{\cos(x+\delta) - \cos(x)}{\delta} = \lim_{\delta \to 0} \frac{-2\sin(x+\frac{\delta}{2}) \sin(\frac{\delta}{2})}{\delta} = -\sin(x)
\]


\vspace{1cm}

\subsection{Производные и пределы}

\subsubsection{Задача 8 (3 балла)}
Функция \( f \) имеет производную в точке \( x = a \). Найди предел
\[
\lim_{n \to \infty} n \left( f\left(a + \frac{1}{n}\right) - f\left(a - \frac{1}{n}\right) \right),
\]
считая \( f'(a) \) известной. 

\textbf{Решение:}

\[
f'(x) = \lim_{\frac{1}{n} \to 0} \frac{f(a+\frac{1}{n}) - f(a)}{\frac{1}{n}} = \lim_{\frac{1}{n} \to 0} \frac{f(a) - f(a-\frac{1}{n})}{\frac{1}{n}} \Rightarrow  2f'(x) = \lim_{n \to \infty} \frac{f(a+\frac{1}{n}) - f(a-\frac{1}{n})}{\frac{1}{n}} 
\]

Тогда 

\[
\lim_{n \to \infty} n \left( f\left(a + \frac{1}{n}\right) - f\left(a - \frac{1}{n}\right) \right) = 2f'(a)
\]
\vspace{1cm}

\subsection{Дифференцируемость в точке}

\subsubsection{Задача 9 (2 балла)}
Исследуй на дифференцируемость в точке \( x = 0 \) функцию
\[
y(x) =
\begin{cases}
x \sin\left(\frac{1}{x}\right), & \text{если } x \neq 0, \\
0, & \text{если } x = 0.
\end{cases}
\]

\textbf{Решение:}

т.к если мы возьмем слева и справа производую, то получим, что конкретного значения у синуса нет, т.к в аргументе данной функции будет бесконечность и результатом будет осциляция между -1 и 1, т.е возникает ситуация, где левосторонняя и правосторонняя производные не равны между собой, следовательно функция недифференциируема в данной точке.

\vspace{1cm}

\subsection{Дифференцируемость и непрерывность}

\subsubsection{Задача 10 (2 балла)}
Найди \( f'_-(0) \) и \( f'_+(0) \) для функции \( f(x) = x - |x| \). Исследуй на дифференцируемость функцию \( f(x) \) в точке \( x = 0 \).

\textbf{Решение:}

\[
f'_-(0) = 2, \quad f'_+(0) = 0.
\]
\vspace{1cm}

\subsection{Построение графиков и производные}

\subsubsection{Задача 11 (0,5 балла)}
Построй график функции
\[
y(x) =
\begin{cases}
0, & \text{если } x \leq 0, \\
4 - x, & \text{если } 0 < x < 3, \\
\frac{1}{4 - x}, & \text{если } x \geq 3.
\end{cases}
\]
Исследуй функцию \( y \) на непрерывность и дифференцируемость на \( \mathbb{R} \).

\textbf{Решение:}

% Здесь вставьте ваше решение задачи 11

\vspace{1cm}

\subsection{Непрерывность и дифференцируемость}

\subsubsection{Задача 12 (4 балла)}
Пусть функция \( f : \mathbb{R} \to \mathbb{R} \) дифференцируема в точке \( x_0 \). Верно ли, что существует окрестность точки \( x_0 \), в которой \( f \) непрерывна?

\textbf{Решение:}
\[
f(x) = \begin{cases}
    x^2, x \in \mathbb{Q}\\
    -x^2, x \in \mathbb{R} \land x \not \in \mathbb{Q} \\
\end{cases}
\]
в точке 0 функция дифференцируема, однако в окрестности она не дифференцируема и прерывна
\[
f'(x) = \lim_{\delta \to 0} \frac{f(\delta)}{\delta}
\]

т.е всё будет зависеть от дельта малого

\vspace{1cm}

\subsection{Геометрический смысл производной}

\subsubsection{Задача 13 (1,5 балла)}
Запиши уравнение той касательной к параболе \( y = 4 - x^2 \), которая:
\begin{enumerate}
    \item[a)] имеет угол наклона с осью Ox, равный \( \frac{3\pi}{4} \);
    \item[b)] параллельна прямой \( y = 2x - 3 \).
\end{enumerate}

\textbf{Решение:}
a)

\[
y=\tan(\frac{3\pi}{4})x+3+0.25 = -x+4.25
\]

b)
\[
y=2x+5
\]
\vspace{1cm}

\subsection{Касательные и их углы}

\subsubsection{Задача 14 (3 балла)}
На параболе \( y = x^2 - 4x + 2 \) найди такие точки, что проведённые в них касательные к параболе проходят через точку \( (4; 1) \). Найди угол между этими касательными.

\textbf{Решение:}
очевидно, что тангенс угла наклона зависит от производной по этой функции, т.е 
\[
k = 2x_0-4
\]
Ур-ие касательной в точке $P(x_0, y_0)$ имеет вид: 

\[
1-x_0^2+4x_0 - 2 = (2x_0-4)(4-x_0)
\]

\[
(x_0-5)(x_0-3)=0
\]

\[
P_1(3, -1) \land P_2(5, 7) \Rightarrow m_1 = 2 \land m_2 = 6
\]

\[
\tan(\theta) = \left| \frac{m_2-m_1}{1+m_1m_2} \right| \Rightarrow \theta = \arctan(\frac{4}{13})
\]

\vspace{1cm}

\subsection{Сопоставление графиков}

\subsubsection{Задача 15 (2 балла)}
Сопоставь график каждой функции из (а)–(г) с графиком её производной из I–IV.
\begin{enumerate}
    \item[a)]
    \item[б)]
    \item[в)]
    \item[г)]
\end{enumerate}

% Здесь, вероятно, необходимо вставить изображения графиков функций и их производных.

\textbf{Решение:}

a - II\\

b - IV\\

c - I\\

d - III\\

\vspace{1cm}

\subsection{Практический смысл производной}

\subsubsection{Задача 16 (1,5 балла)}
Мяч выпал из окна без начальной скорости с высоты \( H = 122{,}5 \) м. Найди:
\begin{enumerate}
    \item[a)] Скорость мяча через 2 секунды после начала полета;
    \item[b)] Скорость мяча в момент, когда он ударится о землю.
\end{enumerate}
Считай, что расстояние, которое мяч пролетит за время \( t \), равно \( s(t) = 4{,}9 t^2 \) (расстояние измеряется в метрах, \( t \) — в секундах).

\textbf{Решение:}

\[
\frac{ds}{dt} = 9.7t = v(t)
\]

\[
v(2) = 19.4
\]

\[
H = s(t) \Rightarrow 122.5 = 4.9t^2 \Rightarrow t = 5 \Rightarrow v(5) = 9.7*5 = 48.5
\]

\vspace{1cm}

\subsection{Скорость и время}

\subsubsection{Задача 17 (1,5 балла)}
Стоимость производства \( x \) единиц товара в рублях равна \( C(x) = 5000 + 10x + 0{,}05x^2 \).
\begin{enumerate}
    \item[1.] Найди среднюю скорость изменения \( C \), когда уровень производства изменился:
    \begin{enumerate}
        \item[a)] с \( x = 100 \) до \( x = 105 \);
        \item[б)] с \( x = 100 \) до \( x = 101 \).
    \end{enumerate}
    \item[2.] Найди мгновенную скорость изменения \( C \), когда уровень производства \( x = 100 \).
\end{enumerate}

\textbf{Решение:}

\[
\frac{dC}{dx} = 0.1x + 10
\]

\[
\overline{\frac{dC}{dx}} = \frac{C(x_2) - C(x_1)}{x_2 - x_1}
\]
\begin{enumerate}
    \item a)20.25 б)20.05
    \item 20
    
\end{enumerate}




\vspace{1cm}

\end{document}
