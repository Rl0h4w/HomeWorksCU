\documentclass[a4paper,12pt]{article}
\usepackage[utf8]{inputenc}
\usepackage[russian]{babel}
\usepackage{amsmath,amsfonts,amssymb}
\usepackage{graphicx}
\usepackage{geometry}
\usepackage{hyperref}
\usepackage{venndiagram}
\usepackage{tikz}
\usetikzlibrary{shapes.geometric, calc}
\usepackage{pgfplots}
\usepackage{float}

% Параметры страницы
\geometry{top=2cm,bottom=2cm,left=2.5cm,right=2.5cm}
\geometry{a4paper, margin=1in}

% Заголовок документа
\title{Домашнее задание}
\author{Студент: Ваше Имя Здесь}
\date{\today}

\begin{document}

% Титульный лист
\begin{titlepage}
    \centering
    \vspace*{1cm}

    \Huge
    \textbf{Домашнее задание}

    \vspace{0.5cm}
    \LARGE
    По курсу: \textbf{Линейная Алгебра}

    \vspace{1.5cm}

    \textbf{Студент: Лохов Ростислав}

    \vfill

    \Large
    АНО ВО Центральный Университет\\
    \vspace{0.3cm}
    \today

\end{titlepage}

% Содержание
\tableofcontents
\newpage

% Основной текст
\section{Задачи для самостоятельного решения}

\subsection{Задача 1}
\textbf{Условие задачи:}
Найди произведение
\[
\left( \begin{array}{cc}
1 & m \\
0 & 1 \\
\end{array} \right)
\left( \begin{array}{cc}
1 & n \\
0 & 1 \\
\end{array} \right);
\]
\[
\left( \begin{array}{cc}
\cos(x) & -\sin(x) \\
\sin(x) & \cos(x) \\
\end{array} \right)
\left( \begin{array}{cc}
\cos(y) & -\sin(y) \\
\sin(y) & \cos(y) \\
\end{array} \right).
\]

\subsection{Задача 2}
\textbf{Условие задачи:}
Найди все матрицы $3 \times 3$, коммутирующие с данной:
\[
\begin{pmatrix}
-3 & 0 & 2 \\
5 & -3 & 5 \\
0 & 0 & -5 \\
\end{pmatrix}.
\]

\subsection{Задача 3}
\textbf{Условие задачи:}
Найди множество матриц $X \in M_{4 \times 4}(\mathbb{R})$, коммутирующих с
\[
A =
\begin{pmatrix}
0 & 1 & 0 & 0 \\
0 & 0 & 1 & 0 \\
0 & 0 & 0 & 1 \\
1 & 0 & 0 & 0 \\
\end{pmatrix}.
\]
\textbf{Решение: }
\[
X =
\begin{pmatrix}
a & b & c & d \\
d & a & b & c \\
c & d & a & b \\
b & c & d & a \\
\end{pmatrix}.
\]

\subsection{Задача 4}
\textbf{Условие задачи:}
Найди матрицу, обратную к данной:
\[
\begin{pmatrix}
2 & 1 & 2 & 3 \\
0 & 1 & -1 & 0 \\
1 & 2 & 0 & 1 \\
0 & 1 & -2 & 0 \\
\end{pmatrix}.
\]


\subsection{Задача 5}
\textbf{Условие задачи:}

Найди $f(A)$ для:
1. $f = x^2 - 2x + 2$, $A =
\begin{pmatrix}
2 & -1 & 1 \\
3 & 1 & 2 \\
1 & -1 & 0 \\
\end{pmatrix}$;
2. $f = \frac{x^2 - 2}{x^2 - 3}$, $A =
\begin{pmatrix}
-1 & 1 \\
3 & 2 \\
\end{pmatrix}$.

\textbf{Решение: }

1.
$A^2 =
\begin{pmatrix}
2 & -4 & 0 \\
11 & -4 & 5 \\
-1 & -2 & -1 \\
\end{pmatrix}$;
 $f = A^2 - 2A + 2I=
 \begin{pmatrix}
0 & -2 & -2 \\
5 & -4 & 1 \\
-3 & 0 & 1 \\
\end{pmatrix}
 $

 2.
$A^2 =
\begin{pmatrix}
4 & 1 \\
3 & 7 \\
\end{pmatrix}$;
$\frac{A^2 - 2I}{A^2 - 3I}=
\begin{pmatrix}
2 & 1 \\
3 & 5 \\
\end{pmatrix}\cdot \frac{1}{\det{(A-3I)}}\cdot 
\begin{pmatrix}
4 & -1 \\
-3 & 1 \\
\end{pmatrix}=\begin{pmatrix}
2 & 1 \\
3 & 5 \\
\end{pmatrix}\cdot \begin{pmatrix}
4 & -1 \\
-3 & 1 \\
\end{pmatrix}=
\begin{pmatrix}
5 & -1 \\
-3 & 2 \\
\end{pmatrix}
$
\subsection{Задача 6}
\textbf{Условие задачи:}
Реши матричное уравнение $AX = B$, где
\[
A = 
\begin{pmatrix}
-1 & 2 & -1 \\
1 & -1 & 2 \\
2 & -3 & 2 \\
\end{pmatrix}, \quad B = 
\begin{pmatrix}
3 & 3 & 4 \\
2 & 4 & 1 \\
-3 & -4 & -4 \\
\end{pmatrix}.
\]

\subsection{Задача 7}
\textbf{Условие задачи:}
Реши матричное уравнение для $X \in M_{3 \times 3}(\mathbb{R})$:
\[
\begin{pmatrix}
-1 & -1 & 2 \\
2 & 1 & -2 \\
-1 & 1 & -1 \\
\end{pmatrix}
\left( X + 
\begin{pmatrix}
3 & 3 & -1 \\
-2 & -1 & 2 \\
1 & 2 & 3 \\
\end{pmatrix} \right)^{-1}
\begin{pmatrix}
1 & -1 & -1 \\
-2 & 3 & 3 \\
1 & -2 & -1 \\
\end{pmatrix} =
\begin{pmatrix}
1 & -2 & -2 \\
1 & -1 & -1 \\
-2 & 2 & 3 \\
\end{pmatrix}.
\]

\textbf{Решение: }

$A\cdot (X+B)^{-1}\cdot C = D$

$(X+B)^{-1}=A^{-1}\cdot D\cdot C^{-1}$

$X=A\cdot D^{-1}\cdot C - B=
\begin{pmatrix}
-1 & 1 & 4 \\
-4 & 6 & 3 \\
8 & -11 & -13 \\
\end{pmatrix}-
\begin{pmatrix}
3 & 3 & -1 \\
-2 & -1 & 2 \\
1 & 2 & 3 \\
\end{pmatrix}=
\begin{pmatrix}

-4 & -2 & 5 \\
-2 & 7 & 1 \\
7 & -13 & -16 \\
\end{pmatrix}
$

\subsection{Задача 8}
\textbf{Условие задачи:}
Пусть даны матрицы:
\[
A = 
\begin{pmatrix}
-4 & -2 & 2 \\
2 & 3 & 1 \\
3 & 4 & 2 \\
-1 & -3 & -2 \\
4 & 2 & -2 \\
\end{pmatrix}, \quad
B = 
\begin{pmatrix}
2 & 0 & 0 & 1 \\
-4 & 0 & 1 & 0 \\
1 & 1 & 0 & 0 \\
\end{pmatrix}.
\]
Найди решения системы $ABx = 0$, $x \in \mathbb{R}^4$. Подумай, как решить задачу не «в лоб».

\textbf{Решение: }
Приведя матрицу к диагональному виду, получим, что строка 5 и 4 линейно зависимы от других строк.
Таким образом матрица $A=
\begin{pmatrix}
-4 & -2 & 2 \\
2 & 3 & 1 \\
3 & 4 & 2 \\
\end{pmatrix}
$

$A\cdot B=\begin{pmatrix}
2 & 2 & -2 & 4 \\
-7 & 1 & 3 & 2 \\
-8 & 2 & 4 & 3\\
\end{pmatrix}=0$

Приведем к диагональному виду:
\begin{pmatrix}
1 & 0 & 0 & 0.5 \\
0 & 1 & 0 & -1 \\
0 & 0 & 1 & 2\\
\end{pmatrix}


таким образом $x = t \cdot
\begin{pmatrix}
-0.5 \\
1 \\
-2 \\
1 \\
\end{pmatrix}, t \in \mathbb{R}$

\subsection{Задача 9}
\textbf{Условие задачи:}
Найди хотя бы 3 матрицы из $M_{2 \times 2}(\mathbb{R})$, помимо $A = I$ и $A = -I$, такие, что $A^2 = I$.
\textbf{Решение: }
1. \( \begin{pmatrix} 1 & 0 \\ 0 & -1 \end{pmatrix} \)
2. \( \begin{pmatrix} -1 & 0 \\ 0 & 1 \end{pmatrix} \)
3. \( \begin{pmatrix} 0 & 1 \\ 1 & 0 \end{pmatrix} \)

\subsection{Задача 10}
\textbf{Условие задачи:}
Покажи, что квадратная матрица не может быть одновременно невырожденной и нильпотентной. Напомним, что нильпотентной матрицей называется матрица $A \in M_{n \times n}(\mathbb{R})$, такая, что для некоторого натурального $m$ выполнено $A^m = 0$.

\textbf{Решение: }
$A^m\cdot A^{-m}= 0\cdot A^{-m}\ne I$ т.е при произведении матрицы на обратную, мы получили 0, а должны были по определению получить единичную матрицу. Вообще есть свойство у нильпотентной - определитель равен 0 всегда, т.е не существует обратной матрицы.


\subsection{Задача 11}
\textbf{Условие задачи:}
Пусть $A \in M_{n \times n}(\mathbb{R})$ — нильпотентная матрица. Покажи, что $I + A$ и $I - A$ обратимы, где $I \in M_{n \times n}(\mathbb{R})$ — единичная матрица (нужно найти явный вид обратной матрицы).

\textbf{Решение: } Здесь есть два способа решения, один через разложение в степенной ряд $\frac{1}{1+x}$, сходящейся |x|<1 и мы можем это использовать т.к степени больше m+1 при $A^m=0$ равны 0 и ряд сходится. Но мне больше импонирует доказательство через собственные числа. По свойству нильпотентных матриц их собственные числа = 0, а у единичной равны 1. Отсюда следует, что собственные числа для I+A равны 1, для 1-A тоже 1,  и т.к определитель равен произведению собственных чисел, отсюда следует, что т.к произведение не равно 0, то обратная матрица к обоим случаям существует.



\subsection{Задача 12}
\textbf{Условие задачи:}
Пусть $A \in M_{m \times n}(\mathbb{R})$. Рассмотрим две системы $Ax = 0$ и $A^T y = 0$. Покажи, что у этих систем одинаковое количество главных переменных.

\textbf{Решение: }
\begin{enumerate}
    \item для любой матрицы А достаточно что кол-во переменных в системе $A^Ty=0$ не превышеет кол-во в $Ax=0$. 
    
    \item Пусть кол-во главных переменных в Ax=0 равно d. Тогда применив метод гаусса для приведения к улучшенному ступенчатому виду получим $B=U_1A$, где $U_1$ - произведение элементарных матриц соответствующие строковым преобразованиям. В матрице В будет ровно d ненулевых строк, каждая начинается с главной переменной. Используя элементарные преобразования столбцов, получим 
    $C=BU_2=\begin{pmatrix}
        I_d & 0\\
        0 & 0\\
    \end{pmatrix}$ где $I_d$ - единичная матрица размености dxd a $U_2$ - произведение элементарных матриц. Рассмотрим $D=AU_2$, $C=U_1AU_2=U_1D$, матрица D имеет не более d ненулевых столбцов.
    
    \item $A^Ty=0$, Транспонируем матрицу D, $D^T=U_2^TA^T$, поскольку матрица D имеет не более d ненулевых столбцов, то транспонированная будет иметь не более d ненулевых строк, приведя к улучшенному ступенчатому виду мы получим, что количество главных переменных не превышает D. Таким же образом применяем к транспонированной A, исходя из этого получаем, что количество главных переменных в $Ax = 0$ не превышает кол-во главных переменных $A^T y = 0$
\end{enumerate}

\subsection{Задача 13}
\textbf{Условие задачи:}
Пусть даны матрицы $A, B \in M_{n \times n}(\mathbb{R})$.
1. Предположим, что $A$ и $B$ нильпотентны и коммутируют друг с другом. Покажи, что $A + B$ тоже нильпотентна.
2. Покажи, что если матрицы $A$ и $B$ нильпотентны, но не коммутируют, то $A + B$ может быть невырожденной, а следовательно, не может быть нильпотентной.
3. Напомним, что $[A,B] = AB - BA$. Предположим, что $A$, $B$ и $[A,B]$ нильпотенты. Кроме того, матрицы $A$ и $[A,B]$ коммутируют, а также $B$ и $[A,B]$ коммутируют. Докажи, что $A + B$ тоже будет нильпотентной.

\textbf{Решение}
\begin{enumerate}
    \item т.к матрицы коммутируют, то можно пойти через биномиальную форму степени. $(A+B)^k=\sum_{i=0}^{k}\binom{k}{i}A^iB^{k-i}$ т.к A и B нильпотентны, то существует такое n, m, что $A^m, B^n$ равны 0, пусть $k\ge m+n$ тогда либо $i\ge m$, либо $k-i\ge n$, следовательно $A^iB^{k-i}=0$, таким образом $(A+B)^k=0$ чтд
    
    \item    \[
   A = \begin{pmatrix}
   0 & 1 \\
   0 & 0
   \end{pmatrix}, \quad
   B = \begin{pmatrix}
   0 & 0 \\
   1 & 0
   \end{pmatrix}
   \] обе нильпотентны, не коммутируют:    \[
   [A, B] = AB - BA = \begin{pmatrix}
   1 & 0 \\
   0 & -1
   \end{pmatrix} \neq 0
   \]
   $\det(A+B) = -1$, т.е обратима и невырождена, а значит не нильпотентна по св-ву нильпотентных матриц.
   \item общий вывод: Если матрицы \( A \), \( B \) и их коммутатор \([A, B] \) нильпотентны, при этом \( A \) и \([A,B] \) коммутируют, а также \( B \) и \([A,B] \) коммутируют, то сумма \( A + B \) также нильпотентна.
\end{enumerate}

\subsection{Задача 14}
\textbf{Условие задачи:}
Пусть $A \in M_{m \times n}(\mathbb{R})$, а $B \in M_{n \times m}(\mathbb{R})$. Покажи, что:
1. $spec_R(AB) \cup \{0\} = spec_R(BA) \cup \{0\}$, то есть спектр $AB$ — то же самое, что спектр $BA$ с точностью до включения нулевого значения.
2. Если $m > n$, то $spec_R(AB) = spec_R(BA) \cup \{0\}$.
3. Если $m = n$, то $spec_R(AB) = spec_R(BA)$.
\textbf{Решение: }
Рассмотрим характеристические многочлены AB и BA:\\
\begin{enumerate}
    
\item $\det(AB - \lambda I_m) = \det(BA - \lambda I_n) \cdot (-\lambda)^{m-n}$\\
$\det(BA - \lambda I_n) = \det(AB - \lambda I_m) \cdot (-\lambda)^{n-m}$, отсюда следует, что собсвенные значения совпадают, доп значения возникают только в случае, если размерности не совпадают и они равны 0. Таким образом $\text{spec}_\mathbb{R}(AB) \cup \{0\} = \text{spec}_\mathbb{R}(BA) \cup \{0\}$
\item Если \( m > n \), то когда матрицы имеют одинаковый размер, мы знаем, что ненулевые значения совпадают, поскольку одна матрица больше по размеру, то она имеет m-n дополнительных значений, которые равны 0.
\item При равных размерностях, из первого пункта следует, что спектры совпадают, за исключением возможного нуля, при разных размерностях, здесь же, размерности совпадают из за чего утверждение верное
\end{enumerate}
\end{document}
