\documentclass[a4paper,12pt]{article}
\usepackage[utf8]{inputenc}
\usepackage[russian]{babel}
\usepackage{amsmath,amsfonts,amssymb}
\usepackage{graphicx}
\usepackage{geometry}
\usepackage{hyperref}
\usepackage{venndiagram}
\usepackage{tikz}
\usetikzlibrary{shapes.geometric, calc}
\usepackage{amsmath}
\usepackage{pgfplots}
\usepackage{float}

% Параметры страницы
\geometry{top=2cm,bottom=2cm,left=2.5cm,right=2.5cm}
\geometry{a4paper, margin=1in}

% Заголовок документа
\title{Домашнее задание}
\author{Студент: Ростислав Алексеевич Лохов}
\date{\today}

\begin{document}

% Титульный лист
\begin{titlepage}
    \centering
    \vspace*{1cm}

    \Huge
    \textbf{Домашнее задание}

    \vspace{0.5cm}
    \LARGE
    По курсу: \textbf{Математический анализ}

    \vspace{1.5cm}

    \textbf{Студент: Ростислав Алексеевич Лохов}

    \vfill

    \Large
    АНО ВО "Центральный Университет"\\
    \vspace{0.3cm}
    \today

\end{titlepage}

% Содержание
\tableofcontents
\newpage

% Основной текст
\section{Теория множеств}

\subsection{Задача 1}
\begin{itemize}
    \item[a)] Задайте множества \(A\) и \(B\) перечислением их элементов:
    \[
    A = \{x \in \mathbb{R} : x^4 - 6x^2 + 8 = 0\}
    \]
    \[
    B = \{x \geq 0 : (x^2 - 6x + 8)(x^4 - 6x^2 + 8) = 0\}
    \]
    \item[б)] Задайте множества \(A \cup B\), \(A \cap B\), \(A \setminus B\), \(B \setminus A\) перечислением их элементов.
\end{itemize}

\textbf{Решение:}

Рассмотрим уравнение для множества \(A\):
\[
x^4 - 6x^2 + 8 = 0 \quad \Rightarrow \quad (x-2)(x+2)(x-\sqrt{2})(x+\sqrt{2}) = 0
\]

Корни этого уравнения:
\[
x_1 = 2, \quad x_2 = -2, \quad x_3 = -\sqrt{2}, \quad x_4 = \sqrt{2}
\]

Таким образом, множество \(A\) имеет вид:
\[
A = \{-2, -\sqrt{2}, \sqrt{2}, 2\}
\]

Теперь рассмотрим уравнение для множества \(B\):
\[
x \geq 0 \quad : \quad (x^2 - 6x + 8)(x^4 - 6x^2 + 8) = 0
\]

Решая это уравнение, получаем:
\[
x_1 = 2, \quad x_2 = 4, \quad x_3 = \sqrt{2}
\]

Таким образом, множество \(B\) имеет вид:
\[
B = \{\sqrt{2}, 2, 4\}
\]

Ответ на пункт а:
\[
A = \{-2, -\sqrt{2}, \sqrt{2}, 2\}
\]
\[
B = \{\sqrt{2}, 2, 4\}
\]

Пункт б:
\[
A\cup B = \{-2, 2, -\sqrt{2}, \sqrt{2}, 4\}
\]

\[
A\cap B = \{2, \sqrt{2}\}
\]

\[
A\setminus B = \{-2, -\sqrt{2}\}
\]

\[
B\setminus A = \{4\}
\]
\vspace{1cm}


\subsection{Задача 2}

Упростите выражения:
\begin{itemize}
    \item[a)] \(\overline{\overline{A} \cup \overline{B} \cup \overline{C}}\);
    \item[б)] \(((A \cup B \cup C) \cap (A \cup B)) \setminus ((A \cup (B \setminus C)) \cap A)\).
\end{itemize}

\textbf{Решение:}

\textbf{а)} Используя законы де Моргана, получаем:
\[
\overline{\overline{A} \cup \overline{B} \cup \overline{C}} = \overline{\overline{A \cap B \cap C}} = A \cap B \cap C
\]

\textbf{б)} Будем постепенно упрощать:
\[
((A \cup B \cup C) \cap (A \cup B)) \setminus ((A \cup (B \setminus C)) \cap A) = (A \cup B) \setminus A = B \setminus A
\]

\vspace{1cm}



\subsection{Задача 3}

В студенческой группе 25 человек. Во время летних каникул 11 из них выезжали в турпоездки за границу, 12 — путешествовали по Камчатке, 15 — отдыхали в Сочи, 6 — путешествовали за границей и по Камчатке, 7 — были и за границей, и в Сочи, 8 — путешествовали по Камчатке и были в Сочи, а трое участвовали во всех трёх поездках. Сколько студентов никуда не выезжало?

\textbf{Решение:}

Обозначим множества студентов, путешествовавших:
\begin{itemize}
    \item $A$ — за границу,
    \item $B$ — по Камчатке,
    \item $C$ — в Сочи.
\end{itemize}

Известно:
\[
|E| = 25, \quad |A| = 11, \quad |B| = 12, \quad |C| = 15,
\]
\[
|A\cap B| = 6, \quad |A \cap C| = 7, \quad |B \cap C| = 8, \quad |A \cap B \cap C| = 3.
\]

Применяя формулу включений-исключений, находим количество студентов, которые путешествовали хотя бы в одно место:

\[
|A \cup B \cup C| = |A| + |B| + |C| - |A \cap B| - |B \cap C| - |A \cap C| + |A \cap B \cap C|,
\]

\[
|A \cup B \cup C| = 11 + 12 + 15 - 6 - 8 - 7 + 3 = 20.
\]

Тогда количество студентов, которые никуда не выезжали, равно:

\[
X = |E| - |A \cup B \cup C| = 25 - 20 = 5.
\]

\textbf{Ответ:} 5 студентов никуда не выезжало.

\subsection{Задача 4}
Докажите, что \(A \triangle B = (A \cup B) \setminus (A \cap B)\):
\begin{itemize}
    \item[a)] с помощью диаграмм Эйлера — Венна;
    \item[б)] не используя диаграммы Эйлера — Венна.
\end{itemize}

\textbf{Решение:}

Рассмотрим определение симметрической разности $A \Delta B$ и множество $(A \cup B) \setminus (A \cap B)$:

1. Симметрическая разность $A \Delta B$ определяется как:
\[
A \Delta B = (A \setminus B) \cup (B \setminus A)
\]
Это множество включает все элементы, которые принадлежат только одному из множеств $A$ или $B$, но не обоим одновременно.

2. $(A \cup B) \setminus (A \cap B)$ представляет собой все элементы, которые принадлежат объединению $A$ и $B$, но не принадлежат их пересечению.

Теперь построим диаграммы Эйлера — Венна для обоих выражений и покажем их эквивалентность.

\begin{center}
\begin{tikzpicture}
    % Draw set A (white)
    \begin{scope}
        \clip (0,0) circle (1.5cm);
        \fill[yellow] (0,0) circle (1.5cm) node[left=2cm] {\Large $A$};
    \end{scope}
    % Draw set B (black)
    \begin{scope}
        \clip (1.5,0) circle (1.5cm);
        \fill[black] (1.5,0) circle (1.5cm) node[right=2cm] {\Large $B$};
    \end{scope}
    % Intersection (yellow)
    \begin{scope}
        \clip (0,0) circle (1.5cm);
        \fill[white] (1.5,0) circle (1.5cm);
    \end{scope}
    \begin{scope}
        \clip (1.5,0) circle (1.5cm);
        \fill[white] (0,0) circle (1.5cm);
    \end{scope}
    % Outline circles
    \draw (0,0) circle (1.5cm);
    \draw (1.5,0) circle (1.5cm);
    % Label below
    \node at (0.75, -2) {$A \Delta B$};
\end{tikzpicture}

\vspace{1cm}

\begin{tikzpicture}
    % Draw set A (white)
    \begin{scope}
        \clip (0,0) circle (1.5cm);
        \fill[yellow] (0,0) circle (1.5cm) node[left=2cm] {\Large $A$};
    \end{scope}
    % Draw set B (black)
    \begin{scope}
        \clip (1.5,0) circle (1.5cm);
        \fill[black] (1.5,0) circle (1.5cm) node[right=2cm] {\Large $B$};
    \end{scope}
    % Intersection (yellow)
    \begin{scope}
        \clip (0,0) circle (1.5cm);
        \fill[white] (1.5,0) circle (1.5cm);
    \end{scope}
    \begin{scope}
        \clip (1.5,0) circle (1.5cm);
        \fill[white] (0,0) circle (1.5cm);
    \end{scope}
    % Outline circles
    \draw (0,0) circle (1.5cm);
    \draw (1.5,0) circle (1.5cm);
    % Label below
    \node at (0.75, -2) {$(A \cup B) \setminus (A \cap B)$};
\end{tikzpicture}
\end{center}

\textbf{Вывод:} Обе диаграммы показывают, что симметрическая разность $A \Delta B$ и множество $(A \cup B) \setminus (A \cap B)$ представляют одну и ту же область на диаграмме, что доказывает их эквивалентность.

Теперь докажем, что $A\Delta B = (A \cup B) \setminus (A \cap B)$:
\[
A\Delta B = (A \setminus B) \cup (B \setminus A)
\]

\[
\forall x \in X: x \in (A \setminus B) \lor x \in (B \setminus A), x \notin A \cap B \leftrightarrow x \in (A\cup B) \ (A\cap B)
\]

\vspace{1cm}

\subsection{Задача 5}

Трое ребят принялись красить квадратный лист бумаги, каждый — в свой цвет. Один закрасил красным 75\% листа, второй закрасил синим 70\% листа, а третий закрасил зелёным 65\% листа. Сколько процентов листа будет заведомо закрашено всеми тремя цветами?

\textbf{Решение:}

Обозначим площадь всего листа за \( |E| = 100\% \). Пусть \( |A| = 75\% \) — площадь, закрашенная красным, \( |B| = 70\% \) — площадь, закрашенная синим, и \( |C| = 65\% \) — площадь, закрашенная зелёным.

Для того чтобы найти площадь, которая закрашена всеми тремя цветами, воспользуемся принципом включения-исключения:

\[
|A \cap B \cap C| = |E| - \left( |E \setminus A| + |E \setminus B| + |E \setminus C| \right)
\]

Сначала находим площади, которые не были закрашены каждым из цветов:

\[
|E \setminus A| = 100\% - 75\% = 25\%
\]
\[
|E \setminus B| = 100\% - 70\% = 30\%
\]
\[
|E \setminus C| = 100\% - 65\% = 35\%
\]

Теперь подставим эти значения в формулу:

\[
|A \cap B \cap C| = 100\% - (25\% + 30\% + 35\%) = 100\% - 90\% = 10\%
\]

Таким образом, \( 10\% \) листа будет заведомо закрашено всеми тремя цветами.


\subsection{Задача 6}
Пусть даны множества \(A\), \(B\), \(C\). Выразите следующие множества через \(A\), \(B\), \(C\) при помощи операций \(\cup\), \(\cap\), \(\setminus\) и \(\triangle\):
\begin{itemize}
    \item[a)] Множество элементов, принадлежащих всем трём множествам;
    \item[б)] Множество элементов, принадлежащих хотя бы двум из множеств \(A\), \(B\), \(C\);
    \item[в)] Множество элементов, принадлежащих ровно двум из множеств \(A\), \(B\), \(C\);
    \item[г)] Множество элементов, принадлежащих хотя бы одному из множеств \(A\), \(B\), \(C\);
    \item[д)] Множество элементов, принадлежащих ровно одному из множеств \(A\), \(B\), \(C\);
    \item[е)] Множество элементов, принадлежащих \(A\), \(B\), но не принадлежащих \(C\);
    \item[ж)] Множество элементов, принадлежащих хотя бы одному из множеств \(A\), \(B\), но не принадлежащих \(C\);
    \item[з)] Множество элементов, принадлежащих ровно одному из множеств \(A\), \(B\), но не принадлежащих \(C\).
\end{itemize}

\textbf{Решение:}
\begin{itemize}
    \item[a)] 
    \[
    A\cap B\cap C
    \]
    
    \item[б)] 
    \[
    (A \cap B) \cup (B \cap C) \cup (C \cap A)
    \]
    
    \item[в)] 
    \[
    ((A \cap B) \cup (B \cap C) \cup (C \cap A)) \setminus (A\cap B\cap C)
    \]
    
    \item[г)] 
    \[
    A\cup B\cup C
    \]
    
    \item[д)] 
    \[
    A \Delta B \Delta C \Delta(A\cap B\cap C)
    \]
    
    \item[е)] 
    \[
    (A\cap B) \setminus C
    \]
    
    \item[ж)] 
    \[
    (A\cup B) \setminus C
    \]
    
    \item[з)] 
    \[
    (A\Delta B) \setminus C
    \]
\end{itemize}

\vspace{1cm}


\section{Элементы логики}

\subsection{Задача 7}
Докажите с помощью таблиц истинности следующие тождества:

\begin{itemize}
    \item[a)] \(A + AB = A\);
    \item[б)] \(A + BC = (A + B)(A + C)\);
    \item[в)] \(\overline{A + B} = \overline{A} \cdot \overline{B}\);
    \item[г)] \(\overline{AB} = \overline{A} + \overline{B}\).
\end{itemize}

\textbf{Решение:}

\begin{itemize}
    \item[a)] Для доказательства \(A + AB = A\) составим таблицу истинности:

\begin{table}[h!]
\centering
\begin{tabular}{|c|c|c|c|c|}
\hline
\(A\) & \(B\) & \(AB\) & \(A + AB\) & \(A\) \\ \hline
0 & 0 & 0 & 0 & 0 \\ \hline
0 & 1 & 0 & 0 & 0 \\ \hline
1 & 0 & 0 & 1 & 1 \\ \hline
1 & 1 & 1 & 1 & 1 \\ \hline
\end{tabular}
\caption{Таблица истинности для \(A + AB = A\)}
\end{table}

Из таблицы видно, что \(A + AB\) всегда равно \(A\), поэтому равенство доказано.

\item[б)] Для доказательства \(A + BC = (A + B)(A + C)\) составим таблицу истинности:

\begin{table}[h!]
\centering
\begin{tabular}{|c|c|c|c|c|c|}
\hline
\(A\) & \(B\) & \(C\) & \(BC\) & \(A + BC\) & \((A + B)(A + C)\) \\ \hline
0 & 0 & 0 & 0 & 0 & 0 \\ \hline
0 & 0 & 1 & 0 & 0 & 0 \\ \hline
0 & 1 & 0 & 0 & 0 & 0 \\ \hline
0 & 1 & 1 & 1 & 1 & 1 \\ \hline
1 & 0 & 0 & 0 & 1 & 1 \\ \hline
1 & 0 & 1 & 0 & 1 & 1 \\ \hline
1 & 1 & 0 & 0 & 1 & 1 \\ \hline
1 & 1 & 1 & 1 & 1 & 1 \\ \hline
\end{tabular}
\caption{Таблица истинности для \(A + BC = (A + B)(A + C)\)}
\end{table}

Из таблицы видно, что \(A + BC\) всегда равно \((A + B)(A + C)\), поэтому равенство доказано.

\item[в)] Для доказательства \(\overline{A + B} = \overline{A} \cdot \overline{B}\) составим таблицу истинности:

\begin{table}[h!]
\centering
\begin{tabular}{|c|c|c|c|c|c|}
\hline
\(A\) & \(B\) & \(A + B\) & \(\overline{A + B}\) & \(\overline{A}\) & \(\overline{A} \cdot \overline{B}\) \\ \hline
0 & 0 & 0 & 1 & 1 & 1 \\ \hline
0 & 1 & 1 & 0 & 1 & 0 \\ \hline
1 & 0 & 1 & 0 & 0 & 0 \\ \hline
1 & 1 & 1 & 0 & 0 & 0 \\ \hline
\end{tabular}
\caption{Таблица истинности для \(\overline{A + B} = \overline{A} \cdot \overline{B}\)}
\end{table}

Из таблицы видно, что \(\overline{A + B}\) всегда равно \(\overline{A} \cdot \overline{B}\), поэтому равенство доказано.

\item[г)] Для доказательства \(\overline{AB} = \overline{A} + \overline{B}\) составим таблицу истинности:

\begin{table}[h!]
\centering
\begin{tabular}{|c|c|c|c|c|c|}
\hline
\(A\) & \(B\) & \(AB\) & \(\overline{AB}\) & \(\overline{A}\) & \(\overline{A} + \overline{B}\) \\ \hline
0 & 0 & 0 & 1 & 1 & 1 \\ \hline
0 & 1 & 0 & 1 & 1 & 1 \\ \hline
1 & 0 & 0 & 1 & 0 & 1 \\ \hline
1 & 1 & 1 & 0 & 0 & 0 \\ \hline
\end{tabular}
\caption{Таблица истинности для \(\overline{AB} = \overline{A} + \overline{B}\)}
\end{table}

Из таблицы видно, что \(\overline{AB}\) всегда равно \(\overline{A} + \overline{B}\), поэтому равенство доказано.
\end{itemize}

\vspace{1cm}



\subsection{Задача 8}
Упростите высказывания и определите их истинность:
\begin{itemize}
    \item[a)] \(\overline{\overline{AB} + BC}\), если \(B\) и \(C\) истинны;
    \item[б)] \(\overline{(\overline{A \Rightarrow C})} \cdot (B + (\overline{C} \Rightarrow A))\), если \(C\) истинно;
    \item[в)] \(\overline{(AB + \overline{AB})} \cdot (A + \overline{B})\), если \(A\) и \(B\) ложны;
    \item[г)] \(\overline{(A + B) \Rightarrow \overline{(B + C)}}\), если \(A\) и \(C\) истинны, а \(B\) ложно.
\end{itemize}

\textbf{Решение:}
\item[а)]
\[
\overline{\overline{AB} + BC} = 0
\]
\item[б)]
\[
(A \Rightarrow C) \cdot (B + (C \Rightarrow A)) = 1
\]
\item[в)]
\[
\overline{(AB + \overline{AB})} \cdot (A + \overline{B}) = 0
\]
\item[г)]
\[
\overline{(A + B) \Rightarrow \overline{(B + C)}} = 1
\]
\vspace{1cm}

\subsection{Задача 9}
Вернувшись домой, Мегрэ позвонил на набережную Орфевр.
\begin{quote}
    — Говорит Мегрэ. Есть новости?

    — Да, шеф. Поступили сообщения от инспекторов. Торранс установил, что если Франсуа был пьян, то либо Этьен убийца, либо Франсуа лжёт. Жуссье считает, что или Этьен убийца, или Франсуа не был пьян и убийство произошло после полуночи. Инспектор Люка попросил передать Вам, что если убийство произошло после полуночи, то либо Этьен убийца, либо Франсуа лжёт.

    — Всё, спасибо. Этого достаточно.

    — Комиссар положил трубку. Он знал, что трезвый Франсуа никогда не лжёт. Теперь он знал всё.
\end{quote}

Рассмотрим следующие высказывания:

\(A = \text{Франсуа был пьян}\);

\(B = \text{Этьен убийца}\);

\(C = \text{Франсуа лжёт}\);

\(D = \text{убийство произошло после полуночи}\).

Используя логические операции, запишите высказывания инспекторов. Составьте произведение этих высказываний и упростите его. Какой вывод сделал комиссар Мегрэ?

\textbf{Решение:}\\
Полное выражение с учетом всех высказываний:
\[
((A \Rightarrow (B \lor C)) \land (B \lor (\overline{A} \land D)) \land (D \Rightarrow (B \lor C))) \land (\overline{A} \land \overline{C})
\]
\[
((\overline{A} \lor B \lor C) \land (B \lor (\overline{A} \land D)) \land (\overline{D} \lor B \lor C)) \land (\overline{A} \land \overline{C})
\]
\[
(\overline{A} \land \overline{C}) \land (B \lor D) \land (\overline{D} \lor B)
\]
\[
\overline{A} \land \overline{C} \land B
\]
Исходя из этого выражения, можно сделать вывод о том, что Этьен был трезв, честен и хладнокровно убил свою жертву:(
\vspace{1cm}

\subsection{Задача 10}
Даны два высказывания:

\(A(x; y) : y > x\) и \(B(x; y) : x^2 + y^2 \geq 4\), определённые при \(x, y \in \mathbb{R}\). Изобразите на координатной плоскости множества истинности следующих высказываний:
\begin{itemize}
    \item[a)] \(A(x; y) + B(x; y)\);
    \item[б)] \(A(x; y) \cdot B(x; y)\);
    \item[в)] \(A(x; y) \Rightarrow B(x; y)\);
    \item[г)] \(B(x; y) \Rightarrow A(x; y)\);
    \item[д)] \(A(x; y) \cdot \overline{B(x; y)}\);
    \item[е)] \(\overline{B(x; y)} \Rightarrow \overline{A(x; y)}\).
\end{itemize}

\textbf{Решение:}
\begin{figure}[H] 
    \centering
    \includegraphics[width=\textwidth]{output.png} 
    \caption{Черным - 0, желтым - 1}
    \label{fig:logic_graphs}
\end{figure}
\vspace{1cm}

\subsection{Задача 11}
Для каждой из следующих теорем сформулируйте обратную, противоположную и противоположную обратную теоремы. Укажите, какие из этих теорем верны:
\begin{itemize}
    \item[a)] Вертикальные углы равны.
    \item[б)] Если сумма углов \(A\) и \(C\) четырёхугольника \(ABCD\) равна \(180^\circ\), то вокруг него можно описать окружность.
\end{itemize}

\textbf{Решение:}
\begin{itemize}
    \item[a)] Прямая: Если углы вертикальные, то они равны.\\
    Обратная: Если углы равны, то они вертикальны. (Не верно, два угла в равнобедренном треугольнике могут быть равны, но не вертикальны.)\\
    Противоположная: Если два угла не равны, то они не вертикальны. (Верно.)\\
    Противоположная обратная: Если углы не вертикальны, то они не равны. (Не верно, два угла могут быть не вертикальными, но при этом равными.)
    \vspace{1cm}

    \item[б)] Прямая: Если сумма углов \(A\) и \(C\) четырёхугольника \(ABCD\) равна \(180^\circ\), то вокруг него можно описать окружность.\\
    Обратная: Если вокруг четырёхугольника \(ABCD\) можно описать окружность, то сумма углов \(A\) и \(C\) равна \(180^\circ\). (Верно.)\\
    Противоположная: Если сумма углов \(A\) и \(C\) четырёхугольника \(ABCD\) не равна \(180^\circ\), то вокруг него нельзя описать окружность. (Верно.)\\
    Противоположная обратная: Если вокруг четырёхугольника \(ABCD\) нельзя описать окружность, то сумма углов \(A\) и \(C\) не равна \(180^\circ\). (Не всегда верно, углы могут случайно суммироваться до \(180^\circ\), но четырёхугольник при этом не будет вписанным.)
\end{itemize}
\vspace{1cm}

\subsection{Задача 12}
Заполните многоточия словами «необходимо», «достаточно», «необходимо и достаточно» так, чтобы получить верные утверждения:
\begin{itemize}
    \item[a)] «Для того чтобы сумма двух натуральных чисел делилась на 2, \ldots, чтобы числа были чётными»;
    \item[б)] «Для того чтобы число делилось на 9, \ldots, чтобы сумма его цифр делилась на 9»;
    \item[в)] «Для того чтобы числа \(x_1\) и \(x_2\) были корнями уравнения \(x^2 + px + q\), \ldots, чтобы \(x_1 x_2 = q\)».
\end{itemize}

\textbf{Решение:}
\begin{itemize}
    \item[a)] «Для того чтобы сумма двух натуральных чисел делилась на 2, \textbf{необходимо и достаточно}, чтобы числа были чётными».
    \vspace{1cm}
    
    \item[б)] «Для того чтобы число делилось на 9, \textbf{необходимо и достаточно}, чтобы сумма его цифр делилась на 9».
    \vspace{1cm}
    
    \item[в)] «Для того чтобы числа \(x_1\) и \(x_2\) были корнями уравнения \(x^2 + px + q\), \textbf{необходимо}, чтобы \(x_1 x_2 = q\)».
\end{itemize}
\vspace{1cm}

\subsection{Задача 13}
Запишите утверждения при помощи кванторов:
\begin{itemize}
    \item[a)] Для любых точек \(A\), \(B\), \(C\) пространства существует плоскость, проходящая через них.
    \item[б)] Для любых точек \(A\), \(B\), \(C\), пространства, не лежащих на одной прямой, существует ровно одна плоскость, проходящая через них.
    \item[в)] Не всякий многочлен имеет хотя бы один действительный корень.
\end{itemize}

\textbf{Решение:}
\begin{itemize}
    \item[a)] \(\forall A \forall B \forall C \, \exists \, P \, (A, B, C \in P)\text{, где P - плоскость, }(A,B,C)\in \mathbb{R}^3 \).
    \vspace{1cm}
    
    \item[б)] \(\forall A \forall B \forall C \, (\neg \exists \, \text{прямая} \, l \, (A, B, C \in l)) \rightarrow \exists! \, \text{плоскость} \, P \, (A, B, C \in P)\).
    \vspace{1cm}
    
    \item[в)] \(\neg \forall P(x) \, \exists x \in \mathbb{R} \, (P(x) = 0)\).
\end{itemize}
\vspace{1cm}

\subsection{Задача 14}
Какое известное вам утверждение означает эта запись:
\[
a \neq 0 \wedge b^2 - 4ac > 0 \Rightarrow \exists x_1, x_2 \in \mathbb{R} : x_1 \neq x_2 \wedge ax_1^2 + bx_1 + c = 0 \wedge ax_2^2 + bx_2 + c = 0?
\]

\textbf{Решение:}
Коэффициент a отвечает за то, что уравнение квадратное, а не линейное, $b^2-ac>0$ означает, что существует два различных действительных корня.\\
Таким образом запись означает, что существуют два различных действительных корня, которые удовлетворяют данному квадратному уравнению.
\vspace{1cm}

\section{Метод математической индукции}

\subsection{Задача 15}
Методом математической индукции докажите, что для любого \(n \in \mathbb{N}\):
\begin{itemize}
    \item[a)] \(1 \cdot 1! + 2 \cdot 2! + \ldots + n \cdot n! = (n+1)! - 1\).
    \item[б)] \(\frac{1}{2} \cdot 2! + \frac{2}{2^2} \cdot 3! + \frac{3}{2^3} \cdot 4! + \ldots + \frac{n}{2^n} \cdot (n+1)! = \frac{(n+2)!}{2^n} - 2\).
\end{itemize}

\textbf{Решение:}
\item[а)] 
Необходимо доказать, что 
\[
\forall n \in \mathbb{N}, \sum_{n=1}^{k}n \cdot n!=(k+1)!-1 
\]
Пусть n=1, тогда
\[
1\cdot 1! = (1+1)!-1 \leftrightarrow 1=1 \text{- Верно}
\]
Предположим, что для некоторого n=k выражение верно, т.е:
\[
\sum_{n=1}^{k}n\cdot n!=(k+1)!-1
\]
Теперь необходимо доказать, что утверждение верно и для $n=k+1$:
\[
\sum_{n=1}^{n=k}n\cdot n! + (k+1)\cdot (k+1)! = (k+1)!-1+(k+1)(k+1)! = (k+1)!\cdot (1+(k+1)) -1
\]
\[
(k+1)!(k+2) - 1 = (k+2)!-1 \text{, что и требовалось доказать}
\]
\item[б)]
Необходимо доказать, что 
\[
\frac{1}{2} \cdot 2! + \frac{2}{2^2} \cdot 3! + \frac{3}{2^3} \cdot 4! + \ldots + \frac{n}{2^n} \cdot (n+1)! = \frac{(n+2)!}{2^n} - 2
\]
Пусть $n=1$, тогда
\[
\frac{1}{2^1}\cdot (1+1)!=1 \leftrightarrow \frac{(1+2)!}{2^1}-2=1 \text{, Верно}
\]
Предположим, что для некоторого $n=k$ выражение верно, т.е:
\[
\sum_{n=1}^{n=k}\frac{n}{2^n}\cdot (n+1)! = \frac{(k+2)!}{2^k}-2
\]
Теперь необходимо доказать, что утверждение верно и для $n=k+1$:
\[
\sum_{n=1}^{n=k}\frac{n}{2^n}\cdot (n+1)! + \frac{k+1}{2^{k+1}}\cdot (k+1+1)!=\frac{(k+2)!}{2^k}-2+\frac{k+1}{2^{k+1}}\cdot (k+2)!
\]
\[
(k+2)!\cdot (\frac{1}{2^k}+\frac{k+1}{2^{k+1}}) - 2 = (k+2)!\cdot \frac{k+3}{2^k}-2 = \frac{(k+3)!}{2^k}-2 \text{, что и требовалось доказать}
\]
\vspace{1cm}

\subsection{Задача 16}
Методом математической индукции докажите, что для любого целого неотрицательного значения \(n\) справедливы утверждения:
\begin{itemize}
    \item[а)] \(4^n - 1 \mid 3\);
    \item[б)] \(n^5 + 9n \mid 5\);
    \item[в)] \(7 \cdot 5^{2n} + 12 \cdot 6^n \mid 19\).
\end{itemize}

\textbf{Замечание.} Обратите внимание, что, так как в условии \(n \geq 0\), база индукции — это проверка утверждения при \(n = 0\).

\textbf{Решение:}
\item[a] 
Необходимо доказать:
\[
4^n - 1 \mid 3\
\]
Пусть $n = 0$:
\[
1-1=0\mid3 \text{, Истина}
\]
Предположим, что для некоторого $n=k$ выражение верно, т.е:
\[
4^k - 1 \mid 3 = 3m
\]
Теперь необходимо доказать, что утверждение верно и для $n=k+1$:
\[
4^{k+1} - 1 \mid 3 = 4*4^k-1 = 4(4^k-1) + 3 = 12m+3= 3\cdot(4m+1), \text{, делится на 3, т.к выражение содержит множитель 3}
\]
\item[б]
Необходимо доказать:
\[
n^5 + 9n \mid 5
\]
Пусть $n=0$:
\[
0\mid 5 \text{, Истина}
\]
Предположим, что для некоторого $n=k$ выражение верно, т.е:
\[
k^5 + 9k \mid 5 = 5m
\]
Теперь необходимо доказать, что утверждение верно и для $n=k+1$:
\[
(k+1)^5+9\cdot (k+1) \mid 5 = k^5+5\cdot k^4+10\cdot = k^5 + 5k^4 + 10k^3 + 10k^2 + 14k + 10
\]
\[
5m + 5k^4+10k^3+10k^2+14k+10 = 5m + 5(k^4+2k^3+2k^2+k+2)\mid 5 \text{, чтд}
\]
\item[в]
Необходимо доказать:
\[
7 \cdot 5^{2n} + 12 \cdot 6^n \mid 19
\]
Пусть $n=0$:
\[
19\mid 19
\]
Предположим, что для некоторого $n=k$ выражение верно, т.е:
\[
7 \cdot 5^{2k} + 12 \cdot 6^k \mid 19 = 19m
\]
Теперь необходимо доказать, что утверждение верно и для $n=k+1$:
\[
7 \cdot 5^{2k + 2} + 12 \cdot 6^{k+1} \mid 19
\]
\[
7\cdot 25\cdot 5^{2k} + 12 \cdot 6\cdot 6^k \equiv 7\cdot 6\cdot 5^{2k}+12\cdot 6\cdot 6^k \pmod{19}
\]
Вынесем $6$
\[
6\cdot 19m \mid 19
\]
\vspace{1cm}

\subsection{Задача 17}
Найдите ошибку в рассуждении.
\begin{quote}
Докажем, что все машины одного цвета. Пусть \(n\) — количество машин.

База индукции (\(n = 1\)). Одна машина одного цвета — верно.

Предположение индукции. Пусть \(n\) произвольно взятых машин имеют один и тот же цвет.

Докажем, что и произвольные \(n + 1\) машин имеют один и тот же цвет. Для этого выделим среди них две группы по \(n\) машин: в первую включим машины 1, 2, 3, \ldots, \(n\), а во вторую — машины 2, 3, \ldots, \(n\), \(n + 1\). По предположению индукции все машины в каждой из групп имеют один и тот же цвет. Так как машина под номером 2 находится в обеих группах, то все машины имеют тот же самый цвет, что и она. Итак, получено, что \(n+1\) машин имеют один и тот же цвет. Переход индукции доказан.
\end{quote}

\textbf{Решение:}
Логическая ошибка заключается в неправильном переходе значений от n=k до n=k+1, при $n\ge 2$ возникает разрыв связи. Допустим в нашем умозаключении была связь между 1 и 2, 2 и 3, но это не означает, что 1 и 3 одинаковых цветов
\vspace{1cm}

\subsection{Задача 18}
Числовая последовательность \(\{F_n\}\), в которой \(F_1 = F_2 = 1\) и для любого натурального \(n\) выполняется равенство \(F_{n+2} = F_{n+1} + F_n\), называется последовательностью Фибоначчи, а её члены — числами Фибоначчи.

Докажите, что
\[
F_{n+m+1} = F_n F_m + F_{n+1} F_{m+1}
\]
для любых натуральных \(n\) и \(m\).

\textbf{Решение:}
Поверим  для $m=1$:
\[
F_{n+2}=F_1\cdot F_n + F_2\cdot F_{n+1} = F_n + F_{n+1} \text{, Истина}
\]
Предположим, что формула верна для $m=k$, докажем, что она верна и для $m=k+1$, т.е:
\[
F_{n+k+2} = F_{n+k+1} + F_{n+k} \leftrightarrow F_{n}\cdot F_{k} + f_{n+1}\cdot f_{k+1} + F_n\cdot F_{k-1}
\]
\[
F_{n}\cdot (F_k+F_{k-1}) + F_{n+1}\cdot (F_{k+1}+F_k)
\]
\[
F_{n}F_k+1 + F_{n+1}F_{k+2}\text{Что и требовалось доказать}
\]
\vspace{1cm}

\subsection{Задача 19}
В некоторой стране каждый город соединён с каждым другим дорогой с односторонним движением. Докажите, что найдётся город, из которого можно добраться в любой другой город.

\textbf{Решение:}
Для n=2, для двух городов A и B существует либо дорога из A в B либо наоборот из B в A. Предположим, что утверждение верно и для n = k, т.е существует город из которого можно добраться в любые другие города, тогда следует рассмотреть n=k+1. Т.е добавляем еще один город. Возможны два варианта:\\
1) В город ведут дороги из уже существующих городов, в таком случае он удовлетворяет условию задачи\\
2) Из города ведут дороги в существующие города.\\
В обоих случаях варианты удовлетворяют условиям. Следовательно, утверждение верно.

\vspace{1cm}

\subsection{Задача 20}
В компании из \(2n + 1\) человек для любых \(n\) человек найдётся отличный от них человек, знакомый с каждым из них. Докажите, что в этой компании есть человек, знающий всех.

\textbf{Решение:}
1) Предположим, что в компании есть \( n + 1 \) попарно знакомых людей. Рассмотрим базу индукции, когда \( n = 1 \). В компании из \( 2 \times 1 + 1 = 3 \) человек (назовем их A, B, C) для любых двух человек (например, A и B) найдётся третий человек (например, C), который знаком с каждым из них. Ясно, что если A и B знакомы, и C знаком с каждым из них, то в компании есть человек (например, C), который знаком со всеми. \\
2) Предположим, что утверждение верно для \( n = k \). То есть, если в компании из \( 2k + 1 \) человек для любых \( k \) человек найдется отличный от них человек, знакомый с каждым из них, то в этой компании есть человек, знающий всех. \\
3) Рассмотрим всех \( 2k + 3 \) человек. Взяв любую группу из \( k + 1 \) человек, мы можем всегда найти человека, который знает всех в этой группе. Так как для каждой подгруппы из \( k + 1 \) человек существует общий знакомый, и этот человек является общим знакомым для всех остальных людей, то это означает, что существует человек, знающий всех в компании.
\vspace{1cm}

\section{Суммирование}

\subsection{Задача 21}
Среди следующих сумм найдите равные:
\begin{itemize}
    \item[a)] \(S_1 = \sum_{k=-10}^{50} k^3\);
    \item[б)] \(S_2 = \sum_{k=10}^{70} k^3\);
    \item[в)] \(S_3 = \sum_{k=10}^{70} (k - 20)^3\);
    \item[г)] \(S_4 = \sum_{k=10}^{70} (k + 20)^3\);
    \item[д)] \(S_5 = \sum_{p=10}^{70} p^3\).
\end{itemize}

\textbf{Решение:}\\
б и д равны и а в равны
\vspace{1cm}

\subsection{Задача 22}
Запишите в виде одной суммы:
\begin{itemize}
    \item[a)] \(3^5 + 4^5 + 5^5 + \ldots + 12^5\);
    \item[б)] \(3^5 + 4^6 + 5^7 + \ldots + 39^41\);
    \item[в)] \(\sum_{k=4}^{n} x^k + x^2 + x^3 + x^{n+1}\);
    \item[г)] \(p^n q + p^{n-1} q^2 + p^{n-2} q^3 + \ldots + p^2 q^{n-1}\);
    \item[д)] \(1 + y + \sum_{j=2}^{n} (j^2 - j + 1) y^j + (n^2 + n + 1) y^{n+1}\);
    \item[е)] \((-5) + (-2) + 1 + 4 + 7 + \ldots + 103\).
\end{itemize}
\textbf{Решение:}
\[
\text{A) } \sum_{i=3}^{i=12}i^5
\]
\[
\text{Б) } \sum_{i=3}^{i=39}i^{i+2}
\]
\[
\text{В) } \sum_{k=2}^{k=n+1}x^k
\]
\[
\text{Г) } \sum_{k=1}^{k=n-1} p^{n-k+1} q^{k}
\]
\[
\text{Д) } \sum_{j=0}^{j=n+1}(j^2-j+1)y^j
\]
\[
\text{Е) } \sum_{k=0}^{k=36}(-5+3*k)
\]
\
\vspace{1cm}

\subsection{Задача 23}
Найдите следующие суммы:
\begin{itemize}
    \item[a)] \(\sum_{j=1}^{n} \frac{1}{(3j-2)(3j+1)}\);
    \item[б)] \(\sum_{k=1}^{n} \frac{1}{(2k-1)(2k+1)(2k+3)}\);
    \item[в)] \(\sum_{p=1}^{n} \frac{p^2}{(2p+1)(2p-1)}\).
\end{itemize}

\textbf{Решение:}\\
A)
\[
\frac{1}{(3j-2)(3j+1)}=\frac{A}{3j-2} + \frac{B}{3j+1}
\]
\[
3jA+A+3jB-2B=1
\]
\[
3j(A+B) + A-2B = 1
\]
\[
A+B = 0 \land A-2B=1
\]
\[
A=-B \land -3B=1
\]
\[
A=\frac{1}{3}\land B=-\frac{1}{3}
\]
\[
\frac{1}{3} \cdot \sum_{j=1}^{n} \frac{1}{3j-2} - \frac{1}{3j+1}
\]
\[
\frac{1}{3} \cdot ((\frac{1}{1} - \frac{1}{4})+(\frac{1}{4}-\frac{1}{7})...(\frac{1}{3n-2}-\frac{1}{3n+1}))=\frac{1}{3} \cdot (1-\frac{1}{3n+1})=\frac{1}{3} - \frac{1}{9n+3} = \frac{n}{3n+1}
\]
Б)
\[
\sum_{k=1}^{n} \frac{1}{(2k-1)(2k+1)(2k+3)} = \sum_{k=1}^{n}\frac{A}{2k-1} + \frac{B}{2k+1} + \frac{C}{2k+3}
\]
\[
4k^2\cdot (A+B+C) + 4k\cdot (2A + B) + 3A + 3B -C = 1
\]
\[
A=-0.25\land B=0.5 \land C=-0.25
\]
\[
\sum_{k=1}^{n} \left(\frac{-0.25}{2k-1} + \frac{0.5}{2k+1} + \frac{-0.25}{2k+3}\right)
\]
\[
\frac{n (2 + n)}{3 (3 + 8 n + 4 n^2)}
\]
В
\[
\sum_{p=1}^{n} \frac{p^2}{(2p+1)(2p-1)}
\]
\[
\frac{1}{4}\cdot \frac{4p^2}{4p^2 - 1} =\frac{1}{4}\cdot( 1 + \frac{1}{4p^2 - 1})
\]
\[
\sum_{p=1}^{n} \frac{p^2}{(2p+1)(2p-1)} = \frac{n}{4} + \frac{1}{4} \sum_{p=1}^{n} \frac{1}{4p^2 - 1}
\]
\[
\frac{1}{8} \sum_{p=1}^{n} \left( \frac{1}{2p-1} - \frac{1}{2p+1} \right)
\]
\[
\frac{n(1+n)}{2 + 4n}
\]
\vspace{1cm}

\subsection{Задача 24}
Найдите двойную сумму \(\sum_{j=1}^{n} \sum_{k=1}^{n} a_{jk}\), если:
\begin{itemize}
    \item[a)] \(a_{jk} = 0\), если \(j \neq k\), и \(a_{jk} = 1\), если \(j = k\);
    \item[б)] \(a_{jk} = j\);
    \item[в)] \(a_{jk} = k - j\);
    \item[г)] \(a_{jk} = |k - j|\).
\end{itemize}

\textbf{Решение:}\\
A) $n$\\
Б) $\frac{n^2(n+1)}{2}$\\
В) $0$\\
Г)
\[
\sum_{k=1}^{j-1} (j - k) = j \cdot (j-1) - \sum_{k=1}^{j-1} k = j \cdot (j-1) - \frac{(j-1) \cdot j}{2} = \frac{j \cdot (j-1)}{2}
\]

\[
\sum_{k=j}^{n} (k - j) = \sum_{k=j}^{n} k - j \cdot (n - j + 1) = \frac{n \cdot (n+1)}{2} - \frac{(j-1) \cdot j}{2} - j \cdot (n - j + 1)
\]

\[
\sum_{k=1}^{j-1} (j - k) + \sum_{k=j}^{n} (k - j) = \frac{j \cdot (j-1)}{2} + \frac{n \cdot (n+1)}{2} - \frac{(j-1) \cdot j}{2} - j \cdot (n - j + 1)
\]

\[
\sum_{k=1}^{j-1} (j - k) + \sum_{k=j}^{n} (k - j) = \frac{n^2}{2} - j^2 + j + \frac{n}{2}
\]
\vspace{1cm}


\subsection{Задача 29}
Пусть \(B_m = \sum_{j=1}^{m} b_j\), \(D_m = \sum_{j=n+1}^{n+m} b_j\). Докажите, что:
\begin{itemize}
    \item[a)] \(\sum_{k=n+1}^{n+p} a_k b_k = \sum_{k=n+1}^{n+p-1} (a_k - a_{k+1})B_k + a_{n+p} B_{n+p} - a_{n+1} B_n\);
    \item[б)] \(\sum_{k=n+1}^{n+p} a_k b_k = \sum_{k=1}^{p-1} (a_{n+k} - a_{n+k+1})D_k + a_{n+p} D_p\).
\end{itemize}

\textbf{Решение:}\\
A)\\
\[
\sum_{k=n+1}^{n+p} a_k b_k = \sum_{k=n+1}^{n+p} a_k (B_k - B_{k-1}) = \sum_{k=n+1}^{n+p} a_k B_k - \sum_{k=n+1}^{n+p} a_k B_{k-1}
\]
\[
= \sum_{k=n+1}^{n+p} a_k B_k - \sum_{k=n}^{n+p-1} a_{k+1} B_k = \sum_{k=n+1}^{n+p-1} (a_k - a_{k+1}) B_k + a_{n+p} B_{n+p} - a_{n+1} B_n
\]

Б\\
\[
\sum_{k=n+1}^{n+p} a_k b_k = \sum_{k=n+1}^{n+p} a_k (B_k - B_{k-1}) = \sum_{k=n+1}^{n+p} a_k B_k - \sum_{k=n+1}^{n+p} a_k B_{k-1}
\]
\[
= \sum_{k=1}^{p} a_{n+k} D_k - \sum_{k=1}^{p} a_{n+k} D_{k-1} = \sum_{k=1}^{p-1} (a_{n+k} - a_{n+k+1})D_k + a_{n+p} D_p
\]
\vspace{1cm}

\subsection{Задача 30}
Есть и общая формула, связывающая суммы \(k\)-х степеней подряд идущих натуральных чисел. Обозначим \(S_n(p) = \sum_{j=1}^{n} j^p\), \(p \in \mathbb{N}\). Докажите, что
\[
\sum_{p=1}^{m} C_p^{m+1} S_n(p) = (n+1)^{m+1} - n - 1. \quad (1)
\]
Найдите с помощью этой формулы \(S_n(1)\), \(S_n(2)\), \(S_n(3)\).

\textbf{Решение:}

\[
\sum_{p=1}^{m} C_p^{m+1} S_n(p) = (n+1)^{m+1} - n - 1
\]

\[
(n+1)^{m+1} = \sum_{p=0}^{m+1} C_p^{m+1} n^p
\]

\[
(j+1)^{m+1} - j^{m+1} = \sum_{p=0}^{m} C_p^{m+1} j^p
\]

\[
\sum_{j=1}^{n} \left[(j+1)^{m+1} - j^{m+1}\right] = \sum_{j=1}^{n} \sum_{p=0}^{m} C_p^{m+1} j^p
\]
\[
(n+1)^{m+1} - 1 = \sum_{p=0}^{m} C_p^{m+1} S_n(p)
\]

\[
\sum_{p=1}^{m} C_p^{m+1} S_n(p) + S_n(0) C_0^{m+1} = (n+1)^{m+1} - 1
\]

\[
\sum_{p=1}^{m} C_p^{m+1} S_n(p) = (n+1)^{m+1} - n - 1
\]
ЧТД

Для \(m=1\):
   \[
   S_n(1) = \frac{n(n+1)}{2}
   \]

Для \(m=2\):
   \[
   S_n(2) = \frac{n(n+1)(2n+1)}{6}
   \]

Для \(m=3\):
   \[
   S_n(3) = \frac{n^2(n+1)^2}{4}
   \]
\vspace{1cm}

\end{document}
