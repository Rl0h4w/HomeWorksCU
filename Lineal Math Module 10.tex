\documentclass[a4paper,12pt]{article}

% Кодировка и язык
\usepackage[utf8]{inputenc}
\usepackage[russian]{babel}

% Математические пакеты
\usepackage{amsmath,amsfonts,amssymb}

% Графика
\usepackage{graphicx}
\usepackage{tikz}
\usetikzlibrary{shapes.geometric, calc}
\usepackage{pgfplots}

% Геометрия страницы
\usepackage{geometry}
\geometry{top=2cm, bottom=2cm, left=2.5cm, right=2.5cm}

% Гиперссылки
\usepackage{hyperref}

% Плавающие объекты
\usepackage{float}

% Дополнительные пакеты
\usepackage{venndiagram}

% Настройки заголовка
\title{Домашнее задание}
\author{Студент: \textbf{Ваше Имя Фамилия}}
\date{\today}

\begin{document}

% Титульный лист
\begin{titlepage}
    \centering
    \vspace*{1cm}

    \Huge
    \textbf{Домашнее задание}

    \vspace{0.5cm}
    \LARGE
    По курсу: \textbf{Линейная алгебра}

    \vspace{1.5cm}

    \textbf{Студент: Ростислав Лохов}

    \vfill

    \Large
    АНО ВО Центральный университет\\
    \vspace{0.3cm}
    \today

\end{titlepage}

% Содержание
\tableofcontents
\newpage

% Основной текст
\section{Евклидовы пространства}

\subsection{Задача 3}
\textbf{Условие задачи:} Найди длины сторон, внутренние углы и площадь треугольника $ABC$ в пространстве $\mathbb{R}^5$ со стандартным скалярным произведением $\langle x, y \rangle = x^Ty$, где 
\[
A = 
\begin{pmatrix}
2 \\ 4 \\ 2 \\ 4 \\ 2
\end{pmatrix}, \quad 
B = 
\begin{pmatrix}
6 \\ 4 \\ 4 \\ 4 \\ 6
\end{pmatrix}, \quad 
C = 
\begin{pmatrix}
5 \\ 7 \\ 5 \\ 7 \\ 2
\end{pmatrix}
\]

\textbf{Решение:}
Найдем норму каждого вектора, полученных путем разности двух точек: $||AB|| = 6$, $||BC|| = 6$, $||CA|| = 6$. Поскольку стороны равны, то равны и углы (по свойству о равностороннем треугольнике). В таком случае площадь будет равна $9\sqrt{3}$ по формуле для равностороннего треугольника ($\frac{a^2\sqrt{3}}{4}$).

\vspace{1cm}

\subsection{Задача 4}
\textbf{Условие задачи:} Найди QR-разложение для следующей матрицы:
\[
A = \begin{pmatrix}
-1 & -1 & 8 \\
-2 & 4 & 1 \\
-2 & 1 & 4 \\
\end{pmatrix}
\]

\textbf{Решение:}
$Q$ — матрица с ортонормированными столбцами, $R$ — верхнетреугольная матрица. Воспользуемся методом Грама-Шмидта для QR-разложения:

\[
u_1 = a_1 = \begin{pmatrix}
-1 \\ -2 \\ -2
\end{pmatrix}
\]

\[
R_{11} = \|u_1\| = 3
\]

\[
q_1 = \frac{u_1}{R_{11}} = \begin{pmatrix}
-\frac{1}{3} \\
-\frac{2}{3} \\
-\frac{2}{3}
\end{pmatrix}
\]

\[
R_{12} = \langle q_1, a_2 \rangle = -3
\]

\[
u_2 = a_2 - R_{12} q_1 = \begin{pmatrix}
-2 \\ 2 \\ -1
\end{pmatrix}
\]

\[
R_{22} = \|u_2\| = 3
\]

\[
q_2 = \frac{u_2}{R_{22}} = \begin{pmatrix}
-\frac{2}{3} \\ \frac{2}{3} \\ -\frac{1}{3}
\end{pmatrix}
\]

\[
R_{13} = -6, \quad R_{23} = -6
\]

\[
u_3 = a_3 - R_{12} q_1 - R_{23} q_2 = \begin{pmatrix}
\frac{2}{3} \\
\frac{1}{3} \\
-\frac{2}{3}
\end{pmatrix}
\]

Таким образом, матрицы $Q$ и $R$ имеют вид:

\[
Q = 
\begin{pmatrix}
-\frac{1}{3} & -\frac{2}{3} & \frac{2}{3} \\
-\frac{2}{3} & \frac{2}{3} & \frac{1}{3} \\
-\frac{2}{3} & -\frac{1}{3} & -\frac{2}{3} \\
\end{pmatrix}, \quad
R = 
\begin{pmatrix}
3 & -3 & -6 \\
0 & 3 & -6 \\
0 & 0 & 3 \\
\end{pmatrix}
\]

\vspace{1cm}

\subsection{Задача 5}
\textbf{Условие задачи:}
\[
B = \begin{pmatrix}
3 & 5 & -4 \\
5 & 9 & -7 \\
-7 & -7 & 6 \\
\end{pmatrix}, \quad
F = \begin{pmatrix}
2 & 1 & 0 \\
-1 & 0 & 1 \\
0 & 1 & 1 \\
\end{pmatrix}, \quad
V = \begin{pmatrix}
3 & 3 & 1 \\
1 & 2 & 2 \\
-1 & -1 & 3 \\
\end{pmatrix}
\]

\textbf{Решение:}
Перемножив каждый вектор на матрицу, получим:
\[
\begin{pmatrix}
1 \\ 1 \\ -1
\end{pmatrix}, \quad
\begin{pmatrix}
-1 \\ -2 \\ 2
\end{pmatrix}, \quad
\begin{pmatrix}
1 \\ 2 \\ -1
\end{pmatrix}
\]
Тогда, сделав перемножение каждого вектора с исходным вектором (до транспонирования и умножения на матрицу), получим 0. В случае, если мы каждый такой вектор дополнительно умножим на такой же, то получим 1, значит данный базис является ортонормированным. Для любого вектора $v$ координаты в базисе переводятся как $\langle v, f_i \rangle$.

Теперь же для нахождения вектора в другом базисе каждый элемент $i$ находится как $\langle v, f_i \rangle$.

\[
v_1 = \begin{pmatrix}
5 \\ -7 \\ 6
\end{pmatrix}, \quad
v_2 = \begin{pmatrix}
6 \\ -9 \\ 8
\end{pmatrix}, \quad
v_3 = \begin{pmatrix}
0 \\ 1 \\ 2
\end{pmatrix}
\]

\vspace{1cm}

\subsection{Задача 7}
\textbf{Условие задачи:} В пространстве $\mathbb{R}^4$ задано стандартное скалярное произведение. Пусть $U = \{y \in \mathbb{R}^4 \mid Ay = 0\}$, где 
\[
A = \begin{pmatrix}
2 & 1 & 1 & 7 \\
1 & 2 & 5 & 5 \\
-2 & -2 & -4 & 8 \\
\end{pmatrix}
\]

Найди ортогональный базис подпространства $U$ и дополни его до ортогонального базиса всего пространства в $\mathbb{R}^4$.

\textbf{Решение:}
Приведем матрицу $A$ к ступенчатому виду:
\[
A = \begin{pmatrix}
1 & 0 & -1 & 0 \\
0 & 1 & 3 & 0 \\
0 & 0 & 0 & 1 \\
\end{pmatrix} \Longleftrightarrow 
\begin{cases}
x_1 = x_3, \\
x_2 = -3x_3, \\
x_4 = 0.
\end{cases}
\]
Тогда базис решений будет:
\[
X = \begin{pmatrix}
1 \\ -3 \\ 1 \\ 0
\end{pmatrix}
\]
Тогда ортогональные векторы к нему:
\[
\begin{pmatrix}
3 \\ 1 \\ 0 \\ 0
\end{pmatrix}, \quad
\begin{pmatrix}
-1 \\ 0 \\ 1 \\ 0
\end{pmatrix}, \quad
\begin{pmatrix}
0 \\ 0 \\ 0 \\ 1
\end{pmatrix}
\]
Дополнить можно, добавив стандартный базисный вектор. Таким образом, полный ортогональный базис $\mathbb{R}^4$:
\[
\left\{
\begin{pmatrix} 1 \\ -3 \\ 1 \\ 0 \end{pmatrix},
\begin{pmatrix} 3 \\ 1 \\ 0 \\ 0 \end{pmatrix},
\begin{pmatrix} -1 \\ 0 \\ 1 \\ 0 \end{pmatrix},
\begin{pmatrix} 0 \\ 0 \\ 0 \\ 1 \end{pmatrix}
\right\}
\]

\vspace{1cm}

\subsection{Задача 8}
\textbf{Условие задачи:} Пусть в пространстве $\mathbb{R}^4$ задано стандартное скалярное произведение и следующий набор векторов:
\[
V = \begin{pmatrix}
1 & -1 & -2 \\
2 & -2 & 2 \\
3 & 3 & -3 \\
-2 & 2 & 1 \\
\end{pmatrix}
\]
Ортогонализируй систему с помощью симметричного метода Гаусса и метода Грама-Шмидта.

\textbf{Решение:}
Воспользуемся сначала методом Грама-Шмидта:
\[
u_k = v_k - \sum_{j=1}^{k-1} \frac{\langle v_k, u_j \rangle}{\langle u_j, u_j \rangle} u_j, \quad e_k = \frac{u_k}{\|u_k\|}
\]
Получим ортогональную систему:
\[
\{u_1, u_2, u_3\} = \left( 
\begin{pmatrix}
1 \\ 2 \\ 3 \\ -2
\end{pmatrix}, \quad
\begin{pmatrix}
-1 \\ -2 \\ 3 \\ 2
\end{pmatrix}, \quad
\begin{pmatrix}
-2 \\ 2 \\ 0 \\ 1
\end{pmatrix}
\right)
\]

Метод симметричного Гаусса:
\[
u_i = v_i - \sum_{j=1}^{i-1} a_{ij} u_j
\]
Получим ортогональную систему:
\[
\{u_1, u_2, u_3\} = \left( 
\begin{pmatrix}
1 \\ 2 \\ 3 \\ -2
\end{pmatrix}, \quad
\begin{pmatrix}
-1 \\ -2 \\ 3 \\ 2
\end{pmatrix}, \quad
\begin{pmatrix}
-2 \\ 2 \\ 0 \\ 1
\end{pmatrix}
\right)
\]

\vspace{1cm}

\textbf{Задача 9:}
Найди матрицу скалярного произведения в $\mathbb{R}^3$, чтобы векторы $f_1, f_2, f_3$ стали ортонормированными:
\[
f_1 = \begin{pmatrix}
-1 \\ -1 \\ 1
\end{pmatrix}, \quad
f_2 = \begin{pmatrix}
1 \\ 2 \\ -2
\end{pmatrix}, \quad
f_3 = \begin{pmatrix}
-1 \\ -2 \\ 1
\end{pmatrix}
\]

\textbf{Решение:}
Условие ортонормированности: $F^T G F = I$, где $G$ — матрица скалярного произведения, $F$ — матрица, составленная из векторов $f_i$.
Попробуем вариант $G = (F F^T)^{-1}$. Тогда:
\[
F F^T = \begin{pmatrix}
3 & 5 & -4 \\
5 & 9 & -7 \\
-4 & -7 & 6 \\
\end{pmatrix}
\]

Вычислим обратную матрицу:
\[
G = (F F^T)^{-1} = \begin{pmatrix}
5 & -2 & 1 \\
-2 & 2 & 1 \\
1 & 1 & 2
\end{pmatrix}
\]

Таким образом, матрица скалярного произведения $G$ выглядит следующим образом:
\[
G = \begin{pmatrix}
5 & -2 & 1 \\
-2 & 2 & 1 \\
1 & 1 & 2
\end{pmatrix}
\]

\subsection{Задача 10}
\textbf{Условие задачи:} Пусть задано евклидово пространство $R_{[x] \leq 3}$ со скалярным произведением $\langle f, g \rangle = \int_{-1}^{1} f(x) g(x) dx$. Методом Грама-Шмидта проведи процесс ортогонализации базиса $\{1, x, x^2, x^3\}$.

\textbf{Решение:}
Используем метод Грама-Шмидта для ортогонализации базиса $\{1, x, x^2, x^3\}$ в пространстве $R_{[x] \leq 3}$ со скалярным произведением $\langle f, g \rangle = \int_{-1}^{1} f(x) g(x) dx$.

Ортогонализированный базис с нормировкой:
\[
e_1 = \frac{\sqrt{2}}{2} \cdot 1
\]
\[
e_2 = \sqrt{\frac{3}{2}} \cdot x
\]
\[
e_3 = \sqrt{\frac{45}{8}} \cdot \left( x^2 - \frac{1}{3} \right)
\]
\[
e_4 = \sqrt{\frac{35}{8}} \cdot \left( x^3 - 0.6x \right)
\]

\vspace{1cm}
\subsection{Задача 11: } 
\textbf{Условие задачи: }Существует ли скалярное произведение на пространстве матриц $n\cdot n(n > 1)$ относительно которго матрица из всех единиц была бы ортогональна любой верхнетреугольной матрице? 

\textbf{Решение: }

Для начала рассмотрим стандартное скалярное произведение. Тогда $\langle A, B \rangle = tr(A^TB) = tr(JU)$ где J - единичная матрица, а U - верхнетреугольная. В таком случае $\langle J, U \rangle = \sum_{i, j} U_{i, j}$ Но сумма элементов верхнетреугольной матрицы U не равна нулю, если сама матрица не равна 0. Предположим, что мы определили скалярное произведение с некоторой матрицей отображений. т.е $\langle J, U \rangle = WJU = \sum_{i, j}w_{i, j} U_{i, j} = 0$ Однако это возможно только если $w_{i, j} = 0, \ \forall i \le j$ но в таком случае скалярное произведение не будет положительно определенным, что противоречит его свойствам. Следовательно не существует такого скалярного произведения, где единичная матрица ортогональна любой верхнетреугольной. 


\vspace{1cm}
\subsection{Задача 12: }
\textbf{Условие задачи: }
Паша и Дима решали задачу. Им надо было составить матрицу квадратичной формы в $\mathbb{R^3}$ для какого нибудь скалярного произведения. Паша написал слудующую матрицу: 

\[
A = \begin{pmatrix}
-1 & -3 & 4 \\
-3 & -7 & 9 \\
4 & 9 & -12 \\
\end{pmatrix}
\] Дима заметил, что эта матрица не соответствует никакому скалярному произведению, и решил ее поправить. Чтобы не обижать друга, он ахочет изменить как можно меньше коэффициентов. Помоги  Диме понять, какое минимальное количество каэффициентов надо изменить, и проведи эти изменения. 


\textbf{Решение: }
Матрица должна быть положительно определенной исходя из условия скалярного произведения. т.е Симметрична и все главные миноры положительны. 

1. Она симметрична.

2. Рассмотрим главные миноры. первый минор - -1, второй - -16, третий - 1. Тогда пусть первый минор будет x. Тогда получим систему. 

\begin{cases}
    x > 0 \\
    -7x-9>0 \\
    3x+4 > 0 \\
\end{cases}

\begin{cases}
    x > 0\\
    x < -\frac{9}{7}\\
    x > -\frac{4}{3}\\
\end{cases}

Видно, что система несовместна, т.е не имеет решений. Тогда пусть второй элемент на диагонали будет y, тогда
\begin{cases}
    x > 0\\
    xy-9 > 0\\
    -(3x+4)(4y+27) > 0
\end{cases}

Все еще нет решений, если проблема состоит в том, что второй минор отрицательный, т.е надо параметризировать второй минор полностью, тогда

\begin{cases}
    x > 0 \\
    -16 y - 3 x (27 + 4 y) + 12 z (6 + z)> 0\\
    7-z^2>0\\
\end{cases}
решением будет x, y, z = 1, 0, 1 т.е нужно изменить всего 4 цифры, что и будет минимальным решением. т.к одна цифра берется для устранения первого минора, вторая и еще две цифры берутся для устранения второго минора и чтобы матрица была симметричной. Пример такой матрицы: \[
\begin{pmatrix}
    1 & 1 & 4 \\ 
    1 & 0 & 9 \\
    4 & 9 & -12 \\
\end{pmatrix}
\]

\vspace{1cm}

\subsection{Задача 13: }
\textbf{ Условие задачи: }
Докажи, что в любых двух подпространствах евклидова пространства можно выбрать ортонормиро- ванные базисы$ e_{1} ,...,e_k cup f 1 ,...,f_m$ таким образом, что $\langle e_{i}, f_{j}\rangle = 0$ при $i \ne j \land \langle e_{i}, f_{i}\rangle \ge 0 $

\textbf{Решение: }
Пусть $ U $ и $ V $ — подпространства евклидова пространства $ E $ размерностей $ k $ и $ m $ соответственно. Нам нужно построить ортонормированные базисы $ \{e_1, \ldots, e_k\} $ для $ U $ и $ \{f_1, \ldots, f_m\} $ для $ V $, такие что:

1. $ \langle e_i, f_j \rangle = 0 $ при $ i \ne j $,
2. $ \langle e_i, f_i \rangle \geq 0 $ для всех $ i $.

Шаг 1 (i = 1):
1. Выберем произвольный единичный вектор $ e_1 \in U $.
2. Рассмотрим проекцию $ v = \text{proj}_V e_1 $ вектора $ e_1 $ на $ V $.

Если $ v = 0 $, то $ e_1 $ ортогонален $ V $, и мы выбираем произвольный единичный вектор $ f_1 \in V $. Тогда $ \langle e_1, f_1 \rangle = 0 $.

Если $ v \ne 0 $, то проецируем $ f_1 = \dfrac{v}{\|v\|} $. Если $ \langle e_1, f_1 \rangle < 0 $, заменим $ f_1 $ на $ -f_1 $, чтобы обеспечить $ \langle e_1, f_1 \rangle \geq 0 $.

Шаг 2 (i > 1):

Предположим, что мы уже построили векторы $ e_1, \ldots, e_{i-1} \in U $ и $ f_1, \ldots, f_{i-1} \in V $, такие что:

Векторы $ e_j $ ортонормированы в $ U $.
Векторы $ f_j $ ортонормированы в $ V $.
$ \langle e_j, f_k \rangle = 0 $ при $ j \ne k $.
$ \langle e_j, f_j \rangle \geq 0 $.

Теперь строим $ e_i $ и $ f_i $:

1. Рассмотрим ортогональное дополнение $ U_i $ к подпространству, натянутому на $ e_1, \ldots, e_{i-1} $, то есть $ U_i = U \cap \left( \text{Span}\{e_1, \ldots, e_{i-1}\} \right)^\perp $.

2. Выберем произвольный единичный вектор $ e_i \in U_i $.

3. Рассмотрите проекцию $ v = \text{proj}_{V_i} e_i $, где $ V_i = V \cap \left( \text{Span}\{f_1, \ldots, f_{i-1}\} \right)^\perp $.

Если $ v = 0 $, то $ e_i $ ортогонален $ V_i $, и мы выбираем произвольный единичный вектор $ f_i \in V_i $. Тогда $ \langle e_i, f_i \rangle = 0 $.

Если $ v \ne 0 $, то положим $ f_i = \dfrac{v}{\|v\|} $. Если $ \langle e_i, f_i \rangle < 0 $, заменим $ f_i $ на $ -f_i $, чтобы обеспечить $ \langle e_i, f_i \rangle \geq 0 $.


чтд
\end{document}
