\documentclass[a4paper,12pt]{article}

% Кодировка и язык
\usepackage[utf8]{inputenc}
\usepackage[russian]{babel}

% Математические пакеты
\usepackage{amsmath,amsfonts,amssymb}

% Графика
\usepackage{graphicx}
\usepackage{tikz}
\usetikzlibrary{shapes.geometric, calc}
\usepackage{pgfplots}

% Геометрия страницы
\usepackage{geometry}
\geometry{top=2cm, bottom=2cm, left=2.5cm, right=2.5cm}

% Гиперссылки
\usepackage{hyperref}

% Плавающие объекты
\usepackage{float}

% Дополнительные пакеты
\usepackage{venndiagram}

% Настройки заголовка
\title{Домашнее задание}
\author{Студент: \textbf{Ваше Имя Фамилия}}
\date{\today}

\begin{document}

% Титульный лист
\begin{titlepage}
    \centering
    \vspace*{1cm}

    \Huge
    \textbf{Домашнее задание}

    \vspace{0.5cm}
    \LARGE
    По курсу: \textbf{Математический Анализ}

    \vspace{1.5cm}

    \textbf{Студент: Ростислав Лохов}

    \vfill

    \Large
    АНО ВО Центральный Университет\
    \vspace{0.3cm}
    \today

\end{titlepage}

% Содержание
\tableofcontents
\newpage

% Основной текст
\section{Производная второго порядка}

\subsection{Задача 2}
\textbf{Решение: }
\[
\frac{ds_1(t)}{dt} = 3t^2+t+1 \land \frac{ds_2(t)}{dt} = 2t^2+6t-5
\]

\[
3t^2+t+1 = 2t^2+6t-5 \Rightarrow t = 2 \land t = 3
\]

\[
\frac{d^2s_1(t)}{dt_{|x=2}} = 13 \land \frac{d^2s_1(t)}{dt_{|x=3}} = 19
\]

\[
\frac{d^2s_2(t)}{dt_{|x=2}} = 14 \land \frac{d^2s_2(t)}{dt_{|x=3}} = 18
\]

\subsection{Задача 3}
\[
\frac{d^2x}{dy^2} = -\frac{12x + 210x^5}{(6x^2+35x^6)^3}
\]

\subsection{Задача 4}
\[
\frac{dx}{dy} = \frac{1}{f'(x)} \Rightarrow \frac{d^2x}{dy^2} = \frac{d(\frac{1}{f'x})}{dy} = \frac{-f''(x)}{f'(x)^3} \Rightarrow \frac{d^3x}{dy^3} = \frac{-\frac{f'''(x)}{f''(x)}\cdot (f'(x)^3 + 3f''(x)^2) \cdot f'(x)}{f'(x)^6}
\]

\section{Производная высших порядков}

\subsection{Задача 6}
\[
f^{(1)} = \frac{1}{x^2+1}
\]

\[
f^{(2)} = - \frac{2x}{(x^2+1)^2} 
\]

\[
f^{(n)} = \sum_{k=0}^n \binom{k}{n} (x^2+1)^k \cdot  (2x)^{n-k}
\]


\[
(-2n+4)\cdot (1+x)^{-2n+6}
\]

\subsection{Задача 7}

\[
f^{(1)} = \frac{-3x^2+8x-6}{(x^2-3x+2)^2}
\]

\[
f^{(n)} = \sum_{k=0}^n \binom{k}{n} (x^2-2)^k \cdot  (((x-1)(x-2))^-1)^{n-k}
\]

\subsection{Задача 8}

\begin{enumerate}
    \item $5(x(x - 3))^{5x} \left(5 \left(\ln(x(x - 3)) + \frac{2x - 3}{x - 3}\right)^2 + \frac{2 + \frac{2(2x - 3)}{x} - \frac{(2x - 3)^2}{x(x - 3)}}{x - 3}\right)$
    \item $2 \left(9x^2 (\sin^2(3x) - \cos^2(3x)) - 12x \sin(3x) \cos(3x) + \cos^2(3x)\right)$
    \item $\sqrt{\frac{3x + 5}{2x + 1}} \left(3 - 2 \frac{3x + 5}{2x + 1}\right) \frac{\left(\frac{3 - 2 \frac{3x + 5}{2x + 1}}{4(3x + 5)} - \frac{3}{2(3x + 5)} - \frac{1}{2x + 1}\right)}{3x + 5}$
    \item $-\frac{x(x - 2)\left(\frac{(2x - 1)^2}{-x^2 + x + 2} + 2\right)}{-x^2 + x + 2} - \frac{4(x - 1)(2x - 1)}{-x^2 + x + 2} + 2 \ln(x^2 - x - 2)$
\end{enumerate}

\subsection{Задача 9}

\[
\sum_{k=0}^{n} \binom{n}{k} sin(x + \frac{\pi k}{2})
\]

Воспользуемся формулой Эйлера:

\[
\sum_{k=0}^{n} \binom{n}{k} \frac{e^{ix + i\frac{\pi k}{2}} + e^{-ix - i\frac{\pi k}{2}}}{2i}
\]

\[
e^{ix} \sum_{k=0}^{n} \binom{n}{k} (e^{0.5 i \pi})^k = (1+i)^n = Im(e^{ix}(1+i)^n)
\]

\[
(1+i)^n = \sqrt{2}^n e^{0.25i\pi n} = sin(x+\frac{\pi n}{4})
\]

\[
\sum_{k=0}^{n} \binom{n}{k} sin(x + \frac{\pi k}{2}) = 2^{0.5n} sin(x+\frac{\pi n}{4})
\]

\subsection{Задача 10}

\[
y = \begin{cases}
    x^3, \ x \ge 0 \\
    -x^3, \ x < 0 
\end{cases}
\]

\[
y' = \begin{cases}
    3x^2, \ x \ge 0 \\
    -3x^2, \ x < 0 
\end{cases}
\]

Левосторонняя производная равна правосторонней равна 0

\[
y'' = \begin{cases}
    6x, \ x \ge 0 \\
    -6x, \ x < 0 
\end{cases}
\]

Левосторонняя производная равна правосторонней равна 0

\[
y''' = \begin{cases}
    6, \ x \ge 0 \\
    -6, \ x < 0 
\end{cases}
\]

Левосторонняя производная не равна правосторонней -6 и 6 соответственно, следовательно производной в точке x=0 не существует.

Таким образом существует производная максимум второго порядка.

\section{Производные высших порядков от параметрически или неявно заданных фукций}

\subsection{Задача 11}

\[
\frac{dx}{dt} = \frac{2x-x^4}{x^6+2x^3+1}
\]

\[
\frac{dy}{dt} = \frac{3x^2}{x^6+2x^3+1}
\]

\[
\frac{dy}{dx} = \frac{(3x^2)(x^6+2x^3+1)}{(x^6+2x^3+1)(2x-x^4)} = \frac{3x^2}{2x-x^4}
\]

\[
\frac{d^2y}{dx^2} = \frac{6x^3+6}{x^6-4x^3+4}
\]

\subsection{Задача 12}

\[
\frac{d^2y}{dx^2} = 4x^3e^{-2x} - 12x^2e^{-2*x} + 6xe^{-2*x}
\]

\subsection{Задача 13}

\[
d^2y = 6xdx^2
\]

\[
d^2y = 30t^4dt^2
\]

\subsection{Задача 14}

\[
3y-3x+3+\arctan(y/x) = 0
\]

\[
\frac{d(3y)}{dx}dx - \frac{d(3x)}{dx}dx + \frac{d(\arctan(y/x))}{dx}dx = \frac{d(0)}{dx}dx
\]

\[
3dy - 3dx + \frac{1}{1+(\frac{y}{x})^2}\cdot \frac{d(\frac{y}{x})}{dx} = 3(dy-dx) + \frac{xdy-ydx}{x^2+y^2} = 0
\]

\[
\frac{dy}{dx} = a = \frac{3}{4}
\]

\[
3a-3+\frac{ax-y}{x^2+y^2} = 0
\]

\[
3\frac{dp}{dx} + \frac{d\frac{dp}{dx}(x^2+y^2)-(xp-y)(2x+2yp)}{(x^2+y^2)^2}
\]

Подставляем, получаем:

\[
d^2y = 0.375dx^2
\]

\subsection{Задача 15}
\begin{enumerate}
    \item $\frac{dy}{dx|_{x=0.25}} = 4\cdot 0.25 - 1 = 0$ - является
    \item $\frac{dy}{dx|_{x=\arccos(4/5)}} = 3\cos(\arccos(4/5)) -4\sin(\arccos(4/5)) = 0$
    \item $y(x) = \begin{cases}
        2x-2, \ x\ge 2 \\
        2x, \ 0 \le x < 2 \\
        0, \ x < 0\\
    \end{cases} \Rightarrow y'(x) = \begin{cases}
        2, \ x\ge 2 \\
        2, \ 0 \le x < 2 \\
        0, \ x < 0\\
    \end{cases}$
\end{enumerate}
чтд

\section{Теорема Ролля}

\subsection{Задача 18}
Т.к Теорема Ролля требует того, чтобы фукнция на этом отрезке была непрерывна и дифференцируема, а тангенс имеет разрыв в точке $\pi/2$

\subsection{Задача 19}
Согласно условию, мы можем выбрать такие m, k, что будет выполнятся $0 \le m < k \le n \Rightarrow f(0)=f(m)=f(k)=f(n)$ согласно условию задачи. Тогда зная, что она определена, и дифференцируема, значение фукнции f в любой точке на отрезке = 0, тогда мы можем повторить процесс до производной n ого порядка. На последнем шаге мы получаем такую точку $\xi$  что $f^{(n)}(\xi) = 0$

\section{Теорема Лагранжа}
\subsection{Задача 23}
Функция $ f $, дифференцируемая на интервале $(a, b)$, не может иметь разрыв второго рода у производной $ f' $, поскольку дифференцируемость $ f $ гарантирует существование и конечность производной в каждой точке интервала. Согласно теореме Лагранжа, значение $ f'(x) $ между любыми двумя точками выражается через среднее изменение функции, что исключает разрывы второго рода. 

Вывод: $ f'(x) $ непрерывна или, в худшем случае, имеет разрывы первого рода, но не второго рода.

\section{Первое достаточное условие экстремума}

\subsection{Задача 28}
Т.к любой равнобедренный треугольник можно вписать в окружность, также третью сторону можно выразить используя теорему косинусов + можно выразить сторону через радиус, тогда

\[
P = 4R \cos(\alpha) + 2R\sin(2\alpha)
\]

\[
\frac{dP}{d\alpha} = 2(2\sin(2\alpha) + 2\cos(\alpha))
\]

\[
2(2\sin(2\alpha) + 2\cos(\alpha)) \Rightarrow \alpha = \frac{3\pi}{2} = 60\deg
\]

Следовательно треугольник равносторонний


\end{document}