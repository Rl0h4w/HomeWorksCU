\documentclass[a4paper,12pt]{article}
\usepackage[utf8]{inputenc}
\usepackage[russian]{babel}
\usepackage{amsmath,amsfonts,amssymb}
\usepackage{graphicx}
\usepackage{geometry}
\usepackage{hyperref}
\usepackage{venndiagram}
\usepackage{tikz}
\usetikzlibrary{shapes.geometric, calc}
\usepackage{pgfplots}
\usepackage{float}

% Параметры страницы
\geometry{top=2cm,bottom=2cm,left=2.5cm,right=2.5cm}
\geometry{a4paper, margin=1in}

% Заголовок документа
\title{Домашнее задание}
\author{Студент: Лохов Ростислав}
\date{\today}

\begin{document}

% Титульный лист
\begin{titlepage}
    \centering
    \vspace*{1cm}

    \Huge
    \textbf{Домашнее задание}

    \vspace{0.5cm}
    \LARGE
    По курсу: \textbf{Линейная Алгебра}

    \vspace{1.5cm}

    \textbf{Студент: Лохов Ростислав}

    \vfill

    \Large
    АНО ВО Центральный Университет\\
    \vspace{0.3cm}
    \today

\end{titlepage}

% Содержание
\tableofcontents
\newpage

% Основной текст
\section{Инструкция}

Синий уровень\\
Возможный максимум — 7 баллов, даже если ты решишь больше задач. Обязательных задач нет.

Красный уровень\\
Возможный максимум — 10 баллов.

Чёрный уровень\\
Для зачёта на чёрном уровне нужно получить зачёт на красном уровне и набрать минимум 4 балла за задачи чёрного уровня. Баллы, полученные на чёрном уровне, можно использовать, чтобы закрыть синий. Если баллов, полученных на чёрном уровне, не хватает для зачёта, их можно использовать для закрытия красного уровня.

В домашнем задании требуется показать методику решения задачи. Задача, где итоговый ответ приведён без обоснования, не будет считаться решённой. Везде нужен точный алгебраический ответ. При выполнении домашней работы используй теорию и алгоритмы, изученные в рамках курса на текущий момент.

\newpage

\section{Задачи}

\subsection{Фундаментальная система решений}

\subsubsection{Задача 1 \hfill 1 балл}
\textbf{Условие задачи:}

Найди базис векторного пространства \( U = \{y \in \mathbb{R}^5 \mid Ay = 0\} \), где
\[
A =
\begin{pmatrix}
2 & 1 & 3 & 0 & 7 \\
1 & 2 & 0 & 1 & 4 \\
0 & 1 & -1 & 1 & 0
\end{pmatrix}.
\]

\textbf{Решение:}
Приведем к диагональному виду: 
\[
A = 
\begin{pmatrix}
1 & 0 & 2 & 0 & 3 \\
0 & 1 & -1 & 0 & 1 \\
0 & 0 & 0 & 1 & -1 \\
\end{pmatrix}
\]

Выразим через свободные и запишем через вектора. $y_1;y_2;y_4$ - ведущие, остальные - свободные. Пусть $y_3=t;y_5=s$

\[
\begin{cases}
y_1 = -2t -3s\\
y_2 = t - s\\
y_3 = t\\
y_4 = s\\
y_5 = s\\
\end{cases}
\]


тогда 
\[
v_1 = \begin{pmatrix}
-2 \\
1 \\
1 \\
0 \\
0 \\
\end{pmatrix}
\]

\[
v_2 = \begin{pmatrix}
-3 \\
-1 \\
0 \\
1 \\
1 \\
\end{pmatrix}
\]

Что и будет базисом векторного пространства решений U






\newpage

\subsubsection{Задача 2 \hfill 1 балл}
\textbf{Условие задачи:}

Определи, можно ли из системы векторов
\[
v_1 =
\begin{pmatrix}
-4 \\
-5 \\
2 \\
3
\end{pmatrix},
\quad
v_2 =
\begin{pmatrix}
0 \\
1 \\
2 \\
1
\end{pmatrix},
\quad
v_3 =
\begin{pmatrix}
5 \\
5 \\
3 \\
-2
\end{pmatrix},
\quad
v_4 =
\begin{pmatrix}
4 \\
5 \\
-2 \\
-3
\end{pmatrix}
\]
выбрать ФСР для системы линейных уравнений
\[
\begin{cases}
x_1 - x_3 + 2x_4 = 0, \\
x_2 - 2x_3 + 3x_4 = 0, \\
2x_1 - 2x_2 + 2x_3 - 2x_4 = 0.
\end{cases}
\]

\textbf{Решение:}

Окей, выберем главные члены - $x_1; x_2$, выразим через них:
\[
\begin{cases}
x_1 = t - 2f \\
x_2 = 2t - 3f \\
x_3 = t \\
x_4 = f \\
\end{cases}
\]

Тогда
\[
v_1 =
\begin{pmatrix}
    1\\
    2\\
    1\\
    0
\end{pmatrix}
\]

\[
v_2 =
\begin{pmatrix}
    -2\\
    -3\\
    0\\
    1
\end{pmatrix}
\]

Внимательно посмотрев на векторы из условия и на наши, можно сказать, что v1 и v2 подходят, а также v2 и v4

\subsubsection{Задача 3 \hfill 2 балла}
\textbf{Условие задачи:}

Найди скелетное разложение матрицы \( A \):
\[
A =
\begin{pmatrix}
1 & 1 & 5 & -1 & 3 \\
1 & 2 & 8 & -2 & 5 \\
1 & 0 & 2 & 1 & -1
\end{pmatrix}.
\]

\textbf{Решение:}

Если я правильно помню что такое скелетное разложение матрицы, то: A = CUR. Приведем к диагональному виду, получим:

\[
\begin{pmatrix}
1 & 0 & 2 & 0 & 1 \\
0 & 1 & 3 & 0 & 0 \\
0 & 0 & 0 & 1 & -2 \\
\end{pmatrix}
\]
Из ступенчатого вида видно, что базисные столбцы - первый, второй и четвертый т.к они содержат ведущие элементы

\[
C =
\begin{pmatrix}
1 & 1 & -1 \\
1 & 2 & -2 \\
1 & 0 & 1 \\
\end{pmatrix}
\]
т.к ранг матрицы совпадает с количеством строк, то можно сделать так: 

\[
A = CU = \begin{pmatrix}
1 & 1 & -1 \\
1 & 2 & -2 \\
1 & 0 & 1 \\
\end{pmatrix} \cdot 
\begin{pmatrix}
1 & 0 & 2 & 0 & 1 \\
0 & 1 & 3 & 0 & 0 \\
0 & 0 & 0 & 1 & -2 \\
\end{pmatrix} = 
\begin{pmatrix}
1 & 0 & 2 & 0 & 1 \\
0 & 1 & 3 & 0 & 0 \\
0 & 0 & 0 & 1 & -2 \\
\end{pmatrix}
\]

\subsubsection{Задача 4 \hfill 2 балла}
\textbf{Условие задачи:}

Найди решения неоднородной системы линейных уравнений \( Ax = b \) через базис решений (ФСР) соответствующей однородной системы \( Ax = 0 \), где
\[
A =
\begin{pmatrix}
1 & 1 & 2 & -1 & -2 \\
-1 & -2 & -1 & 2 & 6 \\
1 & 2 & 1 & -3 & -9
\end{pmatrix},
\quad
b =
\begin{pmatrix}
0 \\
-5 \\
6
\end{pmatrix}.
\]

\textbf{Решение:}
Приведём к диагональному виду: 

\[
\begin{pmatrix}
    1 & 0 & 3 & 0 & 2 & | & -5 \\
    0 & 1 & -1 & 0 & -1 & | & 4 \\ 
    0 & 0 & 0 & 1 & 3 & | & -1 \\
\end{pmatrix}
\]
Выделим главные переменные и параметризируем

\[
\begin{cases}
x_1 = -5 -3s -2t  \\
x_2 = 4 + s + t\\
x_3 = s \\
x_4 = -1 -3t \\
x_5 = t \\
\end{cases}
\]

Тогда общее решение: 

\[
\mathbf{x} =
\begin{pmatrix}
x_1 \\
x_2 \\
x_3 \\
x_4 \\
x_5 \\
\end{pmatrix}
=
\begin{pmatrix}
-5 \\
4 \\
0 \\
-1 \\
0 \\
\end{pmatrix}
+ s
\begin{pmatrix}
-3 \\
1 \\
1 \\
0 \\
0 \\
\end{pmatrix}
+ t
\begin{pmatrix}
-2 \\
1 \\
0 \\
-3 \\
1 \\
\end{pmatrix}, \quad s, t \in \mathbb{R}
\]


\subsubsection{Задача 5 \hfill 1 балл}
\textbf{Условие задачи:}

Найди ранг матрицы
\[
\begin{pmatrix}
5 & 1 & 5 \\
7 & 4 & 9 \\
8 & -1 & 6 \\
5 & 1 & 5
\end{pmatrix}.
\]

\textbf{Решение:}

Приведём к треугольному виду:

\[
\begin{pmatrix}
5 & 1 & 5 \\
0 & \frac{13}{5} & 2 \\
0 & 0 & 0 \\
0 & 0 & 0 \\
\end{pmatrix}
\]
Ранг - наибольшее количество линейно независимых строк, т.е строка 3 и 4 линейно зависимы от 1 и 2, т.е ранг равен 2. 

\subsubsection{Задача 6 \hfill 1 балл}
\textbf{Условие задачи:}

Пусть \( A \in M_n(\mathbb{R}) \) такая, что \( A^2 = A \) и \( \text{rk}(A) = n \). Докажи, что \( A = I \).

\textbf{Решение:}

Домножим слева на обратную от A: $A^{-1}A^2=A^{-1}A\Rightarrow  A(AA^{-1})=I \Rightarrow AI=I \Rightarrow A=I$

\subsubsection{Задача 7 \hfill 2 балла}
\textbf{Условие задачи:}

Пусть \( U \subseteq \mathbb{R}^4 \) — векторное подпространство, \( U = \text{span}(v_1, v_2, v_3, v_4) \), где
\[
v_1 =
\begin{pmatrix}
1 \\
1 \\
4 \\
3
\end{pmatrix},
\quad
v_2 =
\begin{pmatrix}
1 \\
5 \\
5 \\
8
\end{pmatrix},
\quad
v_3 =
\begin{pmatrix}
-2 \\
6 \\
-2 \\
24
\end{pmatrix},
\quad
v_4 =
\begin{pmatrix}
2 \\
-4 \\
3 \\
-19
\end{pmatrix}.
\]
Задай это подпространство в виде \( U = \{y \in \mathbb{R}^4 \mid Ay = 0\} \) для некоторой матрицы \( A \in M_{m \times 4}(\mathbb{R}) \).

\textbf{Решение:}
Запишем в матрицах и приведём к диагональному виду.

\[
V =
\begin{pmatrix}
1 & 1 & -2 & 2 \\
1 & 5 & 6 & -4 \\
4 & 5 & -2 & 3 \\
3 & 8 & 24 & -19
\end{pmatrix}^{T}
\]

\[
\begin{pmatrix}
1 & 0 & 0 & -17 \\
0 & 1 & 0 & 0 \\
0 & 0 & 1 & 5 \\
0 & 0 & 0 & 0
\end{pmatrix}
\]

$dim(ker(A))=3 \Longrightarrow rank(A)=4-3 = 1$ Т.е 1 строку и 4 столбца. Также $Av_1 = 0; Av_2 = 0; Av_3 = 0$. Вектор $A = \begin{pmatrix}
    -17 & 0 & -5 & 1
\end{pmatrix}$

Подпространство \( U \) задаётся матрицей \( A = \begin{pmatrix} 17 & 0 & -5 & 1 \end{pmatrix} \), и $ U = \{ y \in \mathbb{R}^4 \mid 17y_1 - 5y_3 + y_4 = 0 \} $.



\subsubsection{Задача 8 \hfill 2 балла}
\textbf{Условие задачи:}

Пусть \( A \) и \( B \) — квадратные матрицы одного размера. Докажи, что
\[
\text{rk} \begin{pmatrix} A & B \\ 2A & -5B \end{pmatrix} = \text{rk}(A) + \text{rk}(B).
\]

\textbf{Решение:}
После приведения к треугольному виду
\[
\begin{pmatrix}
    A & B\\
    0 & 0
\end{pmatrix}
\]
т.к  ранг блоковой приведенной матрицы равен сумме блоков т.к они независимы и поскольку элементарные преобразования строк не изменяют ранг матрицы, то
\[
\text{rk} \begin{pmatrix} A & B \\ 2A & -5B \end{pmatrix} = \text{rk}(A) + \text{rk}(B).
\] ЧТД

\subsubsection{Задача 9 \hfill 1 балл}
\textbf{Условие задачи:}

Можно ли найти две матрицы \( A, B \in M_{3 \times 4}(\mathbb{R}) \) такие, что \( \text{rk}(A) = \text{rk}(B) \), \( \text{rk}(A - B) = \text{rk}(A) - \text{rk}(B) \) и \( \text{rk}(A + B) = \text{rk}(A) + \text{rk}(B) \)? Если да, приведи пример. Если нет, докажи, что таких матриц не существует.

\textbf{Решение:}

Да, нулевые матрицы )))

\subsubsection{Задача 10 \hfill 1 балл}
\textbf{Условие задачи:}

Найди ранг матрицы \( ABC \), где
\[
A =
\begin{pmatrix}
-2 & 0 & 1 \\
-3 & -2 & 3 \\
0 & 0 & -3
\end{pmatrix},
\quad
B =
\begin{pmatrix}
1 & 2 \\
2 & -1 \\
3 & 3
\end{pmatrix},
\quad
C =
\begin{pmatrix}
-2 & 1 \\
1 & -1
\end{pmatrix}.
\]

\textbf{Решение:}

\[
ABC = \begin{pmatrix}
-3 & 2 \\
1 & -3 \\
9 & 0
\end{pmatrix}
\]

\[
det(ABC) = det(   \begin{pmatrix}
   -3 & 2 \\
   1 & -3
   \end{pmatrix}) = 7
\]

Значит ранг = 2
\subsubsection{Задача 11 \hfill 2 балла}
\textbf{Условие задачи:}

Пусть
\[
A =
\begin{pmatrix}
2 \\
3 \\
1 \\
-2
\end{pmatrix},
\quad
B =
\begin{pmatrix}
-2 & 3 & 1 & -1
\end{pmatrix}.
\]
Выясни, у какого пространства размерность больше: 
\[
V = \{x \in \mathbb{R}^4 \mid ABx = 0\} \quad \text{или} \quad U = \{x \in \mathbb{R}^4 \mid (AB)^{2024}x = 0\}.
\]

\textbf{Решение:}

\[
rk(AB) = 1; dim(ker(AB)) = n - rk(AB) = 3; dim(V) = 3
\]

\[
(AB)^k = A(BA)^{k-1}B; BA = 8; (AB)^k = 8^{k-1}AB = 8^{2023}AB; ABx=0; ker((AB)^2023) = ker(AB) = V
\]

\[
dim(U) = dim(V) = 3
\]


\subsubsection{Задача 12 \hfill 2 балла}
\textbf{Условие задачи:}

Пусть \( A \in M_n(\mathbb{R}) \) — произвольная матрица, \( \hat{A} \) — её присоединённая матрица. Найди \( \text{rk}(\hat{A}) \) в зависимости от \( \text{rk}(A) \).

\textbf{Решение:}
$A\cdot adj(a) = det(A)\cdot I$ Возможны несколько случаев
\begin{enumerate}
    \item Матрица невырождена и ранг А = n, значит ранги равны
    \item Матрица вырождена и равнг равен n-1, значит существует единственное решение Ax=0. Из формулы выше получаем, что произведение = 0, это означает, что все столбцы адьюктивной матрицы находятся в ядре А, размер ядра равен 1 по теореме о ранге, и поскольку rk(A) = n-1, существует хотя бы один ненулевой минор порядка n-1, что гарантирует что adj(A) не является нулвевой матрицей, а значит rk(adj(A)) = 1
    \item rk(A) < n-1, значит все миноры порядка n-1 равны нулю и ранг равен 0.
\end{enumerate}

Таким образом: 

\[
\text{rk}(\hat{A}) =
\begin{cases}
n, & \text{если } \text{rk}(A) = n, \\
1, & \text{если } \text{rk}(A) = n - 1, \\
0, & \text{если } \text{rk}(A) < n - 1.
\end{cases}
\]

\subsubsection{Задача 13 \hfill 2 балла}
\textbf{Условие задачи:}

Дана матрица из \( M_{4 \times 4}(\mathbb{R}) \). В ней разрешается взять произвольную квадратную подматрицу \( 3 \times 3 \) на любых соседних строках и соседних столбцах и повернуть её против часовой стрелки на \( 90^\circ \). Например,
\[
\begin{pmatrix}
* & a & b & c \\
* & d & e & f \\
* & g & h & i \\
* & * & * & *
\end{pmatrix}
\quad \rightarrow \quad
\begin{pmatrix}
* & c & f & i \\
* & b & e & h \\
* & a & d & g \\
* & * & * & *
\end{pmatrix}.
\]
На сколько максимально может измениться ранг матрицы?

\textbf{Решение:}

На 1. 

Окей, поворот подматрицы может повлиять на линейную зависимость строк или столбцов, а значит на ранг. Поскольку изменяется лишь часть матрицы, то изменение ранга ограничено. допустим мы нашли что на 1 - возможно изменение: 

\[
A = \begin{pmatrix}
    1& 2& 3& 4 \\
    5& 6 &7 &8 \\
    9 &10 &11& 12 \\
    13 &14 &15& 16 \\
\end{pmatrix}
\]
при повороте левой верхней подматрицы будет изменеие на 1. Окей, рассмотрим тогда матрицу D = |B-A|, т.е разницу, нули будут только в тех местах, где совпадают B и A, т.е там где центр подматрицы и везде где не подматрица, тогда изменение ранга может быть 0, 1, 2. т.е $rank(D) \ge |rank(B) - rank(A)| \Rightarrow 2 \ge |rank(B) - rank(A)$

\subsubsection{Задача 14 \hfill 2 балла}
\textbf{Условие задачи:}

Дана матрица \( A \in M_{4 \times 4}(\mathbb{R}) \) с минимальным многочленом \( f_{\text{min}} = t^2 - 5t \). Известно, что \( \text{rk}(5I - A) = 3 \). Найди \( \text{rk}(A) \) и характеристический многочлен \( \chi_A \).

\textbf{Решение:}
Собственные значения матрицы 0 и 5, окей, попробуем сделать такую матрицу чтобы подходило под условие. Сумма степеней многочлена будет равна размерности. Тогда пусть А - матрица, покажем ее:

\[
A = 
\begin{pmatrix}
    5& 0 &0& 0\\
    0 &0 &0 &0\\
    0 &0 &0 &0\\
    0 &0& 0 &0\\
\end{pmatrix}
\]
Подходит под условие задачи, размерность 4 на 4, минимальный многочлен = $t^2-5t$ Покажем, что подходит под условие для ранга: 

\[
5I-A = 
\begin{pmatrix}
    0& 0 &0& 0\\
    0 &5 &0 &0\\
    0 &0 &5 &0\\
    0 &0& 0 &5\\
\end{pmatrix}
\]
Подходит, ранг = 3, значит $rk(A) = 1 ;\chi_A = t^3(t-5)$ 

\end{document}
