\documentclass[a4paper,12pt]{article}

% Кодировка и язык
\usepackage[utf8]{inputenc}
\usepackage[russian]{babel}

% Математические пакеты
\usepackage{amsmath,amsfonts,amssymb}

% Графика
\usepackage{graphicx}
\usepackage{tikz}
\usetikzlibrary{shapes.geometric, calc}
\usepackage{pgfplots}

% Геометрия страницы
\usepackage{geometry}
\geometry{top=2cm, bottom=2cm, left=2.5cm, right=2.5cm}

% Гиперссылки
\usepackage{hyperref}

% Плавающие объекты
\usepackage{float}

% Дополнительные пакеты
\usepackage{venndiagram}

% Настройки заголовка
\title{Домашнее задание}
\author{Студент: \textbf{Ростислав Лохов}}
\date{\today}

\begin{document}

% Титульный лист
\begin{titlepage}
    \centering
    \vspace*{1cm}

    \Huge
    \textbf{Домашнее задание}

    \vspace{0.5cm}
    \LARGE
    По курсу: \textbf{Линейная Алгебра}

    \vspace{1.5cm}

    \textbf{Студент: Ростислав Лохов}

    \vfill

    \Large
    АНО ВО Центральный университет\\
    \vspace{0.3cm}
    \today

\end{titlepage}

% Содержание
\tableofcontents
\newpage

% Основной текст
\section{Сингулярное разложение (SVD)}

\subsection{Задача 4}
\textbf{Условие задачи:}

\[
S = AA^T = \begin{pmatrix}
    6 & -5 \\
    -5 & 6 \\
\end{pmatrix}
\]

\[
v_1 = \begin{pmatrix}
    -\frac{\sqrt{2}}{2} \\ \frac{\sqrt{2}}{2}
\end{pmatrix} \quad \text{и} \quad v_2 = \begin{pmatrix}
\frac{\sqrt{2}}{2} \\ \frac{\sqrt{2}}{2}
\end{pmatrix}
\]

\[
S = \begin{pmatrix}
    -\frac{\sqrt{2}}{2} & \frac{\sqrt{2}}{2} \\
    \frac{\sqrt{2}}{2} & \frac{\sqrt{2}}{2} \\
\end{pmatrix} \begin{pmatrix}
    11 & 0 \\
    0 & 1 \\
\end{pmatrix} \begin{pmatrix}
    -\frac{\sqrt{2}}{2} & \frac{\sqrt{2}}{2} \\
    \frac{\sqrt{2}}{2} & \frac{\sqrt{2}}{2} \\
\end{pmatrix}
\]

\[
V = \begin{pmatrix}
    1 & -1 \\
    2 & -1 \\
    -1 & 2 
\end{pmatrix} \begin{pmatrix}
    -\frac{\sqrt{2}}{2} & \frac{\sqrt{2}}{2} \\
    \frac{\sqrt{2}}{2} & \frac{\sqrt{2}}{2} \\
\end{pmatrix} \begin{pmatrix}
    \sqrt{11} & 0 \\
    0 & 1 \\
\end{pmatrix} = \begin{pmatrix}
    -\frac{\sqrt{22}}{2} & 0 \\
    -\frac{3\sqrt{22}}{22} & \frac{\sqrt{2}}{2} \\
    \frac{3\sqrt{22}}{22} & \frac{\sqrt{2}}{2}
\end{pmatrix}
\]

\[
A = U\Sigma V^T = \begin{pmatrix}
    1 & 2 & -1 \\
    -1 & -1 & 2 \\
\end{pmatrix}
\]

Ближайшая матрица по норме Фробениуса ранга 1:

\[
B_1 = \begin{pmatrix}
    -\frac{\sqrt{2}}{2} & \frac{\sqrt{2}}{2} \\
    \frac{\sqrt{2}}{2} & \frac{\sqrt{2}}{2} \\
\end{pmatrix} \sqrt{11} V^T = \begin{pmatrix}
    1 & \frac{3 + \sqrt{11}}{2} & \frac{-3 + \sqrt{11}}{2} \\
    -1 & \frac{-3 + \sqrt{11}}{2} & \frac{3 + \sqrt{11}}{2}
\end{pmatrix}
\]

\subsection{Задача 5}
\[
A = \begin{pmatrix}
    -1 & 3 & 3 \\
    3 & -4 & -4 \\
    3 & -4 & -4 \\
\end{pmatrix}
\]

\[
AA^T = \begin{pmatrix}
    19 & -27 & -27 \\
    -27 & 41 & 41 \\
    -27 & 41 & 41 \\
\end{pmatrix}
\]

\[
\det(AA^T-\lambda I) = (\lambda-100)(\lambda-1)\lambda = 0
\]

\[
V = \begin{pmatrix}
    -2 & 3 & 0 \\
    3 & 1 & -1 \\
    3 & 1 & 1 \\
\end{pmatrix}
\]

\[
V_{\text{норм}} = \begin{pmatrix}
    -\frac{\sqrt{22}}{11} & \frac{3\sqrt{11}}{11} & 0 \\
    \frac{3\sqrt{22}}{22} & \frac{\sqrt{11}}{11} & -\frac{\sqrt{2}}{2} \\
    \frac{3\sqrt{22}}{22} & \frac{\sqrt{11}}{11} & \frac{\sqrt{2}}{2} \\
\end{pmatrix}
\]

SVD:

\[
\begin{pmatrix}
    \frac{\sqrt{22}}{11} & \frac{3\sqrt{11}}{11} & 0 \\
    \frac{3\sqrt{22}}{22} & \frac{\sqrt{11}}{11} & -\frac{\sqrt{2}}{2} \\
    \frac{3\sqrt{22}}{22} & \frac{\sqrt{11}}{11} & \frac{\sqrt{2}}{2} \\
\end{pmatrix} \begin{pmatrix}
    10 & 0 & 0 \\
    0 & 1 & 0 \\
    0 & 0 & 1 \\
\end{pmatrix} \begin{pmatrix}
    -\frac{\sqrt{22}}{11} & \frac{3\sqrt{11}}{11} & 0 \\
    \frac{3\sqrt{22}}{22} & \frac{\sqrt{11}}{11} & -\frac{\sqrt{2}}{2} \\
    \frac{3\sqrt{22}}{22} & \frac{\sqrt{11}}{11} & \frac{\sqrt{2}}{2} \\
\end{pmatrix}^T = \begin{pmatrix}
    -1 & 3 & 3 \\
    3 & -4 & -4 \\
    3 & -4 & -4 \\
\end{pmatrix}
\]

\subsection{Задача 7}

\[
A = \begin{pmatrix}
    3 & 1 & 1 \\
    1 & 3 & -1 \\
    1 & -1 & 3 \\
\end{pmatrix}
\]

Можно подобрать $B$, просто очень внимательно посмотрев:

\[
B = \begin{pmatrix}
    1 & 1 & 1 \\
    1 & 1 & -1 \\
    1 & -1 & 1 \\
\end{pmatrix}
\]

\subsection{Задача 10}
\[
\varphi: v \mapsto v\varphi^* 
\]

\[
\varphi \varphi^* = \varphi^*\varphi
\]

\[
\varphi(x) = Ax, \varphi^*(x) = Bx = A^Tx \implies AA^T = A^TA
\]

Пусть $u$ - собственный вектор для $A$:

\[
Au = \lambda u
\]

\[
AA^Tu = A^TAu = \lambda A^Tu
\]

\[
A^Tu \text{ - собственный для } A
\]

\[
\varphi^*(u) \text{ - собственный для } \varphi \text{ c собственным значением } \lambda
\]

$S$ - собственное подпространство $\varphi$ с собственным значением $\lambda$, выберем ортонормированный базис в $S$ из собственных векторов.

Докажем для всех базисных векторов:

$\forall v_i: \varphi^*v_i = \lambda v_i$

Рассмотрим $v_1$ (для остальных доказывается аналогично)

$\varphi^*v_1$ - собственный вектор $\varphi^*v_1 \in S$

$\varphi^*v_1 = \beta_1 v_1 + \beta_2 v_2 + \cdots + \beta_s v_s$

Рассмотрим:

$\langle v_1, \varphi v_1 \rangle = \lambda \langle v_1, v_1 \rangle = \lambda$

$\langle \varphi v_1, v_1 \rangle = \beta_1 \implies \beta_1 = \lambda$ 

Рассмотрим $\forall i \neq 1$:

$\langle v_1, \varphi v_i \rangle = \lambda \langle v_i, v_1 \rangle = 0$

$\langle \varphi v_1, v_i \rangle = \beta_i \implies \beta_i = 0 \implies \varphi^*v_1 = \lambda v_1 \implies \forall v \in S: \varphi^*v = \lambda v$

Что и требовалось доказать.

\subsection{Задача 11}
$U$, $S$, $V$ - Рон, Гарри и Гермиона соответственно. 

$A = USV$

$\det(USV - \lambda I)$

\[
A' = C^{-1}AC = \begin{pmatrix}
    0.5 & 2.5 \\
    -0.5 & -0.5 \\
\end{pmatrix}
\]

$A'=USV^T=UQAQ^TV^T \implies A'A'^T = US^2U^T$

\[
A'A'^T = \begin{pmatrix}
    6.5 & -1.5 \\
    -1.5 & 0.5 \\
\end{pmatrix}
\]

\[
\det(A'A'^T - \lambda I) = 4\lambda^2 - 28\lambda + 4 = 0
\]

\[
\text{spec} \left\{\sqrt{\frac{7\pm\sqrt{45}}{2}}\right\}
\]

\end{document}