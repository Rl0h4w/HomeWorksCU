\documentclass[a4paper,12pt]{article}
\usepackage[utf8]{inputenc}
\usepackage[russian]{babel}
\usepackage{amsmath,amsfonts,amssymb}
\usepackage{graphicx}
\usepackage{geometry}
\usepackage{hyperref}
\usepackage{venndiagram}
\usepackage{tikz}
\usetikzlibrary{shapes.geometric, calc}
\usepackage{pgfplots}
\usepackage{float}

% Параметры страницы
\geometry{top=2cm,bottom=2cm,left=2.5cm,right=2.5cm}
\geometry{a4paper, margin=1in}

% Заголовок документа
\title{Домашнее задание}
\author{Студент: Ваше Имя Здесь}
\date{\today}

\begin{document}

% Титульный лист
\begin{titlepage}
    \centering
    \vspace*{1cm}

    \Huge
    \textbf{Домашнее задание}

    \vspace{0.5cm}
    \LARGE
    По курсу: \textbf{Математический Анализ}

    \vspace{1.5cm}

    \textbf{Студент: Лохов Ростислав}

    \vfill

    \Large
    АНО ВО Центральный Университет\\
    \vspace{0.3cm}
    \today

\end{titlepage}

% Содержание
\tableofcontents
\newpage

% Основной текст

\section{Предел монотонной последовательности}

\subsection{Задача 1}
\textbf{Условие задачи:}
Докажи, что последовательность ${x_n}$ имеет предел и найди его, если $x_1=13, x_{n+1}=\sqrt{12+x_n}$

Найди $\sup_{n \in \mathbb{N}} x_n \quad \inf_{n \in \mathbb{N}} x_n$

\textbf{Решение:}
Предположим, что последовательность сходится, тогда $L=\sqrt{12+L}$, получаем $L^2-12-L=0, L=4$, отрицательным быть не можем т.к у нас множество значений принадлежит множеству неотрицательных действительных чисел, $\sup = 13, \inf = 4$ т.к последовательность монотонно убывает($x_n \le x_{n+1} \Rightarrow x_n \le \sqrt{12+x_n} \Rightarrow x_n= 4, -3, \text{при} x_n \ge 4 \text{ монотонно убывает}$) (последовательность вложенных корней) и по теореме Великого Карла Вайерштрасса если последовательность ограничена и монотонно убывает, то она имеет предел.$\sup = 13, \inf = 0$ 
\vspace{1cm}

\subsection{Задача 2}

\textbf{Условие задачи:}
Последовательность $\{x_n\}$ задана следующим образом:
\[
x_1 > 0, \quad x_{n+1} = \frac{1}{2} \left( x_n + \frac{a}{x_n} \right), \quad \text{где } a > 0.
\]
Докажи, что данная последовательность сходится и найди её предел. Также найди $\sup_{n \in \mathbb{N}} x_n$ и $\inf_{n \in \mathbb{N}} x_n$.

\textbf{Решение:}

\[
x_{n+1}=\frac{x_n^2+a}{2x_n}
\]

Предположим, что последовательность сходится, тогда, $L=0.5(L+\frac{a}{L}), L=\sqrt{a}$

Исследуем монотонность

$x_{n}-x_{n+1}=\frac{x_n^2-a}{2x_n}$ рассмотрим три случая, меньше корня, больше и равен корню.
\begin{enumerate}
    \item $x_n > \sqrt{a}$, тогда последовательность монотонно возрастает, ограничена снизу $\sqrt{a}$, супремум $x_0$, инфинум $\sqrt{a}$
    \item $x_n < \sqrt{a}$, тогда последовательность монотонно убывает, ограничена сверху $\sqrt{a}$, супремум $\sqrt{a}$, инфинум $x_0$
    \item $x_n = \sqrt{a}$ последовательность постоянна и предел равен инфинуму равен супремуму равен $\sqrt{a}$
    Таким образом последовательность монотонна, ограничена и сходится к $\sqrt{a}$
\end{enumerate}


\vspace{1cm}

\subsection{Задача 3}


\textbf{Условие задачи:}
Найди предел
\[
\lim_{n \to \infty} \sum_{k=1}^{n} \frac{2k - 1}{2^k}.
\]

\textbf{Решение:}
\[
\sum_{k=1}^{n} \frac{2k - 1}{2^k} = 2 \sum_{k=1}^{n} \frac{k}{2^k} - \sum_{k=1}^{n} \frac{1}{2^k}.
\]

1. \( S_1 = \sum_{k=1}^{n} \frac{k}{2^k} \)
2. \( S_2 = \sum_{k=1}^{n} \frac{1}{2^k} \)
Т.к вторая сумма - сумма ряда геометрической прогрессии с первым членом 0.5 и знаменателем 0.5, то
\[
\lim_{n \to \infty} S_2 = 1.
\]
распишем вторую сумму: 
\[
S = \frac{1}{2} + \frac{2}{2^2} + \frac{3}{2^3} + \frac{4}{2^4} + \cdots
\]
\[
\frac{S}{2} = \frac{1}{2^2} + \frac{2}{2^3} + \frac{3}{2^4} + \cdots
\]
\[
S - \frac{S}{2} = \left( \frac{1}{2} + \frac{2}{2^2} + \frac{3}{2^3} + \frac{4}{2^4} + \cdots \right) - \left( \frac{1}{2^2} + \frac{2}{2^3} + \frac{3}{2^4} + \cdots \right)
\]
\[
\frac{S}{2} = \frac{1}{2} + \frac{1}{2^2} + \frac{1}{2^3} + \frac{1}{2^4} + \cdots
\]
\[
\frac{S}{2} = \sum_{k=1}^{\infty} \frac{1}{2^k} = \frac{\frac{1}{2}}{1 - \frac{1}{2}} = 1
\]
\[
\frac{S}{2} = 1 \implies S = 2
\]
\[
\lim_{n \to \infty} \sum_{k=1}^{n} \frac{2k - 1}{2^k} = 3
\]

\vspace{1cm}

\subsection{Задача 4}


\textbf{Условие задачи:}
Докажи, что
\[
\lim_{n \to \infty} \sqrt[n]{a} = 1 \quad \text{при } a > 1.
\]

\textbf{Решение:}
Тут типо просто прологарифмировать т.к логарифм непрерывен. Распишем
\[
\lim_{n \to \infty} \frac{1}{n} \ln a = 0.
\]

\[
\ln \left( \lim_{n \to \infty} \sqrt[n]{a} \right) = 0.
\]

\[
\ln \left( \lim_{n \to \infty} \sqrt[n]{a} \right) = 0.
\]

\[
\lim_{n \to \infty} \sqrt[n]{a} = 1.
\]

\vspace{1cm}

\subsection{Задача 5}


\textbf{Условие задачи:}
Найди предел
\[
\lim_{n \to \infty} \sum_{k=1}^{n} \frac{1}{\sqrt{n^2 + k}}.
\]

\textbf{Решение:}
\[
\sum_{k=1}^{n} \frac{1}{\sqrt{n^2 + k}} = \frac{1}{n} \sum_{k=1}^{n} \frac{1}{\sqrt{1 + \frac{k}{n^2}}}=1
\] (Вынесли 1/n, затем при довольно больших n мы получаем под корнем 1, и сумма сокращается до n, далее получаем n/n, что равно 1)
\vspace{1cm}

\subsection{Задача 6}


\textbf{Условие задачи:}
Останется ли верным утверждение теоремы о зажатой последовательности, если
\begin{enumerate}[а)]
    \item заменить нестрогие неравенства строгими: $\forall n \in \mathbb{N}\ x_n < y_n < z_n$;
    \item потребовать, чтобы неравенства выполнялись не для всех членов последовательности, а начиная с некоторого номера: $\exists N : \forall n \geq N\ x_n \leq y_n \leq z_n$?
\end{enumerate}

\textbf{Решение:}

Для обоих случаев верно, я хочу испытать себя и записать оба доказательства полностью в кванторах:

\begin{enumerate}
    \item   \[
            \left( \forall n \in \mathbb{N}\ (x_n < y_n < z_n) \right) \wedge \left( \lim_{n \to \infty} x_n = \lim_{n \to \infty} z_n = L \right) \Rightarrow \left( \lim_{n \to \infty} y_n = L \right)
            \]
    \item   \[
            \left( \exists N \in \mathbb{N}\ \forall n \geq N\ (x_n \leq y_n \leq z_n) \right) \wedge \left( \lim_{n \to \infty} x_n = \lim_{n \to \infty} z_n = L \right) \Rightarrow \left( \lim_{n \to \infty} y_n = L \right)
            \]
\end{enumerate}

\vspace{1cm}

\subsection{Задача 7}


\textbf{Условие задачи:}
Попробуем обобщить теорему о зажатой последовательности на случай, когда пределы $\{x_n\}$ и $\{z_n\}$ различны: даны последовательности $\{x_n\}$, $\{y_n\}$, $\{z_n\}$ такие, что
\[
\forall n \in \mathbb{N}\ x_n \leq y_n \leq z_n,
\]
и при этом
\[
\lim_{n \to \infty} x_n = A, \quad \lim_{n \to \infty} z_n = C.
\]
Отсюда следует, что
\[
A \leq \lim_{n \to \infty} y_n \leq C.
\]
Верно ли приведённое утверждение? Если нет, то что нужно исправить, чтобы оно стало верным?

\textbf{Решение:}
нет, т.к без гарантии того что $y_n$ сходится. Необходимо добавить в условие то, что последовательность сходится
пример: $x_n = \frac{1}{n}, y_n = (-1)^n, z_n = 1$
\vspace{1cm}

\subsection{Задача 8}


\textbf{Условие задачи:}
Найди пределы:
\begin{enumerate}[а)]
    \item \[
    \lim_{n \to \infty} \frac{125^{\frac{1}{n}}-1}{625^{\frac{1}{n}}-1}
    \]
    \item \[
    \lim_{n \to \infty} \left(\frac{3}{1-27^{\frac{1}{n}}}-\frac{4}{1-81^{\frac{1}{n}}} \right).
    \]
\end{enumerate}

\textbf{Решение:}
Можем также представить в виде логарифма:

a)
\[
\lim_{n \to \infty} \frac{125^{\frac{1}{n}} - 1}{625^{\frac{1}{n}} - 1} \approx \lim_{n \to \infty} \frac{\frac{\ln 125}{n}}{\frac{\ln 625}{n}} = \frac{\ln 125}{\ln 625}=0.75
\]

б)
Честно говоря, если в пункте а это был просто быстрый способ посчитать предел, то в пункте б, я не нахожу никакого другого более короткого и красивого способа, чем через логарифмирование и представление через ряды тейлора. Лопеталем тоже можно решить, но с производной второго порядка не охота возиться. Поэтому вот так: 
\[
1 - a^{1/n} \approx -\frac{\ln a}{n} - \frac{(\ln a)^2}{2n^2} + \dots
\]

\[
\frac{3}{1 - 27^{\frac{1}{n}}} \approx \frac{3}{ -\frac{\ln 27}{n} - \frac{(\ln 27)^2}{2n^2}} \approx -\frac{3n}{\ln 27} + \frac{3}{2} + \dots
\]

\[
\frac{4}{1 - 81^{\frac{1}{n}}} \approx \frac{4}{ -\frac{\ln 81}{n} - \frac{(\ln 81)^2}{2n^2}} \approx -\frac{4n}{\ln 81} + 2 + \dots
\]

\[
\frac{3}{1 - 27^{\frac{1}{n}}} - \frac{4}{1 - 81^{\frac{1}{n}}} \approx \left(-\frac{3n}{\ln 27} + \frac{3}{2}\right) - \left(-\frac{4n}{\ln 81} + 2\right)
\]

\[
-\frac{3n}{\ln 27} + \frac{3}{2} + \frac{4n}{\ln 81} - 2
\]

\[
-\frac{3n}{3\ln 3} + \frac{4n}{4\ln 3} + \frac{3}{2} - 2 = -\frac{n}{\ln 3} + \frac{n}{\ln 3} + \frac{3}{2} - 2 = \frac{3}{2} - 2 = -\frac{1}{2}
\]
\vspace{1cm}

\subsection{Задача 9}


\textbf{Условие задачи:}
Докажи, что
\[
\lim_{n \to \infty}  (n!)^{\frac{1}{n}} = +\infty.
\]

\textbf{Решение:}
Если открыть википедию и почитать про апроксимации интересных штук в математике, то можно хорошо доказывать некие сложные штуки, как например эту. Есть формула Стирлинга. Докажем через неё:
\[
n! = \sqrt{2\pi n} \left( \frac{n}{e} \right)^n \left( 1 + \frac{1}{12n} + \frac{1}{288n^2} - \frac{139}{51840n^3} + \cdots \right)
\]
Получается, что при n стремящимся к бесконечности, правый множитель можно откинуть, он равен 1.
   \[
   \lim_{n \to \infty} (n!)^{\frac{1}{n}} = \lim_{n \to \infty} (\sqrt{2\pi n})^{\frac{1}{n}} \cdot \frac{n}{e} = 1 \cdot \lim_{n \to \infty} \frac{n}{e} = +\infty
   \]
как можно заметить, это будет бесконечностю. Опустил доказательство факта, что $n^{1/n}$ стремится к 1 при стремлении n к бесконечности. Математика удивительная вещь, как можно связать $\e, \pi$, факториал в одной формуле, это же реально круто(аж муражки по коже были когда узнал про такие формулы, которые связывают константы)
\end{document}
