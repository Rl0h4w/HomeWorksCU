\documentclass[a4paper,12pt]{article}
\usepackage[utf8]{inputenc}
\usepackage[russian]{babel}
\usepackage{amsmath,amsfonts,amssymb}
\usepackage{graphicx}
\usepackage{geometry}
\usepackage{hyperref}
\usepackage{venndiagram}
\usepackage{tikz}
\usetikzlibrary{shapes.geometric, calc}
\usepackage{pgfplots}
\usepackage{float}

% Параметры страницы
\geometry{top=2cm,bottom=2cm,left=2.5cm,right=2.5cm}
\geometry{a4paper, margin=1in}

% Заголовок документа
\title{Домашнее задание}
\author{Студент: Лохов Ростислав}
\date{\today}

\begin{document}

% Титульный лист
\begin{titlepage}
    \centering
    \vspace*{1cm}

    \Huge
    \textbf{Домашнее задание}

    \vspace{0.5cm}
    \LARGE
    По курсу: \textbf{Математический Анализ}

    \vspace{1.5cm}

    \textbf{Студент: Лохов Ростислав}

    \vfill

    \Large
    АНО ВО Центральный Университет\\
    \vspace{0.3cm}
    \today

\end{titlepage}

% Содержание
\tableofcontents
\newpage

% Основной текст
\section{Задачи для самостоятельного решения}

\subsection{Задача 1}
\textbf{Условие задачи:} Найди собственные значения и собственные векторы линейных операторов, заданных в некотором базисе матрицами:
\begin{enumerate}
    \item 
    \[
    \begin{pmatrix}
    2 & -1 & 2 \\
    5 & -3 & 3 \\
    -1 & 0 & -2
    \end{pmatrix};
    \]
    \item 
    \[
    \begin{pmatrix}
    4 & -5 & 2 \\
    5 & -7 & 3 \\
    6 & -9 & 4
    \end{pmatrix}.
    \]
\end{enumerate}

\textbf{Решение: }

а)
\[
\det(A-\lambda I) = \lambda^3+3\lambda^2+3\lambda + 1 = 0 \Longleftrightarrow (\lambda + 1)^3=0
\]

\[
(A+I)x = 0
\]

\[
x = t \begin{pmatrix}
    -1 \\
    -1 \\
    1
\end{pmatrix}
\]

б) 
\[
\det(A-\lambda I) = \lambda^2(-\lambda + 1) = 0
\]

$\lambda = 0$
\[
x = t \begin{pmatrix}
    1 \\
    2 \\
    3
\end{pmatrix}
\]

$\lambda = 1$
\[
x = t \begin{pmatrix}
    1 \\ 
    1 \\
    1 
\end{pmatrix}
\]

\subsection{Задача 2}
\textbf{Условие задачи:} Являются ли операторы в $\mathbb{R}^4$ со следующими матрицами диагонализируемыми?
\begin{enumerate}
    \item 
    \[
    A =
    \begin{pmatrix}
    1 & 3 & 5 & 0 \\
    -1 & 2 & 1 & 4 \\
    -1 & 0 & 0 & 0 \\
    0 & 0 & 0 & 0
    \end{pmatrix};
    \]
    \item 
    \[
    B =
    \begin{pmatrix}
    1 & 2 & 3 & 0 \\
    -1 & -2 & 1 & 0 \\
    -1 & 0 & 0 & 0 \\
    0 & 0 & 0 & 0
    \end{pmatrix}.
    \]
\end{enumerate}
Если нет, то объясни почему. Если да, то найди:
\begin{enumerate}
    \item базис, в котором оператор имеет диагональный вид;
    \item сам диагональный вид.
\end{enumerate}
Здесь пропуски в матрицах - это нули.

\subsection{Задача 3}
\textbf{Условие задачи:} Какие из следующих матриц подобны? В случаях подобия укажи, с помощью какой матрицы можно из одной получить другую:
\begin{enumerate}
    \item 
    \[
    \begin{pmatrix}
    1 & 0 \\
    0 & 0
    \end{pmatrix}
    \text{ и }
    \begin{pmatrix}
    0 & 0 \\
    0 & 1
    \end{pmatrix};
    \]
    \item 
    \[
    \begin{pmatrix}
    3 & 1 \\
    1 & 2
    \end{pmatrix}
    \text{ и }
    \begin{pmatrix}
    3 & -1 \\
    2 & 1
    \end{pmatrix};
    \]
    \item 
    \[
    \begin{pmatrix}
    2 & 1 \\
    1 & 1
    \end{pmatrix}
    \text{ и }
    \begin{pmatrix}
    2 & 1 \\
    0 & 1
    \end{pmatrix}.
    \]
\end{enumerate}

\subsection{Задача 4}
\textbf{Условие задачи:} Для каждого из следующих линейных операторов в $\mathbb{R}^3$ найди его матрицу в стандартном базисе, собственные значения и собственные подпространства для вычисленных собственных значений.
\begin{enumerate}
    \item Отражение относительно плоскости $xy$.
    \item Ортогональная проекция на плоскость $xz$.
    \item Поворот против часовой стрелки вокруг положительной оси $x$ на угол $90^\circ$.
\end{enumerate}

\subsection{Задача 5}
\textbf{Условие задачи:} Для диагонализируемой матрицы $A$ найти $\sin(A)$ и $\cos(A)$, если
\[
A =
\begin{pmatrix}
3 & -1 \\
2 & 0
\end{pmatrix}.
\]
Проверь, выполняется ли равенство $\sin^2(A) + \cos^2(A) = I$.

\textbf{Решение: }

\[
\det(A-\lambda I) = 0 \Leftrightarrow (\labda - 2)(\lambda - 1) = 0
\]

собственные векторы: 
\[
v_1 = \begin{pmatrix}
    1 \\
    1
\end{pmatrix}
\]

\[
v_2 = \begin{pmatrix}
    1 \\
    2
\end{pmatrix}
\]

\[
PDP^{-1} = \begin{pmatrix}
    1 & 1 \\
    1 & 2 
\end{pmatrix} \cdot \begin{pmatrix}
    2 & 0 \\
    0 & 1 
\end{pmatrix} \cdot \begin{pmatrix}
    2 & -1 \\
    -1 & 1 
\end{pmatrix}
\]


\[
\sin^2(A) + \cos^2(A) = I \Longleftrightarrow P(\sin^2(D) + \cos^2(D))P^{-1} = I = \Longleftrightarrow PIP^{-1} = I
\]


\subsection{Задача 6}
\textbf{Условие задачи:} Пусть $V = C^\infty(\mathbb{R})$ — множество бесконечно дифференцируемых функций на вещественной прямой.
\begin{enumerate}
    \item Проверь, что $V$ является векторным пространством.
    \item Проверь, что отображение $\phi: V \to V$, заданное по правилу $f \mapsto f''$ (взятие второй производной), является линейным оператором. \emph{В каком месте мы пользуемся тем, что функции бесконечное число раз дифференцируемы?}
    \item Проверь, что векторы $\sin(ax)$ и $\cos(bx)$ являются собственными для $\phi$. Найди соответствующие собственные значения.
\end{enumerate}

\textbf{Решение: }

1) сумма двух бесконечно дифф функций является бесконечно дифф

2) Умноежние функции на скаляр сохраняет дифф

3) Ассоциативность достигается по свойствам функций

4) Коммутативность таже следует из свойств сложений фукнций

5) существует функция 0(x)=0 для всех x которая является нулевым элементом

6) существование обратного элемента также достигается f + (-f) = 0

7) дистрибутивность умножения на скаляр относительно сложения векторов также достигается

8) дистрибутивность умножения на скаляр относительно сложения скаляров выполняется

9) ассоциативность умножения на скаляр выполняется

10) существование единичного элемента умножения 1f=f 

11) $\phi(f+g) = (f+g)`` = f`` + g`` = \phi(f)+\phi(g)$

12) $\phi(\alpha f) = (\alhpa f)`` = \alpha f`` = \alpha \phi(f)$

таким образом V является векторным пространством и обеспечена замкнутость оператора $\phi$ в пространстве V

13) $\sin(ax): f``(x)=-a^2(\sin(ax))=-a^2f$ - собственное значение - $-a^2$

14) $\cos(bx): f``(x)= -b^2\cos(bx) = -b^2f$ - собственное значение  - $-b^2$

таким образом простейшие триг функции являются собственными векторами оператора $\phi$

\subsection{Задача 7}
\textbf{Условие задачи:} Докажи, что в пространстве $\mathbb{R}[x]_{\leq n}$ линейный оператор $f(x) \mapsto f(ax + b)$ имеет множество собственных значений $1, a, \dots, a^n$.

\textbf{Решение: }

\[
T(x^k) = (ax+b)^k = \sum_{i=0}^{k}\binom{k}{i}a^ib^{k-i}x^i
\]

\[
[T] = \begin{pmatrix}
1 & \binom{1}{0} b & \binom{2}{0} b^2 & \dots & \binom{n}{0} b^n \\
0 & a & \binom{2}{1} a b & \dots & \binom{n}{1} a b^{n-1} \\
0 & 0 & a^2 & \dots & \binom{n}{2} a^2 b^{n-2} \\
\vdots & \vdots & \vdots & \ddots & \vdots \\
0 & 0 & 0 & \dots & a^n \\
\end{pmatrix}
\]

Поскольку матрица верхнетреугольная, то её собственныве значения равны элементам главной диагонали т.е $1, a, a^2...a^n$

\subsection{Задача 8}
\textbf{Условие задачи:} Пусть $\phi: V \to V$ — обратимый оператор. Докажи, что у операторов $\phi$ и $\phi^{-1}$ собственные векторы совпадают.

\textbf{Решение: }
\[
\varphi(\mathbf{v}) = \lambda \mathbf{v}, \quad \lambda \neq 0
\]

\[
\varphi^{-1}(\varphi(\mathbf{v})) = \varphi^{-1}(\lambda \mathbf{v}) \\
\Rightarrow \mathbf{v} = \lambda \varphi^{-1}(\mathbf{v})
\]

\[
\varphi^{-1}(\mathbf{v}) = \mu \mathbf{v}, \quad \text{где} \quad \mu = \frac{1}{\lambda}
\]

\[
\varphi(\varphi^{-1}(\mathbf{v})) = \varphi(\mu \mathbf{v}) \\
\Rightarrow \mathbf{v} = \mu \varphi(\mathbf{v})
\]

\[
\varphi(\mathbf{v}) = \lambda \mathbf{v}, \quad \text{где} \quad \lambda = \frac{1}{\mu}
\]

Таким образом существует связь между оператором и обратным к нему, притом собственные значения обратного оператора являются обратными собственными значениями исходного оператора

\subsection{Задача 9}
\textbf{Условие задачи:} В пространстве $\mathbb{R}^3$ заданы векторы:
\[
v_1 =
\begin{pmatrix}
1 \\
-1 \\
-1
\end{pmatrix}, \quad
v_2 =
\begin{pmatrix}
-1 \\
2 \\
2
\end{pmatrix}, \quad
v_3 =
\begin{pmatrix}
-1 \\
2 \\
3
\end{pmatrix}.
\]
Можно ли в этом пространстве найти линейный оператор $\phi$, такой, что
\begin{itemize}
    \item $v_1$ является собственным вектором с собственным значением $1$,
    \item $v_3$ является собственным вектором с собственным значением $2$,
    \item выполнено равенство $\phi(v_2) = v_1 + v_2$?
\end{itemize}
Если можно, то найди матрицу этого оператора в стандартном базисе. Если нельзя, то объясни почему.

k\textbf{Решение: }

\[
\phi(v_1) - v_1 \land \phi(v_2) = v_1 + v_2 \land \phi(v_3) = 2v_3
\]

\[
P = \begin{pmatrix}
    1 & -1 & -1 \\
    -1 & 2 & 2 \\
    -1 & 2 & 3 
\end{pmatrix}
\]

\[
P^{-1} = \begin{pmatrix}
    2 & 1 & 0 \\
1 & 2 & -1 \\
0 & -1 & 1 \\
\end{pmatrix}
\]

\[
\varphi = \begin{pmatrix}
    1 & 1 & 0 \\
    0 & 1 & 0 \\
    0 & 0 & 2 \\
\end{pmatrix}
\]

\[
A = P\cdot \phi \cdot P^{-1} = \begin{pmatrix}
    2 & 3 & -2 \\
    -1 & -3 & 3 \\
    -1 & -5 & 5 \\
\end{pmatrix}
\]

\subsection{Задача 10}
\textbf{Условие задачи:} Определи ЖНФ (Жорданова нормальная форма) матрицы оператора $\phi: \mathbb{C}^3 \to \mathbb{C}^3$, определённого правилом $\phi(x) = Ax$, где:
\[
A =
\begin{pmatrix}
% Вставьте матрицу здесь.
\end{pmatrix}.
\]

\textbf{Решение: }
а) найдем хар многочлен: $(\lambda^2-2)^2-0$ Найдем собственный вектор: $y=2x$ Алгебраическая кратность - 2, геометрическая - 1. Тогда Жорданова матрица: 

\[
\begin{pmatrix}
    2 & 1 \\
    0 & 2 \\
\end{pmatrix}
\]

б) хар многочлен: $(\lambda-(1+\sqrt{2}))(\lambda-(1-\sqrt{2}))$. Т.к собственные значения различны, то геом кратность 1, т.е жорданова форма будет диагональной: 

\[
\begin{pmatrix}
    1+\sqrt{2} & 0 \\
    0 & 1-\sqrt{2}
\end{pmatrix}
\]

\subsection{Задача 11}
\textbf{Условие задачи:} Возведи матрицу в степень, используя ЖНФ.
\begin{enumerate}
    \item 
    \[
    \begin{pmatrix}
    1 & 1 \\
    -1 & 3
    \end{pmatrix}^{50};
    \]
    \item 
    \[
    \begin{pmatrix}
    7 & -4 \\
    14 & -8
    \end{pmatrix}^{64}.
    \]
\end{enumerate}

\textbf{Решение: }

a) $(\lambda - 2)^2 $ алгебраическая кратность 2, собств вектор y=x, гк < ак, тогда 

\[
J = \begin{pmatrix}
    2 & 1 \\
    0 & 2 \\
\end{pmatrix}
\]


проблема в том, что вектор у нас один, попробуем дополнить чтобы был линейно независим
\[
P =
\begin{pmatrix}
    1 & 0 \\
    1 & 1 \\
\end{pmatrix}
\]

\[
A^{50} = PJ^{50}P^{-1} = 2^{49} \cdot \begin{pmatrix}
    -48 & 50 \\
    -50 & 52 \\
\end{pmatrix}
\]

б)
\[
det(b - \lambda I) = \lambda(\lambda + 1)
\]
y = 7/4 x
y = 2x
\[
\begin{pmatrix}
    4 & 7 \\
    7 & 2\\
\end{pmatrix}
\]

\[
B = PJ^{64}P^{-1} = \begin{pmatrix}
    4 & 1 \\
    7 & 2 \\
\end{pmatrix} \cdot \begin{pmatrix}
    0 & 0 \\
    0 & 1 \\
\end{pmatrix}^{64} \cdot \begin{pmatrix}
    2 & -1 \\
    -7 & 4\\
\end{pmatrix}
\]

\[
B = \begin{pmatrix}
    -7 & 4 \\
    -14 & 8 \\
\end{pmatrix}
\]



\subsection{Задача 12}
\textbf{Условие задачи:} На базовой кафедре базовых исследований базовой линейной алгебры базового университета, базированного в столице, сделали базовое открытие нового базового вида оператора. Оператор $\phi: \mathbb{R}^3 \to \mathbb{R}^3$, заданный по правилу $\phi(x) = Ax$, где
\[
A =
\begin{pmatrix}
-2 & 3 & -2 \\
0 & -2 & 0 \\
0 & 0 & -2
\end{pmatrix},
\]
называется базированным, если любой ненулевой вектор $v \in V$ может быть дополнен до жорданового базиса. Определи, является ли данный оператор $\phi$ базированным.

\textbf{Решение: }

\[
A + 2I = \begin{pmatrix}
    0 & 3 & -2 \\
    0 & 0 & 0 \\
    0 & 0 & 0 \\
\end{pmatrix}
\]

\[
v = x_1 \begin{pmatrix}
    1 \\
    0\\
    0\\
\end{pmatrix}
+
x_3 \begin{pmatrix}
    0 \\
    \frac{2}{3} \\
    1
\end{pmatrix}
\]

\[
J =
\begin{pmatrix}
    -2 & 1 & 0 \\
    0 & -2 & 0 \\
    0 & 0 & -2 
\end{pmatrix}

Оператор называется базированным если любой ненулевой вектор может быть дополнен до жорданового базиса. 
\]

\subsection{Задача 13}
\textbf{Условие задачи:} Пусть $\varphi: V \to V$ — обратимый оператор в комплексном векторном пространстве. Вырази все собственные значения оператора $(\varphi^2 - \varphi - I)$ через собственные значения $\varphi$.

\textbf{Решение: }

\[
(\phi^2-\phi-I)(v) = \phi^2(v)-\phi(v)-v=\lambda^2v-\lambda v - v = (\lambda^2 - \lambda - 1)v
\]
Обобщим

\[
\Lambda' = \{ \mu \in \mathbb{C} \mid \mu = \lambda^2 - \lambda - 1, \ \lambda \in \Lambda \}
\]

\subsection{Задача 14}
\textbf{Условие задачи:} Найди комплексные собственные значения следующих матриц $A, B \in M_{4 \times 4}(\mathbb{R})$.
\begin{enumerate}
    \item 
    \[
    A =
    \begin{pmatrix}
    0 & 1 & 0 & 0 \\
    0 & 0 & 1 & 0 \\
    0 & 0 & 0 & 1 \\
    1 & 0 & 0 & 0
    \end{pmatrix};
    \]
    \item 
    \[
    B =
    \begin{pmatrix}
    0 & 1 & 0 & 1 \\
    1 & 0 & 1 & 0 \\
    0 & 1 & 0 & 1 \\
    1 & 0 & 1 & 0
    \end{pmatrix}.
    \]
\end{enumerate}
a) 
\[
det(A-\lambda I) = \lambda^4 - 1 = 0 \Leftrightarrow \lambda_k = e^{i\frac{\pi k}{2}}, k = 0, 1, 2, 3
\]
1, -1, i, -i

б)
\[
\det(B-\lambda I) = (-\lamba)(-\lambda^3+2\lambda) = \lambda^4-4\lambda^2
\]
2, -2, 0, 0

\end{document}
