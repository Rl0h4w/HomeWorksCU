\documentclass[a4paper,12pt]{article}
\usepackage[utf8]{inputenc}
\usepackage[russian]{babel}
\usepackage{amsmath,amsfonts,amssymb}
\usepackage{graphicx}
\usepackage{geometry}
\usepackage{hyperref}
\usepackage{venndiagram}
\usepackage{tikz}
\usetikzlibrary{shapes.geometric, calc}
\usepackage{pgfplots}
\usepackage{float}

% Параметры страницы
\geometry{top=2cm,bottom=2cm,left=2.5cm,right=2.5cm}
\geometry{a4paper, margin=1in}

% Заголовок документа
\title{Домашнее задание}
\author{Студент: Лохов Ростислав}
\date{\today}

\begin{document}

% Титульный лист
\begin{titlepage}
    \centering
    \vspace*{1cm}

    {\Huge\textbf{Домашнее задание}}

    \vspace{0.5cm}
    {\LARGE По курсу: \textbf{Линейная Алгебра}}

    \vspace{1.5cm}

    \textbf{Студент: Лохов Ростислав}

    \vfill

    {\Large АНО ВО Центральный Университет\\
    \vspace{0.3cm}
    \today}

\end{titlepage}

% Содержание
\tableofcontents
\newpage

% Основной текст
\section{Векторные пространства}

\subsection{Задача 1}

\textbf{Условие задачи:}

Даны векторы
\[
\mathbf{a}_1 = \begin{pmatrix}1 \\ 2 \\ 3\end{pmatrix},\quad
\mathbf{a}_2 = \begin{pmatrix}0 \\ 1 \\ 1\end{pmatrix},\quad
\mathbf{a}_3 = \begin{pmatrix}3 \\ 8 \\ 11\end{pmatrix},\quad
\mathbf{a}_4 = \begin{pmatrix}1 \\ 1 \\ 3\end{pmatrix},\quad
\mathbf{a}_5 = \begin{pmatrix}1 \\ 3 \\ 3\end{pmatrix}.
\]

Среди этих векторов найди базис их линейной оболочки и вырази оставшиеся вектора через базисные.

\textit{Указание:} постарайся ответить на оба вопроса, используя один и тот же процесс приведения к ступенчатому виду с помощью метода Гаусса.

\vspace{0.5cm}

\subsection{Задача 2}

\textbf{Условие задачи:}

Проверь, являются ли следующие векторы линейно независимыми. Если являются, то дополни их до базиса всего пространства $\mathbb{R}^4$:

\[
\mathbf{v}_1 = \begin{pmatrix}2 \\ -3 \\ 1 \\ -1\end{pmatrix},\quad
\mathbf{v}_2 = \begin{pmatrix}1 \\ -1 \\ 3 \\ -2\end{pmatrix} \in \mathbb{R}^4.
\]

\textit{Указание:} постарайся ответить на оба вопроса, используя один и тот же процесс приведения к ступенчатому виду с помощью метода Гаусса.

\vspace{0.5cm}

\subsection{Задача 3}

\textbf{Условие задачи:}

В пространстве $\mathbb{R}^3$ заданы векторы

\[
\mathbf{f}_1 = \begin{pmatrix}1 \\ -3 \\ 1\end{pmatrix},\quad
\mathbf{f}_2 = \begin{pmatrix}2 \\ -5 \\ 1\end{pmatrix},\quad
\mathbf{f}_3 = \begin{pmatrix}-3 \\ 6 \\ -1\end{pmatrix},\quad
\mathbf{v}_1 = \begin{pmatrix}-1 \\ 3 \\ 1\end{pmatrix},\quad
\mathbf{v}_2 = \begin{pmatrix}2 \\ -1 \\ 1\end{pmatrix}.
\]

\begin{enumerate}
    \item Проверь, что $\mathbf{f}_1$, $\mathbf{f}_2$, $\mathbf{f}_3$ формируют базис в $\mathbb{R}^3$.
    \item Найди матрицу перехода от стандартного базиса $\{\mathbf{e}_1, \mathbf{e}_2, \mathbf{e}_3\}$ к базису $\{\mathbf{f}_1, \mathbf{f}_2, \mathbf{f}_3\}$ и матрицу перехода в обратном направлении.
    \item Найди координаты векторов $\mathbf{v}_1$ и $\mathbf{v}_2$ в базисе $\{\mathbf{f}_1, \mathbf{f}_2, \mathbf{f}_3\}$.
\end{enumerate}

\vspace{0.5cm}

\subsection{Задача 4}

\textbf{Условие задачи:}

Рассмотрим в качестве векторного пространства $V = \{f : \mathbb{R} \rightarrow \mathbb{R}\}$. Проверь, лежит ли заданный вектор в указанной линейной оболочке. Найди его координаты в указанном базисе.

\begin{enumerate}
    \item $\sin\left(x + \dfrac{\pi}{3}\right)$ в базисе $\{\sin(x),\ \cos(x)\}$.
    \item $\dfrac{1}{x^3 - 7x^2 + 12x}$ в базисе $\left\{\dfrac{1}{x},\ \dfrac{1}{x - 3},\ \dfrac{1}{x - 4}\right\}$.
\end{enumerate}

\vspace{0.5cm}

\subsection{Задача 5}

\textbf{Условие задачи:}

Дополни систему многочленов $\{x^5 + x^4,\ x^5 - 3x^3,\ x^5 + 2x^2,\ x^5 - x\}$ до базиса пространства $\mathbb{R}[x]_{\leq 5}$.

\vspace{0.5cm}

\subsection{Задача 6}

\textbf{Условие задачи:}

Докажи, что система матриц $E_1$, $E_2$, $E_3$, $E_4$ образует базис пространства $M_{2 \times 2}(\mathbb{R})$. Построй другой базис этого пространства так, чтобы ни одна из его матриц не была линейной комбинацией каких-либо двух матриц из $E_1$, $E_2$, $E_3$, $E_4$.

\[
E_1 = \begin{pmatrix}1 & 2 \\ 2 & 1\end{pmatrix},\quad
E_2 = \begin{pmatrix}2 & 1 \\ 2 & 1\end{pmatrix},\quad
E_3 = \begin{pmatrix}1 & 2 \\ 1 & 2\end{pmatrix},\quad
E_4 = \begin{pmatrix}2 & 2 \\ 1 & 1\end{pmatrix}.
\]

\textbf{Решение:}
\begin{enumerate}
    \item Выправим в строку, запишем в виде перемножения матрицы на вектор: 
    
    $\begin{pmatrix}1 & 2 & 1 & 2\\
                    2 & 1 & 2 & 2\\
                    2 & 2 & 1 & 1\\
                    1 & 1 & 2 & 1\\\end{pmatrix} \cdot 
                    \begin{pmatrix}\lambda_1\\\lambda_2\\\lambda_3\\\lambda_4 \end{pmatrix}=0$

                    
    $\det(M)$ = -6, значит система не имеет линейно зависимых векторов, только при тривиальных $\lambda$
\end{enumerate}

\vspace{0.5cm}5

\subsection{Задача 7}

\textbf{Условие задачи:}

В линейной оболочке $\mathrm{span}(\mathbf{v}_1, \mathbf{v}_2, \mathbf{v}_3) \subset \mathbb{R}^3$, где

\[
\mathbf{v}_1 = \begin{pmatrix}-4 \\ 7 \\ -1\end{pmatrix},\quad
\mathbf{v}_2 = \begin{pmatrix}6 \\ -7 \\ -9\end{pmatrix},\quad
\mathbf{v}_3 = \begin{pmatrix}-4 \\ 5 \\ 5\end{pmatrix},
\]

выдели базис и дополни его до базиса всего пространства $\mathbb{R}^3$.

\textit{Указание:} используй один процесс приведения к ступенчатому виду методом Гаусса.


\textbf{Решение: }
$\begin{pmatrix}-4 & 7 & -1 \\
                6 & -7 & -9 \\
                -4 & 5 & 5 \\\end{pmatrix}
                =
\begin{pmatrix}1 & 0 & -5 \\
                0 & 1 & -3 \\
                0 & 0 & 0 \\\end{pmatrix}$
$v_4= \begin{pmatrix}
     1 \\ 1 \\ 0 \\
\end{pmatrix}    $
\vspace{0.5cm}

\subsection{Задача 8}

\textbf{Условие задачи:}

Пусть $\mathbb{R}[x]_{\leq n}$ — множество всех многочленов с вещественными коэффициентами степени не больше $n$. Покажи, что системы $\{1,\ x,\ x^2,\ \ldots,\ x^n\}$ и $\{1,\ x - a,\ (x - a)^2,\ \ldots,\ (x - a)^n\}$, где $a \in \mathbb{R}$, являются базисами в $\mathbb{R}[x]_{\leq n}$. Найди матрицы перехода от первого базиса ко второму и от второго к первому.

\textbf{Решение: }
$P(x)=ax^n+bx^{n-1}+...+1 \in \mathbb{R}[x]_{\leq n}$, всего n+1 размерность. Первая система - стандартный базис, т.к данный базис порождает все возможные многочлены n+1 размерности. Вторая система также является стандартным базисом, т.к тоже порождает всё пространство из многочленов.  Матрица \( M_{\mathcal{B} \to \mathcal{C}} \):
  \[
  \left[ M_{\mathcal{B} \to \mathcal{C}} \right]_{k,j} = \begin{cases}
  \binom{j}{k} (-a)^{j - k} & \text{если } k \leq j, \\
  0 
  \end{cases}
  \]
  где \( k, j = 0, 1, \dots, n \).

Матрица \( M_{\mathcal{C} \to \mathcal{B}} \):
  \[
  \left[ M_{\mathcal{C} \to \mathcal{B}} \right]_{k,j} = \begin{cases}
  \binom{j}{k} a^{j - k} & \text{если } k \leq j, \\
  0 
  \end{cases}
  \]
  где \( k, j = 0, 1, \dots, n \).
\vspace{0.5cm}

\subsection{Задача 9}

\textbf{Условие задачи:}

Пусть $V$ — векторное пространство, $e_1,\ldots,e_n \in V$ — линейно независимая система. Найди все возможные базисы линейной оболочки среди следующих векторов:

\[
e_1 - e_2,\quad e_2 - e_3,\quad \ldots,\quad e_{n-1} - e_n,\quad e_n - e_1.
\]
\textbf{Решение:}
сумма всех векторов = 0, т.е линейно зависима, значит размерность будет n-1. Проверим это. Предположим мы исключаем вектор $v_k$, тогда $v_1+v_2+...+v_{k-1}+v{k+1}+...+v_n=-v_k$, отсюда следует, что ни одна из линейных комбинаций не будет равна 0 кроме тривиальной комбинации.

\vspace{0.5cm}

\subsection{Задача 10}

\textbf{Условие задачи:}

Пусть $V$ — векторное пространство. Покажи, что

\begin{enumerate}
    \item Для любого числа $\lambda \in \mathbb{R}$ выполнено $\lambda \cdot 0 = 0 \in V$.
    \item Если для $\lambda \in \mathbb{R}$ и $v \in V$ выполнено $\lambda v = 0$, то либо $\lambda = 0$, либо $v = 0$.
    \item Для любого вектора $v \in V$ выполнено $-v = (-1) \cdot v$.
\end{enumerate}

\textbf{Решение: }
\begin{enumerate}

\item Воспользуемся распределительностью умножения скаляра относительно сложения: $\lambda (v_1+v_2)=\lambda v_1 + \lambda v_2; v_1, v_2 \in V, \lambda \in \mathbb{R}$, пусть $v_1, v_2 = 0, 0$, тогда $\lambda 0 = \lambda 0 + \labda 0$, отсюда следует $0 = \lambda 0$

\item При $\lambda = 0$ равенство выполняется автоматически. При $\lambda \ne 0$ существует опрабный скаляр, $\lambda^{-1}(\lambda v)=\lambda^{-1} 0$, поскольку $\lambda^{-1}\lambda = 1$ Получаем $1v=v=0$ чтд

\item по условию существования $v+(-v)=0$, рассмотрим скаляр -1 и v, по распределительности умножения скаляра относительно сложения векторов получим: $(−1)⋅v+1⋅v=(−1+1)⋅v=0⋅v=0$ и $(−1)⋅v+1⋅v=(−1)⋅v+v$, таким образом $(−1)⋅v+v=0$ отсюда следует, что при сравнении с обратной, получаем $−v=(−1)⋅v$
\end{enumerate}
\vspace{0.5cm}

\subsection{Задача 11}

\textbf{Условие задачи:}

Введём на множестве положительных вещественных чисел $V = \mathbb{R}_{>0}$ следующие операции:

\begin{itemize}
    \item «Сложение» $v \oplus u = vu$, где $v, u \in V$ (справа обычное умножение чисел).
    \item «Умножение на скаляр» $\lambda \ast v = v^\lambda$, где $\lambda \in \mathbb{R}$, $v \in V$.
\end{itemize}

Проверь, что множество $\mathbb{R}_{>0}$ с операциями $\oplus$ и $\ast$ является векторным пространством над $\mathbb{R}$. Определи, какая у него размерность.

\textbf{Решение: }
Чтобы првоерить является ли множество векторным пространством над \mathbb{R} проверим аксиомы
\begin{enumerate}
    \item Замкнутость отсносительно сложения выполняется т.к произведение положительных чисел положительно
    \item Коммутативность сложения также выполняется $v⊕u=v⋅u=u⋅v=u⊕v$
    \item Ассоциативность сложения также выполняется $(v⊕u)⊕w=(v⋅u)⋅w=v⋅(u⋅w)=v⊕(u⊕w)$
    \item Наличие нейтрального элемента относительно сложения также выполняется: $v⋅e=v⟹e=1$
    \item Наличие обратного элемента относительно сложения также выполняется $v⋅w=1\rightarroww= v1\cdot $
    \item Замкнутость отсносительно умножения на скаляр также выполняется т.к вектор >0 и для любого $\lambda$ это выполняется
    \item Дистрибутивность относительно сложения векторов также выполняется $\lambda * (v⊕u)=\lambda*(v\cdot u) = (v\cdot u)^\lambda = v^\lambda \cdot u^\lambda = (\lambda * v)⊕(\labmda * u)$
    \item Дистрибутивность относительно сложения скаляров также выполняется: $\forall \lambda, \mu \in \mathbb{R} & v \in V$: $(\lambda + \mu)*v = v^{\lambda + \mu}=v^\lambda \cdot v^\mu = (\lambda * v)⊕(\mu * v)$
    \item Ассоциативность умножения скаляров также выполняется: $\lambda * (\mu*v)=\lambda*(v^\mu)=(v^\mu)^\lambda=v^{\lambda \mu}=(\lambda \mu)*v$
    \item Нейтральный элемент относительно умножения на скаляр также существует: $1*v=v^1=v$
    
    Все аксиомы векторного пространства выполняются с заданными операциями, следовательно (V,⊕,∗) является векторным пространством над $\mathbb{R}$
\end{enumerate}

\vspace{0.5cm}

\subsection{Задача 12}

\textbf{Условие задачи:}
Рассмотрим в качестве векторного пространства $V = \{f : \mathbb{R} \rightarrow \mathbb{R}\}$ с обычными операциями сложения и умножения на число. Роль $0$ в таком пространстве играет функция, которая во всех точках равна $0$. Проверь на линейную независимость следующие функции:

\begin{enumerate}
    \item $1,\ \sin(x),\ \ldots,\ \sin^n(x)$.
    \item $\sin(x),\ \cos(x),\ \sin(2x),\ \cos(2x),\ \ldots,\ \sin(nx),\ \cos(nx)$.
\end{enumerate}
\textbf{Решение: }
$\sin^n(x) = \frac{1}{2^n} \sum_{k=0}^{n} \binom{n}{k} (-1)^k \cos\left((n - 2k)x\right)$
т.к $\cos(kx)$ для различных k являются линейно независимыми, то 
\[
\begin{cases}
a_0 + \frac{a_2}{2^2} \binom{2}{0} (-1)^0 + \ldots + \frac{a_n}{2^n} \binom{n}{k} (-1)^k = 0 \\
a_1 = 0 \\
\frac{a_2}{2^2} \binom{2}{1} (-1)^1 + \ldots = 0 \\
\vdots \\
\frac{a_n}{2^n} \binom{n}{k} (-1)^k = 0 \\
\end{cases}
\]

2. т.к $\sin(kx)$ и $\cos(kx)$ в зависимости от разных k линейно независимы и линейная комбинация таких функций всегда независима(по свойству тригонометрических полиномов)
\vspace{0.5cm}

\subsection{Задача 13}

\textbf{Условие задачи:}

Рассмотрим в качестве векторного пространства $V = \{f : \mathbb{R} \rightarrow \mathbb{R}\}$ с обычными операциями сложения и умножения на число. Построй континуальное семейство функций $\{f_\lambda\ |\ \lambda \in \mathbb{R}\}$ такое, что любой конечный набор $f_{\lambda_1},\ldots,f_{\lambda_n}$ (все $\lambda_i$ различны) будет линейно независимым.

\textbf{Решение: }
$f_\lambda=\sin(\lambda x)$ пользуясь теоремой о том, что функции с различными частотами линейно независимы, кроме того, они ортогональны, можно сделать выводы о том, что они линейно независ имы в пространстве V.
\vspace{0.5cm}

\subsection{Задача 14}

\textbf{Условие задачи:}

Пусть $V$ — векторное пространство над $\mathbb{R}$ и $U_1,\ldots,U_k \subset V$ — подпространства такие, что $V = \bigcup\limits_{i=1}^k U_i$. Покажи, что найдётся $j$ такое, что $V = U_j$.

\vspace{0.5cm}
Не понял подсказок семинаристов, как избежать "Покрытия". Поэтому постараюсь доказать как чувствую. Пусть V имеет  размерность n. Любое подпространство $U_i$ либо совпадает с V(размерность n), либо имеет меньшую размерность. "Покрытие": Если все $U_i$ имеют размерность меньше n, то каждое из них не покрывает некие участки в V. Обьединение конечного числа таких подпространств не может охватить все участки в V. т.к всегда найдутся такие участки не покрытые $U_i$. Отсюда следует что для того, чтобы обьединение подпространств покрывало всё пространсто V должно существовать хотя бы одно подпространство $U_j$ с размерностью n, т.е $U_j=V$
\end{document}
