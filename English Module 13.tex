\documentclass[a4paper,12pt]{article}

% Кодировка и язык
\usepackage[utf8]{inputenc}
\usepackage[russian]{babel}

% Математические пакеты
\usepackage{amsmath,amsfonts,amssymb}

% Графика
\usepackage{graphicx}
\usepackage{tikz}
\usetikzlibrary{shapes.geometric, calc}
\usepackage{pgfplots}

% Геометрия страницы
\usepackage{geometry}
\geometry{top=2cm, bottom=2cm, left=2.5cm, right=2.5cm}

% Гиперссылки
\usepackage{hyperref}

% Плавающие объекты
\usepackage{float}

% Дополнительные пакеты
\usepackage{venndiagram}

% Настройки заголовка
\title{Домашнее задание}
\author{Студент: \textbf{Ростислав Лохов}}
\date{\today}

\begin{document}

% Титульный лист
\begin{titlepage}
    \centering
    \vspace*{1cm}

    \Huge
    \textbf{Homework}

    \vspace{0.5cm}
    \LARGE
    По курсу: \textbf{English}

    \vspace{1.5cm}

    \textbf{Студент: Rostislav Lokhov}

    \vfill

    \Large
    ANO HE Central University\\
    \vspace{0.3cm}
    \today

\end{titlepage}

% Содержание
\tableofcontents
\newpage

% Основной текст
\section{}

\subsection{Задача 1}
\textbf{Условие задачи:}

% Условие задачи
\begin{itemize}
    \item[a)] Complete the text with the past simple, past continuous, or past perfect simple form of the verbs in brackets. You need to use passive voice twice.
\end{itemize}

\textbf{Решение:}

A few months ago, Mike \textbf{was working} on his computer at home when he \textbf{noticed} something strange. His email account \textbf{was being logged in} to from a different device that he didn’t recognize. He quickly \textbf{changed} his password, but a few days later, he \textbf{realised} that his social media accounts \textbf{were hacked} as well. Mike \textbf{felt} really frustrated and \textbf{decided} to investigate. He \textbf{installed} a tracking program on his computer to see if anyone \textbf{accessed} his accounts without his permission. In a few days, when he \textbf{checked} the program, he \textbf{discovered} that someone had been using his passwords the whole time. It turned out that Mike’s passwords \textbf{had been stolen} months before, while he \textbf{was using} public Wi-Fi at a café. He \textbf{didn't realize} it until now.

\subsection{Задача 2}
\textbf{Условие задачи:}

% Условие задачи
\begin{itemize}
    \item[a)] Watch a video with some terms explained by non-technical people. Do the respondents know what the words mean? Put a tick (\(\checkmark\)) or a cross (\(\times\)) in the table.
\end{itemize}

\textbf{Решение:}


\begin{tabular}{|c|c|c|c|}
\hline
IT Concept & Respondent 1 & Respondent 2 & Respondent 3 \\ \hline
Gif       & \(\checkmark\) & \(\times\)    & \(\checkmark\) \\ \hline
Wiki      & \(\times\)    & \(\checkmark\) & \(\times\)    \\ \hline
Pixel     & \(\checkmark\) & \(\times\)    & \(\checkmark\) \\ \hline
Byte      & \(\times\)    & \(\checkmark\) & \(\checkmark\) \\ \hline
Cookie    & \(\times\)    & \(\times\)    & \(\checkmark\) \\ \hline
404 Error & \(\times\)    & \(\checkmark\) & \(\times\)    \\ \hline
Phishing  & \(\times\)    & \(\checkmark\) & \(\checkmark\) \\ \hline
RSS       & \(\checkmark\) & \(\times\)    & \(\times\)    \\ \hline
\end{tabular}

\textbf{Пояснения:}

1. \textbf{Gif:} Respondents misunderstood it as a video instead of an image format.
2. \textbf{Wiki:} Some respondents incorrectly thought it referred to a specific website.
3. \textbf{Pixel:} Misunderstood as a camera feature rather than the smallest unit of an image.
4. \textbf{Byte:} Confusion with other data units.
5. \textbf{Cookie:} Misinterpreted as something unrelated to web browsing.
6. \textbf{404 Error:} Mistaken as a general error, not "page not found."
7. \textbf{Phishing:} Misunderstood as legitimate account security.
8. \textbf{RSS:} Described incorrectly as software rather than a web feed format.

\end{document}