\documentclass[a4paper,12pt]{article}

% Кодировка и язык
\usepackage[utf8]{inputenc}
\usepackage[russian]{babel}

% Математические пакеты
\usepackage{amsmath,amsfonts,amssymb}

% Графика
\usepackage{graphicx}
\usepackage{tikz}
\usetikzlibrary{shapes.geometric, calc}
\usepackage{pgfplots}

% Геометрия страницы
\usepackage{geometry}
\geometry{top=2cm, bottom=2cm, left=2.5cm, right=2.5cm}

% Гиперссылки
\usepackage{hyperref}

% Плавающие объекты
\usepackage{float}

% Дополнительные пакеты
\usepackage{venndiagram}

% Настройки заголовка
\title{Домашнее задание}
\author{Студент: \textbf{Ваше Имя Фамилия}}
\date{\today}

\begin{document}

% Титульный лист
\begin{titlepage}
    \centering
    \vspace*{1cm}

    \Huge
    \textbf{Homework}

    \vspace{0.5cm}
    \LARGE
    По курсу: \textbf{English}

    \vspace{1.5cm}

    \textbf{Студент: Rostislav Lokhov}

    \vfill

    \Large
    ANO HE Central university\\
    \vspace{0.3cm}
    \today

\end{titlepage}

% Содержание
\tableofcontents
\newpage

% Основной текст
\section{Solution}

\subsection{Task 1}
\textbf{Problem Stste:}

\begin{itemize}
    \item[a)] Прочитать блог-пост и написать ответ (10 баллов).
    \item[б)] Выбрать правильные варианты в предложениях (5 баллов).
\end{itemize}

\textbf{Решение:}

\textbf{a) Ответ на блог-пост:}

\textit{Hi there,}  

\textit{I completely understand how frustrating it can be to lose motivation and feel overwhelmed, especially when juggling demanding modules. It’s great that you’ve identified the challenges and reached out for help—this is the first step toward improvement.}  

\textit{Here are a few strategies that might work for you:}  
\begin{itemize}
    \item \textit{Break tasks into smaller chunks: Start with just 15 minutes of focused study, then gradually increase the time.}
    \item \textit{Set specific goals: For example, “Complete two pages of notes in the next hour.” Small wins can boost motivation.}
    \item \textit{Reward yourself: Promise yourself a small treat after finishing a task—it could be a short walk or a favorite snack.}
    \item \textit{Try digital detox strategies: Keep your phone in another room or use the "focus mode" feature.}
    \item \textit{Join a study group: Collaborating with others might make studying less monotonous.}
\end{itemize}

\textit{Remember, it’s okay to have off days. Stay consistent, and you’ll see progress. You’ve got this!}  

\textbf{b) Выбор правильных вариантов:}

\begin{enumerate}
    \item All the clips \textbf{have been uploaded} to the website.
    \item I can’t use my laptop because \textbf{it’s being repaired} right now.
    \item The video \textbf{was filmed} in Mexico City.
    \item Were those documents \textbf{sent} by email or by post?
    \item Not many houses \textbf{are being built} at the moment.
    \item Are those toys \textbf{made} by hand?
    \item We \textbf{weren’t employed} by the government until 2018.
    \item That piano \textbf{hasn’t been played} for years.
\end{enumerate}

\vspace{1cm}

\subsection{Задача 2}
\textbf{Условие задачи:}

Заполнить пропуски правильными формами активного или пассивного залога.

\textbf{Решение:}

\begin{enumerate}
    \item Many students say they \textbf{feel} overwhelmed at the start of the semester, but this issue can \textbf{be managed} by developing a clear study schedule.
    \item The email with project deadlines \textbf{has not been sent} yet, but the task instructions \textbf{have already been uploaded} on the course website.
    \item In most courses, students \textbf{are encouraged} to work in groups, but individual tasks \textbf{are assigned} at the end of each week.
    \item Last semester, students \textbf{prepared} questions in advance for the presentation on data encryption which \textbf{was given} by an expert from the IT industry.
    \item The concept of data flow \textbf{was taught} in the previous lesson, and it \textbf{will be reviewed} again during the final exam.
    \item We \textbf{haven’t submitted} our group project yet, but the research data \textbf{has already been collected}.
    \item At the university, new methods of data communication \textbf{are developed} every year, and students \textbf{test} them as part of their coursework.
    \item I \textbf{started} working on my term paper last week, and I \textbf{found} most of the sources online.
\end{enumerate}

\vspace{1cm}

\end{document}
