\documentclass[a4paper,12pt]{article}

% Кодировка и язык
\usepackage[utf8]{inputenc}
\usepackage[russian]{babel}

% Математические пакеты
\usepackage{amsmath,amsfonts,amssymb}

% Графика
\usepackage{graphicx}
\usepackage{tikz}
\usetikzlibrary{shapes.geometric, calc}
\usepackage{pgfplots}

% Геометрия страницы
\usepackage{geometry}
\geometry{top=2cm, bottom=2cm, left=2.5cm, right=2.5cm}

% Гиперссылки
\usepackage{hyperref}

% Плавающие объекты
\usepackage{float}

% Дополнительные пакеты
\usepackage{venndiagram}

% Настройки заголовка
\title{Домашнее задание}
\author{Студент: \textbf{Ваше Имя Фамилия}}
\date{\today}

\begin{document}

% Титульный лист
\begin{titlepage}
    \centering
    \vspace*{1cm}

    \Huge
    \textbf{Домашнее задание}

    \vspace{0.5cm}
    \LARGE
    По курсу: \textbf{Математический Анализ}

    \vspace{1.5cm}

    \textbf{Студент: Ростислав Лохов}

    \vfill

    \Large
    АНО ВО Центральный университет\\
    \vspace{0.3cm}
    \today

\end{titlepage}

% Содержание
\tableofcontents
\newpage

% Основной текст
\section{Дифференцирование}

\subsection{Задача 5}
\textbf{Решение: }
\[
\frac{dy}{dx} = 2\sqrt{4-x^2}
\]

\[
2\sqrt{3} \land 2
\]

\vspace{1cm}

\subsection{Задача 6:}
\textbf{Решение: }
а)
\[
dy = \frac{3u^2 \cdot du \cdot v - u^3 \cdot dv}{v^2} = \frac{3u^2 v \, du - u^3 \, dv}{v^2}
\]

б)
\[
dy = \frac{1}{\cot(\frac{v^2}{u})}d(cot\frac{v^2}{u}) = -\frac{1}{\cot(\frac{v^2}{u})}\frac{1}{\sin^2(\frac{v^2}{u})}d(\frac{v^2}{u}) = -\frac{1}{\cot(\frac{v^2}{u})}\frac{1}{\sin^2(\frac{v^2}{u})} \frac{2vudv-v^2du}{u^2} = \frac{2vudv-v^2du}{u^2\sin(\frac{v^2}{u})\cos(\frac{v^2}{u})}
\]

в)
\[
dy = -\sin(v^u)(v^n\ln(v)du + uv^{n-1}dv)
\]


\subsection{Задача 8: }
\textbf{Решение: }
Асимптоты задаются уравнениями $y=\pm \frac{bx}{a}$

уравнение касательной $\frac{xx_0}{a^2}-\frac{yy_0}{b^2}=1$

Подставим асимптоту в уравнение касательной, пусть это будет точкой А. Подставим Другую асимптоту в уравнение касательной, пусть будет точкой B. Тогда: 

\[
A = \left(\frac{a^2b}{x_0b-ay_0}, \frac{ab^2}{x_0b-ay_0} \right) \land B = \left( \frac{a^2 b}{x_0 b + a y_0}, -\frac{a b^2}{x_0 b + a y_0} \right) \land O(0, 0)
\]

\[
S = 0.5|x_Ay_B - x_By_A| = 0.5| \frac{a^2 b}{x_0 b - a y_0} \cdot \left( -\frac{a b^2}{x_0 b + a y_0} \right) + \frac{a^2 b}{x_0 b + a y_0} \cdot \frac{a b^2}{x_0 b - a y_0} |= 0.5|-\frac{a^3 b^3}{(x_0 b - a y_0)(x_0 b + a y_0)} - \frac{a^3 b^3}{(x_0 b - a y_0)(x_0 b + a y_0)}|
\]

\[
x_0^2 = - \frac{a^2y_0^2}{b^2}=a^2 
\]

\[
S = 0.5|-2ab|=ab
\]

\subsection{Задача 12: }
\textbf{Решение: }

\[
\frac{dy}{dx} = \frac{2|x|^{-1/3}}{3} \cdot sign(x)+\frac{2}{3}|y|^{-1/3}\cdot sign(y)\frac{dy}{dx} = 0 \Rightarrow \frac{dy}{dx} = -(\frac{y}{x})^{\frac{1}{3}}
\]

Пусть $x=acos^3(\theta), y=asin^3(\theta)$

\[
\frac{dy}{dx} = -\tan(\theta)
\]

\[
y - y_0 = m(x-x_0) \Rightarrow y-a\sin^3\theta = -\tan(\theta)(x-a\cos^3(\theta)) \Rightarrow x\sin(\theta) + y\cos(\theta) = a\sin(\theta)\cos(\theta)
\]

При x=0 y=0 точки пересечения будут $A(acos(\theta), 0) \land B(0, a\sin(\theta))$

$L = \sqrt{(a\cos(\theta) - 0)^2+(0-a\sin(\theta))^2} = a$

\subsection{Задача 14: }
\textbf{Решение: }
$y=2a(x-a)+a^2 = 2ax-a^2$

$a_1+a_2 = 2x_0 \land a_1a_2=y_0, \land k = 2a \Rightarrow k_1k_2 = 4y_0 \Rightarrow y_0 = -0.25$
Геометрическое место точек образует прямую y=-0.25

\subsection{Задача 16: }
\textbf{Решение: }
\[
\begin{cases}
    x = t^3 + 2t^2 + t\\
    y = -2+3t-t^3\\
    1 < t < +\infty\\
\end{cases}
\]
\[
\frac{dx}{dt} = 3t^2+4t+1 \land
\frac{dy}{dt} = 3 - 3t^2 \Rightarrow \frac{dx}{dy} = \frac{1+4t+3t^2}{3-3t^2} = \frac{3t+1}{3-3t}
\]
\end{document}