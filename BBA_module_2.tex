\documentclass[a4paper,12pt]{article}
\usepackage[T2A]{fontenc}   % Поддержка кириллицы
\usepackage[utf8]{inputenc} % Кодировка UTF-8
\usepackage[russian]{babel} % Языковые настройки
\usepackage{amsmath}        % Математические формулы
\usepackage{amssymb}        % Дополнительные математические символы
\usepackage{graphicx}       % Для вставки изображений
\usepackage{geometry}       % Настройка полей страницы
\usepackage{hyperref}       % Гиперссылки в документе
\usepackage{booktabs}       % Улучшенные таблицы
\usepackage{array}          % Дополнительные возможности для таблиц

\geometry{
  left=2.5cm,
  right=2cm,
  top=2cm,
  bottom=2cm
}

\begin{document}

% Титульный лист
\begin{titlepage}
  \centering
  \vspace*{\fill}
  {\LARGE\bfseries Улучшение клиентского пути и повышение NPS в компании Мяу Гав\par}
  \vspace{1cm}
  {\Large Анализ и предложения по оптимизации\par}
  \vspace{2cm}
  {\large Автор: Иван Иванов\par}
  {\large Дата: \today\par}
  \vspace*{\fill}
\end{titlepage}

% Оглавление
\tableofcontents
\newpage

\section*{Введение}
\addcontentsline{toc}{section}{Введение}

Цель данного документа — проанализировать текущий клиентский путь (CJM) и показатели NPS компании Мяу Гав, предложить инициативы по их улучшению, а также оценить влияние этих инициатив на юнит-экономику компании.

\section{Анализ текущего клиентского пути и NPS}

На основе предоставленных данных проведём анализ каждого этапа клиентского пути и соответствующих оценок NPS.

\subsection{Таблица анализа CJM}

\begin{table}[h!]
\centering
\begin{tabular}{p{6cm} p{2.5cm} p{6cm}}
\toprule
\textbf{Этап} & \textbf{Оценка NPS} & \textbf{Комментарий} \\
\midrule
1. Просматривает ассортимент & Отлично & Яркий и уместный дизайн сайта, соответствующий потребностям ЦА \\
2. Использует фильтры для поиска товара & Хорошо & Навигация не лучшая, но фильтры работают исправно \\
3. Добавляет товары в корзину & Средне & Удовлетворительно \\
4. Вводит адрес и детали доставки & Средне & Удовлетворительно \\
5. Выбирает вариант оплаты & Плохо & При повторном заказе нужно вводить данные заново \\
6. Клиенту звонят из поддержки & Хорошо & Голос приятный, доброжелательный, скрипт отчитан \\
7. Происходит урегулирование цены & Средне & Политика ценообразования сработала не лучшим образом \\
8. Ожидает доставку (2 дня) & Средне & Удовлетворительно \\
9. Получает заказ & Средне & Питомцу нравится корм, но для повторной покупки необходимо пройти весь путь заново \\
\bottomrule
\end{tabular}
\caption{Анализ этапов клиентского пути}
\end{table}

\subsection{Выводы из анализа}

\textbf{Сильные стороны:}

\begin{itemize}
  \item Отличный дизайн сайта и презентация ассортимента.
  \item Хорошая работа службы поддержки.
\end{itemize}

\textbf{Области для улучшения:}

\begin{itemize}
  \item Необходима оптимизация навигации и поиска товаров.
  \item Упрощение процесса оформления заказа, особенно для повторных покупок.
  \item Улучшение процесса оплаты и сохранения данных клиентов.
  \item Прозрачность политики ценообразования.
\end{itemize}

\section{Предложенные инициативы по улучшению CJM}

На основе проведённого анализа предлагаем следующие 10 инициатив:

\begin{enumerate}
  \item \textbf{Улучшение навигации и системы фильтров на сайте.}
  \item \textbf{Оптимизация процесса добавления товаров в корзину.}
  \item \textbf{Автоматическое сохранение адреса и деталей доставки.}
  \item \textbf{Улучшение процесса выбора варианта оплаты.}
  \item \textbf{Внедрение функции <<Повторить заказ>>.}
  \item \textbf{Оптимизация политики ценообразования.}
  \item \textbf{Сокращение времени доставки.}
  \item \textbf{Персонализация рекомендаций.}
  \item \textbf{Внедрение программы лояльности.}
  \item \textbf{Упрощение процесса оформления заказа.}
\end{enumerate}

\section{Влияние инициатив на NPS}

Каждая из предложенных инициатив призвана улучшить определённые аспекты клиентского опыта, что в совокупности должно привести к повышению NPS.

\begin{itemize}
  \item \textbf{Инициатива 1:} Повысит удовлетворённость от поиска товаров.
  \item \textbf{Инициатива 2:} Снизит время на оформление заказа.
  \item \textbf{Инициатива 3:} Повысит удобство для повторных покупателей.
  \item \textbf{Инициатива 4:} Снизит количество брошенных корзин.
  \item \textbf{Инициатива 5:} Увеличит лояльность постоянных клиентов.
  \item \textbf{Инициатива 6:} Уменьшит недовольство ценовой политикой.
  \item \textbf{Инициатива 7:} Повысит удовлетворённость от скорости доставки.
  \item \textbf{Инициатива 8:} Увеличит средний чек и удовлетворённость.
  \item \textbf{Инициатива 9:} Стимулирует повторные покупки.
  \item \textbf{Инициатива 10:} Упростит процесс покупки, повысив общую удовлетворённость.
\end{itemize}

\section{Выбор топ-3 инициатив и их обоснование}

\subsection{Инициатива 1: Внедрение функции <<Повторить заказ>>}

\textbf{Обоснование:} Значительно упрощает процесс для постоянных клиентов, повышая их лояльность и удовлетворённость.

\subsection{Инициатива 2: Автоматическое сохранение адреса и деталей доставки}

\textbf{Обоснование:} Устраняет необходимость повторного ввода данных, что улучшает пользовательский опыт и снижает время оформления заказа.

\subsection{Инициатива 3: Улучшение процесса выбора варианта оплаты}

\textbf{Обоснование:} Снижает барьеры на пути к завершению покупки, уменьшая количество отказов на последнем этапе.

\section{Расчёт юнит-экономики для выбранных инициатив}

\subsection{Исходные данные}

\begin{itemize}
  \item \textbf{Стоимость за клик (CPC):} 50 ₽
  \item \textbf{Кликабельность (CTR):} 2\%
  \item \textbf{Общее количество показов рекламы:} 5000
  \item \textbf{Средняя выручка с клиента (ARPC):} 9000 ₽
  \item \textbf{Валовая маржа:} 35\%
  \item \textbf{Отток клиентов (Churn Rate):} 60\%
  \item \textbf{Конверсия пользователя в покупателя:} 60\%
\end{itemize}

\subsection{Текущая юнит-экономика}

\textbf{1. Количество кликов:}

\[
\text{Клики} = 5000 \times 0{,}02 = 100 \text{ кликов}
\]

\textbf{2. Количество покупателей:}

\[
\text{Покупатели} = 100 \times 0{,}6 = 60 \text{ покупателей}
\]

\textbf{3. Затраты на привлечение клиентов (CAC):}

\[
\text{Общие затраты} = 100 \times 50 = 5000 \, \text{₽}
\]

\[
\text{CAC} = \frac{5000}{60} \approx 83{,}33 \, \text{₽}
\]

\textbf{4. Средняя продолжительность жизни клиента:}

\[
\text{Средняя продолжительность жизни} = \frac{1}{0{,}6} \approx 1{,}67 \text{ периода}
\]

\textbf{5. Lifetime Value (LTV):}

\[
\text{LTV} = 9000 \times 0{,}35 \times 1{,}67 \approx 5250 \, \text{₽}
\]

\textbf{6. Юнит-экономика:}

\[
\text{Unit\_Economics} = 5250 - 83{,}33 \approx 5166{,}67 \, \text{₽}
\]

\subsection{Расчёты для выбранных инициатив}

\subsubsection{Инициатива 1: Внедрение функции <<Повторить заказ>>}

\textbf{Изменения:} Churn Rate снизится на 3 п.п. (до 57\%)

\textbf{1. Новая средняя продолжительность жизни:}

\[
\text{Средняя продолжительность жизни}_{\text{новая}} = \frac{1}{0{,}57} \approx 1{,}754 \text{ периода}
\]

\textbf{2. Новый LTV:}

\[
\text{LTV}_{\text{новый}} = 9000 \times 0{,}35 \times 1{,}754 \approx 5527{,}5 \, \text{₽}
\]

\textbf{3. Новая юнит-экономика:}

\[
\text{Unit\_Economics}_{\text{новая}} = 5527{,}5 - 83{,}33 \approx 5444{,}17 \, \text{₽}
\]

\textbf{4. Изменение юнит-экономики:}

\[
\Delta \text{Unit\_Economics} = 5444{,}17 - 5166{,}67 = 277{,}5 \, \text{₽}
\]

\subsubsection{Инициатива 2: Автоматическое сохранение адреса и деталей доставки}

\textbf{Изменения:} Churn Rate снизится на 2 п.п. (до 58\%)

\textbf{1. Новая средняя продолжительность жизни:}

\[
\text{Средняя продолжительность жизни}_{\text{новая}} = \frac{1}{0{,}58} \approx 1{,}724 \text{ периода}
\]

\textbf{2. Новый LTV:}

\[
\text{LTV}_{\text{новый}} = 9000 \times 0{,}35 \times 1{,}724 \approx 5427 \, \text{₽}
\]

\textbf{3. Новая юнит-экономика:}

\[
\text{Unit\_Economics}_{\text{новая}} = 5427 - 83{,}33 \approx 5343{,}67 \, \text{₽}
\]

\textbf{4. Изменение юнит-экономики:}

\[
\Delta \text{Unit\_Economics} = 5343{,}67 - 5166{,}67 = 177 \, \text{₽}
\]

\subsubsection{Инициатива 3: Улучшение процесса выбора варианта оплаты}

\textbf{Изменения:} Конверсия увеличится на 1 п.п. (до 61\%)

\textbf{1. Новое количество покупателей:}

\[
\text{Покупатели}_{\text{новые}} = 100 \times 0{,}61 = 61 \text{ покупатель}
\]

\textbf{2. Новый CAC:}

\[
\text{CAC}_{\text{новый}} = \frac{5000}{61} \approx 81{,}97 \, \text{₽}
\]

\textbf{3. Юнит-экономика остаётся прежней:}

\[
\text{Unit\_Economics}_{\text{новая}} = 5250 - 81{,}97 \approx 5168{,}03 \, \text{₽}
\]

\textbf{4. Изменение юнит-экономики:}

\[
\Delta \text{Unit\_Economics} = 5168{,}03 - 5166{,}67 = 1{,}36 \, \text{₽}
\]

\subsection{Выводы по расчётам}

Наиболее значительное влияние на юнит-экономику оказывает \textbf{Инициатива 1} с увеличением юнит-экономики на 277{,}5 ₽.

\section{Рекомендации}

Рекомендуется в первую очередь внедрить \textbf{Инициативу 1} — функцию <<Повторить заказ>>, так как она имеет наибольший потенциал для повышения удовлетворённости клиентов и улучшения финансовых показателей компании.

\section*{Заключение}
\addcontentsline{toc}{section}{Заключение}

Внедрение предложенных инициатив позволит существенно улучшить клиентский путь, повысить NPS и улучшить юнит-экономику компании. Это приведёт к увеличению лояльности клиентов, снижению оттока и росту прибыли.

\section*{Список литературы}
\addcontentsline{toc}{section}{Список литературы}

\begin{enumerate}
  \item Собственные аналитические данные компании Мяу Гав.
  \item Методические материалы по улучшению клиентского пути и юнит-экономики.
\end{enumerate}

\end{document}
