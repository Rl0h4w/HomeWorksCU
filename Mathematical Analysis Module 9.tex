\documentclass[a4paper,12pt]{article}

% Кодировка и язык
\usepackage[utf8]{inputenc}
\usepackage[russian]{babel}

% Математические пакеты
\usepackage{amsmath,amsfonts,amssymb}

% Геометрия страницы
\usepackage{geometry}
\geometry{top=2cm, bottom=2cm, left=2.5cm, right=2.5cm}

% Плавающие объекты
\usepackage{float}

% Настройки заголовка
\title{Домашнее задание}
\author{Студент: \textbf{Ваше Имя Фамилия}}
\date{\today}

\begin{document}

% Титульный лист
\begin{titlepage}
    \centering
    \vspace*{1cm}

    \Huge
    \textbf{Домашнее задание}

    \vspace{0.5cm}
    \LARGE
    По курсу: \textbf{Математический анализ}

    \vspace{1.5cm}

    \textbf{Студент: Ростислав Лохов}

    \vfill

    \Large
    АНО ВО Центральный университет\\
    \vspace{0.3cm}
    \today
\end{titlepage}

% Содержание
\tableofcontents
\newpage

\section{Дифференцирование. Часть 2}

\subsection{Задача 1 (0,5 балла)}
\textbf{Условие задачи:} \\
Через точки пересечения прямой \( y = a \) и параболы \( y = x^2 \) проведены касательные к параболе. Найдите \( a \), если известно, что эти касательные пересекаются под прямым углом.

\textbf{Решение:} \\
% Здесь вставьте решение задачи

\vspace{1cm}

\subsection{Задача 2 (1 балл)}
\textbf{Условие задачи:} \\
Докажите, что площадь треугольника, образованного касательной к графику функции \( y = \frac{5}{x} \) (\( x > 0 \)) и координатными осями, не зависит от выбора точки касания.

\textbf{Решение:} \\
% Здесь вставьте решение задачи

\vspace{1cm}

\subsection{Задача 3 (1 балл)}
\textbf{Условие задачи:} \\
Для функции \( y(x) = \sqrt{x} \) в точке \( (1; 1) \) найдите:
\begin{itemize}
    \item[а)] уравнение касательной;
    \item[б)] уравнение нормали;
    \item[в)] уравнение касательной к обратной функции;
    \item[г)] уравнение нормали к обратной функции.
\end{itemize}

\textbf{Решение:} \\
% Здесь вставьте решение задачи

\vspace{1cm}

\subsection{Задача 4 (1 балл)}
\textbf{Условие задачи:} \\
При каком условии кубическая парабола \( y = x^3 + px + q \) касается оси \( Ox \)?

\textbf{Решение:} \\
% Здесь вставьте решение задачи

\vspace{1cm}

\subsection{Задача 5 (2 балла)}
\textbf{Условие задачи:} \\
Найдите приближённые значения следующих выражений с помощью дифференциала:
\begin{itemize}
    \item[а)] \( \sqrt[3]{123} \);
    \item[б)] \( \log_3 23 \);
    \item[в)] \( \sin \frac{39\pi}{120} \);
    \item[г)] \( \arccos(0{,}4) \).
\end{itemize}

\textbf{Решение:} \\
% Здесь вставьте решение задачи

\vspace{1cm}

\subsection{Задача 6 (1,5 балла)}
\textbf{Условие задачи:} \\
Площадь прямоугольника уменьшается со скоростью \( 40 \ \text{см}^2/\text{с} \), причём длинная сторона прямоугольника остаётся в два раза длиннее короткой стороны. С какой скоростью меняется периметр прямоугольника в момент, когда площадь прямоугольника равна \( 5000 \ \text{см}^2 \)?

\textbf{Решение:} \\
% Здесь вставьте решение задачи

\vspace{1cm}

\subsection{Задача 7 (1,5 балла)}
\textbf{Условие задачи:} \\
Геостационарный спутник на высоте \( 35 \ 800 \ \text{км} \) следит за развитием штормовой системы. Спутник всегда находится над точкой \( L \) на земле. Угол \( \theta \) — это угол между горизонталью спутника и прямой от него до штормовой системы. Известно, что штормовая система удаляется от спутника, при этом угол \( \theta \) уменьшается со скоростью \( 2,88 \times 10^{-4} \ \text{радиан в час} \). Найдите скорость движения штормовой системы в момент, когда угол \( \theta \) равен \( 0,726 \ \text{радиан} \).

\textbf{Решение:} \\
% Здесь вставьте решение задачи

\vspace{1cm}

\subsection{Задача 8 (0,5 балла)}
\textbf{Условие задачи:} \\
Найдите производную функции на множестве её существования:
\begin{itemize}
    \item[а)] \( y(x) = \cos x^3 \);
    \item[б)] \( y = \cos^3 x \).
\end{itemize}

\textbf{Решение:} \\

a) \[
\frac{dy}{dx} = -3x^3 \sin(x)
\]

б) \[
\frac{dy}{dx} = -3\cos(x)^2\sin(x)
\]

\vspace{1cm}

\subsection{Задача 9 (1,5 балла)}
\textbf{Условие задачи:} \\
Найдите производную функции на множестве её существования:
\begin{itemize}
    \item[а)] \( y(x) = 6x^{5/3} + x^{-2} - \frac{3}{x} \);
    \item[б)] \( y(x) = \sqrt[3]{x^3} - \sqrt{x} \);
    \item[в)] \( y(x) = x \tan^2 x + \log_3 x^2 \);
    \item[г)] \( y(x) = x \arctan \sqrt{x} \);
    \item[д)] \( y(x) = \frac{\arccos x}{\arcsin x} \);
    \item[е)] \( y(x) = (x^2 - 2x)e^{x^2} \).
\end{itemize}

\textbf{Решение:} \\

\begin{itemize}
    \item[а)] $10x^{2/3}+5x^4-3$
    \item[б)] $3\sqrt{3}x^2-\frac{1}{2\sqrt{x}}$
    \item[в)] $\tan^2(x)+2x\tan(x)\sec^2(x)+\frac{2\log_3(x)}{x\ln(3)}$
    \item[г)] $\arctan(\sqrt{x}) + \frac{\sqrt{x}}{2+2x}$
    \item[д)] $\frac{\arccos(x)-\arcsin(x)}{\sqrt{1-x^2}}$
    \item[е)] $e^{x^{2}}(2x^3-4x^2+2x-2)$
\end{itemize}

\vspace{1cm}

\subsection{Задача 10 (1 балл)}
\textbf{Условие задачи:} \\
Графики функций \( f \) и \( g \) показаны на рисунке. Пусть \( u = f \cdot g \), \( v = \frac{f}{g} \). Найдите:
\begin{itemize}
    \item[а)] \( u'(1) \);
    \item[б)] \( v'(5) \).
\end{itemize}

\textbf{Решение:} \\
% Здесь вставьте решение задачи

\vspace{1cm}

\subsection{Задача 11 (1 балл)}
\textbf{Условие задачи:} \\
Графики функций \( f \) и \( g \) показаны на рисунке. Пусть \( u(x) = f(g(x)) \), \( v(x) = g(f(x)) \), \( w(x) = g(g(x)) \). Найдите производные, если они существуют:
\begin{itemize}
    \item[а)] \( u'(1) \);
    \item[б)] \( v'(1) \);
    \item[в)] \( w'(1) \).
\end{itemize}

\textbf{Решение:} \\
% Здесь вставьте решение задачи

\vspace{1cm}

\subsection{Задача 12 (1 балл)}
\textbf{Условие задачи:} \\
Найдите производную обратной функции к \( y(x) = x + \frac{x^5}{5} \) в точках \( y = 0 \), \( y = \frac{6}{5} \).

\textbf{Решение:} \\

Условия существования обратной: функция дифференцируема в этих точках, функция строго монотонна, производная в окрестности точек не равна 0.

\[
\frac{dy}{dx} = 1 + x^4
\]

\[
y'(0) = 1 \land y'(1) = 2
\]

\[
y^{-1}(0)' = 1 \land y^{-1}(1)' = 0.5
\]
\vspace{1cm}

\subsection{Задача 13 (1 балл)}
\textbf{Условие задачи:} \\
Напишите уравнение касательной и нормали к графику функции \( y = \cos 2x - 2 \sin x \) в точке \( (\pi; 1) \). Верно ли, что искомая касательная имеет только одну точку пересечения с графиком функции \( y = \cos 2x - 2 \sin x \)?

\textbf{Решение:} \\

ур-е касательной
\[
f(x_0) + \frac{dy}{dx|_{x_0}}\cdot (x-x_0)
\]

ур-е нормали
\[
f(x_0)  - \frac{1}{\frac{dy}{dx}|_{x_0}}\cdot (x-x_0)
\]

\[
\frac{dy}{dx} = -2\sin(2x)-2\cos(x)
\]

\[
\frac{dy}{dx|_{\pi}} = 2
\]

ур-е касательной
\[
y=2x-2\pi + 1
\]

ур-е нормали
\[
y=-0.5x+0.5\pi + 1
\]


Нахождение точек касания:

рассмотрим разницу двух функций - первоначальную и ее касательную: она непрерывна. Рассмотрим  производную, для нахождения точек экстреумма

\[
\frac{dy}{dx} = 2(\sin(2x)+\cos(x)+1)
\]

\[
x = \frac{\pi}{2} + \pi k \lor x = \frac{7\pi}{6} + 2\pi k \lor x = \frac{-\pi}{6} + 2\pi k
\]

таким образом если разместить 3 корня от 0 до повторяющегося с предыдущим, мы получим что существует 2 корня т.е совпадения с нешей касательной ровно 2.




\vspace{1cm}

\subsection{Задача 14 (3 балла)}
\textbf{Условие задачи:} \\
Постройте пример функции, отличной от постоянной, график которой имеет бесконечное количество точек пересечения с касательной в любой окрестности точки касания.

\textbf{Решение:} \\

f(x) = 
\begin{cases}
    x^2 \sin\left(\frac{1}{x}\right), & \text{если } x \ne 0 \\
    0, & \text{если } x = 0
\end{cases}

\vspace{1cm}

\subsection{Задача 15 (0,5 балла)}
\textbf{Условие задачи:} \\
Определите среднюю скорость изменения функции \( y = \sin\left(\frac{1}{x}\right) \) на отрезке \( \left[\frac{2}{\pi}; \frac{6}{\pi}\right] \). Найдите мгновенную скорость изменения \( y \) в точке \( x = \frac{4}{\pi} \).

\textbf{Решение:} \\

\[
\overline{v} = \frac{0.5 - 1}{4\pi} = \frac{-\pi}{8}
\]

\[
\frac{dy}{dx|_{\frac{\pi}{4}}} = \frac{-1}{x^2}\cos(\frac{1}{x}) = -\frac{\sqrt{2}\pi^2}{32}
\]

\vspace{1cm}

\subsection{Задача 16 (2 балла)}
\textbf{Условие задачи:} \\
Функция \( f \) имеет ненулевую производную в точке \( x = 0 \). Найдите предел:
\[
\lim_{x \to 0} \frac{f(x)e^x - f(0)}{f(x) \cos x - f(0)},
\]
считая \( f'(0) \) и \( f(0) \) известными величинами.

\textbf{Решение:} \\
По определению производной:

\[
f(\Delta x) = f(0) + f'(0)\Delta x + o(\Delta x)
\]

\[
e^{\Delta x} = 1 + \Delta x + o(\Delta x)
\]

\[
\cos(\Delta x) = 1 - \frac{(\Delta x)^2}{2} + o(\Delta x^2)
\]


\[
\lim_{\Delta x \to 0} \frac{(f(0)+f ′(0)\Delta x+o(\Delta x))(1+\Delta x+o( \Delta x))−f(0)}{(f(0)+f′(0)x+o(\Delta x))(1− \frac{(\Delta x)^2}{2})−f(0)} = \lim_{\Delta x \to 0} \frac{f(0)\Delta x+f′(0)\Delta x+o(\Delta x)}{f′(0)\Delta x+o(\Delta x^2).} = 1 + \frac{f(0)}{f'(0)}
\]

\vspace{1cm}

\subsection{Задача 17 (1 балл)}
\textbf{Условие задачи:} \\
Найдите производные функций:
\begin{itemize}
    \item[а)] \( y(x) = \ln |\sin x| \);
    \item[б)] \( y(x) = \log_x 2x \);
    \item[в)] \( y(x) = \cosh^2 x + \sinh^2 x \);
    \item[г)] \( y(x) = \tanh^2 x + \frac{1}{\cosh^2 x} \).
\end{itemize}

\textbf{Решение:} \\

\begin{itemize}
    \item[а)] $\cot(x)$
    \item[б)] $0$
    \item[в)] $2\sinh(2x)$
    \item[г)] $0$
\end{itemize}


\vspace{1cm}

\subsection{Задача 18 (1 балл)}
\textbf{Условие задачи:} \\
Найдите производные функций:
\begin{itemize}
    \item[а)] \( y(x) = x^{x^x}, \ x > 0 \);
    \item[б)] \( y(x) = (\cosh x)^{e^x}, \ x \in \mathbb{R} \).
\end{itemize}

\textbf{Решение:} \\
a)
\[
\ln(\ln(y)) = x \cdot \ln(x) + \ln(\ln(x))
\]

\[
\frac{d\ln(\ln(y))}{dx} = \ln(x) + 1 + \frac{1}{\ln(x)} \frac{d}{dx} \cdot \ln(x)
\]

\[
\frac{d\ln(\ln(y))}{dx} = \ln(x) + 1 + \frac{1}{x\ln(x)}
\]

\[
\frac{dy}{dx}\frac{1}{y\ln(y)} = \ln(x) + 1 + \frac{1}{x\ln(x)}
\]

\[
\frac{dy}{dx} = x^{x^x}x^x\ln(x)(\ln(x)+1+\frac{1}{x\ln(x)}) = x^{x^x+x-1}\cdot (x(\ln(x))^2+1)
\]


\vspace{1cm}

\subsection{Задача 19 (0,5 балла)}
\textbf{Условие задачи:} \\
Вычислите производную функции:
\[
y(x) = \begin{cases}
\frac{\tan x}{x}, & \text{если } x \neq 0, \\
1, & \text{если } x = 0
\end{cases}
\]
в точке \( x = 0 \), зная, что \( \tan t - t = o(t^2) \).

\textbf{Решение:} \\

\[
\frac{dy}{dx} =\lim_{x \to 0} \frac{y(x) - y(0)}{x - 0} = \lim_{x \to 0} \frac{y(x)-1}{x} = \lim_{x \to 0} \frac{\tan(x) - x}{x^2} = \frac{o(x^2)}{x^2} = 0
\]


\vspace{1cm}

\subsection{Задача 20 (1,5 балла)}
\textbf{Условие задачи:} \\
Исследуйте на непрерывность и дифференцируемость в точке \( x = 0 \) функцию:
\[
y(x) = \begin{cases}
|x|^\alpha \sin \frac{1}{x}, & \text{если } x \neq 0, \\
0, & \text{если } x = 0.
\end{cases}
\]

\textbf{Решение:} \\
Условие непрерывности - $\lim_{x \to 0}y(x)=y(0)$

оценим на зажатость в нуле

\[
-|x|^a \le |x|^a \sin(\frac{1}{x}) \le |x|^a
\]

По теореме о зажатой функции, предел существует в нуле и равен 0, т.е функция непрерывна в 0

Вычислим предел в 0: 

\[
\frac{dy}{dx|_0} = \lim_{x \to 0} \frac{|x|^a\sin(\frac{1}{x})}{x}
\]

Оценим по модулю:

\[
|\frac{|x|^a\sin(\frac{1}{x})}{x}| \le |x|^{a-1}
\]

Предел существует только при а>1 т.к при равных значениях оценка стремится к 1, при меньших к бесконечности. Таким образом дифференцируема при а>1 в точке 0

\vspace{1cm}

\subsection{Задача 21 (1 балл)}
\textbf{Условие задачи:} \\
Вычислите производную функции:
\[
y(x) = \arcsin x \sqrt{1 - x^2} + \ln \sqrt{\frac{1 - x}{1 + x}}.
\]

\textbf{Решение:} \\
\[
\frac{dy}{dx} = \frac{1}{\sqrt{1-x^2}}\cdot \frac{-x}{\sqrt{1-x^2}} + \sqrt{1-x^2}\cdot \frac{1}{\sqrt{1-x^2}} + 0.5 \cdot (\frac{-1}{1-x} - \frac{1}{1+x}) = 1 - \frac{x\arcsin(x)}{\sqrt{1-x^2}} - \frac{1}{1-x^2}
\]

\vspace{1cm}

\subsection{Задача 22 (3 балла)}
\textbf{Условие задачи:} \\
Исследуйте на непрерывность производную функции:
\[
y(x) = \begin{cases}
|x|^\alpha \sin \frac{1}{x}, & \text{если } x \neq 0, \\
0, & \text{если } x = 0
\end{cases}
\]
в точке \( x = 0 \).

\textbf{Решение:} \\
Как уже ранее выясняли:

\[
\frac{dy}{dx|_0} = \frac{|x|^{a}\sin(\frac{1}{x})}{x}
\]

необходимо чтобы предел в нуле функции был равен значению функции в нуле. Значение мы знаем около нуля, не знаем производную


\[
\frac{dy}{dx} = a|x|^{a-1}\cdot sgn(x)\cdot 
\sin(\frac{1}{x}) - |x|^{a-2}\cos(\frac{1}{x})
\]

при а > 2 после оценки второго слагаемого.


\vspace{1cm}

\subsection{Задача 23 (5 баллов)}
\textbf{Условие задачи:} \\
Постройте пример функции:
\begin{itemize}
    \item[а)] имеющей производную только в одной точке;
    \item[б)] имеющей производную только в точках \( x_1, x_2, \ldots, x_n \);
    \item[в)] имеющей производную только в счётном числе точек;
    \item[г)] непрерывной, но не дифференцируемой в точках \( x_1, x_2, \ldots, x_n \);
    \item[д)] непрерывной, но не дифференцируемой в бесконечном числе точек.
\end{itemize}

\textbf{Решение:} \\

а)
f(x)=
\begin{cases}
    x^2\sin(\frac{1}{x}), если x\ne 0\\
    0, x = 0
\end{cases}

б)
f(x) = 
\begin{cases}
    (x-x_i)^2, если\ x=x_i\\
    |x-x_i|,\ иначе
\end{cases}

в)
f(x) = 
\begin{cases}
    |x-x_i| , если\ x=x_i\\
    x^2, \ иначе
\end{cases}

г)
f(x) =
\begin{cases}
    |x|\sin(\frac{1}{x}), x\ne0\\
    0, x=0
\end{cases}
\vspace{1cm}

\subsection{Задача 24 (2 балла)}
\textbf{Условие задачи:} \\
Найдите производную функции:
\[
y(x) = \frac{(x^2 - 3x)^5}{(x - 3)^3 (2x + 1)^4}
\]
на множестве её существования.

\textbf{Решение:} \\
\[
\ln(f(x)) = 5\ln(x) + 2\ln(x-3)-4\ln(2x+1)
\]

\[
\frac{d\ln(f)}{dx} = \frac{5}{x} + \frac{2}{x-3} - \frac{8}{2x-1}
\]

\[
\frac{df}{dx} = \frac{x^5(x-3)^2}{(2x+1)^4}\left(\frac{5}{x} + \frac{2}{x-3} - \frac{8}{2x-1}\right)
\]

\[
\frac{df}{dx} = \frac{x^4(x-3)(6x^2+x-15)}{(2x+1)^5}
\]


\vspace{1cm}

\end{document}
