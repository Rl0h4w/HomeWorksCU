\documentclass[a4paper,12pt]{article}

% Кодировка и язык
\usepackage[utf8]{inputenc}
\usepackage[russian]{babel}

% Математические пакеты
\usepackage{amsmath,amsfonts,amssymb}

% Графика
\usepackage{graphicx}
\usepackage{tikz}
\usetikzlibrary{shapes.geometric, calc}
\usepackage{pgfplots}

% Геометрия страницы
\usepackage{geometry}
\geometry{top=2cm, bottom=2cm, left=2.5cm, right=2.5cm}

% Гиперссылки
\usepackage{hyperref}

% Плавающие объекты
\usepackage{float}

% Дополнительные пакеты
\usepackage{venndiagram}

% Настройки заголовка
\title{Домашнее задание}
\author{Студент: \textbf{Ваше Имя Фамилия}}
\date{\today}

\begin{document}

% Титульный лист
\begin{titlepage}
    \centering
    \vspace*{1cm}

    \Huge
    \textbf{Домашнее задание}

    \vspace{0.5cm}
    \LARGE
    По курсу: \textbf{Математический Анализ}

    \vspace{1.5cm}

    \textbf{Студент: Ростислав Лохов}

    \vfill

    \Large
    АНО ВО Центральный университет\\
    \vspace{0.3cm}
    \today

\end{titlepage}

% Содержание
\tableofcontents
\newpage

% Основной текст
\section{Применение формулы Тейлора}

\subsection{Задача 5}

Разложим в ряд Маклорена каждое слагаемое:

\[
\frac{x + \frac{x^3}{3}+O(x^6)-x}{x-\frac{x^3}{6}+\frac{x^5}{120}+O(x^6)-(x+\frac{x^3}{6}+\frac{x^5}{120}+O(x^6))} =  -1
\]

\subsection{Задача 6}

\[
\frac{x + \frac{x^3}{3}+O(x^6)-\frac{x}{1+x^2}}{x-\frac{x^3}{6}+\frac{x^5}{120}+O(x^6)-(x+\frac{x^3}{6}+\frac{x^5}{120}+O(x^6))} = -4
\]

\subsection{Задача 7}

разложим каждое слагаемое в Маклорена

\[
\frac{x-\frac{x^3}{6}+\frac{3x^5}{40}+O(x^5)+1-x+1.5x^2-2.5x^3+4.375x^4+O(x^5)-(1+1.5x^2+0.375x^4+O(x^6))}{\frac{x^3}{6}+\frac{11x^5}{120}+O(x^5)} = -16
\]

\subsection{Задача 8}
\[
(2-\frac{2x^2}{3}+\frac{2x^4}{9}+O(x^6)-(1-\frac{2x^2}{3}-\frac{2x^4}{9}+O(x^6)))^{\frac{1}{x^4}-\frac{5}{6x^2}+\frac{1}{24}+O(x^2)} = (\frac{4x^4}{9}+1)^{O(x^2)+\frac{1}{x^4}-\frac{5}{6x^2}+\frac{1}{24}}=e^{\frac{4}{9}}
\]

\subsection{Задача 9}
\[
(1-\frac{x}{2}+frac{11x^2}{24}+O(x^2)+\frac{x}{2}-\frac{5x^2}{8}+O(x^2))^{\frac{1}{x^2}-\frac{2}{3}+O(x^2)} = \frac{1}{\sqrt[6]{e}}
\]

\subsection{Задача 10}
\[
(1+x+\frac{x^2}{2}+\frac{x^3}{2}+o(x^3)-x-\frac{x^2}{2}-\frac{x^3}{3}+o(x^3))^{\frac{3}{x^3}-\frac{3}{2x}-\frac{53x}{140}+O(x^2)} = \sqrt{e}
\]

\subsection{Задача 11}

\[
x(x+1-\frac{1}{2x}+\frac{1}{2x^2}+o(\frac{1}{x^2})-(2x+1-\frac{1}{4x}+\frac{1}{8x^2}+o(\frac{1}{x^2}))+x)=-0.25
\]

\subsection{Задача 14}

\[
S = 20\cdot 1 + \frac{2\cdot 1^2}{2} = 21
\]

\subsection{Задача 15}
Разложим в ряд Маклорена, т.к l гораздо меньше L:

\[
E = \frac{q}{L^2} - (\frac{q}{L^2} - \frac{2q}{L^3}l + \frac{3q}{L^4}l^2 - \frac{4q}{L^5}l^3 + ...)
\]

чтд: в ведущем порядке поле пропорционально $\frac{1}{L^3}$.




\end{document}
