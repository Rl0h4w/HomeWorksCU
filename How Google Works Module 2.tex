\documentclass[a4paper,12pt]{article}
\usepackage[utf8]{inputenc}
\usepackage[russian]{babel}
\usepackage{amsmath,amsfonts,amssymb}
\usepackage{graphicx}
\usepackage{geometry}
\usepackage{hyperref}
\usepackage{venndiagram}
\usepackage{tikz}
\usetikzlibrary{shapes.geometric, calc}
\usepackage{pgfplots}
\usepackage{float}
\usepackage{listings}
\usepackage{hyperref}

\lstset{
    language=Python,
    basicstyle=\ttfamily\small,
    keywordstyle=\color{blue},
    stringstyle=\color{red},
    commentstyle=\color{gray},
    morecomment=[l][\color{magenta}]{\#},
    numbers=left,
    numberstyle=\tiny\color{gray},
    stepnumber=1,
    numbersep=10pt,
    backgroundcolor=\color{lightgray!10},
    tabsize=4,
    showspaces=false,
    showstringspaces=false
}

% Параметры страницы
\geometry{top=2cm,bottom=2cm,left=2.5cm,right=2.5cm}
\geometry{a4paper, margin=1in}

% Заголовок документа
\title{Домашнее задание}
\author{Студент: Лохов Ростислав Алексеевич}
\date{\today}

\begin{document}

% Титульный лист
\begin{titlepage}
    \centering
    \vspace*{1cm}

    \Huge
    \textbf{Домашнее задание №2}

    \vspace{0.5cm}
    \LARGE
    По курсу: \textbf{How Google Works}

    \vspace{1.5cm}

    \textbf{Студент: Лохов Ростислав Алексеевич}

    \vfill

    \Large
    АНО ВО Центральный Университет\\
    \vspace{0.3cm}
    \today

\end{titlepage}

% Содержание
\tableofcontents
\newpage

% Основной текст
\section{Матричная разминка}

\subsection{Задача 1}
\textbf{Условие задачи:}
\begin{itemize}
    \item[а)] Найдите транспонированные матрицы \( A^T \) и \( B^T \).
    \item[б)] Вычислите произведения \( A \cdot x_1 \) и \( B \cdot x_2 \).
    \item[в)] Вычислите выражение \( -\frac{1}{2} A \cdot x_1 + 3 B \cdot x_2 \).
\end{itemize}

\textbf{Решение:}

\textbf{а) Решение:}
\begin{itemize}
       \[
   A^T = \begin{pmatrix}
   2 & 0 & 0 \\
   4 & -2 & 0 \\
   2 & 2 & 4
   \end{pmatrix}, \quad
   B^T = \begin{pmatrix}
   \frac{1}{3} & \frac{1}{3} & 1 \\
   \frac{5}{6} & 0 & -\frac{1}{6} \\
   0 & 0 & -1
   \end{pmatrix}
   \]
\end{itemize}

\textbf{б) Решение:}
\begin{itemize}
   \[
   A \cdot x_1 = \begin{pmatrix} 11 \\ -3 \\ 2 \end{pmatrix}, \quad B \cdot x_2 = \begin{pmatrix} \frac{4}{3} \\ -\frac{1}{3} \\ -\frac{5}{3} \end{pmatrix}
   \]
\end{itemize}

\textbf{в) Решение:}
\begin{itemize}
   \[
   -\frac{1}{2} A \cdot x_1 + 3 B \cdot x_2 = \begin{pmatrix} -1.5 \\ 0.5 \\ -6 \end{pmatrix}
   \]
\end{itemize}



\subsection{Задача 2}
\textbf{Условие задачи:}
\begin{itemize}
    \item[а)] Построй матрицу смежности \( M \) для графа 1. Считай, что вершины упорядочены по алфавиту: \( v_1 = A \), \( v_2 = B \) и так далее.
    \item[б)] Транспонируй матрицу \( M \) и построй граф, соответствующий матрице смежности \( M^T \).
    \item[в)] Рассмотрим произвольные графы \( G_1 \) и \( G_2 \) с матрицами смежности \( M \) и \( M^T \), соответственно. Как эти графы связаны между собой?
\end{itemize}

\textbf{Решение:}

\textbf{а) Матрица смежности \( M \):}
\[
M = \begin{pmatrix}
0 & 1 & 1 & 1 & 0 & 0 & 0 & 0 \\
0 & 0 & 0 & 0 & 1 & 0 & 0 & 0 \\
0 & 0 & 0 & 0 & 1 & 0 & 0 & 0 \\
0 & 0 & 0 & 0 & 1 & 0 & 0 & 0 \\
1 & 0 & 0 & 0 & 0 & 0 & 0 & 0 \\
0 & 0 & 0 & 1 & 0 & 0 & 0 & 0 \\
0 & 0 & 0 & 1 & 0 & 0 & 0 & 0 \\
0 & 0 & 0 & 1 & 0 & 0 & 0 & 0
\end{pmatrix}
\]

\textbf{б) Транспонированная матрица \( M^T \):}
\[
M^T = \begin{pmatrix}
0 & 0 & 0 & 0 & 1 & 0 & 0 & 0 \\
1 & 0 & 0 & 0 & 0 & 0 & 0 & 0 \\
1 & 0 & 0 & 0 & 0 & 0 & 0 & 0 \\
1 & 0 & 0 & 0 & 0 & 1 & 1 & 1 \\
0 & 1 & 1 & 1 & 0 & 0 & 0 & 0 \\
0 & 0 & 0 & 0 & 0 & 0 & 0 & 0 \\
0 & 0 & 0 & 0 & 0 & 0 & 0 & 0 \\
0 & 0 & 0 & 0 & 0 & 0 & 0 & 0
\end{pmatrix}
\]

\textbf{в) Связь между графами \( G_1 \) и \( G_2 \):}
Граф \( G_2 \), который описывается транспонированной матрицей \( M^T \), представляет собой граф, в котором все рёбра графа \( G_1 \) обращены в противоположном направлении. То есть, если в графе \( G_1 \) существовала дуга из вершины \( v_i \) в вершину \( v_j \), то в графе \( G_2 \) существует дуга из вершины \( v_j \) в вершину \( v_i \). Графы \( G_1 \) и \( G_2 \) называют **обратными** по отношению друг к другу.


\subsection{Задача 3}
\textbf{Условие задачи:}
\begin{itemize}
    \item[a)] Найди все собственные вектора матрицы A с собственным значением λ = 1.
\end{itemize}

\textbf{Решение:}

  \[
   \vec{v} = \begin{pmatrix} 1 \\ 3 \end{pmatrix}
   \]

\vspace{1cm}

\section{Матричная форма PageRank}

\subsection{Задача 4}
\textbf{Условие задачи:}
\begin{itemize}
    \item[a)] Составь матрицу голосования P и вектор авторитетностей a(0) для графа 2.
    \item[б)] Проведи вычисления двух итераций алгоритма Basic PageRank в матричной форме и найди вектора a(1) и a(2). В решении обязательно покажи свои вычисления.
\end{itemize}

\textbf{Решение:}

\textbf{а) Решение:}
\begin{itemize}
    \item[1.] \begin{pmatrix}
0.0 & 0.5 & 0.5 & 0.0 & 0.0 & 0.0 & 0.0 & 0.0 \\
0.0 & 0.0 & 0.0 & 0.5 & 0.5 & 0.0 & 0.0 & 0.0 \\
0.0 & 0.0 & 0.0 & 0.0 & 0.0 & 0.5 & 0.5 & 0.0 \\
0.5 & 0.0 & 0.0 & 0.0 & 0.0 & 0.0 & 0.0 & 0.5 \\
0.5 & 0.0 & 0.0 & 0.0 & 0.0 & 0.0 & 0.0 & 0.5 \\
1.0 & 0.0 & 0.0 & 0.0 & 0.0 & 0.0 & 0.0 & 0.0 \\
1.0 & 0.0 & 0.0 & 0.0 & 0.0 & 0.0 & 0.0 & 0.0 \\
1.0 & 0.0 & 0.0 & 0.0 & 0.0 & 0.0 & 0.0 & 0.0
\end{pmatrix}

\end{itemize}
\textbf{б) Решение:}
\vspace{1cm}
\begin{itemize}
    \item[a(1)] [4.0, 0.5, 0.5, 0.5, 0.5, 0.5, 0.5, 1.0]
    \item[a(2)] [2.5, 2.0, 2.0, 0.25, 0.25, 0.25, 0.25, 0.5]
\end{itemize}

\subsection{Задача 5}
\textbf{Условие задачи:}
\begin{itemize}
    \item[a)] Может ли вектор x (определен ниже) быть предельным значением вектора авторитетностей в алгоритме Basic PageRank на графе 2? Если да, то можем ли мы заранее утверждать, что алгоритм
Basic PageRank при увеличении количества итераций k будет сходиться к вектору x?
\end{itemize}



\textbf{а) Решение:}
\begin{verbatim}
def transpose(P):
    result = [[P[j][i] for j in range(len(P))] for i in range(len(P[0]))]
    return result

def multiply(matrix, vector):
    result = [0 for i in range(len(matrix))]
    for i in range(len(matrix)):
        result[i] = sum(matrix[i][j] * vector[j] for j in range(len(vector)))
    return result

def page_rank(M, a_0, k):
    P = [[0. if sum(M[i]) == 0 else 1/sum(M[i]) if M[i][j] else 0 for j in range(len(M[0]))] for i in range(len(M))]
    P_t = transpose(P)
    last = a_0
    for i in range(k):
        last = multiply(P_t, last)
    return last

if __name__=="__main__":
    M = [
        [0, 1, 1, 0, 0, 0, 0, 0],
        [0, 0, 0, 1, 1, 0, 0, 0],
        [0, 0, 0, 0, 0, 1, 1, 0],
        [1, 0, 0, 0, 0, 0, 0, 1],
        [1, 0, 0, 0, 0, 0, 0, 1],
        [1, 0, 0, 0, 0, 0, 0, 0],
        [1, 0, 0, 0, 0, 0, 0, 0],
        [1, 0, 0, 0, 0, 0, 0, 0]
    ]
    a_0 = [1 for i in range(len(M))]
    k = 20000

    print([ i * 13 for i in page_rank(M, a_0, k)])
\end{verbatim}
Вот что выведет алгоритм: [32.00000000000001, 16.000000000000007, 16.000000000000007, 8.0, 8.0, 8.0, 8.0, 8.000000000000002]\\
что и является тем самым вектором
\vspace{1cm}
\subsection{Задача 6}
\textbf{Условие задачи:}
Из-за отсутствия исходящих ребер у одной из вершин суммарная авторитетность страниц в графе уменьшается: уже на первой итерации суммарная авторитетность страниц уменьшается на 1, а в итоге может сойтись к нулю. У этой проблемы есть простой фикс: если у вершины X нет исходящих ребер, то во время голосования X распределяет свою авторитетность равномерно на все вершины графа, включая себя.

\begin{itemize}
    \item[a)] Как наличие в графе вершины без исходящих ребер отражается на матрице голосования \(P\)?
    \item[б)] Как нужно изменить матрицу \(P\), чтобы реализовать предложенный фикс?
    \item[в)] Приведи интерпретацию предложенного фикса в терминах Random Surfer Model.
\end{itemize}

\textbf{Решение:}

\textbf{А.} Если вершина X не имеет исходящих ребер, это означает, что все вероятности перехода из X в другие вершины равны нулю. Это приводит к тому, что соответствующая строка в матрице голосования \( P \) будет состоять только из нулей. В результате, вся авторитетность, накопленная вершиной X, исчезает в последующих итерациях алгоритма, что ведет к снижению общей суммы авторитетности в графе.

\textbf{Б.} Чтобы реализовать фикс, нужно сделать следующее изменение в матрице \( P \): если у вершины X нет исходящих ребер, нужно заменить её строку в матрице \( P \) на строку, в которой каждая вершина (включая саму вершину X) получает равномерную долю авторитетности от X. То есть, вместо строки, содержащей нули, каждая ячейка строки будет равна \( \frac{1}{n} \), где \( n \) — количество вершин в графе.

\textbf{В.} В модели Random Surfer фикс можно интерпретировать следующим образом: если случайный серфер попадает на страницу без исходящих ссылок, то вместо того, чтобы "застрять" на этой странице, он случайным образом перемещается на любую другую страницу графа (включая саму себя) с равной вероятностью. Это гарантирует, что авторитетность не теряется и общее распределение вероятностей остаётся корректным.

\subsection{Задача 7}
\textbf{Условие задачи:}
Рассмотрим проблему второго типа графов. Вершины F и G обмениваются авторитетностью и не отдают её другим вершинам. В пределе вся авторитетность сконцентрируется на вершинах F и G, что приведёт к некорректным результатам.

\begin{itemize}
    \item[a)] Как нужно изменить матрицу \( P \), чтобы реализовать предложенный фикс?
    \item[б)] Приведи интерпретацию предложенного фикса в терминах Random Surfer Model.
    \item[в)] Сформулируй полную версию алгоритма PageRank в терминах матричного умножения.
\end{itemize}

\textbf{Решение:}

\textbf{А.} Изменённая матрица \( P_s \):
\[
P_s = s \cdot P + \frac{1 - s}{n} \cdot E
\]

где \( P \) — исходная матрица переходов, \( n \) — количество вершин, \( E \) — единичная матрица.

\textbf{Б.} В Random Surfer Model: с вероятностью \( s \) серфер переходит по ссылкам, с вероятностью \( 1 - s \) — на случайную страницу.

\textbf{В.} Полная версия алгоритма PageRank:
\[
a^{(k)} = s \cdot P^T \cdot a^{(k-1)} + \frac{1 - s}{n} \cdot \mathbf{1}
\]

\subsection{Задача 8}
\textbf{Условие задачи:}
Реализуй модифицированную версию алгоритма PageRank, которую ты разработал-(а) в задачах
6 и 7, на Python или другом языке. Установи параметр s = 0.85 и проведи k = 1, 000 итераций
алгоритма для графов на рисунке 3. В качестве решения:

\begin{itemize}
    \item[a)] Получи значения авторитетностей после одной итерации голосования с точностью до двух знаков
после запятой.
    \item[б)] Получи предельные значения авторитетностей с точностью до двух знаков после запятой.
    \item[в)] В LMS приложи ссылку на Google Colab, открытый для чтения, Jupyter Notebook или другой
файл, в котором ты проводил(-а) вычисления.
\end{itemize}

\textbf{Решение:}
\href{https://colab.research.google.com/drive/1H1bMEVws8LbPNbGYGrFPi2vDS5lYgGls?usp=sharing}{COLAB}
\vspace{1cm}
\subsection{Задача 9}
\textbf{Условие задачи:}
Примени эти теоремы к модифицированному алгоритму PageRank и сформулируй теорему о том,
что модифицированный метод PageRank для любого графа сходится к ненулевому вектору авторитетностей.

\textbf{Решение:}
Теорема (о сходимости модифицированного PageRank):

Пусть \( P \in \mathbb{R}^{n \times n} \) — стохастическая матрица переходов графа, \( s \in (0, 1) \) — scaling factor, и \( M = s \cdot P^T + \frac{1 - s}{n} \mathbf{1} \) — модифицированная матрица PageRank.

Тогда для любого начального вектора \( a^{(0)} = \frac{1}{n} \mathbf{1} \), последовательность векторов авторитетности:

\[
a^{(k)} = M \cdot a^{(k-1)}, \quad k \geq 1,
\]

сходится к единственному собственному вектору \( a^* \), такому что:
- \( M \cdot a^* = a^* \) (собственное значение \( \lambda = 1 \)),
- \( a^* \in \mathbb{R}^n \), все элементы положительны и \( \sum_{i=1}^n a_i^* = 1 \).

Доказательство:
1. \( M \) является положительной и стохастической матрицей (по определению).\\
2. По теореме Перрона-Фробениуса существует единственный собственный вектор \( a^* \) с \( \lambda = 1 \), где все элементы \( a^* \) положительны.\\
3. По теореме о Power Method последовательность \( a^{(k)} = M^k \cdot a^{(0)} \) сходится к \( a^* \).

Таким образом, модифицированный алгоритм PageRank сходится к уникальному вектору авторитетностей \( a^* \), независимо от начального состояния.

\subsection{Задача 10}
\textbf{Условие задачи:}
Рассмотрим произвольный направленный граф G c матрицей смежности M. Самостоятельно разберись, как умножаются матрицы, и ответь на вопросы:
\begin{itemize}
    \item[a) ] Пусть матрица N равна произведению матрицы M на себя: $N = M\cdot M$ Какой физический смысл
    имеет значение произвольного элемента $N_{ij}$ ?
    \item[б) ] Пусть теперь $N=\prod_{n=1}^{k}M$ Какой физический смысл имеет значение элемента $N_{ij}$ ?
\end{itemize}
\textbf{Решение:}
\begin{itemize}
    \item[a) ] Количество маршрутов длины 2 из $i$ вершины в $j$ 
    \item[б) ] Количество маршрутов длины k из $i$ вершины в $j$
\end{itemize}
\end{document}
