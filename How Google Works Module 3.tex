\documentclass[a4paper,12pt]{article}
\usepackage[utf8]{inputenc}
\usepackage[russian]{babel}
\usepackage{amsmath,amsfonts,amssymb}
\usepackage{graphicx}
\usepackage{geometry}
\usepackage{hyperref}
\usepackage{venndiagram}
\usepackage{tikz}
\usetikzlibrary{shapes.geometric, calc}
\usepackage{pgfplots}
\usepackage{float}

% Параметры страницы
\geometry{top=2cm,bottom=2cm,left=2.5cm,right=2.5cm}
\geometry{a4paper, margin=1in}

% Заголовок документа
\title{Домашнее задание}
\author{Студент: Лохов Ростислав Алексеевич}
\date{\today}

\begin{document}

% Титульный лист
\begin{titlepage}
    \centering
    \vspace*{1cm}

    \Huge
    \textbf{Домашнее задание}

    \vspace{0.5cm}
    \LARGE
    По курсу: \textbf{How Google Works}

    \vspace{1.5cm}

    \textbf{Студент: Лохов Ростислав Алексеевич}

    \vfill

    \Large
    АНО ВО Центральный Университет\\
    \vspace{0.3cm}
    \today

\end{titlepage}

% Содержание
\tableofcontents
\newpage

% Основной текст
\section{Задачи базового уровня}

\subsection{Задача 1 (0.5 + 0.5 балла)}
\textbf{Условие задачи:}

А. В аукционе первой цены участвуют 8 кандидатов. Их ставки собраны в таблице 1. Кто победит в аукционе и какую стоимость заплатит?

\begin{center}
\begin{tabular}{|c|c|}
\hline
Кандидат & Ставка \\
\hline
A & 10 \\
B & 8 \\
C & 12 \\
D & 7 \\
E & 9 \\
F & 2 \\
G & 8 \\
H & 0 \\
\hline
\end{tabular}
\end{center}

Б. В аукционе второй цены участвуют 8 кандидатов. Их ставки собраны в таблице 2. Кто победит в аукционе и какую стоимость заплатит?

\begin{center}
\begin{tabular}{|c|c|}
\hline
Кандидат & Ставка \\
\hline
A & 18 \\
B & 14 \\
C & 3 \\
D & 2 \\
E & 9 \\
F & 7 \\
G & 17 \\
H & 20 \\
\hline
\end{tabular}
\end{center}

\textbf{Решение:}
\begin{itemize}
    \item [a)] C: 12
    \item [б)] H: 18
\end{itemize}

\vspace{1cm}

\subsection{Задача 2 (0.5 + 0.5 балла)}
\textbf{Условие задачи:}

В аукционе второй цены участвуют 4 кандидата. Все участники ведут себя честно и делают ставки равные ценности, которую товар имеет для них: $b_i = v_i$, $i \in \{A,B,C,D\}$. Их ставки собраны в таблице 3.

\begin{center}
\begin{tabular}{|c|c|}
\hline
Кандидат & Ставка \\
\hline
A & 2 \\
B & 4 \\
C & 6 \\
D & 8 \\
\hline
\end{tabular}
\end{center}

А. Чему равна суммарная ценность, полученная участниками по результатам аукциона?

Б. Представь, что к аукциону присоединяется участник E и делает ставку $b_E = 10$. Чему будет равна суммарная ценность участников $\{A,B,C,D\}$ по результатам нового аукциона? Как изменение суммарной ценности связано с ценой, которую заплатит $E$?

\textbf{Решение:}

\begin{itemize}
    \item[a)] 8
    \item[б)] 0, т.к Е стал победителем
\end{itemize}
\vspace{1cm}

\subsection{Задача 3 (1 балл)}
\textbf{Условие задачи:}

В аукционе второй цены участвуют 2 кандидата, которые действуют независимо и не могут вступать в сговор. Организатор аукциона не готов продавать свой товар меньше чем за $s$, поэтому он становится третьим участником и делает ставку $s$. Если его ставка победит, то он останется с товаром и заплатит сам себе итоговую цену $p$ (фактически, организатор ничего не заплатит и оставит товар у себя).

Могут ли участники повысить свою выгоду, отклонившись от честной стратегии, то есть сделав ставку $b \neq v$?

\textbf{Решение:}

\begin{itemize}
    \item [a)] в случае если b>v>s, то нет
    \item [б)] в случае если b<v<s, то да
\end{itemize}

\vspace{1cm}


\subsection{Задача 4 (1 балл)}
\textbf{Условие задачи:}

По запросу «Авиабилеты в Москву» в поисковой выдаче выделены три слота под рекламу. Вероятности кликов на этих слотах равны 0.6 (слот \#1), 0.5 (слот \#2) и 0.1 (слот \#3), соответственно. На показ по этому запросу претендуют четыре рекламодателя. Их ставки за клик собраны в таблице 4.

\begin{center}
\begin{tabular}{|c|c|}
\hline
Кандидат & Ставка за клик \\
\hline
A & 40 \\
B & 20 \\
C & 10 \\
D & 5 \\
\hline
\end{tabular}
\end{center}

Распредели слоты с помощью аукциона VCG и посчитай, какую цену заплатит каждый рекламодатель.

\textbf{Решение:}

\begin{itemize}
    \item[a)]Распределение: Слот #1: A (ставка 40) Слот #2: B (ставка 20) Слот #3: C (ставка 10)
    \item[б)] VCG: A - 6.5; B - 4.5; C - 0.5
\end{itemize}

\vspace{1cm}

\subsection{Задача 5 (0.25 + 0.25 + 0.5 балла)}
\textbf{Условие задачи:}

По запросу «Авиабилеты в Красноярск» в поисковой выдаче выделены два слота под рекламу. Вероятности кликов на этих слотах равны 0.4 (слот \#1) и 0.3 (слот \#2). На показ по этому запросу претендуют четыре рекламодателя. Их ставки собраны в таблице 5.

\begin{center}
\begin{tabular}{|c|c|}
\hline
Кандидат & Ставка за клик \\
\hline
A & 40 \\
B & 30 \\
C & 10 \\
D & 1 \\
\hline
\end{tabular}
\end{center}

А. Распредели слоты с помощью аукциона VCG и посчитай, какую цену заплатит каждый рекламодатель.

Б. Поисковая система решает выделить по этому запросу еще один рекламный слот. Вероятность клика по новому слоту равна 0.2, а его появление не влияет на два других слота. Распредели слоты с помощью аукциона VCG и посчитай, какую цену заплатит каждый рекламодатель.

В. Чему равен доход поисковой системы в случае А и Б? Было ли решение выделить третий слот верным?

\textbf{Решение:}

\begin{itemize}
    \item [a)] A = 6; B = 3
    \item [б)] A = 4.2; B=1.2; C= 0.2
    \item [в)] в случае а - доход 9, в случае б - доход 5.6, решение неверное
\end{itemize}
\vspace{1cm}

\subsection{Задача 6 (1.5 + 0.5 балла)}
\textbf{Условие задачи:}

Докажи, что в аукционе второй цены оптимальная стратегия для каждого участника — быть честным. Независимо от того, как себя ведут другие участники, участнику $i$ выгодно делать ставку, равную ценности, которую товар имеет для него: $b_i = v_i$.

А. Докажи, что участник $i$ не может увеличить свою выгоду за счет выбора ставки $b'_i > v_i$.

Б. Докажи, что участник $i$ не может увеличить свою выгоду за счет выбора ставки $b'_i < v_i$.

Подсказка: Обрати внимание, что выбор ставки влияет на то, победил ли участник в аукционе или нет, но не влияет на размер оплаты в случае победы.

\textbf{Решение:}

\begin{itemize}
    \item [a)] Ставка \(b'_i > v_i\)\\
    Может привести к победе и переплате (выгода отрицательна, если товар стоит меньше, чем заплатил).\\
    Если проиграет, выгода не изменится.

    \item [б)] Ставка \(b'_i < v_i\)\\  
    Может привести к проигрышу и упущению выгодного товара (выгода теряется).\\
    Если всё равно победит, выгода не изменится.\\
    Таким образом, честная ставка \(b_i = v_i\) всегда оптимальна.
\end{itemize}

\vspace{1cm}

\section{Задачи среднего уровня}

\subsection{Задача 7 (1 балл)}
\textbf{Условие задачи:}

Рассмотрим аукцион второй цены, где n = 100 кандидатов участвуют в аукционе с целью получения прибыли от перепродажи товара. Каждый участник оценивает товар как $v_i = v^* + x$, где $x$ — случайная ошибка. Верно ли, что стратегия быть честным остается оптимальной в такой постановке?

\textbf{Решение:}

Оптимальная стратегия в аукционе второй цены с ошибкой в оценке \( v^* \) — честная ставка \( b_i = v_i \).\\
1. Пусть \( \forall i \quad v_i = v^* + x_i \), где \( x_i \) — случайная ошибка.\\
2. \( \forall i, j \quad b_i > b_j \implies i \) выигрывает аукцион и платит \( b_j \).\\
3. Если \( \exists b'_i > v_i \), то \( \exists j \quad P(v^* < b_j) > P(v^* < v_i) \), и \( i \) рискует переплатить.\\
4. Если \( \exists b'_i < v_i \), то \( \exists j \quad P(v^* > b'_i) > P(v^* > v_i) \), и \( i \) рискует упустить товар.\\
5. \( \forall i \quad b_i = v_i \) минимизирует как \( P(\text{переплата}) \), так и \( P(\text{упущенная выгода}) \).\\

Следовательно, \( \forall i \quad b_i = v_i \) — оптимальная стратегия.
\vspace{1cm}

\subsection{Задача 8 (1 + 1 балл)}
\textbf{Условие задачи:}

Построй пример, в котором одному из рекламодателей выгодно отклониться от честной стратегии в аукционе GSP. Приведи расчеты. Затем примени аукцион VCG к этому примеру и покажи, что в случае VCG у рекламодателей нет мотивации отклоняться от честной стратегии.

\textbf{Решение:}

\item[a)] Рассмотрим 3 рекламодателей \(A\), \(B\) и \(C\) с реальными ценностями кликов:
\(v_A = 10\), \(v_B = 8\), \(v_C = 6\)
Вероятности кликов по слотам:
\(p_1 = 0.4\), \(p_2 = 0.3\), \(p_3 = 0.2\)\\
Честная стратегия (GSP):\\
Ставки \(b_A = 10\), \(b_B = 8\), \(b_C = 6\).\\
Цены за клик:\\
\(A\) платит \(b_B = 8\), \(B\) платит \(b_C = 6\).\\
Выгоды:\\
\(A: (10-8)\cdot 0.4 = 0.8\)\\
\(B: (8-6)\cdot 0.3 = 0.6\)\\
Отклонение \(B\):\\
Если \(B\) снижает ставку до \(b_B = 5\), он получает слот 3:\\
\(A\) платит \(b_C = 6\), \(C\) платит \(b_B = 5\), \(B\) платит 0.\\
Новые выгоды:\\
\(A: 1.6\), \(B: 1.6\) (увеличение), \(C: 0.3\)\\

\item[б)] Применим аукцион VCG:\\
\(A\) платит \(5\), \(B\) платит \(1.8\), \(C\) платит \(0\).\\
Выгоды:\\
\(A: (10-5)\cdot 0.4 = 2\)\\
\(B: (8-1.8)\cdot 0.3 = 1.86\)\\
\(C: 6 \cdot 0.2 = 1.2\)\\
В VCG отклоняться невыгодно — максимальная выгода при честной стратегии. 

\vspace{1cm}

\subsection{Задача 9 (3 балла)}
\textbf{Условие задачи:}

Докажи, что в аукционе VCG оптимальная стратегия для каждого участника — быть честным. Используй обозначения $V_{-i}$ и другие при доказательстве.

\textbf{Решение:}

% Вставьте решение здесь

\vspace{1cm}

\end{document}
