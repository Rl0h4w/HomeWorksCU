\documentclass[a4paper,12pt]{article}
\usepackage[utf8]{inputenc}
\usepackage[russian]{babel}
\usepackage{geometry}
\usepackage{amsmath}

\geometry{left=2cm, right=2cm, top=2cm, bottom=2cm}

\title{Решение Задач}
\author{Лохов Ростислав}
\date{\today}

\begin{document}

\maketitle

\section*{Задача 1 (4 балла)}

\textbf{Условие:} 

Необходимо определить, сколько дополнительных рейсов в день требуется, чтобы обеспечить 100\%-ную загрузку гостиницы. 

\textbf{Дано:}
\begin{enumerate}
    \item Максимальное количество посадочных мест в самолёте: 100.
    \item Местные жители, командировочные и военные продолжают выкупать 80\% всех мест.
    \item Следовательно, туристам достаётся 20\% мест, то есть 20 мест на рейс.
    \item В тёплый сезон (июль-октябрь) средняя загрузка гостиницы составляет 40\%.
    \item В гостинице в одной половине номеров проживает по 2 человека, а в другой — по 1 человеку.
\end{enumerate}

\textbf{Решение:}

\begin{enumerate}
    \item \textbf{Определение необходимого количества туристов для 100\%-ной загрузки гостиницы:}
    
    Пусть \( H \) — общая вместимость гостиницы в человеко-днях. При текущей загрузке 40\%:
    \[
    \text{Текущие туристы} = 0.4H
    \]
    Для достижения 100\% загрузки требуется:
    \[
    \text{Необходимые туристы} = H
    \]
    \[
    \Delta \text{Туристов} = H - 0.4H = 0.6H
    \]
    
    \item \textbf{Количество дополнительных туристических мест в день:}
    
    Предположим, что гостиница рассчитана на \( D \) дней в тёплый сезон. Тогда ежедневно требуется:
    \[
    \Delta \text{Туристов в день} = \frac{0.6H}{D}
    \]
    
    Однако, без конкретных значений \( H \) и \( D \) можно выразить отношение:
    \[
    \text{Необходимые дополнительные места в день} = 0.6 \times \text{Ежедневная вместимость гостиницы}
    \]
    
    \item \textbf{Количество туристических мест, предоставляемых одним рейсом:}
    
    \[
    \text{Туристических мест в одном рейсе} = 20
    \]
    
    \item \textbf{Количество рейсов, необходимых для обеспечения дополнительной загрузки:}
    
    \[
    \text{Необходимое количество рейсов} = \frac{0.6H}{20}
    \]
    
    \item \textbf{Поскольку гостиница изначально имеет 40\% загрузки при 1 рейсе в день, необходимая загрузка для 100\%:}
    
    \[
    \frac{100\%}{40\%} = 2.5
    \]
    То есть требуется 2.5 раза больше туристических мест.
    
    \item \textbf{Расчёт дополнительных рейсов:}
    
    \[
    \text{Необходимые рейсы в день} = 2.5 \times 1 = 2.5
    \]
    Округляем до целого числа:
    \[
    \text{Необходимые рейсы в день} = 3
    \]
    \[
    \text{Дополнительные рейсы} = 3 - 1 = 2
    \]
\end{enumerate}

\textbf{Ответ:} Необходимо добавить \textbf{2 дополнительных рейса} в день для достижения 100\%-ной загрузки гостиницы.

\section*{Задача 2 (3 балла)}

\textbf{Условие:} 

Необходимо составить драфт письма в государственный орган с просьбой о субсидировании дополнительных рейсов.

\textbf{Решение:}

\begin{flushleft}
\textbf{Министерство Транспорта Российской Федерации} \\
ул. Новая Басманная, д. 20 \\
Москва, 123456 \\
\\
\textbf{Тема:} Запрос на субсидирование дополнительных рейсов для повышения туристического потока на остров \\
\\
Уважаемый [Имя Отчество Получателя], \\
\\
Меня зовут [Ваше Имя], я представляю гостиницу "[Название Гостиницы]" на острове [Название Острова]. Наша гостиница является одной из ведущих в регионе, отличаясь высоким уровнем сервиса, уникальным расположением и разнообразием предоставляемых услуг. Мы гордимся тем, что наши гости выбирают нас за уютную атмосферу, комфортные номера и широкий спектр развлечений. \\
\\
В настоящее время мы сталкиваемся с проблемой низкой загрузки гостиницы в туристический сезон из-за ограниченного количества рейсов на остров. Ежедневный рейс из Южно-Сахалинска предоставляет всего 20 туристических мест, что обеспечивает лишь 40\% загрузку наших номеров. Это ограничение негативно сказывается не только на нашем бизнесе, но и на экономическом развитии всего региона, снижая доходы местных предприятий и уменьшая туристическую привлекательность острова. \\
\\
В связи с вышеизложенным, мы обращаемся к Вам с просьбой о предоставлении субсидии на организацию дополнительных рейсов в тёплый сезон (июль-октябрь). Увеличение количества рейсов позволит повысить доступность острова для туристов, что приведет к увеличению загрузки гостиницы до 100\%, стимулирует развитие инфраструктуры и создаст новые рабочие места. Государственная поддержка в данном вопросе будет способствовать укреплению туристического потенциала региона и увеличению налоговых поступлений. \\
\\
Мы уверены, что реализация данного проекта принесет значительные экономические выгоды как для нашего бизнеса, так и для региона в целом. Готовы предоставить дополнительную информацию и обсудить детали сотрудничества в удобное для Вас время. \\
\\
Надеемся на Ваше положительное решение и поддержку в реализации нашей инициативы. \\
\\
С уважением, \\
[Ваше Имя] \\
[Должность] \\
[Контактная информация] \\
[Электронная почта] \\
[Телефон]
\end{flushleft}

\section*{Задача 3 (3 балла)}

\textbf{Условие:} 

Разработать план Б, изучив альтернативную логистику для туристов на остров и сделать её более комфортной.

\textbf{Решение:}

\begin{enumerate}
    \item \textbf{Инициатива:} \\
    \textbf{Использование морских судов} (выбор вида транспорта и решение проблемы ограниченного сообщения) Предлагается организовать регулярные круизные рейсы между портом Ванино (материк) и портом Холмск (Сахалин). Круизные суда обеспечат комфортное и привлекательное путешествие для туристов, что поможет решить проблему ограниченного сообщения и привлечь больше посетителей на остров.
    
    \item \textbf{Описание идеи:}
    \begin{itemize}
        \item \textbf{Частота рейсов:} Еженедельные рейсы в летний сезон (с мая по сентябрь) и раз в две недели в остальное время года.
        \item \textbf{Вместимость:} Судно рассчитано на 200–300 пассажиров, что позволит обслуживать значительное количество туристов.
        \item \textbf{Удобства: } На борту будут предоставлены комфортабельные каюты различных классов, рестораны с местной кухней, развлекательные программы и экскурсионные предложения, что сделает поездку привлекательной и удобной для туристов.
    \end{itemize}
    
    \item \textbf{Потенциальный спонсор/партнёр:} \\
    \textbfПартнёром может выступить крупная туристическая компания, специализирующаяся на круизных путешествиях, например, «Сахалинское морское пароходство» (SASCO) Компания имеет опыт в организации морских перевозок и заинтересована в расширении туристических услуг.
    
    \item \textbf{Выгоды для партнёра/спонсора:}
    \begin{enumerate}
        \item Выход на новый туристический рынок и привлечение клиентов, интересующихся путешествиями на Сахалин.
        \item Дополнительный источник дохода от продажи билетов и предоставления услуг на борту.
        \item Повышение репутации компании как организатора уникальных и комфортных путешествий.
        \item Возможность получения государственных субсидий или грантов для развития туристической инфраструктуры.
    \end{enumerate}
    
    \item \textbf{Анализ рисков и способы их минимизации:}
    \begin{enumerate}
        \item \textbf{Риск 1: Неблагоприятные погодные условия} Штормы и сильные ветры могут привести к отмене рейсов.
        \begin{itemize}
            \item \textbf{Минимизация:} Использование современных судов, способных безопасно передвигаться в сложных погодных условиях, и разработка гибкого расписания с возможностью переноса рейсов.
        \end{itemize}
        \item \textbf{Риск 2: Риск низкого спроса в межсезонье}Снижение количества туристов в осенне-зимний период может привести к убыткам.
        \begin{itemize}
            \item \textbf{Минимизация:} Разработка специальных предложений и скидок для привлечения туристов в межсезонье, а также организация тематических круизов (например, новогодние туры).
        \end{itemize}
    \end{enumerate}
\end{enumerate}

\textbf{Заключение:} \\
Реализация данного проекта позволит улучшить транспортную доступность Сахалина, повысить комфорт туристов и привлечь новых посетителей на остров.

\end{document}
