\documentclass[a4paper,12pt]{article}

% Кодировка и язык
\usepackage[utf8]{inputenc}
\usepackage[russian]{babel}

% Математические пакеты
\usepackage{amsmath,amsfonts,amssymb}

% Графика
\usepackage{graphicx}
\usepackage{tikz}
\usetikzlibrary{shapes.geometric, calc}
\usepackage{pgfplots}

% Геометрия страницы
\usepackage{geometry}
\geometry{top=2cm, bottom=2cm, left=2.5cm, right=2.5cm}

% Гиперссылки
\usepackage{hyperref}

% Плавающие объекты
\usepackage{float}

% Дополнительные пакеты
\usepackage{venndiagram}

% Настройки заголовка
\title{Домашнее задание}
\author{Студент: \textbf{Ваше Имя Фамилия}}
\date{\today}

\begin{document}

% Титульный лист
\begin{titlepage}
    \centering
    \vspace*{1cm}

    \Huge
    \textbf{Домашнее задание}

    \vspace{0.5cm}
    \LARGE
    По курсу: \textbf{Название Курса}

    \vspace{1.5cm}

    \textbf{Студент: Ваше Имя Фамилия}

    \vfill

    \Large
    Название Вашего Университета\\
    \vspace{0.3cm}
    \today

\end{titlepage}

% Содержание
\tableofcontents
\newpage

% Основной текст
\section{Название Раздела}

\subsection{Задача 1}
\textbf{Условие задачи:}

% Здесь вставьте условие задачи
\begin{itemize}
    \item[a)] Условие пункта а)
    \item[б)] Условие пункта б)
\end{itemize}

\textbf{Решение:}

% Здесь вставьте решение задачи

\vspace{1cm}

\subsection{Задача 2}
\textbf{Условие задачи:}

% Здесь вставьте условие задачи
\begin{itemize}
    \item[a)] Условие пункта а)
    \item[б)] Условие пункта б)
\end{itemize}

\textbf{Решение:}

% Здесь вставьте решение задачи

\vspace{1cm}

% Добавляйте новые задачи по аналогии
% \subsection{Задача 3}
% \textbf{Условие задачи:}
% 
% % Условие задачи
% 
% \textbf{Решение:}
% 
% % Решение задачи
% 
% \vspace{1cm}

\section{Другой Раздел}

\subsection{Задача 3}
\textbf{Условие задачи:}

% Здесь вставьте условие задачи

\textbf{Решение:}

% Здесь вставьте решение задачи

\vspace{1cm}

% Продолжайте добавлять разделы и задачи по мере необходимости

\end{document}
