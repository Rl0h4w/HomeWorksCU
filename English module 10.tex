\documentclass[a4paper,12pt]{article}

% Кодировка и язык
\usepackage[utf8]{inputenc}
\usepackage[russian]{babel}

% Математические пакеты
\usepackage{amsmath,amsfonts,amssymb}

% Графика
\usepackage{graphicx}
\usepackage{tikz}
\usetikzlibrary{shapes.geometric, calc}
\usepackage{pgfplots}

% Геометрия страницы
\usepackage{geometry}
\geometry{top=2cm, bottom=2cm, left=2.5cm, right=2.5cm}

% Гиперссылки
\usepackage{hyperref}

% Плавающие объекты
\usepackage{float}

% Дополнительные пакеты
\usepackage{venndiagram}

% Настройки заголовка
\title{Домашнее задание}
\author{Студент: \textbf{Ваше Имя Фамилия}}
\date{\today}

\begin{document}

% Титульный лист
\begin{titlepage}
    \centering
    \vspace*{1cm}

    \Huge
    \textbf{Домашнее задание}

    \vspace{0.5cm}
    \LARGE
    По курсу: \textbf{English}

    \vspace{1.5cm}

    \textbf{Студент: Ростислав Лохов}

    \vfill

    \Large
    АНО ВО Центральный университет\\
    \vspace{0.3cm}
    \today

\end{titlepage}

% Содержание
\tableofcontents
\newpage

% Основной текст
\section{Stress Management Homework}

\subsection{Условие задачи}

\textbf{Stress Management}\\
HOMEWORK UNIT 3 103\\
READING \& WRITING\\
Study the Student’s Guide to Stress Management and answer the questions:

\begin{enumerate}
    \item In your opinion, why are the stress statistics for university students so high?
    \item Which of the ways to manage stress mentioned in the guide have you tried? What was the result?
    \item Which of the ways to manage stress management mentioned in the guide would you like to try? Why?
    \item Which of the ways to manage stress mentioned in the guide would be completely useless for you? Why?
\end{enumerate}

\textbf{The College Student's Guide to STRESS MANAGEMENT}

Take an already busy life that may include work and family obligations, add college classes and studying, sprinkle in exams, budgeting, and other interests, and then try to have a social life on top of it all. It's easy to see why college students are so stressed.

Let's look at some ways students can alleviate stress, succeed in college, and live healthy, balanced lives.

\subsubsection*{STRESS STATS}
\begin{itemize}
    \item \textbf{Anxiety and Stress Are the Top 2 Mental Health Concerns Facing College Students.}
    \item \textbf{44\%} of college students experience symptoms of depression and anxiety.
    \item \textbf{75\%} of students dealing with depression and anxiety are reluctant to seek assistance.
    \item \textbf{60\%} During the 2020–2021 academic year, over 60\% of students met the criteria for one or more mental health problems.
\end{itemize}

\subsubsection*{Ways to Manage Stress}
\begin{itemize}
    \item \textbf{Exercise:} This is one of the best activities you can engage in to reduce stress. Exercise produces endorphins, the feel-good chemicals in the brain that act as natural painkillers and that can lower stress levels. Try walking, jogging, or yoga.
    \item \textbf{Eat Well:} Did you know that an unhealthy diet can affect your ability to manage stress levels? Equip your body with the nutrition it needs to fight stress. Avoid high-fat, low-sugar foods, and go easy on the caffeine.
    \item \textbf{Have an Outlet:} Find a new hobby, play sports, join a club, paint, draw, garden — something that gives you an outlet from the tension of everyday life.
    \item \textbf{Build a Support System:} Surround yourself with family and friends who lift you up, encourage you, listen without judgment, and provide sound perspective.
    \item \textbf{Make a Plan:} Prioritize your obligations each week, and then schedule time for each — time for studying, working, family and friends, and yourself.
    \item \textbf{Think Positively:} Your thoughts create your reality, and it's time to turn negative thinking around. Try saying positive affirmations such as:
    \begin{itemize}
        \item I am relaxed and calm. I can handle this situation with ease.
        \item No matter the obstacles, I will rise to the challenge.
    \end{itemize}
    \item \textbf{Meditate:} Meditation is a simple way to lower stress that you can do anywhere and at any time. Begin with a simple technique such as deep breathing, do a guided meditation (find these on YouTube), or repeat a mantra.
    \item \textbf{Journal:} Journaling can help you process life's problems and deal with everyday stress.
    \item \textbf{Try Aromatherapy:} Aromatherapy is the use of essential oils to improve one's physical and emotional well-being. Examples include Lavender, Jasmine, Lemon, Bergamot, Clary Sage, Ylang Ylang.
\end{itemize}

\subsection{Решение задачи}

\textbf{Ответ:}

In my opinion, the stress statistics for university students are so high because they have many responsibilities to juggle. They have to study, meet deadlines, sometimes work part-time jobs, and adjust to living away from home. This can be overwhelming and lead to high stress levels.

I have tried exercising and building a support system to manage stress. Going for walks or doing yoga helps me relax. Talking to friends and family makes me feel supported and less anxious.

I would like to try meditation because I think it could help me stay calm and focused. Learning to meditate might help me handle stress better during exams.

Aromatherapy would be completely useless for me because I don't believe that using essential oils would reduce my stress. I prefer more active methods like exercise.
\end{document}
