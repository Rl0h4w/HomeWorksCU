\documentclass[a4paper,12pt]{article}

% Кодировка и язык
\usepackage[utf8]{inputenc}
\usepackage[russian]{babel}

% Математические пакеты
\usepackage{amsmath,amsfonts,amssymb}

% Графика
\usepackage{graphicx}
\usepackage{tikz}
\usetikzlibrary{shapes.geometric, calc}
\usepackage{pgfplots}

% Геометрия страницы
\usepackage{geometry}
\geometry{top=2cm, bottom=2cm, left=2.5cm, right=2.5cm}

% Гиперссылки
\usepackage{hyperref}

% Плавающие объекты
\usepackage{float}

% Дополнительные пакеты
\usepackage{venndiagram}

% Настройки заголовка
\title{Домашнее задание}
\author{Студент: \textbf{Ростислав Лохов}}
\date{\today}

\begin{document}

% Титульный лист
\begin{titlepage}
    \centering
    \vspace*{1cm}

    \Huge
    \textbf{Домашнее задание}

    \vspace{0.5cm}
    \LARGE
    По курсу: \textbf{Линейная Алгебра}

    \vspace{1.5cm}

    \textbf{Студент: Ростислав Лохов}

    \vfill

    \Large
    АНО ВО Центральный университет\\
    \vspace{0.3cm}
    \today

\end{titlepage}

% Содержание
\tableofcontents
\newpage

% Основной текст
\section{Задачи: Домашнее задание}

\subsection{ЗАДАЧА 1 \hfill \textbf{(2 балла)}}
\textbf{Условие задачи:}

Для подпространств \( U, V \subseteq \mathbb{R}^3 \) найдите базис их суммы и пересечения, если
\[
V = \text{span} \left\{
\begin{pmatrix}
1 \\
2 \\
1
\end{pmatrix},
\begin{pmatrix}
1 \\
1 \\
-1
\end{pmatrix},
\begin{pmatrix}
1 \\
3 \\
3
\end{pmatrix}
\right\},
\]
\[
U = \text{span} \left\{
\begin{pmatrix}
1 \\
2 \\
2
\end{pmatrix},
\begin{pmatrix}
2 \\
3 \\
-1
\end{pmatrix},
\begin{pmatrix}
1 \\
1 \\
-3
\end{pmatrix}
\right\}.
\]

\textbf{Решение:}

найдем базис для подпространства V:

\[
\begin{pmatrix}
    1 & 2 & 1 \\
    1 & 1 & -1 \\
    1 & 3 & 3 \\
\end{pmatrix}
\]

Несложно заметить, что третья строка является линейной комбинацией первой и второй строки причем ранг матрицы - 2. тогда базисом будет:

V = \[
\begin{pmatrix}
    1 \\
    2 \\
    1 \\
\end{pmatrix}, 
\begin{pmatrix}
    1 \\ 
    1 \\
    -1\\
\end{pmatrix}
\]

Найдём базис для пространства U:

\[
\begin{pmatrix}
    1 & 2 & 2 \\
    2 & 3 & -1 \\
    1 & 1 & -3\\
\end{pmatrix}
\]

третья строка является линейной комбинацией первой и второй строки, ранг матрицы равен двум, тогда

U = 
\[
\begin{pmatrix}
    1 \\
    2 \\
    2 \\
\end{pmatrix}, 
\begin{pmatrix}
    2 \\
    3 \\ 
    -1 \\
\end{pmatrix}
\]

тогда суммой будут являтся :

U + V = \[
\begin{pmatrix}
    1 \\
    2 \\
    2 \\
\end{pmatrix}, 
\begin{pmatrix}
    2 \\
    3 \\ 
    -1 \\
\end{pmatrix}
\begin{pmatrix}
    1 \\ 
    1 \\
    -1\\
\end{pmatrix}
\]
вектор 1 2 1 будет линейно зависим от других

рассмотрим пресечение пространств, т.е нам нужно найти такую линейную комбинацию совпадения базисов:

\[
a \begin{pmatrix}1 \\ 2 \\ 2\end{pmatrix} + b \begin{pmatrix}2 \\ 3 \\ -1\end{pmatrix} = c \begin{pmatrix}1 \\ 2 \\ 1\end{pmatrix} + d \begin{pmatrix}1 \\ 1 \\ -1\end{pmatrix}
\]

\[
\begin{pmatrix}
    1& 2& -1& -1 \\
    2& 3& -2& -1 \\
    2& -1& -1& 1\\
\end{pmatrix} \cdot \begin{pmatrix}
    a \\
    b \\
    c \\
    d \\
\end{pmatrix} = 0 \Rightarrow \begin{pmatrix}
    a \\
    b \\
    c \\
    d \\
\end{pmatrix} = a\cdot
\begin{pmatrix}
    1 \\
    1 \\
    2 \\
    1
\end{pmatrix}
\]

т.е один вектор.

\vspace{1cm}

\subsection{ЗАДАЧА 2 \hfill \textbf{(2 балла)}}
\textbf{Условие задачи:}

Для подпространств \( U, V \subseteq \mathbb{R}^4 \) найдите базис их суммы и пересечения, если
\[
V = \text{span} \left\{
\begin{pmatrix}
2 \\
1 \\
4 \\
1
\end{pmatrix},
\begin{pmatrix}
-2 \\
-3 \\
3 \\
3
\end{pmatrix}
\right\},
\]
\[
U = \left\{ y \in \mathbb{R}^4 \mid
\begin{pmatrix}
0 & 2 & 1 & 1 \\
3 & 1 & 2 & 1
\end{pmatrix}
y = 0
\right\}.
\]

\textbf{Решение:}

% Здесь вставьте решение задачи

\vspace{1cm}

\subsection{ЗАДАЧА 3 \hfill \textbf{(1,5 балла)}}
\textbf{Условие задачи:}

Проверьте, лежит ли подпространство \( U \) внутри подпространства \( W \):

1. \( U = \text{span} \left\{
\begin{pmatrix}
3 \\
1 \\
-3
\end{pmatrix},
\begin{pmatrix}
1 \\
-2 \\
3
\end{pmatrix}
\right\} \),
\( W = \text{span} \left\{
\begin{pmatrix}
2 \\
3 \\
-6
\end{pmatrix},
\begin{pmatrix}
1 \\
5 \\
-9
\end{pmatrix}
\right\} \);

2. \( U = \text{span} \left\{
\begin{pmatrix}
-1 \\
2 \\
1
\end{pmatrix},
\begin{pmatrix}
-3 \\
3 \\
1
\end{pmatrix}
\right\} \),
\( W = \left\{ y \in \mathbb{R}^3 \mid
\begin{pmatrix}
-3 & -6 & 9 \\
2 & 4 & -6
\end{pmatrix}
y = 0
\right\} \);

3. \( U = \left\{ y \in \mathbb{R}^3 \mid
\begin{pmatrix}
-1 & 2 & 3 \\
2 & 3 & 1
\end{pmatrix}
y = 0
\right\} \),
\( W = \left\{ y \in \mathbb{R}^3 \mid
\begin{pmatrix}
3 & 1 & -2 \\
5 & 4 & -1
\end{pmatrix}
y = 0
\right\} \).

\textbf{Решение:}

% Здесь вставьте решение задачи

\vspace{1cm}

\subsection{ЗАДАЧА 4 \hfill \textbf{(2,5 балла)}}
\textbf{Условие задачи:}

Пусть матрица \( A \in M_{3 \times 3}(\mathbb{R}) \) задана следующим образом:
\[
A =
\begin{pmatrix}
2 & 4 & -2 \\
-4 & -5 & 1 \\
-4 & 1 & -9
\end{pmatrix}.
\]
1. Найдите \( LU \)-разложение для \( A \), в котором \( L \) — нижнетреугольная с 1 на диагонали, а \( U \) — верхнетреугольная.
2. Найдите \( LU \)-разложение для \( A \), в котором \( L \) — нижнетреугольная, а \( U \) — верхнетреугольная с 1 на диагонали.
3. Найдите обратные матрицы к \( L \) и \( U \) из первого пункта.
4. Найдите \( A^{-1} \), используя обратные к \( L \) и \( U \) из первого пункта.
5. Решите систему \( Ax = B \), используя \( LU \)-разложение, где
\[
B =
\begin{pmatrix}
-2 & 2 \\
2 & 1 \\
3 & 3
\end{pmatrix}.
\]

\textbf{Решение:}

\[
\begin{pmatrix}
    1& 0& 0 \\
    2& 1& 0 \\
    2& 0& 1 \\
\end{pmatrix}^{-1} \cdot 
\begin{pmatrix}
    0.5& 0& 0 \\
    0& \frac{1}{3}& 0 \\
    0& 0& 1 \\ 
\end{pmatrix}^{-1} \cdot 
\begin{pmatrix}
    1 & 0 & 0 \\
    0 & 1 & 0 \\
    0 & -9 & 1 \\
\end{pmatrix}^{-1} \cdot
\begin{pmatrix}
    1 & 0 & 0 \\
    0 & 1 & 0 \\
    0 & 0 & -0.25 \\
\end{pmatrix}^{-1} \cdot\begin{pmatrix}
    1& 2& -1 \\
    0& 1& -1 \\
    0& 0& 1 \\
\end{pmatrix}= A
\]

\[
L'U' = A \Longleftrightarrow
\begin{pmatrix}
    2 & 0 & 0 \\
    -4 & 3 & 0 \\
    -4 & 9 & -4 \\
\end{pmatrix} \cdot \begin{pmatrix}
    1& 2& -1 \\
    0& 1& -1 \\
    0& 0& 1 \\
\end{pmatrix} = A
\]

Можно поменять, достаточно домножить на матрицу состоящую из неединичной из разложения, получаем для смены единиц: $L=L'\cdot D^{-1}\land U = D U'$
 Тогда 
\[
  L \cdot U = \begin{pmatrix}
   1 & 0 & 0 \\
   -2 & 1 & 0 \\
   -1 & 3 & 1 \\
   \end{pmatrix}
   \cdot
   \begin{pmatrix}
   2 & 4 & -2 \\
   0 & 3 & -3 \\
   0 & 0 & -4 \\
   \end{pmatrix} = A
\]

\[
L^{-1} = 
\begin{pmatrix} 
    1 & 0 & 0\\
    2 & 1 & 0 \\
    -5 & -3 & 1 \\
\end{pmatrix} \land
U^{-1} = \begin{pmatrix}
\frac{1}{2} & -\frac{2}{3} & \frac{1}{4} \\
0 & \frac{1}{3} & -\frac{1}{4} \\
0 & 0 & -\frac{1}{4} \\
\end{pmatrix}
\]

\[
A^{-1} = U^{-1}L^{-1} = \begin{pmatrix}
-\frac{11}{6} & -\frac{17}{12} & \frac{1}{4} \\
\frac{23}{12} & \frac{13}{12} & -\frac{1}{4} \\
\frac{5}{4} & \frac{3}{4} & -\frac{1}{4} \\
\end{pmatrix}
\]

\vspace{1cm}

\subsection{ЗАДАЧА 5 \hfill \textbf{(2 балла)}}
\textbf{Условие задачи:}

Для оператора \( \varphi: \mathbb{R}^4 \to \mathbb{R}^4 \), заданного матрицей
\[
A =
\begin{pmatrix}
4 & -3 & 3 & -1 \\
2 & -3 & 3 & 1 \\
-1 & -2 & 2 & 3 \\
4 & -3 & 3 & -1
\end{pmatrix},
\]
найдите базис пересечения \( \text{Im} \varphi \cap \ker \varphi \) и суммы \( \text{Im} \varphi + \ker \varphi \).

\textbf{Решение:}
Окей, рассмотрим линейно независимые от других столбцы и строки одновременно, по матрице видно, что строка 3 зависит от строки 2, с коэффициентом -1, видно, что строка 2 выражается, через строку 1 и 4, тогда лн строки - 1 и 4. Что касается столбцов, то столбцец 2 эквивалентен столбцу 3, и столбец 4 выражается как линейная комбинация 1 и 2

\[
A = \begin{pmatrix}
    4 & -3 \\
    2 & -3 \\
    -1 & -2 \\
    4 & -3 \\
\end{pmatrix} \land

A = \begin{pmatrix}
    4 & 2 & -1 & 4 \\
    -1 & 1 & 3 & -1\\
\end{pmatrix}
\]

\[
Ax = 0 \leftrightarrow
\begin{pmatrix}
    1& 0& 0& -1 \\
    0& 1& -1& -1 \\
    0& 0& 0& 0\\
    0& 0& 0& 0\\
\end{pmatrix} x = 0 \Rightarrow x_1 = x_4 \land x_2 = x_3 + x_4  \land x_3 = x_3 \land x_4 = x_4\Rightarrow
    X = x_3\begin{pmatrix}
        0 \\ 1 \\ 1 \\ 0
    \end{pmatrix} + x_4 \begin{pmatrix}
        1 \\ 1 \\ 0 \\ 1
    \end{pmatrix}
\]

\[
Im = span\left(\begin{pmatrix}
    4\\
    2\\
    -1\\
    4\\
\end{pmatrix}, \begin{pmatrix}
    -3 \\
    -3 \\
    -2 \\
    -3 \\
\end{pmatrix}\right)
\]

\[
Im(\varphi) \cap ker(\varphi) \leftrightarrow a\begin{pmatrix}
    4\\
    2\\
    -1\\
    4\\
\end{pmatrix} + b\begin{pmatrix}
    -3 \\
    -3 \\
    -2 \\
    -3 \\
\end{pmatrix} = c \begin{pmatrix}
        0 \\ 1 \\ 1 \\ 0
    \end{pmatrix} + d \begin{pmatrix}
        1 \\ 1 \\ 0 \\ 1
    \end{pmatrix} \Rightarrow 
\]

к улучшеному ступенчатому виду:

\[
\begin{pmatrix}
    1 & 0 & 0& -0.4 \\
    0 & 1 & 0& -0.2 \\
    0 & 0 & 1& 0.8 \\
    0 & 0 & 0 & 0  \\
\end{pmatrix}\
x_1 = 0.4x_4 \land x_2 = 0.2x_4 \land x_3 = -0.8 x_4 \land x_4 = x_4 \Longleftrightarrow
X = x_4\begin{pmatrix}
    0.4 \\
    0.2 \\
    -0.8 \\
    1
\end{pmatrix}
\]

\[
Im(\varphi) + Ker(\varphi) = \left(\begin{pmatrix}
    4\\
    2\\
    -1\\
    4\\
\end{pmatrix}, \begin{pmatrix}
    -3 \\
    -3 \\
    -2 \\
    -3 \\
\end{pmatrix},\begin{pmatrix}
        1 \\ 1 \\ 0 \\ 1
    \end{pmatrix} \right)
\]


\vspace{1cm}

\subsection{ЗАДАЧА 6 \hfill \textbf{(1,5 балла)}}
\textbf{Условие задачи:}

Пусть заданы линейные отображения \( \varphi \) и \( \psi \):
\[
\varphi: \mathbb{R}^2 \to \mathbb{R}^4, \quad \varphi(x) = Ax, \quad \text{где}
\]
\[
A =
\begin{pmatrix}
-3 & -5 \\
1 & 4 \\
3 & 4 \\
-1 & 1
\end{pmatrix},
\]
\[
\psi: \mathbb{R}[x]_{\leq 2} \to \mathbb{R}^4, \quad \psi(f) =
\begin{pmatrix}
2f(0) - 2f'(0) - 4f(1) \\
-3f(0) + 3f'(0) + 6f(1) \\
7f(0) - 3f'(0) - 2f(1) \\
-2f(0) + 2f'(0) + 4f(1)
\end{pmatrix}.
\]
Найдите матрицу линейного оператора \( \phi: \mathbb{R}^4 \to \mathbb{R}^4 \), такого, что \( \text{Im} \phi = \text{Im} \varphi + \text{Im} \psi \). Матрицу нужно записать в стандартном базисе.

\textbf{Решение:}
\[
\psi = \{1, x, x^2\}
\]

$e_1: f(0) = 1, f'(0)=0, f(1) = 1$

$e_2: f(0)=0, f'(0)=1, f(1) = 1$

$e_3: f(0) = 0, f'(0)=0, f(1)=1$

\[
B = \left(\begin{pmatrix}
    -2 \\ 3 \\ 5 \\ 2
\end{pmatrix}, \begin{pmatrix}
    -6 \\ 9 \\ -5 \\ 6
\end{pmatrix}, \begin{pmatrix}
    -4 \\ 6 \\ -2 \\ 4
\end{pmatrix} \right)
\]

Видно, что третий вектор образован от первых двух, т.е сумма 1/5 на вектор 1 + 3/5 на вектор 2

\[
B = \left(\begin{pmatrix}
    -2 \\ 3 \\ 5 \\ 2
\end{pmatrix}, \begin{pmatrix}
    -6 \\ 9 \\ -5 \\ 6
\end{pmatrix}\right)
\]

\[
M = Im(\phi) + Im(\psi)  = \begin{pmatrix}
    -3& -5& -2& -6 \\
    1& 4& 3& 9 \\
    3& 4& 5& -5 \\
    -1& 1& 2& 6 \\
\end{pmatrix}
\]
Видно, что четвертый столбец, образован от линейной комбинации трех остальных( -5 * 1 вектор + 5 * второй вектор - 2 * третий вектор) Таким образом:

\[
M = Im(\phi) + Im(\psi)  = \begin{pmatrix}
    -3& -5& -2 \\
    1& 4& 3 \\
    3& 4& 5 \\
    -1& 1& 2 \\
\end{pmatrix}
\]

\vspace{1cm}

\subsection{ЗАДАЧА 7 \hfill \textbf{(2 балла)}}
\textbf{Условие задачи:}

Пусть заданы два линейных отображения \( \varphi: \mathbb{R}^4 \to \mathbb{R}^2 \) и \( \psi: \mathbb{R}^4 \to \mathbb{R}^2 \) по правилам
\[
\varphi(x) = Ax \quad \text{и} \quad \psi(x) = Bx,
\]
где
\[
A =
\begin{pmatrix}
1 & 2 & -2 & -1 \\
1 & -1 & -2 & 2
\end{pmatrix},
\quad
B =
\begin{pmatrix}
-3 & -2 & 3 & 2
\end{pmatrix}.
\]
Найдите матрицу какого-нибудь линейного оператора \( \phi: \mathbb{R}^4 \to \mathbb{R}^4 \) в стандартном базисе, такого, что \( \ker \varphi \cap \ker \psi = \ker \phi \).

\textbf{Решение:}

\[
A = \begin{pmatrix}
    1 & 2 & -2 & -1 \\
    0 & 1 & 0 & -1 \\
\end{pmatrix}
x_1 = -x_4 + 2x_3  \land x_2 = x_4 \land x_3 = x_3 \land x_4 = x_4
X = x_4\begin{pmatrix}
    -1 \\ 1 \\ 0 \\1
\end{pmatrix} + x_3 \begin{pmatrix}
    2 \\ 0 \\ 1 \\ 0
\end{pmatrix}
\]

\[
\ker(\phi) = \left( 
    \begin{pmatrix}
        -1 \\ 
        1 \\ 
        0 \\ 
        1
    \end{pmatrix}, 
    \begin{pmatrix}
        2 \\ 
        0 \\ 
        1 \\ 
        0
    \end{pmatrix} 
\right) \land \ker(\psi) = \left( 
    \begin{pmatrix} 
        -\dfrac{2}{3} \\ 
        1 \\ 
        0 \\ 
        0 
    \end{pmatrix}, 
    \begin{pmatrix} 
        1 \\ 
        0 \\ 
        1 \\ 
        0 
    \end{pmatrix}, 
    \begin{pmatrix} 
        \dfrac{2}{3} \\ 
        0 \\ 
        0 \\ 
        1 
    \end{pmatrix} 
\right)
\]

Подставляем, в матрицу перехода B ядро получаем

\[
ker(\phi) \cap ker(\psi) = span(\begin{pmatrix}
    1 \\
    1 \\
    1 \\
    1 \\
\end{pmatrix})
\]

теперь нам нужно найти матрицу ядром которой будет это пересчение. Допустим

\[
\phi =
\begin{pmatrix}
1 & -1 & 0 & 0 \\
1 & 0 & -1 & 0 \\
1 & 0 & 0 & -1 \\
0 & 0 & 0 & 0 \\
\end{pmatrix}.
\]

очевидно, что ядром такой матрицы будет единичный вектор.


\vspace{1cm}

\subsection{ЗАДАЧА 8 \hfill \textbf{(3 балла)}}
\textbf{Условие задачи:}

В пространстве \( V = \mathbb{R}^3 \) заданы два линейных оператора \( \varphi, \psi: V \to V \). Известно, что оператор \( \varphi \) в базисе \( g_1, g_2, g_3 \) задан матрицей \( A \), где
\[
g_1 =
\begin{pmatrix}
-1 \\
1 \\
-2
\end{pmatrix},
\quad
g_2 =
\begin{pmatrix}
2 \\
-1 \\
3
\end{pmatrix},
\quad
g_3 =
\begin{pmatrix}
-1 \\
-1 \\
-1
\end{pmatrix},
\]
\[
A =
\begin{pmatrix}
-1 & 3 & -4 \\
-1 & 3 & -4 \\
-2 & 3 & -2
\end{pmatrix}.
\]
Оператор \( \psi \) в базисе \( f_1, f_2, f_3 \) задан матрицей \( B \), где
\[
f_1 =
\begin{pmatrix}
1 \\
-2 \\
2
\end{pmatrix},
\quad
f_2 =
\begin{pmatrix}
-2 \\
3 \\
-2
\end{pmatrix},
\quad
f_3 =
\begin{pmatrix}
2 \\
-2 \\
1
\end{pmatrix},
\]
\[
B =
\begin{pmatrix}
1 & 7 & -9 \\
3 & 6 & -9 \\
3 & 1 & -3
\end{pmatrix}.
\]
Найдите:
1. Матрицу оператора \( \varphi \circ \psi \) в стандартном базисе.
2. Базис пространства \( \ker \varphi + \ker \psi \).
3. Базис пространства \( \text{Im} \varphi \cap \text{Im} \psi \).

\textbf{Решение:}
Для того чтобы найти стадартный базис, достаточно сделать произведение базиса на матрицу на обратный базис, т.е:

\[
GAG^{-1} = \begin{pmatrix}
    2 & 1 & -1 \\
    1 & -1 & -2 \\
    2 & 1 & -1 \\
\end{pmatrix}
\]

\[
FBF^{-1} = \begin{pmatrix}
    1 & -1 & -1 \\
    -1 & 2 & 3 \\
    -1 & 1 & 1 \\
\end{pmatrix}
\]

\[
GAG^{-1} \cdot FBF^{-1} = \begin{pmatrix}
    2 & -1& 0 \\
    4 & -5& -6 \\
    2 & -1& 0 \\
\end{pmatrix}
\]

\[
ker(\phi) = \begin{pmatrix}
    1 & 0.5 & -0.5 \\
    0 & -1/5 & -1.5 \\
\end{pmatrix} \Leftarrow x_1 = x_3 \land x_2 = -x_3 \land x_3 = x_3 \Leftarrow X = x_3 \begin{pmatrix}
    1 \\ -1 \\ 1
\end{pmatrix}
\]

\[
ker(\psi) = \begin{pmatrix}
    -1 \\ -2 \\ 1
\end{pmatrix}
\]

\[
ker(\psi) + ker(\phi) = \left\{\, (1,\ -1,\ 1),\ (-1,\ -2,\ 1) \,\right\}
\]

\[
ker(\psi) \cap ker(\phi) = (0, 3, 0)
\]


\vspace{1cm}

\subsection{ЗАДАЧА 9 \hfill \textbf{(1 балл)}}
\textbf{Условие задачи:}

В пространстве \( \mathbb{R}^4 \) задана билинейная форма \( \beta: \mathbb{R}^4 \times \mathbb{R}^4 \to \mathbb{R} \) по правилу
\[
(x, y) \mapsto x^T By,
\]
где
\[
B =
\begin{pmatrix}
0 & 1 & -1 & 0 \\
1 & -3 & 1 & -1 \\
-1 & 1 & 5 & 4 \\
0 & -1 & 4 & 2
\end{pmatrix}.
\]
И пусть
\[
U = \text{span} \left\{
\begin{pmatrix}
3 \\
4 \\
1 \\
-3
\end{pmatrix},
\begin{pmatrix}
3 \\
5 \\
1 \\
-4
\end{pmatrix}
\right\}.
\]
Найдите базис \( U \cap U^{\perp} \) и \( U + U^{\perp} \).

\textbf{Решение:}

\[
\beta(u_1, y) = u_1^TBy = 3y_1 -5y_2 -6y_3 -6y_4 = 0 
\]
\[
\beta(u_2, y) = u_2^TBy = 4y_1 -7y_2 -9y_3 -9y_4 = 0
\]


\[
U^{\perp} = \text{span} \left( \begin{pmatrix}
    -3 \\ -3 \\ 1 \\ 0
\end{pmatrix}, \begin{pmatrix}
    -3 \\ -3 \\ 0 \\ 1
\end{pmatrix} \right)
\]
\[
U \cap U^{\perp} = \left( \begin{pmatrix}
    -3 \\ -3 \\ -1 \\ 2
\end{pmatrix} \right)
\]

 \[ U + U^{\perp} =
   \left\{ 
   \begin{pmatrix} -3 \\ -3 \\ -1 \\ 2 \end{pmatrix}, 
   \begin{pmatrix} 3 \\ 4 \\ 1 \\ -3 \end{pmatrix}, 
   \begin{pmatrix} -3 \\ -3 \\ 1 \\ 0 \end{pmatrix} 
   \right\}
   \]


\vspace{1cm}

\subsection{ЗАДАЧА 10 \hfill \textbf{(2 балла)}}
\textbf{Условие задачи:}

Пусть
\[
\varphi: \mathbb{R}[x]_{\leq 3} \to \mathbb{R}^2, \quad \varphi(f) =
\begin{pmatrix}
f(1) \\
f''(0)
\end{pmatrix},
\]
и
\[
\psi: \mathbb{R}[x]_{\leq 2} \to \mathbb{R}[x]_{\leq 3}, \quad \psi(f) = \int_0^x f(t) \, dt.
\]
Найдите базис \( \text{Im} \psi \cap \ker \varphi \).

\textbf{Решение:}

\[
Im(\psi) = \left[Ax^3 + Bx^2 + Cx | A, B, C \in \mathbb{R}\right]
\]

\[
ker(\phi) = \begin{cases}
    A + B + C = 0 \\
    2B = 0
\end{cases}
\]

\[
ker(\phi) \cap Im(\psi) = A(x^3-x)
\]

\vspace{1cm}

\subsection{ЗАДАЧА 11 \hfill \textbf{(2 балла)}}
\textbf{Условие задачи:}

Рассмотрим векторное пространство последовательностей
\[
V = \left\{ (a_0, a_1, \dots, a_n, \dots) \mid a_i \in \mathbb{R}, \, a_n = a_{n-1} + a_{n-2} + \dots + a_{n-100} \ \forall n \geq 100 \right\}.
\]
Зададим в нём следующие два подпространства:
\[
U = \left\{ (a_0, a_1, \dots, a_n, \dots) \in V \mid a_{100} = a_{101} \right\},
\]
\[
W = \left\{ (a_0, a_1, \dots, a_n, \dots) \in V \mid a_{200} = 0 \right\}.
\]
Найдите размерности \( U \cap W \) и \( U + W \).

\textbf{Решение:}

Очевидно, что V имеет размерность 100.  Рассмотрим пространство U : $a_{101} = a_{100}...+a_1 = 2a_{99} + ... + 2a_1 + a_0$  Тогда при условии $a_{100} = a_{101}$  приводит к тому, что $a_0 + ... + a_{99} = a_0 + 2a_1 + ... + 2a_{99} \Rightarrow a_1 + ... a_{99} = 0$ Следовательно размерность 99

Теперь рассмотрим W, его размерность также будет 99, т.к $a_{200}=0$ является одним линейным уравнением на начальных 100 элементах. 

Тогда размерностью пересечения этих двух пространств будте 100 - 2 = 98 т.к оба условия являются линейными и независимыми. Тогда сумма будет являтся разностью суммы размерностей двух пространств и размерности пересечений, т.е 99 + 99 - 98  = 100. Таким образом сумма этих подпространств даёт V

\vspace{1cm}

\subsection{ЗАДАЧА 12 \hfill \textbf{(2 балла)}}
\textbf{Условие задачи:}

В пространстве \( \mathbb{R}^4 \) задана симметричная билинейная форма \( \beta: \mathbb{R}^4 \times \mathbb{R}^4 \to \mathbb{R} \) по правилу
\[
(x, y) \mapsto x^T By,
\]
где
\[
B =
\begin{pmatrix}
8 & -5 & 3 & -2 \\
-5 & 3 & -3 & 2 \\
3 & -3 & -9 & 6 \\
-2 & 2 & 6 & -4
\end{pmatrix}.
\]
Для оператора \( \varphi: \mathbb{R}^4 \to \mathbb{R}^4 \) определим форму \( \beta_\varphi \) по правилу
\[
\beta_\varphi(x, y) = \beta(\varphi(x), \varphi(y)).
\]
Определите максимальную размерность подпространства \( \ker \beta + \ker \beta_\varphi \) для всех возможных обратимых операторов \( \varphi: \mathbb{R}^4 \to \mathbb{R}^4 \).

\textbf{Решение:}

Найдем размерность ядра. Для начала решим Bx = 0. Получим после приведения к ступенчатому виду:

\[
\begin{pmatrix}
    1& 0 &6& -4 \\
    0 &1 &9& -6 \\
    0& 0 &0 &0 \\
    0 &0& 0 &0
\end{pmatrix}
\] т.е размерность ядра будет 2. По свойству обратного оператора, он сохраняет размерность пространства оператора. Тогда $U = ker(\beta) \land V = ker(\beta_\phi) \Rightarrow dim(U+V) = dim(U) + dim(V) - dim(U\cap V) $ в таком случае зная первые два слагаемых для максимизации стоит уменьшить размерность пространства пересечений. Наилучшим образом это будет достигаться в нулевом векторе. Тогда размерность суммы подпространств будет равна 4. т.е $dim(U + V) = 4$

\vspace{1cm}

\subsection{ЗАДАЧА 13 \hfill \textbf{(2 балла)}}
\textbf{Условие задачи:}

Пусть в \( \mathbb{R}^4 \) задан вектор \( v \), а линейные операторы \( \varphi \) и \( \psi \) заданы матрицами \( A \) и \( A^T \):
\[
A =
\begin{pmatrix}
4 & 4 & -5 & -10 \\
-2 & -2 & 4 & 7 \\
0 & 0 & 4 & 4 \\
0 & 0 & -2 & -2
\end{pmatrix},
\quad
v =
\begin{pmatrix}
-6 \\
4 \\
-5 \\
6
\end{pmatrix}.
\]
Выясните, верно ли, что
\[
\mathbb{R}^4 = \text{Im} \varphi^{2024} + \ker \psi^{2025} \quad \text{и} \quad \text{Im} \varphi^{2024} \cap \ker \psi^{2025} = \{0\}.
\]
Если это верно, то представьте вектор \( v \) в виде суммы \( v = u + w \), где \( u \in \text{Im} \varphi^{2024} \) и \( w \in \ker \psi^{2025} \), и покажите, что такие \( u \) и \( w \) единственные.

\textbf{Решение:}

После приведения к ступенчатому виду матрицы А получаем:

\[
\begin{pmatrix}
    1&1 & 0& 0\\
    0& 0& 1& 0\\
    0& 0& 0& 1\\
    0& 0& 0& 0\\
\end{pmatrix}
\]
т.е ранг 3, размерность ядра - 1.  как у оператора $\phi $ так и у опрератора $\psi$


\vspace{1cm}

\end{document}
