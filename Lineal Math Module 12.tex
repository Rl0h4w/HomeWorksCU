\documentclass[a4paper,12pt]{article}

% Кодировка и язык
\usepackage[utf8]{inputenc}
\usepackage[russian]{babel}

% Математические пакеты
\usepackage{amsmath,amsfonts,amssymb}

% Графика
\usepackage{graphicx}
\usepackage{tikz}
\usetikzlibrary{shapes.geometric, calc}
\usepackage{pgfplots}
\pgfplotsset{compat=1.18} % Добавлено для устранения предупреждения

% Геометрия страницы
\usepackage{geometry}
\geometry{top=2cm, bottom=2cm, left=2.5cm, right=2.5cm}

% Гиперссылки
\usepackage{hyperref}

% Плавающие объекты
\usepackage{float}

% Дополнительные пакеты
\usepackage{venndiagram}

% Настройки заголовка
\title{Домашнее задание}
\author{Студент: \textbf{Ваше Имя Фамилия}}
\date{\today}

\begin{document}

% Титульный лист
\begin{titlepage}
    \centering
    \vspace*{1cm}

    \Huge
    \textbf{Домашнее задание}

    \vspace{0.5cm}
    \LARGE
    По курсу: \textbf{Линейная Алгебра}

    \vspace{1.5cm}

    \textbf{Студент: Ростислав Лохов}

    \vfill

    \Large
    АНО ВО Центральный университет\\
    \vspace{0.3cm}
    \today

\end{titlepage}

% Содержание
\tableofcontents
\newpage

% Основной текст
\section{Объёмы и движения в евклидовом пространстве}

\subsection{Задача 1}

\[
G = \begin{pmatrix}
    7 & -4 \\
    -4 & 7 \\
\end{pmatrix}
\]

\[
V_1 = \sqrt{\det(G)} = \sqrt{33}
\]

\[
G = \begin{pmatrix}
    7 & -4 & -3 \\
    -4 & 7 & -1 \\
    -3 & -1 & 4 \\
\end{pmatrix}
\]

\[
V_2 = \sqrt{\det(G)} = \sqrt{38}
\]

\subsection{Задача 3}

а) \( ||v_1|| = \sqrt{15} \land ||v_2|| = \sqrt{13} \)

б) \( \cos(\varphi) = \arccos\left(\frac{-7}{\sqrt{195}}\right) \)

в) \( ||v_1 - v_2|| = \sqrt{42} \)

\subsection{Задача 7}

Пусть 
\[
v_1 = \begin{pmatrix}
    2 \\ 0 \\ 0 \\ 0
\end{pmatrix}
\]
Тогда 
\[
v_2 = \begin{pmatrix}
    e \\ f \\ g \\ h
\end{pmatrix}
\]

\[
\langle v_1, v_2 \rangle = 2e = 3 \Rightarrow e = 1.5 \land \langle v_2, v_2 \rangle = f^2 + g^2 + h^2 = 4.75 \Rightarrow h = 0 \land f = \frac{\sqrt{19}}{2} \land g = 0
\]

\[
v_2 = \begin{pmatrix}
    1.5 \\ \frac{\sqrt{19}}{2} \\ 0 \\ 0
\end{pmatrix}
\]

\[
v_3 = \begin{pmatrix}
    k \\ l \\ m \\ n
\end{pmatrix}
\]

\[
\langle v_1, v_3 \rangle = 1 \Rightarrow k = 0.5 \land \langle v_3, v_3 \rangle = 7 \Rightarrow l^2 + m^2 + n^2 = 6.75
\]

\[
v_3 = \begin{pmatrix} 0.5 \\ \dfrac{13}{2\sqrt{19}} \\ \dfrac{\sqrt{1634}}{38} \\ 0 \end{pmatrix}
\]

\subsection{Задача 8}

\[
\det(A-\lambda I) = 0 \Rightarrow 3\lambda^2 +2\lambda^2 -2\lambda - 3 = 0 \Rightarrow \lambda_1 = 1, \quad \lambda_2 = \dfrac{-5 + i\sqrt{11}}{6}, \quad \lambda_3 = \dfrac{-5 - i\sqrt{11}}{6}
\]

\[
v_1 = \begin{pmatrix}
    1 \\ 3 \\ 1
\end{pmatrix} \Rightarrow e_1 = \begin{pmatrix}
    \dfrac{1}{\sqrt{11}} \\[6pt]
    \dfrac{3}{\sqrt{11}} \\[6pt]
    \dfrac{1}{\sqrt{11}}
\end{pmatrix}
\]
Отбросим комплексное решение, поскольку поворот в вещественном поле. Тогда выберем произвольные векторы ортогональные \( e_1 \) и ортонормируем с помощью Грама-Шмидта.

Пусть \( u_1 = \begin{pmatrix} 1 \\ 0 \\ -1 \end{pmatrix} \), тогда \( \langle e_1, u_1 \rangle = 0 \), т.е. ортогонален, после нормирования получим \( e_2 = \begin{pmatrix} \dfrac{1}{\sqrt{2}} \\ 0 \\ -\dfrac{1}{\sqrt{2}} \end{pmatrix} \).

Пусть \( u_3 = \begin{pmatrix} -3 \\ 2 \\ -3 \end{pmatrix} \), тогда \( \langle e_1, u_3 \rangle = 0 \), нормируем и составляем матрицу ортонормированного оператора в каноническом виде. 

\[
Q =
\begin{pmatrix}
\dfrac{1}{\sqrt{11}} & \dfrac{1}{\sqrt{2}} & -\dfrac{3}{\sqrt{22}} \\
\dfrac{3}{\sqrt{11}} & 0 & \dfrac{2}{\sqrt{22}} \\
\dfrac{1}{\sqrt{11}} & -\dfrac{1}{\sqrt{2}} & -\dfrac{3}{\sqrt{22}}
\end{pmatrix}
\]

\[
A' = Q^T A Q
\]

\[
A' =
\begin{pmatrix}
1 & 0 & 0 \\
0 & -\dfrac{5}{6} & -\dfrac{\sqrt{11}}{6} \\
0 & \dfrac{\sqrt{11}}{6} & -\dfrac{5}{6}
\end{pmatrix}
\]

\subsection{Задача 9}

\[
Av_1 = v_2
\]

Нормализируем вектора, получаем третий вектор как ортогональный первым двум, находится через векторное произведение. Строим базис.

\[
A =
\begin{bmatrix}
-\dfrac{2\sqrt{17}}{17} & \dfrac{5\sqrt{1122}}{1122} & \dfrac{7\sqrt{66}}{66} \\
\dfrac{3\sqrt{17}}{17} & -\dfrac{8\sqrt{1122}}{561} & \dfrac{2\sqrt{66}}{33} \\
\dfrac{2\sqrt{17}}{17} & \dfrac{29\sqrt{1122}}{1122} & \dfrac{\sqrt{66}}{66}
\end{bmatrix}
\]

\subsection{Задача 10}

Матрицы неотрицательны, нормированны, также должно выполняться \( Q^T Q = I \), таким образом данным условиям удовлетворяют только перестановочные матрицы.

\subsection{Задача 11}

\[
G = \begin{pmatrix}
2 & 0 \\
0 & \dfrac{2}{3} \\
\end{pmatrix}.
\]

\[
V = \sqrt{\det(G)} = \sqrt{\dfrac{4}{3}}
\]

\[
G = \begin{pmatrix}
2 & 0 & \dfrac{2}{3} \\
0 & \dfrac{2}{3} & 0 \\
\dfrac{2}{3} & 0 & \dfrac{2}{5} \\
\end{pmatrix}.
\]

\[
V = \sqrt{\det(G)} = \sqrt{\dfrac{32}{135}}
\]

\subsection{Задача 12}

\[
G = \begin{pmatrix}
    2 & 3 & 1 \\
    3 & 10 & -4 \\
    1 & -4 & 6 \\
\end{pmatrix}
\]

\[
\begin{pmatrix}
    1 & 0 & 2 \\
    0 & 1 & -1 \\
    0 & 0 & 0 \\
\end{pmatrix}
\]

$v_1, v_2$ - базисные, $v_3$ выражается через $v_1, v_2$ следовательно его произведение зависит только от скалярных произведений $v_1, v_2$

Дополним до базиса $R^3$  $e_3 = (0, 0, 1)$ матрица грамма для $v_1, v_2, e_3$:

\[
G =
\begin{pmatrix}
2 & 3 & 0 \\
3 & 10 & 0 \\
0 & 0 & 1 \\
\end{pmatrix}
\]

\[
C_{fe} = \begin{pmatrix}
    2 & 1 & 0 \\
    1 & 2 & 0 \\
    -2 & -3 & 1 \\
\end{pmatrix}
\]

\[
C_{fe}^{-1} = \begin{pmatrix}
    2/3 & -1/3 & 0 \\
    1/3 & 2/3 & 0 \\
    -1/3 & 4/3 & 1 \\
\end{pmatrix}
\]

\[
B = G(e_1, e_2, e_3) = C_{fe}^T G(v_1, v_2, e_3) \cdot C_{fe} = \begin{pmatrix}
    7/9 & -5/9 & 1/3 \\
    -5/9 & 46/9 & 4/3 \\
    1/3 & 4/3 & 1 \\
\end{pmatrix}
\]

\subsection{Задача 13}

\[
\det(v_1|v_2) = -7
\]

\[
\det(BA) = 0
\]

Просуммируем главные миноры второго порядка для определения коэффициентов при втором члене характеристического многочлена, получаем, что все собственные значения - 0, 2, -2. Мы знаем, что характеристический многочлен от коммутирующих матриц не меняется за исключением нулей. Тогда:

\[
\det(v_1|v_2) \det(AB) = 28
\]

\subsection{Задача 14}
$||v_1|| = \sqrt{14} \land ||v_2|| = \sqrt{14} \land ||v_3|| = \sqrt{114} \land ||v_4|| = \sqrt{114}$

$||u_1|| = \sqrt{14} \land ||u_2|| = \sqrt{114}$

$u_1$ может получиться только из $v_1 \lor v_2$

$u_2$ может получиться только из $v_3 \lor v_4$

$\langle u_1, u_2 \rangle = \langle v_1, v_3 \rangle \Rightarrow \varphi (v_1) = u_1 \land \varphi(v_3) = u_2$

\[
\begin{pmatrix}
    3 & -2 & 8 & 7 \\
    2 & -1 & 5 & 4 \\
    -1 & 2 & -5 & -7 \\
\end{pmatrix} \ Rightarrow \begin{pmatrix}
    1 & 2 & 0 & -3 \\
    0 & 1 & -1 & -2 \\
    0 & 0 & 0 & 0 \\
\end{pmatrix}
\]

\[
v_2 \to 2u_1 - u_2 = \begin{pmatrix}
    -3 \\ -1 \\ 2 
\end{pmatrix}  = u_3
\]

\[
v_4 \to -3 u_1 + 2 u_2 = \begin{pmatrix}
    8 \\ 5 \\ -5
\end{pmatrix} = u_4
\]

\end{document}
