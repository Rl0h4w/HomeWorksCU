\documentclass[a4paper,12pt]{article}
\usepackage[utf8]{inputenc}
\usepackage[russian]{babel}
\usepackage{amsmath,amsfonts,amssymb}
\usepackage{graphicx}
\usepackage{geometry}
\usepackage{hyperref}
\usepackage{venndiagram}
\usepackage{tikz}
\usetikzlibrary{shapes.geometric, calc}
\usepackage{pgfplots}
\usepackage{float}
\usepackage{enumitem} % Для настройки списков

% Параметры страницы
\geometry{top=2cm,bottom=2cm,left=2.5cm,right=2.5cm}
\geometry{a4paper, margin=1in}

% Настройки гиперссылок
\hypersetup{
    colorlinks=true,
    linkcolor=blue,
    filecolor=magenta,      
    urlcolor=cyan,
}

% Заголовок документа
\title{Домашнее задание}
\author{Студент: Лохов Ростислав}
\date{\today}

\begin{document}

% Титульный лист
\begin{titlepage}
    \centering
    \vspace{1cm}

    \Huge
    \textbf{Домашнее задание}

    \vspace{0.5cm}
    \LARGE
    По курсу: \textbf{Математический Анализ}

    \vspace{1.5cm}

    \textbf{Студент: Лохов Ростислав}

    \vfill

    \Large
    АНО ВО Центральный Университет\\
    \vspace{0.3cm}
    \today

\end{titlepage}

% Содержание
\tableofcontents
\newpage

% Основной текст
\section{Непрерывность функции}

% Задача 1
\subsection{Задача 1}
\textbf{(1,5 балла)} \\
Найди односторонние пределы функции в заданной точке.
\begin{enumerate}[label=\alph*)]
    \item $f(x) = \dfrac{x - 4}{x^2 - 6x + 8}$, $x_0 = 4$;
    \item $f(x) = 2^{\dfrac{1}{x}}$, $x_0 = 0$;
    \item 
    \[
    f(x) =
    \begin{cases}
        3 - 2x, & \text{если } x \leq 1, \\
        3x - 5, & \text{если } x > 1,
    \end{cases}
    \]
    $x_0 = 1$.
\end{enumerate}
\textbf{Решение:}


\subsection{Задача 2}
\textbf{(0,5 балла)} \\
Доопредели функцию
\[
f(x) =
\begin{cases}
    \dfrac{\sqrt{1 + 2x} - 1}{\sqrt{3}}, & x = 0, \\
    \sqrt{1 + 2x} - 1, & x \neq 0,
\end{cases}
\]
так, чтобы функция стала непрерывной.
\textbf{Решение:}

% Задача 3
\subsection{Задача 3}
\textbf{(1 балл)} \\
Найди значения $a$ и $k$, при которых функция
\[
f(x) =
\begin{cases}
    x + 2k, & \text{если } x < a, \\
    -x - k, & \text{если } a \leq x \leq a + 5, \\
    2x + k, & \text{если } x > a + 5,
\end{cases}
\]
будет непрерывной.
\textbf{Решение:}

% Задача 4
\subsection{Задача 4}
\textbf{(1 балл)} \\
Пусть функции $f(x)$ и $g(x)$ разрывны в точке $x = x_0$. Что можно сказать о непрерывности функций
\begin{enumerate}[label=\alph*)]
    \item $f(x) + g(x)$;
    \item $f(x)g(x)$?
\end{enumerate}
\textbf{Решение:}

% Задача 5
\subsection{Задача 5}
\textbf{(0,5 балла)} \\
Найди точку разрыва функции
\[
f(x) = \dfrac{x}{|x|}
\]
и определи её тип.
\textbf{Решение:}

% Задача 6
\subsection{Задача 6}
\textbf{(1 балл)} \\
Найди точки разрыва функции $f(x)$ и определи их тип, если
\[
f(x) =
\begin{cases}
    \cos \dfrac{\pi x}{2}, & \text{если } |x| \leq 1, \\
    |x - 1|, & \text{если } |x| > 1.
\end{cases}
\]
\textbf{Решение:}

% Задача 7
\subsection{Задача 7}
\textbf{(1 балл)} \\
Докажи, что данное уравнение имеет хотя бы один корень, лежащий между $a$ и $b$.
\begin{enumerate}[label=\alph*)]
    \item $x^5 - 3x = 1$, $a = 1$, $b = 2$;
    \item $(3 - x) \cdot 3^x = 3$, $a = 2$, $b = 3$.
\end{enumerate}
\textbf{Решение:}

% Задача 8
\subsection{Задача 8}
\textbf{(1 балл)} \\
Пусть функция $f(x)$ непрерывна на $[a; b]$. Докажи, что для любых точек $x_1, x_2 \in [a,b]$ существует $c \in [a; b]$:
\[
f(c) = \dfrac{1}{2} \left( f(x_1) + f(x_2) \right).
\]
\textbf{Решение:}

% Задача 9
\subsection{Задача 9}
\textbf{(0,5 балла)} \\
Найди все значения параметров $a$ и $b$, при которых функция
\[
f(x) =
\begin{cases}
    \dfrac{x^2 - 4}{x - 2}, & \text{если } x < 2, \\
    a x^2 - b x + 3, & \text{если } 2 < x < 3, \\
    2x - a + b, & \text{если } x \geq 3,
\end{cases}
\]
непрерывна на $\mathbb{R}$.
\textbf{Решение:}
\[
f(x) =
\begin{cases}
    x+2, & \text{если } x < 2, \\
    a x^2 - b x + 3, & \text{если } 2 < x < 3, \\
    2x - a + b, & \text{если } x \geq 3,
\end{cases}
\]

Рассматриваем  разносторонние пределы в точках $x_0=2; x_0=3$:

По определению функция называется непрерывной на $\mathbb{R}$ Если функция определена на $\mathbb{R}$ и необходимо обеспечить непрерывность на границах областей.

Тогда рассмотрим левосторонний и правосторннний предел в точке 2, они будут равны 4 и $4a-2b+3$.  

Рассмотрим также левосторонний и правосторонний предел в точке 3, они будут равны $9a-3b+3$, $6-a+b$

Теперь по условию непрерывности получаем: 

\begin{cases}
    4a - 2b = 1 \\
    10a - 4b = 3
\end{cases}

$a = b = 0.5$

% Задача 10
\subsection{Задача 10}
\textbf{(1 балл)} \\
\begin{enumerate}[label=\alph*)]
    \item Выясни, можно ли определить значение $f(0)$ так, чтобы
    \[
    f(x) = \dfrac{1}{1 - e^{1/x}} \quad (x \neq 0)
    \]
    стала непрерывной на $\mathbb{R}$?
    \item Тот же вопрос для функции
    \[
    g(x) = \dfrac{x}{1 - e^{1/x}}, \quad x \neq 0.
    \]
\end{enumerate}
\textbf{Решение:}

a) Рассмотрим левосторонний предел и правосторонний - 1 и $\infty$ соответственно, тогда разрыв второго рода, который нельзя устранить

б) Рассмотрим также, левосторонний и правосторонний предел будут равны 0 и 0 соответственно, тогда функция просто не определена в точке 0, значит можно устранить.

% Задача 11
\subsection{Задача 11}
\textbf{(3 балла)} \\
Существует ли функция, определенная на $\mathbb{R}$ и непрерывная ровно в трёх точках?

\textbf{Решение: }
Да, модифицированная дирихле

f(x) = 
\begin{cases}
    1, x \in \mathbb{Q} \\
    \tan(\cosh(x)), x \in \mathbb{I} \land x\in (-3, -1.5) \\
    0, x \in \mathbb{I} \land x \in (-\infty, -3) \land (-1.5, \infty)
\end{cases}


% Задача 12
\subsection{Задача 12}
\textbf{(0,5 балла)} \\
Существует ли функция, имеющая счётное число разрывов второго рода?

\textbf{Решение:}

Разрыв второго рода - нет предела в точке предела, существует односторонний предел равный бесконечости.

f(x) = tanh(x)tan(h)

% Задача 13
\subsection{Задача 13}
\textbf{(0,5 балла)} \\
Существует ли функция, имеющая несчётное число разрывов второго рода?

\textbf{Решение:}

Да, функция Дирихле - 1, если x рационально, иначе 0 если иррационально без рационального множества

% Задача 14
\subsection{Задача 14}
\textbf{(1 балл)} \\
Определи точки разрыва и исследуй характер этих точек функции
\[
f(x) =
\begin{cases}
    \dfrac{\dfrac{1}{x} - \dfrac{1}{x+1}}{\dfrac{1}{x-1} - \dfrac{1}{x}}, & \text{если } x \in \mathbb{R} \setminus \{ -1, 1, 0 \}, \\
    0, & \text{если } x \in \{ -1, 1, 0 \}.
\end{cases}
\]

\textbf{Решение:}

f(x)=
\begin{cases}
    \frac{x-1}{x+1}, x\notin{-1, 1, 0} \\
    0, x \in {-1, 1, 0}
\end{cases}

\begin{enumerate}
    \item $x_0$ = -1, левосторонний и правосторонний предел будут равны $+\infty$ и $-\infty$  соответственно - разрыв второго рода
    \item $x_0$ = 0, левосторонний и правосторонний предел будут равны $-1$ разрыв первого рода
    \item $x_0$ = 1, левосторонний и правосторонний предел будут равны 0, равны пределу в этой точке, значит разрыва нет и функция непрерывна в этой точке
    
\end{enumerate}

% Задача 15
\subsection{Задача 15}
\textbf{(1 балл)} \\
Докажи, что уравнение $x^5 - 3x = 1$ имеет не менее трёх корней на $\mathbb{R}$.
\textbf{Решение:}
очевидно, что область значений функции - $\mathbb{R}$
Тогда пользуясь теоремой о промежуточном значении для нуля:

\[
f \in [a, b], f(a)\ne f(b); f(a)\cdot f(c) < 0 \Rightarrow \exists c \in [a, b]: f(c) = 0
\]

Тогда рассмотрим в точках -2, -1, 0, 1, 2, f(x) = -27, 1, -1, -3, 25. в таком случае знак сменяется трижды, т.е от -2 до -1 существует 0, от -1 до 0 существует 0, от 1 до 2 существует 0. чтд

% Задача 16
\subsection{Задача 16}
\textbf{(2 балла)} \\
Функция $f$ непрерывна и периодична с периодом $T > 0$. Докажи, что существует точка $x_0$ такая, что

\[
f\left(x_0 + \dfrac{T}{2}\right) = f(x_0).
\]

\textbf{Решение:}

\[
f(x+T)=f(x)
\]


\[
g(x) = f(x_0) - f(x_0 + \frac{T}{2})
\]

поскольку f непрерывна, то и g непрерывна

\[
g(x+T/2) = -(f(x+T/2) + f(x))=-g(x) 
\]

\[
g(T/2) = f(T/2) - f(0) = -g(0)
\]

По теореме о промежуточном значении получаем, что от 0 до T/2 будет 0


% Задача 17
\subsection{Задача 17}
\textbf{(2 балла)} \\
Функция $f : (0; 1) \to \mathbb{R}$ непрерывна, и $(f(x))^2 = 1$ для всех $x \in (0; 1)$. Докажи, что либо $f \equiv 1$, либо $f \equiv -1$.
\textbf{Решение:}

$f(x)=\pm 1$

Тогда по теореме о промежуточных значениях т.к фукнция непрерывна то она принимает значения от -1 до 1, но это не возможно т.к единственные допустимые значения это $\pm 1$ а промежуточных значений не существует. следовательно либо 1 либо -1 $\forall x \in (0, 1)$

% Задача 18
\subsection{Задача 18}
\textbf{(0,5 балла)} \\
Верно ли утверждение леммы о вложенных отрезках, если заменить отрезки интервалами, то есть взять систему
\[
(a_1; b_1) \supset (a_2; b_2) \supset (a_3; b_3) \supset \dots \supset (a_k; b_k) \supset \dots
\]
\textbf{Решение:}

Проблема с открытыми интервалами, в случае если длины интервалов стемятся к нулю, но не включают конечную точку, то пересечение этих интервалов может оказаться пустым, например $\lim_{k \to \infty} = (-\frac{1}{k}; \frac{1}{k})=(0; 0)$

% Задача 19
\subsection{Задача 19}
\textbf{(3 балла)} \\
Определи точки разрыва и исследуй характер этих точек функции
\[
f(x) =
\begin{cases}
    \sin(\pi x), & \text{если } x \in \mathbb{Q}, \\
    0, & \text{если } x \in \mathbb{R} \setminus \mathbb{Q}.
\end{cases}
\]
\textbf{Решение:}

при целых числах мы получаем непрерывность

при нецелых числах мы получаем два варианта: 

Если мы стремимся к x через рациональные числа, то получаем $\sin(\pi x)$ иначе 0. Поскольку $\sin(\pi x)$ при x нецелых различны, то это означает, что существует разрыв первого рода в каждоый такой точке

% Задача 20
\subsection{Задача 20}
\textbf{(1,5 балла)} \\
Найди асимптоты графиков функций:
\begin{enumerate}[label=\alph*)]
    \item $y = \dfrac{(x + 3)^3}{(x - 1)^4}$;
    \item $y = \dfrac{(x + 3)^3}{(x - 1)^3}$;
    \item $y = \dfrac{(x + 3)^3}{(x - 1)^2}$;
    \item $y = \dfrac{(x + 3)^3}{x - 1}$;
    \item $y = \dfrac{3x - 1}{\sqrt[4]{81x^4 + 230x^2 + 10}}$;
    \item $y = \sqrt{16x^2 + x} - \sqrt{x + 1}$;
    \item $y = \sqrt{16x^2 + x} - x - 1$. 
\end{enumerate}
\textbf{Решение:}

\begin{enumerate}[label=\alph*)]
    \item вертикальная - x= 1, горизонтальная y=0
    \item вертикальная - x=1, горизонтальная y=1
    \item вертикальная - x=1, наклонная - y=x
    \item поделим, получим $x^2+10x+\frac{64}{x-1}+37$ рассмотрим внимательно - при $\lim_{x \to 1}$ получаем $\infty$ или разрыв второго порядка. т.е есть вертикальная асимптота x=1, на бесконечности ведет себя как $x^2+10x+37$ т.е нелинейная ассимптота.
    \item $\frac{x(3-1/x)}{|x|\sqrt[4]{81+230/x^2+10/x^4}}$ на $+\infty$ ведёт себя как 1, на $-\infty$ ведёт себя как -1
    \item $\frac{16x^2-1}{\sqrt{16x^2+x}+\sqrt{x+1}}=\frac{x^2(16-1/x^2)}{|x|(\sqrt{16+1/x}+\sqrt{1/x+1/x^2})}$ $k = 4, b = \sqrt{16x^{2}+x}-\sqrt{x+1}+4x$ 
    \item найдём производную, посчитаем ёё пределы: $\frac{32x+1}{\sqrt{16x^2+x}}-1$ асимптоты при плюс минус бесконечности будут равны f(x)=3 и f(x)=-5 соответственно
\end{enumerate}

% Задача 21
\subsection{Задача 21}
\textbf{(3 балла)} \\
Найди асимптоты кривых:
\begin{enumerate}[label=\alph*)]
    \item $x = \dfrac{t^2}{1 - 3t}$, $y = \dfrac{t^3}{1 - 3t}$;
    \item $x = \dfrac{1}{t- t^2} $, $y = \dfrac{1}{t- t^3} $;
    \item $x = \dfrac{e^{2t}}{t - 1}$, $y = \dfrac{e^{-2t}}{t - 1}$.
\end{enumerate}
\textbf{Решение:}

\begin{enumerate}
    \item $\lim_{t \to 1/3}x(t), y(t)=\infty$
    $\lim_{t \to \infty}x(t), y(t) = -\infty$
    $\lim_{t \to \infty} = \infty$
    нет асимптот
    \item при стремлении t к бесконечности y(t)=0
    при стремлении t к -1 x(t)=-0.5. горизонтальная - y=0, вертикальная x=-0.5
    \item вертикальная x=0 и горизонтальная y = 0 при t стремящемся к - бесконечности и +бесконечности соответственно
\end{enumerate}

% Задача 22
\subsection{Задача 22}
\textbf{(1 балл)} \\
Докажи, что функция $f(x) = \sin x$ равномерно непрерывна на множестве $\mathbb{R}$.
\textbf{Решение:}

Определение непрерывности: 
\[
(\forall \varepsilon > 0)(\exists \delta > 0)(\forall x, y \in \mathbb{R}): |x-y| < \delta \Rightarrow |\sinx - \siny| < \varepsilon
\]

\[
2|\sin(\frac{x+y}{2})\cdot \cos(\frac{x-y}{2})| \le 2|\sin(\frac{x-y}{2})|\le |x-y|
\]

Если взять $\delta = \varepsilon$ то при $|x-y|<\delta$

$|sin(x)-sin(y)| \le |x-y| < \delta = \varepsilon$

% Задача 23
\subsection{Задача 23}
\textbf{(1 балл)} \\
Докажи, что $f(x) = \sqrt{x}$ равномерно непрерывна на $E = [0; +\infty)$.
\textbf{Решение:}

\[
|\sqrt{x} - \sqrt{y}| = \frac{|x - y|}{\sqrt{x} + \sqrt{y}}
\]

\[
\sqrt{x} + \sqrt{y} \geq \sqrt{x} \geq \sqrt{x - y} \geq \sqrt{|x - y|}
\]

\[
|\sqrt{x} - \sqrt{y}| = \frac{|x - y|}{\sqrt{x} + \sqrt{y}} \leq \frac{|x - y|}{\sqrt{|x - y|}} = \sqrt{|x - y|}
\]

\[
\sqrt{|x - y|} < \varepsilon \quad \Rightarrow \quad |x - y| < \varepsilon^2
\]

\[
|\sqrt{x} - \sqrt{y}| \leq \sqrt{|x - y|} < \sqrt{\delta} = \varepsilon
\]

\[
|x - y| < \delta \implies |\sqrt{x} - \sqrt{y}| < \varepsilon
\]

чтд


% Задача 24
\subsection{Задача 24}
\textbf{(1 балл)} \\

% Задача 25
\subsection{Задача 25}
\textbf{(1 балл)} \\
Исследуй функцию $f(x) = x \sin x$ на равномерную непрерывность на множестве $E = (0; +\infty)$.
\textbf{Решение:}

% Задача 26
\subsection{Задача 26}
\textbf{(1 балл)} \\
\begin{enumerate}[label=\alph*)]
    \item Функции $f$ и $g$ не являются равномерно непрерывными на $E$. Что можно сказать насчёт равномерной непрерывности их суммы $f + g$ на $E$?
    \item Функции $f$ и $g$ равномерно непрерывны на $E$. Что можно сказать о равномерной непрерывности их произведения $fg$ на $E$?
\end{enumerate}
\textbf{Решение:}

а) Пусть $E = R: f(x)=x^2; g(x) = -x^2$ обе не равномерно непрерывны на R, но сумма их непрерывна и равна 0, с другой стороны сумма $f(x)=x^2; g(x)=x^2 \Rightarrow g(x)+f(x)=2x^2$ тоже не является равномерно непрерывной

б) Пусть $E = R: f(x)=x; g(x)=x \Rightarrow fg(x) = x^2$ не является равномерно непрерывной, но если мы введем ограничение на E, то их произведение будет непрерывно 

% Задача 27
\subsection{Задача 27}
\textbf{(4 балла)} \\
Является ли функция $f(x) = \dfrac{\arctan x}{x}+ \cos x^3$ равномерно непрерывной на множествах:
\begin{enumerate}[label=\alph*)]
    \item $E_1 = (0; 1)$;
    \item $E_2 = (1; +\infty)$.
\end{enumerate}
\textbf{Решение:}

a)
рассмотрим первое слагаемое: $\arctan(x)$ непрерывна на (0;1) и ограничена, т.к $0 < \arctan(x) < \pi/4$ При $x \to 0+ \Rightarrow \frac{\arctan(x)}{x} = 1$ 

Косинус - гладкая функция на R, непрерывна и равномерно непрерывно на подмножествах R

Сумма непрерывных функций является равномерно непрерывной, значит f(x) непрерывна на E

б)
$\arctan(x)$ возрастает и ограничен сверху $\pi/2$ поэтому $\lim_{x \to +\infty}\frac{\arctan(x)}{x}=0$
$\cos(x^3)$ также равномерно непрерывная поскольку увеличивается частота с увеличением моделя аргумента,  можно посмотреть на производную $-3x^2\sin(x^3)$ и она не будет ограничена на E. Следовательно т.к обе функции равномерно непрерывны на E их сумма также равномерно непрерывна.


% Задача 28
\subsection{Задача 28}
\textbf{(3 балла)} \\
Найди модуль непрерывности и исследуй с помощью него функцию $f(x) = \dfrac{1}{x^2}$ на равномерную непрерывность на множествах:
\begin{enumerate}[label=\alph*)]
    \item $E_1 = (0; 1)$;
    \item $E_2 = (5; +\infty)$.
\end{enumerate}
\textbf{Решение:}
а) $|f(x)-f(y)|=\frac{|x-y||x+y|}{x^2y^2}<\frac{2|x-y}{x^2y^2}$ при x стремящемся к 0, знаменатель стремится гораздо бысрее чем числитель, т.е при разности аргументов стремящейся к 0 разность функций стремится к бесконечности. Т.е нет

б) Предположим что y больше или равен x, тогда: $     |f(x) - f(y)| < \dfrac{2|x - y|y}{x^2 y^2} = \dfrac{2|x - y|}{x^2 y} \Longrightarrow    |f(x) - f(y)| < \dfrac{2|x - y|}{25 \cdot 5} = \dfrac{2|x - y|}{125}$ Тогда достаточно взять $\delta < \frac{125\varepsilon}{2}$ что означает, что функция является равномерно непрерывной


\end{document}
