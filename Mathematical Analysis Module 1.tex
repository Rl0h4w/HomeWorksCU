\documentclass[a4paper,12pt]{article}
\usepackage[utf8]{inputenc}
\usepackage[russian]{babel}
\usepackage{amsmath,amsfonts,amssymb}
\usepackage{graphicx}
\usepackage{geometry}
\usepackage{hyperref}
\usepackage{venndiagram}
\usepackage{tikz}
\usetikzlibrary{shapes.geometric, calc}
\usepackage{pgfplots}
\usepackage{float}

% Параметры страницы
\geometry{top=2cm,bottom=2cm,left=2.5cm,right=2.5cm}
\geometry{a4paper, margin=1in}

% Заголовок документа
\title{Домашнее задание}
\author{Студент: Ваше Имя Здесь}
\date{\today}

\begin{document}

% Титульный лист
\begin{titlepage}
    \centering
    \vspace*{1cm}

    \Huge
    \textbf{Домашнее задание}

    \vspace{0.5cm}
    \LARGE
    По курсу: \textbf{Математический анализ}

    \vspace{1.5cm}

    \textbf{Студент: Лохов Ростислав Алексеевич}

    \vfill

    \Large
    АНО ВО Центральный университет\\
    \vspace{0.3cm}
    \today

\end{titlepage}

% Содержание
\tableofcontents
\newpage

% Основной текст
\section{Последовательности}

\subsection{Задача 1}
\textbf{Условие задачи:}
Какие из чисел \(a = 1215\), \(b = 12 555\) — члены последовательности \(x_n = 5 \cdot 3^{2n-3}\)?
\textbf{Решение:}
\begin{itemize}
    \item[a)] $1215 = 3^5*5$ - да, при n=4
    \item[б)] $12555 = 3^4*5*31$ - нет т.к есть 31 в множителях числа
\end{itemize}
\vspace{1cm}

\subsection{Задача 2}
\textbf{Условие задачи:}
\begin{itemize}
    \item[a)] Дай определение последовательности, ограниченной снизу.
    \item[б)] С помощью кванторов запиши, что последовательность не ограничена снизу.
\end{itemize}
\textbf{Решение:}
\begin{itemize}
    \item[a)] $\exists M \in \mathbb{R} : \forall n \in \mathbb{N}, \ x_n \geq M$
    \item[б)] $\nexists M \in \mathbb{R} : \forall n \in \mathbb{N}, \ x_n \geq M$
\end{itemize}


\vspace{1cm}

\subsection{Задача 3}
\textbf{Условие задачи:}
Укажи, какие из следующих утверждений эквивалентны тому, что последовательность ограничена сверху Какие из утверждений являются достаточными для того, чтобы последовательность
была ограничена сверху?:
\begin{itemize}
    \item[a)] \( \exists C \in \mathbb{R} : \forall n \in \mathbb{N} \ x_n \leq C \);
    \item[б)] \( \exists C \in \mathbb{R}, C > 0 : \forall n \in \mathbb{N} \ x_n < C \);
    \item[в)] \( \exists C \in \mathbb{R}, C < 0 : \forall n \in \mathbb{N} \ x_n < C \);
    \item[г)] \( \forall C \in \mathbb{R}, C > 0 \ \forall n \in \mathbb{N} \ x_n < C \);
    \item[д)] \( \exists C \in \mathbb{R} : \exists N \in \mathbb{N} : \forall n \in \mathbb{N}, n > N \ x_n < C \);
    \item[е)] \( \exists C \in \mathbb{R} : \forall n \in \mathbb{N} |x_n| < C \).
\end{itemize}

\vspace{1cm}
\textbf{Решение:}
Эквивалентны: a\\
Достаточны: а, б, е
\subsection{Задача 4}
\textbf{Условие задачи:}
Какие из следующих последовательностей являются ограниченными сверху? Ограниченными снизу? Ограниченными? Приведи обоснование.
\begin{itemize}
    \item[a)] \(a_n = \cos(\pi n)\);
    \item[б)] \(b_n = n \sin n\);
    \item[в)] \(c_n = \frac{3 + 5n}{6 - 4n}\);
    \item[г)] \(d_n = \sum_{k=1}^{n} \frac{3}{k(k+3)}\);
    \item[д)] \(f_n = \sum_{k=1}^{n} \frac{\sqrt{k}}{100}\).
\end{itemize}
\textbf{Решение:}
\begin{itemize}
    \item[a)] $a_n = \cos(\pi n)$ $\cos$ - функция периодичная и ограничена как снизу так и сверху значением $\pm 1$
    \item[б)] $b_n = n \sin(n)$ т.к $\sin$ ограничен и n > 0, то получаем, что функция не ограничена ни снизу ни сверху
    \item[в)] $c_n = \frac{3+5n}{6-4n}$ проверим, при n=1, получим, 4, обе функции являются линейными одна из которых убывает Ограничена сверху и снизу значениями -6.5, 4
    \item[г)] \(d_n = \sum_{k=1}^{n} \frac{3}{k(k+3)}=\sum_{k=1}^{n}(\frac{1}{k}-\frac{1}{k+3})\), ограничен сверху и снизу ззначениями 0.75, 11/6 ;

    \item[д)] Корень из числа при натуральных k - число положительное, а значит, достигает минимума при n=1, ограничена снизу значением 0.01
\end{itemize}
\vspace{1cm}

\subsection{Задача 5}
\textbf{Условие задачи:}
Какие из следующих последовательностей ограничены, а какие — нет:
\begin{itemize}
    \item[a)] \(a_n = \sum_{k=1}^{n} \frac{1}{n+k}\);
    \item[б)] \(b_n = \sum_{k=n+1}^{2n} \frac{1}{k-n}\);
    \item[в)] \(c_n = \frac{1}{n^2} \sum_{j=1}^{n} \sum_{k=1}^{j} k\);
    \item[г)] \(d_n = \frac{1}{n^3} \sum_{j=1}^{n} \sum_{k=1}^{j} k\);
    \item[д)] \(f_n = \frac{1}{(n+1)!} \sum_{k=1}^{n} k \cdot k!\).
\end{itemize}
\textbf{Решение:}
\begin{itemize}
    \item[a)] ограничена 0 и $\ln(2)$   сумма ряда от а до б это примерно интеграл по этим же значениям. те $\int_{k=1}^{n}\frac{1}{n+k}dk=ln(2)$
    \item[б)] Путем нехитрых преобразований получаем, что $b_n = \sum_{k=1}^{n}\frac{1}{k}$ что в свою очередь имеет ограничение снизу
    \item[в)] ограничен снизу значением 1:
    \[
    \frac{1}{2n^2}\sum_{j=1}^{n}\frac{j(j+1)}{2}
    \]
    \[
    \frac{1}{2n^2}(\frac{n(n+1)}{2})(\frac{2n+4}{3})
    \]
    \[
    \quad \frac{n^2 + 3n + 2}{6n}
    \]
    \[
    \frac{n}{6} + 0.5 + \frac{1}{3n}
    \]
    что в свою очередь является возрастающей функцией при n > 0.
    \item[г)] тоже самое сокращение, но получаем $\frac{1}{6}+\frac{1}{2n}+\frac{1}{3n^2}$ - убывает при n>0, ограничена сверху.
    \item[д)] ограничена 0 и 0.5
    \[
    \frac{1}{(n+1)!}\sum_{k=1}^{n}k!(k+1)-k!
    \]
    \[
    \frac{1}{(n+1)!}\sum_{k=2}^{n+1}(k!)-\sum_{k=1}^{k}k!
    \]
    \[
    \frac{(n+1)!-1}{n+1!}
    \]
    \[
    1-\frac{1}{(n+1)!}
    \]
    max = 1 при n->inf; min = 0.5 при n = 1
\end{itemize}
\vspace{1cm}

\subsection{Задача 6}
\textbf{Условие задачи:}
Последовательность \(\{a_n\}\) ограничена. Последовательность \(\{b_n\}\) удовлетворяет условию \( \exists C > 0 : \forall n \in \mathbb{N} |b_n| > C \). Верно ли, что последовательность \( \left\{ \frac{a_n}{b_n} \right\} \) ограничена? Приведи обоснование.
\textbf{Решение:}
\begin{itemize}
    \item[a)] $\exists C:  \forall n \in \mathbb{N} |a_n| \le M, M > 0$\\
    $\exists C: \forall n \in \mathbb{N} |b_n| > C$\\
    $\frac{a_n}{b_n}\le \frac{M}{C}$\\
    $\frac{M}{C}$ - конечное число, отсюда следует, что $\frac{a_n}{b_n}$ ограничена
\end{itemize}
\vspace{1cm}

\subsection{Задача 7}
\textbf{Условие задачи:}
Напиши, что последовательность \( \{a_n\} \) ограничена, используя кванторы. Построй отрицание этого утверждения. Используя записанное отрицание, докажи, что последовательность \(b_n = \frac{n^2}{n+3}\) не является ограниченной.
\textbf{Решение:}
\begin{enumerate}
    \item \[
    \exists M \in \mathbb{R} \, \left( M > 0 \land \forall n \in \mathbb{N} \, (|a_n| \leq M) \right)
    \]
    
    \[
    \overline{\exists M \in \mathbb{R} \, \left( M > 0 \land \forall n \in \mathbb{N} \, (|a_n| \leq M) \right)}
    \]

    \[
    |\frac{n^2}{n+3}| > M
    \]
    
    \[
    \frac{n^2}{n+3} > M
    \]

    \[
    n^2-Mn-3M>0, n>0
    \]

    \[
    n=\frac{M+\sqrt{M^2+12M}}{2}>0
    \]

    \[
    M+\sqrt{M^2+12M}>0, всегда при M>0
    \]
    ЧТД
    
\end{enumerate}

\vspace{1cm}

\subsection{Задача 8}
\textbf{Условие задачи:}
Используя кванторы, сформулируй утверждения:
\begin{itemize}
    \item[a)] последовательность \(\{x_n\}\) не является строго возрастающей;
    \item[б)] последовательность \(\{y_n\}\) не является убывающей;
    \item[в)] последовательность \(\{z_n\}\) не является монотонной.
\end{itemize}
\begin{itemize}
    \item[a)] $\forall n \in \mathbb{N}: X_{n+1}\le X_{n}$
    \item[б)] $\forall n \in \mathbb{N}: X_{n+1}\ge X_{n}$
    \item[в)] $\exists n, m \in \mathbb{N}: z_{n+1}>z_{n} и z_{m+1}<z_{m}$
\end{itemize}
\vspace{1cm}

\subsection{Задача 9}
\textbf{Условие задачи:}
Даны последовательности \(\{x_n\}, \{y_n\}, \{u_n\}, \{v_n\}\), причём \(u_n = x_n + y_n\), \(v_n = x_n y_n\). Докажи или опровергни утверждения:
\begin{itemize}
    \item[a)] Чтобы последовательность \( \{u_n\} \) была строго возрастающей, достаточно, чтобы последовательности \(\{x_n\}\) и \(\{y_n\}\) были строго возрастающими.
    \item[б)] Чтобы последовательность \( \{u_n\} \) была строго убывающей, необходимо, чтобы последовательности \(\{x_n\}\) и \(\{y_n\}\) были убывающими.
    \item[в)] Чтобы последовательность \( \{v_n\} \) была строго возрастающей, достаточно, чтобы последовательности \(\{x_n\}\) и \(\{y_n\}\) были строго возрастающими.
    \item[г)] Чтобы последовательность \( \{v_n\} \) была строго возрастающей, необходимо, чтобы последовательности \(\{x_n\}\) и \(\{y_n\}\) были строго возрастающими.
    \item[д)] Чтобы последовательность \( \{u_n\} \) была строго убывающей, необходимо, чтобы хотя бы одна из последовательностей \(\{x_n\}\) или \(\{y_n\}\) была строго убывающей.
\end{itemize}
\textbf{Решение: }
\begin{itemize}
    \item[a) ] Да, сумма двух возрастающих функци строго возрастает
    \item[б) ] Нет, сумма двух убывающих функций не обязательно строго убывает(одна может убывать медленнее другой)
    \item[в) ] Нет, произведение двух возрастающих функций не обязательно строго возрастает
    \item[г) ] Нет, произведение двух убывающих функций не убывает (например одна из функций может быть отрицательной)
    \item[д) ] Нет, скорость одной функции может быть больше чем другой()
\end{itemize}
\vspace{1cm}

\subsection{Задача 10}
\textbf{Условие задачи:}
Исследуй последовательности на монотонность:
\begin{itemize}
    \item[a)] \(a_n = 5n^2 - 15n - 3\);
    \item[б)] \(b_n = 5n^2 + 15n - 3\);
    \item[в)] \(c_n = \tan \frac{1}{n}\);
    \item[г)] \(d_n = \tan \frac{5}{n}\);
    \item[д)] \(f_n = (-1)^n\);
    \item[е)] \(g_n = -5n + 2(-1)^n\);
    \item[ж)] \(h_n = -2n + 3(-1)^n\).
\end{itemize}
\begin{itemize}
    \item[a)] т.к $n_в$ = 15/10=1.5, что меньше чем 1 - минимальное натуральное, и коэффициент перед квадратом аргумента больше 0, то получаем, что последовательность от 1 до 2 монотонна, от 2 - возрастает
    
    \item[б) ] $n_в$ = -15/10, -1.5, что меньше, чем 1, а т.к $n \in \mathbb{N}$ и ветви вверх - монотонно возрастает со значения n=1
    
    \item[в) ] бесконечно убывает т.к тангенс на интервале от 0 до $pi$ строго возрастает, а 1/n строго убывает, то tan(1/n) убывает

    \item[г) ] аналогично предыдущему заданию

    \item[д) ] страшная функция, не возрастает и не убывает

    \item[e) ] убывает

    \item[ж) ] убывает 
\end{itemize}
\vspace{1cm}

\subsection{Задача 11}
\textbf{Условие задачи:}
Докажи, что последовательность \(\{x_n\}\) строго монотонна, начиная с некоторого номера, если:
\begin{itemize}
    \item[a)] \(x_n = \left( \frac{3}{2} \right)^n - 1000n\);
    \item[б)] \(x_n = \ln(n^2 + 3n + 1) - 2 \ln n\).
\end{itemize}
\textbf{Решение: }\\
\textbf{Определение строго возростающей последовательности: $\forall n \in \mathbb{N} x_{n+1}>x_n}$}\\
\textbf{Определение строго убывающей последовательности: $\forall n \in \mathbb{N} x_{n+1}<x_n}$}

\begin{itemize}
    \item[a) ] Исследование возрастания \[
    1.5^n-1000n<1.5\cdot 1.5^n-1000-1000^n
    \]

    \[
    1.5^n>2000, n>18
    \]  

    Исследование убывания \[
    1.5^n<2000, n < 18
    \]

    Т.е при 0 < n < 18 строго убывает, при n > 18 строго возрастает
    \item[б) ] Исследование возрастания \[
    \ln(1+\frac{3}{n+1}+\frac{1}{(n^2+2n+1)}) > \ln(1+\frac{3}{n}+\frac{1}{n^2})
    \]

    \[
    \ln(\frac{n^2+5n+5}{n^2+2n+1})>\ln(\frac{n^2+3n+1}{n^2})
    \]
    
    \[
    3n^2+5n+1<0, n < 0 - не существует
    \]

    Исследование убывания\[
    3n^2+5n+1>0, n > 0
    \]
    Т.е строго убывает при n > 0
\end{itemize}

\vspace{1cm}

\subsection{Задача 12}
\textbf{Условие задачи:}
Найди наибольший член последовательности:
\begin{itemize}
    \item[a)] \(x_n = \frac{2023^n}{n!}\);
    \item[б)] \(x_n = \frac{n^{40}}{2^n}\);
    \item[в)] \(x_n = 2^{n+100} - 3^{n-100}\).
\end{itemize}

\textbf{Решение: }
\begin{itemize}
    \item[а)] найдем такое n где функция возрастает или убывает $\frac{2023^{n+1}\cdot n!}{2023^n\cdot (n+1)!}=\frac{2023}{n+1}$ следовательно достигает максимума при n = 2022
    \item[б)] $\frac{(n+1)^{40}\cdot 2^n}{n^{40}\cdot 2\cdot 2^{n}}=(1+\frac{1}{n})^{40}\cdot 0.5$ \\
    возьмем логарифм обоих частей, т.к степень довольно большая, то получим $n=\frac{40}{\ln(2)}=57$ Получаем, что 57 число где функция достигает максимума.
    \item[в)] Найдем производную и приравняем к 0, получим $\ln 2 \cdot 2^{n+100} - \ln 3 \cdot 3^{n-100} = 0$\\
    далее $\left(\frac{3}{2}\right)^n = \frac{\ln 2}{\ln 3} \cdot 2^{100} \cdot 3^{100}$\\
    выражаем n: $\frac{\ln\left(\frac{\ln 2}{\ln 3} \cdot 2^{100} \cdot 3^{100}\right)}{\ln\left(\frac{3}{2}\right)} = 440$
\end{itemize}

\vspace{1cm}

\subsection{Задача 13}
\textbf{Условие задачи:}
Найди решение линейного рекуррентного уравнения \(x_{n+1} = 3x_n - 1\), удовлетворяющее условию \(x_1 = -2\).
\textbf{Решение: }
\[
x_{n+1} - ax_{n} = \gamma_n
\]
Решим без неоднородного члена
\[
\lambda^{n+1}-3\lambda^n=0, \lambda \ne 0, \lambda = 3
\]

\[
x_n^{(h)}=C*3^n, \text{h - значение того, что это частное решение(homogenous)}
\]
Найдем теперь частное решение(particular)
\[
x_{n+1} - 3x_n = -1
\]

\[
x_n^{(p)} = A
\]
Подставим, получим
\[
A-3A=-1
\]

\[
x_n^{(p)} = 0.5
\]
Используя теорему о том, что общее решение - сумма частного и общего решения, получим:
\[
x_n = C*3^n+0.5
\]
Используя начальное условие, что $x_1=-2$
\[
x_n = -\frac{5}{6}*3^n+0.5
\]
\vspace{1cm}

\subsection{Задача 14}
\textbf{Условие задачи:}
Найди решение линейного рекуррентного уравнения \(x_{n+2} = -4x_{n+1} - 3x_n\), удовлетворяющее условиям \(x_1 = 2\), \(x_2 = 4\).
\textbf{Решение: }
\[
\lambda^2+4\lambda+3=0, \lambda_1 = -3, \lambda_2=-1
\]

\[
x_n = C_1\cdot (-3)^n+ C_2\cdot (-1)^n
\]
далее подставляем в систему решения $x_1$ и $x_2$, я решил подбором, получил, что $c_1$ = 1, $c_2$ = -5
\[
x_n = (-3)^n - 5(-1)^n.
\]
\vspace{1cm}

\subsection{Задача 15}
\textbf{Условие задачи:}
Найди общее решение линейного рекуррентного уравнения \(x_{n+2} = 2x_{n+1} + 3x_n - 3n\).
\textbf{Решение:}
Общее:
\[
\lambda^2-2\lambda -3 = 0, \lambda = -1, \lambda =  3
\]

\[
x_n = C_1 \cdot 3^n + c_2 \cdot (-1)^n
\]
Частное, Пусть решение имеет вид An+B = $x_n^{(p)}$, тогда
\[
x_{n+2}^{(p)} = 2x_{n+1}^{(p)} + 3x_n^{(p)} - 3n
\]

\[
An + 2A + B = 5An + 5B - 3n
\]
выедляем в систему при одинаковых степенях n:
\[
\begin{cases}
A = 5A - 3 \quad \text{(коэффициент при \(n\))} \\
2A + B = 5B \quad \text{(свободный член)}
\end{cases}
\]
\[
x_n^{(p)} = \frac{3}{4}n + \frac{3}{8}
\]

\[
x_n = x_n^{(h)} + x_n^{(p)} = C_1 \cdot 3^n + C_2 \cdot (-1)^n + \frac{3}{4}n + \frac{3}{8}
\]
решим при условии, что знаем $x_1$ и $x_2$:
\[
x_n = \frac{1}{4} \cdot 3^n - \frac{1}{8} \cdot (-1)^n + \frac{3}{4}n + \frac{3}{8}
\]
\vspace{1cm}

\subsection{Задача 16}
\textbf{Условие задачи:}
Найди решение линейного рекуррентного уравнения \(x_{n+2} + 8x_{n+1} + 16x_n = (-4)^n (-64 + 96n + 192n^2)\), удовлетворяющее условиям \(x_1 = 36\), \(x_2 = -240\).
\textbf{Решение: }
Общее:
\[
\lambda^2+8\lambda + 16 = 0, \lambda = -4
\]

\[
x_n^{(h)} = (C_1+C_2n)\cdot (-4)^n
\]
Возникает проблема в том, что обе части имеют $(-4)^n$ и будет некое совпадение, чего быть не должно, т.к не сможем найти коэффициенты частного решения, кратность данного корня - 2, так что надо домножить на $n^2$, чтобы избежть дублирования.
\[
x_n^{(p)} = n^2\cdot (-4)^n \cdot (An^2+Bn+C)
\]

$
(-4)^{n+2} (A (n+2)^4 + B (n+2)^3 + C (n+2)^2) + 8 \cdot (-4)^{n+1}$\\$ (A (n+1)^4 + B (n+1)^3 + C (n+1)^2) + 16 \cdot$\\$ (-4)^n (A n^4 + B n^3 + C n^2) = (-4)^n (192n^2 + 96n - 64)
$

\[
16 (A (n+2)^4 + B (n+2)^3 + C (n+2)^2) -32 (A (n+1)^4 + B (n+1)^3 + C (n+1)^2) +16 (A n^4 + B n^3 + C n^2)
\]
\[
 = 192n^2 + 96n -64
\]

\[
192A n^2 +384A n +96B n +224A +96B +32C =192n^2 +96n -64
\]

\[
\begin{cases}
192A =192 & \text{(коэффициент при \(n^2\))} \\
384A +96B =96 & \text{(коэффициент при \(n\))} \\
224A +96B +32C =-64 & \text{(свободный член)}
\end{cases}
\]


\[
A=1, \quad B=-3, \quad C=0
\]

\[
x_n^{(p)} = (-4)^n (n^4 -3n^3 )
\]

\[
x_n = x_n^{(h)} + x_n^{(p)} = (C_1 + C_2 n ) (-4)^n + (-4)^n (n^4 -3n^3 )
\]

найдем $C_1$ и $C_2$, получим:
\[
x_n = (-4)^n ( -7 +0 \cdot n +n^4 -3n^3 ) = (-4)^n (n^4 -3n^3 -7 )
\]
\vspace{1cm}

\subsection{Задача 17}
\textbf{Условие задачи:}
Найди решение линейного рекуррентного уравнения \(x_{n+2} = -5x_{n+1} - 4x_n + 10n^2 + 14n - 121\), удовлетворяющее условиям \(x_1 = -18\), \(x_2 = 33\).
\textbf{Решение:}
Общее:
\[
\lambda^2+5\lambda + 4 = 0, \lambda = -1, \lambda = -4
\]

\[
C_1(-1)^n+C_2(-4)^n=0
\]
Частное
\[
x_n^{(p)}=An^2+Bn-C
\]

\[
x_n^{(p+2)} +5x_n^{(p+1)} +4 x_n^{(p)} = A(n+2)^2+B(n+2)-C +5(A(n+1)^2+B(n+1)-C) + 4(An^2+Bn-C) = 10n^2 + 14n - 121
\]

\[
x_n^{(p)} = An^2 + Bn + C = 1 \cdot n^2 + 0 \cdot n -13 = n^2 -13
\]
Общее решение: 
\[
x_n = x_n^{(h)} + x_n^{(p)} = C_1 \cdot (-1)^n + C_2 \cdot (-4)^n + n^2 -13
\]
После того как найдем  $C_1$ и $C_2$ зная x1 и x2 получим: 
\[
x_n = -6(-1)^n +3(-4)^n +n^2 -13
\]

\vspace{1cm}

\subsection{Задача 18}
\textbf{Условие задачи:}
Найди общее решение линейного рекуррентного уравнения \(x_{n+2} = 6x_{n+1} - 36x_n\).
\textbf{Решение: }
\[
\lambda^2-6\lambda + 36 = 0, \lambda = 3\pm 3i\sqrt{3}
\]

\[
x_n^{(h)} = 6^n\cdot (C_1\cos{(n\arctan{\frac{\sqrt{27}}{3}}})+C_2\sin{(n\arctan{\frac{\sqrt{27}}{3}}}))
\]

\[
x_n = 6^n \left( C_1 \cos\left( \frac{\pi}{3} n \right ) + C_2 \sin\left( \frac{\pi}{3} n \right ) \right )
\]
\vspace{1cm}

\subsection{Задача 19}
\textbf{Условие задачи:}
Найди общее решение линейного рекуррентного уравнения \(x_{n+2} = -4x_{n+1} - 16x_n + (61n + 70) \cdot 5^n\).
\textbf{Решение: }
\[
\lambda^2 + 4\lambda + 16 = 0, \lambda = -2 \pm 2i\sqrt{3}
\]

\[
r=4, \theta=-\pi/3=\pi/3
\]

\[
x_n^{(h)} = 4^n(C_1\cos{(\frac{n\pi}{3})} + C_2(\sin(\frac{3\pi}{3})))
\]
Частное:
\[
x_n^{(p)} = (An+B)\cdot 5^n
\]

\[
x_{n+2}^{(p)} + 4x_{n+1}^{(p)} +16x_{n}^{(p)} = (A(n+2)+B)\cdot 5^{n+2} + 4(A(n+1)+B)\cdot 5^{n+1} + 16(An+B)\cdot 5^n = (61n + 70) \cdot 5^n
\]
\[
25An + 50A + 25B = (-36A +61)n + (-20A -36B) +70.
\]

\[
\begin{cases}
25A = -36A + 61 & \text{(коэффициент при } n), \\
50A + 25B = -20A -36B +70 & \text{(свободный член)}.
\end{cases}
\]
B=0, A=1
\[
x_n^{(p)} = (An + B) \cdot 5^n = (1 \cdot n + 0) \cdot 5^n = n \cdot 5^n.
\]

\[
x_n = 4^n \left( C_1 \cos\left( \frac{\pi}{3} n \right ) + C_2 \sin\left( \frac{\pi}{3} n \right ) \right ) + n \cdot 5^n,
\]
\vspace{1cm}

\end{document}
